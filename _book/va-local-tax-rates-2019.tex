% Options for packages loaded elsewhere
\PassOptionsToPackage{unicode}{hyperref}
\PassOptionsToPackage{hyphens}{url}
%
\documentclass[
]{book}
\usepackage{amsmath,amssymb}
\usepackage{lmodern}
\usepackage{ifxetex,ifluatex}
\ifnum 0\ifxetex 1\fi\ifluatex 1\fi=0 % if pdftex
  \usepackage[T1]{fontenc}
  \usepackage[utf8]{inputenc}
  \usepackage{textcomp} % provide euro and other symbols
\else % if luatex or xetex
  \usepackage{unicode-math}
  \defaultfontfeatures{Scale=MatchLowercase}
  \defaultfontfeatures[\rmfamily]{Ligatures=TeX,Scale=1}
\fi
% Use upquote if available, for straight quotes in verbatim environments
\IfFileExists{upquote.sty}{\usepackage{upquote}}{}
\IfFileExists{microtype.sty}{% use microtype if available
  \usepackage[]{microtype}
  \UseMicrotypeSet[protrusion]{basicmath} % disable protrusion for tt fonts
}{}
\makeatletter
\@ifundefined{KOMAClassName}{% if non-KOMA class
  \IfFileExists{parskip.sty}{%
    \usepackage{parskip}
  }{% else
    \setlength{\parindent}{0pt}
    \setlength{\parskip}{6pt plus 2pt minus 1pt}}
}{% if KOMA class
  \KOMAoptions{parskip=half}}
\makeatother
\usepackage{xcolor}
\IfFileExists{xurl.sty}{\usepackage{xurl}}{} % add URL line breaks if available
\IfFileExists{bookmark.sty}{\usepackage{bookmark}}{\usepackage{hyperref}}
\hypersetup{
  pdftitle={Virginia Local Tax Rates, 2019},
  pdfauthor={Stephen C. Kulp, Weldon Cooper Center for Public Service, University of Virginia},
  hidelinks,
  pdfcreator={LaTeX via pandoc}}
\urlstyle{same} % disable monospaced font for URLs
\usepackage{color}
\usepackage{fancyvrb}
\newcommand{\VerbBar}{|}
\newcommand{\VERB}{\Verb[commandchars=\\\{\}]}
\DefineVerbatimEnvironment{Highlighting}{Verbatim}{commandchars=\\\{\}}
% Add ',fontsize=\small' for more characters per line
\usepackage{framed}
\definecolor{shadecolor}{RGB}{248,248,248}
\newenvironment{Shaded}{\begin{snugshade}}{\end{snugshade}}
\newcommand{\AlertTok}[1]{\textcolor[rgb]{0.94,0.16,0.16}{#1}}
\newcommand{\AnnotationTok}[1]{\textcolor[rgb]{0.56,0.35,0.01}{\textbf{\textit{#1}}}}
\newcommand{\AttributeTok}[1]{\textcolor[rgb]{0.77,0.63,0.00}{#1}}
\newcommand{\BaseNTok}[1]{\textcolor[rgb]{0.00,0.00,0.81}{#1}}
\newcommand{\BuiltInTok}[1]{#1}
\newcommand{\CharTok}[1]{\textcolor[rgb]{0.31,0.60,0.02}{#1}}
\newcommand{\CommentTok}[1]{\textcolor[rgb]{0.56,0.35,0.01}{\textit{#1}}}
\newcommand{\CommentVarTok}[1]{\textcolor[rgb]{0.56,0.35,0.01}{\textbf{\textit{#1}}}}
\newcommand{\ConstantTok}[1]{\textcolor[rgb]{0.00,0.00,0.00}{#1}}
\newcommand{\ControlFlowTok}[1]{\textcolor[rgb]{0.13,0.29,0.53}{\textbf{#1}}}
\newcommand{\DataTypeTok}[1]{\textcolor[rgb]{0.13,0.29,0.53}{#1}}
\newcommand{\DecValTok}[1]{\textcolor[rgb]{0.00,0.00,0.81}{#1}}
\newcommand{\DocumentationTok}[1]{\textcolor[rgb]{0.56,0.35,0.01}{\textbf{\textit{#1}}}}
\newcommand{\ErrorTok}[1]{\textcolor[rgb]{0.64,0.00,0.00}{\textbf{#1}}}
\newcommand{\ExtensionTok}[1]{#1}
\newcommand{\FloatTok}[1]{\textcolor[rgb]{0.00,0.00,0.81}{#1}}
\newcommand{\FunctionTok}[1]{\textcolor[rgb]{0.00,0.00,0.00}{#1}}
\newcommand{\ImportTok}[1]{#1}
\newcommand{\InformationTok}[1]{\textcolor[rgb]{0.56,0.35,0.01}{\textbf{\textit{#1}}}}
\newcommand{\KeywordTok}[1]{\textcolor[rgb]{0.13,0.29,0.53}{\textbf{#1}}}
\newcommand{\NormalTok}[1]{#1}
\newcommand{\OperatorTok}[1]{\textcolor[rgb]{0.81,0.36,0.00}{\textbf{#1}}}
\newcommand{\OtherTok}[1]{\textcolor[rgb]{0.56,0.35,0.01}{#1}}
\newcommand{\PreprocessorTok}[1]{\textcolor[rgb]{0.56,0.35,0.01}{\textit{#1}}}
\newcommand{\RegionMarkerTok}[1]{#1}
\newcommand{\SpecialCharTok}[1]{\textcolor[rgb]{0.00,0.00,0.00}{#1}}
\newcommand{\SpecialStringTok}[1]{\textcolor[rgb]{0.31,0.60,0.02}{#1}}
\newcommand{\StringTok}[1]{\textcolor[rgb]{0.31,0.60,0.02}{#1}}
\newcommand{\VariableTok}[1]{\textcolor[rgb]{0.00,0.00,0.00}{#1}}
\newcommand{\VerbatimStringTok}[1]{\textcolor[rgb]{0.31,0.60,0.02}{#1}}
\newcommand{\WarningTok}[1]{\textcolor[rgb]{0.56,0.35,0.01}{\textbf{\textit{#1}}}}
\usepackage{longtable,booktabs,array}
\usepackage{calc} % for calculating minipage widths
% Correct order of tables after \paragraph or \subparagraph
\usepackage{etoolbox}
\makeatletter
\patchcmd\longtable{\par}{\if@noskipsec\mbox{}\fi\par}{}{}
\makeatother
% Allow footnotes in longtable head/foot
\IfFileExists{footnotehyper.sty}{\usepackage{footnotehyper}}{\usepackage{footnote}}
\makesavenoteenv{longtable}
\usepackage{graphicx}
\makeatletter
\def\maxwidth{\ifdim\Gin@nat@width>\linewidth\linewidth\else\Gin@nat@width\fi}
\def\maxheight{\ifdim\Gin@nat@height>\textheight\textheight\else\Gin@nat@height\fi}
\makeatother
% Scale images if necessary, so that they will not overflow the page
% margins by default, and it is still possible to overwrite the defaults
% using explicit options in \includegraphics[width, height, ...]{}
\setkeys{Gin}{width=\maxwidth,height=\maxheight,keepaspectratio}
% Set default figure placement to htbp
\makeatletter
\def\fps@figure{htbp}
\makeatother
\setlength{\emergencystretch}{3em} % prevent overfull lines
\providecommand{\tightlist}{%
  \setlength{\itemsep}{0pt}\setlength{\parskip}{0pt}}
\setcounter{secnumdepth}{5}
\usepackage{booktabs}
\ifluatex
  \usepackage{selnolig}  % disable illegal ligatures
\fi
\usepackage[]{natbib}
\bibliographystyle{apalike}

\title{Virginia Local Tax Rates, 2019}
\usepackage{etoolbox}
\makeatletter
\providecommand{\subtitle}[1]{% add subtitle to \maketitle
  \apptocmd{\@title}{\par {\large #1 \par}}{}{}
}
\makeatother
\subtitle{38th Annual Edition: Information for All Cities and Counties and Selected Incorporated Towns}
\author{Stephen C. Kulp, Weldon Cooper Center for Public Service, University of Virginia}
\date{2021}

\begin{document}
\maketitle

{
\setcounter{tocdepth}{1}
\tableofcontents
}
\hypertarget{introduction}{%
\chapter*{Introduction}\label{introduction}}
\addcontentsline{toc}{chapter}{Introduction}

\hypertarget{foreward}{%
\section*{Foreward}\label{foreward}}
\addcontentsline{toc}{section}{Foreward}

This is the thirty-eighth edition of the Cooper Center's annual publication about the tax rates of Virginia's local governments. In addition to information about tax rates, the publication contains details about tax administration, valuation methods, and due dates. There is also information on water and sewer rates, waste disposal charges and numerous other aspects of local government finance. This comprehensive guide to local taxes is based on information gathered in the spring, summer, and early fall of 2019. The study includes all of Virginia's 38 independent cities and 95 counties and 118 of the 190 incorporated towns. The included towns account for 92 percent of the Commonwealth's population in towns\(^1\). The study also contains information from several outside sources, including two Department of Taxation studies, 2019 Legislative Summary and The 2017 Assessment/Sales Ratio Study, as well as Department of Taxation information on the assessed value of real estate by type of property. We also used the Auditor of Public Accounts' Comparative Report of Local Government Revenues and Expenditures, Year Ended June 30, 2018, the Commission on Local Governments' Report on Proffered Cash Payments and Expenditures by Virginia's Counties, Cities and Towns, 2017-2018, and the Depart-ment of Housing and Community Development's Virginia Enterprise Zone Program 2018 Grant Year Annual Report.

\hypertarget{organization-of-the-book}{%
\section*{Organization of the Book}\label{organization-of-the-book}}
\addcontentsline{toc}{section}{Organization of the Book}

The study is divided into 26 sections. Section 1 is a reprint of the ``Local Tax Legislation'' section of the Department of Taxation's 2019 Legislative Summary. The original Department of Taxation report is available at its website\(^2\). Sections 2 through 26 cover specific taxes, fees, service charges, cash proffers, enterprise zones, and financial documents on the web. Most of the data came from a detailed web-based ques-tionnaire sent to all cities, counties, and incorporated towns (see Appendix A for a printed version). Appendix B provides a listing of names, phone numbers, and email addresses, when available, of respondents and non-respondents to the questionnaire. Appendix C shows the percentage share of total local taxes represented by each specific tax for each locality based on data from the Auditor of Public Accounts for fiscal year 2018. Information is provided for each city and county and for 38 populous incorporated towns. Finally, Appendix D contains 2018 population estimates for cities, counties and towns from the Cooper Center's Demographics Research Group. The population information is provided to give readers some perspective on the relative size of localities.

Please note that the web addresses provided in this publication were good at the time this text was printed. However, some links are unstable and may not work with certain browsers or they may be modified or withdrawn subsequent to publication.

\hypertarget{about-the-survey}{%
\section*{About the Survey}\label{about-the-survey}}
\addcontentsline{toc}{section}{About the Survey}

In 2019, localities could choose between online or printed versions of the questionnaire. The Cooper Center has made its best efforts to accurately refl ect in this report the responses of localities based on the survey or follow-up queries.

In the tables three dots (\ldots) are used to show there was no response and ``N/A'' is used to indicate ``not applicable.'' Readers may use the telephone/email list in Appendix B to contact local offi cials in order to obtain clarification and additional detail.

\hypertarget{some-components-of-local-taxes}{%
\section*{Some Components of Local Taxes}\label{some-components-of-local-taxes}}
\addcontentsline{toc}{section}{Some Components of Local Taxes}

This book deals mainly with local sources of revenue for local governments. Though localities might also receive federal and state resources, an important part of local funding comes from local sources. The Auditor of Public Accounts, Comparative Report of Local Government Revenues and Expenditures provides data on these local sources. The fol-lowing analysis uses the data from their report for the year ended June 30, 2018.

A common misperception is that nearly all local tax revenue comes from the real property tax. True, the real property tax is the dominant source, accounting for 61.9 percent of city-county tax revenue in FY 2018, the latest year available (see text table below). But three other taxes---the personal property tax, the local option sales and use tax, and the business license tax---together accounted for 24.5 percent of total tax revenue. The remaining 14.6 percent of tax revenue came from more than a dozen other taxes.

\begin{table}

\caption{\label{tab:unnamed-chunk-2}Sources of Virginia Local Government Tax Revenue, FY 2018}
\centering
\begin{tabular}[t]{l|l|r}
\hline
Tax & Amount (\$) & \% of Total\\
\hline
Total taxes & \$17,967,385,766 & 100.00\\
\hline
Real property & \$10,946,877,675 & 60.93\\
\hline
Personal property & \$2,370,758,768 & 13.19\\
\hline
Local option sales and use & \$1,239,855,163 & 6.90\\
\hline
Business license & \$771,958,263 & 4.30\\
\hline
Restaurant meals & \$612,940,580 & 3.41\\
\hline
Public service corporation property & \$412,121,081 & 2.29\\
\hline
Consumer utility & \$327,627,947 & 1.82\\
\hline
Hotel and motel room & \$244,412,96 & 1.36\\
\hline
Machinery and tools & \$233,076,157 & 1.30\\
\hline
Motor vehicle license & \$197,705,384 & 1.10\\
\hline
Recordation and will & \$126,458,487 & 0.70\\
\hline
Bank stock & \$117,199,137 & 0.65\\
\hline
Other local taxes & \$92,124,397 & 0.51\\
\hline
Tobacco & \$65,150,996 & 0.36\\
\hline
Coal, oil, and gas & \$28,510,002 & 0.16\\
\hline
Admission & \$21,815,169 & 0.12\\
\hline
Franchise license & \$16,362,103 & 0.09\\
\hline
Merchants' Capital & \$14,301,188 & 0.08\\
\hline
Penalties and interest & \$128,130,305 & 0.71\\
\hline
\end{tabular}
\end{table}

There are six localities where the real property tax is not dominant. Bath and Surry counties have large power plants that pay public service corporation property taxes that overwhelm other sources. Buchanan County has rich mineral deposits subject to local severance taxes that exceed the real property tax. Covington City and Alleghany County receive large shares of revenue from machinery and tools taxes on MeadWestvaco's large paperboard manufacturing facility. Finally, the small city of Norton, the least populous independent city in Virginia\(^3\) (3,908 in 2018) received almost as much money from the local option sales and use tax as from the real property tax. In the remaining 127 cities and counties where the real property tax is dominant, its relative importance varies from 30.3 percent of total tax revenue in Galax City to 78.8 percent in Lancaster County (see Appendix C).

Thirty-six cities (two cities--Hopewell and Petersburg--did not provide information for the 2018 Comparative Report) and 95 counties imposed four of the taxes shown in the previous table---the real property tax, the personal property tax, the local option sales and use tax, and the public service corporation property tax. Most, but not all, localities imposed recordation and will taxes, consumer utility taxes, motor vehicle license taxes, and taxes on manufacturers' machinery and tools. Nonetheless, as shown in the next text table, there are a number of taxes, a few of them signifi cant sources of revenue, which are not levied by all localities. Also, some of the taxes are used so sparingly that their revenue yield is very low.

\begin{table}

\caption{\label{tab:unnamed-chunk-3}Number of Virginia Localities Imposing Taxes by Type, FY 2018}
\centering
\begin{tabular}[t]{l|r|r|r}
\hline
Tax & Cities & Counties & Total\\
\hline
Real property & 36 & 95 & 131\\
\hline
Personal property & 36 & 95 & 131\\
\hline
Local option sales and use & 36 & 95 & 131\\
\hline
Public service corporation property & 36 & 95 & 131\\
\hline
Consumer utility & 36 & 92 & 128\\
\hline
Recordation and wills & 32 & 93 & 125\\
\hline
Motor vehicle license & 32 & 86 & 118\\
\hline
Machinery and tools property & 31 & 85 & 116\\
\hline
Bank stock & 36 & 64 & 100\\
\hline
Hotel and motel room & 32 & 67 & 99\\
\hline
Business license & 36 & 52 & 88\\
\hline
Restaurant meals & 36 & 49 & 85\\
\hline
Franchise license & 11 & 37 & 48\\
\hline
Merchants’ capital & 1 & 43 & 44\\
\hline
Tobacco & 29 & 2 & 31\\
\hline
Admission & 18 & 3 & 21\\
\hline
Coal, oil, and gas & 1 & 6 & 7\\
\hline
Other local taxes & 23 & 49 & 72\\
\hline
\end{tabular}
\end{table}

There are three major reasons why local governments do not to impose some taxes: (1) The locality lacks a tax base for a particular tax (e.g., a locality must have a bank in order to apply a bank stock tax and a locality must have taxable mineral deposits to impose coal, oil, and gas taxes). (2) The locality is faced with state restrictions (e.g., county excise taxes on hotel and motel room rental have tax rate restrictions imposed by the state; county restaurant meals taxes must be approved in a voter referendum; tobacco taxes are permitted in only two counties; and county admissions taxes are subject to many restrictions). In regard to the busi-ness, professional, and occupational license tax (BPOL tax), counties must choose either the BPOL tax or the merchants' capital tax. Counties are not permitted to impose a business license tax within the boundaries of an incorporated town situated within the county without permission of the town. This means that counties with large shares of business activity within towns are motivated to impose a merchants' capital tax that can be applied countywide. (3) The locality chooses not to impose a permitted tax (e.g., Richmond City, a community with a large cigarette manufacturing plant, has not adopted a consumer tobacco tax even though all cities are granted the authority to levy such a tax).

\hypertarget{partnership-with-lexisnexis}{%
\section*{Partnership with LexisNexis}\label{partnership-with-lexisnexis}}
\addcontentsline{toc}{section}{Partnership with LexisNexis}

The Weldon Cooper Center for Public Service is partner-ing with the publisher LexisNexis to produce the annual Tax Rates books. The Cooper Center still prepares and distributes the survey and writes up the results. LexisNexis publishes the book and fulfi lls orders from interested parties. This arrangement allows us to concentrate on providing the most accurate and up-to-date information about Virginia tax rates and leverages LexisNexis' considerable expertise in production and distribution of the annual volume. We hope the arrangement will lead to continued improvements in our Virginia Local Tax Rates series.

\hypertarget{study-personnel}{%
\section*{Study Personnel}\label{study-personnel}}
\addcontentsline{toc}{section}{Study Personnel}

Stephen C. Kulp, Research Specialist at the Center for Economic and Policy Studies, was responsible for work on the project. He refined the new database, administered the survey, translated the results into tables, checked relevant code sections, assisted with the development of the web-based questionnaire, and made appropriate changes in the text. Jennifer Nelson, of the Cooper Center's Publications Section, designed the cover. Cooper Center employee Albert W. Spengler, who authored this study for a number of years prior to 1991, laid the foundation for the study when it was his responsibility.

The strong support for this publication by the Virginia Association of Counties and the Virginia Municipal League helps ensure our continued efforts to provide this resource as a basic reference on Virginia local taxes.

\hypertarget{final-comments}{%
\section*{Final Comments}\label{final-comments}}
\addcontentsline{toc}{section}{Final Comments}

The Cooper Center is grateful to the many local officials throughout the commonwealth who supplied the survey information presented in this study. Their willingness to provide information and their patience in answering follow-up questions is what makes this book successful. The high response rates could not have been achieved without their cooperation. Corrections to the text or suggestions for possible changes or additions to future editions can be made using the email address and phone number listed below.

Stephen C. Kulp

Research Specialist

Center for Economic and Policy Studies

Weldon Cooper Center for Public Service

University of Virginia

Charlottesville

February 2020

\(^1\) Locality population figures are based on estimates developed by the Demographics Research Group of the Weldon Cooper Center for Public Service. See Appendix D.

\(^2\) \url{https://tax.virginia.gov/legislative-summary-reports}

\(^3\) Weldon Cooper Center for Public Service, University of Virginia. \url{https://demographics.coopercenter.org/population-estimates-age-sex-race-hispanic-towns/}

\hypertarget{summary-of-legislative-changes-in-local-taxation-in-2019}{%
\chapter{Summary of Legislative Changes in Local Taxation in 2019}\label{summary-of-legislative-changes-in-local-taxation-in-2019}}

\hypertarget{general-provisions}{%
\section{General Provisions}\label{general-provisions}}

\hypertarget{local-license-tax-on-mobile-food-units}{%
\subsection{Local License Tax on Mobile Food Units}\label{local-license-tax-on-mobile-food-units}}

Senate Bill 1425 (Chapter 791) provides that when the owner of a new business that operates a mobile food unit has paid a license tax as required by the locality in which the mobile food unit is registered, the owner is not required to pay a license tax to any other locality for conducting business from such mobile food unit in such a locality.\(^1\)

This exemption from paying the license tax in other localities expires two years after the payment of the initial license tax in the locality in which the mobile food unit is registered. During the two year exemption period, the owner is entitled to exempt up to three mobile food units from license taxation in other localities. However, the owner of the mobile food unit is required to register with the Commissioner of the Revenue or Director of Finance in any locality in which he conducts business from such mobile food unit, regardless of whether the owner is exempt from paying license tax in the locality.

This Act defines ``mobile food unit'' as a restaurant that is mounted on wheels and readily moveable from place to place at all times during operation. It also defines ``new business'' as a business that locates for the first time to do business in a locality. A business will not be deemed a new business based on a merger, acquisition, similar business combination, name change, or a change to its business form.

Without the exemption provided in this Act, localities are authorized to impose business, professional and occupational license (BPOL) taxes upon local businesses. Generally, the BPOL tax is levied on the privilege of engaging in business at a definite place of business within a Virginia locality. Businesses that are mobile, however, can be subject to license taxes or fees in multiple localities in certain situations.

Effective: July 1, 2019

Added: § 58.1-3715.1

\hypertarget{local-gas-road-improvement-tax-extension-of-sunset-provision}{%
\subsection{Local Gas Road Improvement Tax; Extension of Sunset Provision}\label{local-gas-road-improvement-tax-extension-of-sunset-provision}}

House Bill 2555 (Chapter 24) and Senate Bill 1165 (Chapter 191) extend the sunset date for the local gas road improve-ment tax from January 1, 2020 to January 1, 2022. The authority to impose the local gas road improvement tax was previously scheduled to sunset on January 1, 2020.

The localities that comprise the Virginia Coalfield Economic Development Authority may impose a local gas road improvement tax that is capped at a rate of one percent of the gross receipts from the sale of gases severed within the locality. Under current law, the revenues generated from this tax are allocated as follows: 75\% are paid into a special fund in each locality called the Coal and Gas Road Improvement Fund, where at least 50\% are spent on road improvements and 25\% may be spent on new water and sewer systems or the construction, repair, or enhancement of natural gas systems and lines within the locality; and the remaining 25\% of the revenue is paid to the Virginia Coal-fi eld Economic Development Fund. The Virginia Coalfi eld Economic Development Authority is comprised of the City of Norton, and the Counties of Buchanan, Dickenson, Lee, Russell, Scott, Tazewell, and Wise.

Effective: July 1, 2019

Amended: § 58.1-3713

\hypertarget{private-collectors-authorized-for-use-by-localities-to-collect-delinquent-debts}{%
\subsection{Private Collectors Authorized for Use by Localities to Collect Delinquent Debts}\label{private-collectors-authorized-for-use-by-localities-to-collect-delinquent-debts}}

Senate Bill 1301 (Chapter 271) allows a local treasurer to employ private collection agents to assist with the collection of delinquent amounts due other than delinquent local taxes that have been delinquent for a period of three months or more and for which the appropriate statute of limitations has not run.

Effective: July 1, 2019

Amended: § 58.1-3919.1

\hypertarget{real-property-tax}{%
\section{Real Property Tax}\label{real-property-tax}}

\hypertarget{real-property-tax-exemptions-for-elderly-and-disabled-computation-of-income-limitation}{%
\subsection{Real Property Tax Exemptions for Elderly and Disabled: Computation of Income Limitation}\label{real-property-tax-exemptions-for-elderly-and-disabled-computation-of-income-limitation}}

House Bill 1937 (Chapter 16) provides that, if a locality has established a real estate tax exemption for the elderly and handicapped and enacted an income limitation related to the exemption, it may exclude, for purposes of calculating the income limitation, any disability income received by a family member or nonrelative who lives in the dwelling and who is permanently and totally disabled.

Under current law, if a locality's tax relief ordinance establishes an annual income limitation, the computation of annual income is calculated by adding together the income received during the preceding calendar year of the owners of the dwelling who use it as their principal residence; and the owners' relatives who live in the dwelling, except for those relatives living in the dwelling and providing bona fide caregiving services to the owner whether such relatives are compensated or not; and at the option of each locality, nonrelatives of the owner who live in the dwelling except for bona fide tenants or bona fide caregivers of the owner, whether compensated or not.

Effective: July 1, 2019

Amended: § 58.1-3212

\hypertarget{real-property-tax-exemption-for-elderly-and-disabled-improvements-to-a-dwelling}{%
\subsection{Real Property Tax Exemption for Elderly and Disabled: Improvements to a Dwelling}\label{real-property-tax-exemption-for-elderly-and-disabled-improvements-to-a-dwelling}}

House Bill 2150 (Chapter 736) and Senate Bill 1196 (Chapter 737) clarify the definition of ``dwelling,'' for purposes of the real property tax exemption for owners who are 65 years of age or older or permanently and totally disabled, to include certain improvements to the exempt land and the land on which the improvements are situated. These Acts define the term ``dwelling'' to include an improvement to the land that is not used for a business purpose but is used to house certain motor vehicles or household goods.

Under current law, in order to be granted real property tax relief, qualifying property must be owned by and occupied as the sole dwelling of a person who is at least 65 years of age, or, if the local ordinance provides, any person with a permanent disability. Dwellings jointly held by spouses, with no other joint owners, qualify if either spouse is 65 or over or permanently and totally disabled.

Effective: July 1, 2019

Amended: § 58.1-3210

\hypertarget{real-property-tax-partial-exemption-from-real-property-taxes-for-flood-mitigation-efforts}{%
\subsection{Real Property Tax: Partial Exemption from Real Property Taxes for Flood Mitigation Efforts}\label{real-property-tax-partial-exemption-from-real-property-taxes-for-flood-mitigation-efforts}}

Senate Bill 1588 (Chapter 754) enables a locality to provide by ordinance a partial exemption from real property taxes for flooding abatement, mitigation, or resiliency efforts for improved real estate that is subject to recurrent flooding, as authorized by an amendment to Article X, Section 6 of the Constitution of Virginia that was adopted by the voters on November 6, 2018.

This act provides that exemptions may only be granted for qualifying flood improvements that do not increase the size of any impervious area and are made to qualifying structures or to land. ``Qualifying structures'' are defined as structures that were completed prior to July 1, 2018 or were completed more than 10 years prior to the completion of the improvements. For improvements made to land, the improvements must be made primarily for the benefit of one or more qualifying structures. No exemption will be authorized for any improvements made prior to July 1, 2018.

A locality is granted the authority to (i) establish flood protection standards that qualifying flood improvements must meet in order to be eligible for the exemption; (ii) determine the amount of the exemption; (iii) set income or property value limitations on eligibility; (iv) provide that the exemption shall only last for a certain number of years; (v) determine, based upon flood risk, areas of the locality where the exemption may be claimed; and (vi) establish preferred actions for qualifying for the exemption, including living shorelines.

Effective: July 1, 2019

Amended: § 58.1-3228.1

\hypertarget{real-property-tax-exemption-for-certain-surviving-spouses}{%
\subsection{Real Property Tax: Exemption for Certain Surviving Spouses}\label{real-property-tax-exemption-for-certain-surviving-spouses}}

House Bill 1655 (Chapter 15) and Senate Bill 1270 (Chapter 801) allow surviving spouses of disabled veterans to continue to qualify for a real property tax exemption regardless of whether the surviving spouse moves to a different residence, as authorized by an amendment to subdivision (a) of Section 6-A of Article X of the Constitution of Virginia that was adopted by the voters on November 6, 2018. If a surviving spouse was eligible for the exemption but lost such eligibility due to a change in residence, then the surviving spouse is eligible for the exemption again, beginning January 1, 2019.

These Acts also clarify that the real property tax exemptions for spouses of service members killed in action and spouses of certain emergency service providers killed in the line of duty continue to apply regardless of the spouse's moving to a new principal residence.

Effective: Taxable years beginning on or after January 1, 2019

Amended: §§ 58.1-3219.5, 3219.9, and 3219.14

\hypertarget{land-preservation-special-assessment-optional-limit-on-annual-increase-in-assessed-value}{%
\subsection{Land Preservation; Special Assessment, Optional Limit on Annual Increase in Assessed Value}\label{land-preservation-special-assessment-optional-limit-on-annual-increase-in-assessed-value}}

House Bill 2365 (Chapter 22) authorizes localities that employ use value assessments for certain classes of real property to provide by ordinance that the annual increase in the assessed value of eligible property may not exceed a specified dollar amount per acre.

Effective: July 1, 2019

Amended: § 58.1-3231

\hypertarget{virginia-regional-industrial-act-revenue-sharing-composite-index}{%
\subsection{Virginia Regional Industrial Act: Revenue Sharing; Composite Index}\label{virginia-regional-industrial-act-revenue-sharing-composite-index}}

House Bill 1838 (Chapter 534) requires that the Department of Taxation's calculation of the true values of real estate and public service company property component of the Commonwealth's educational composite index of local ability-to-pay take into account arrangements by localities entered into pursuant to the Virginia Regional Industrial Facilities Act, whereby a portion of tax revenue is initially paid to one locality and redistributed to another locality. This Act requires such calculation to properly apportion the percentage of tax revenue ultimately received by each locality.

Effective: July 1, 2021

Amended: § 58.1-6407

\hypertarget{real-estate-with-delinquent-taxes-or-liens-appointment-of-special-commissioner-increase-required-value}{%
\subsection{Real Estate with Delinquent Taxes or Liens: Appointment of Special Commissioner; Increase Required Value}\label{real-estate-with-delinquent-taxes-or-liens-appointment-of-special-commissioner-increase-required-value}}

House Bill 2060 (Chapter 541) increases the assessed value of a parcel of land that could be subject to appointment of a special commissioner to convey the real estate to a locality as a result of unpaid real property taxes or liens from \$50,000or less to \$75,000 or less in most localities. In the Cities of Norfolk, Richmond, Hopewell, Newport News, Petersburg, Fredericksburg, and Hampton, this Act increases the threshold from \$100,000 or less to \$150,000 or less.

Effective: July 1, 2019

Amended: § 58.1-3970.1

\hypertarget{real-estate-with-delinquent-taxes-or-liens-appointment-of-special-commissioner-in-the-city-of-martinsville}{%
\subsection{Real Estate with Delinquent Taxes or Liens; Appointment of Special Commissioner in the City of Martinsville}\label{real-estate-with-delinquent-taxes-or-liens-appointment-of-special-commissioner-in-the-city-of-martinsville}}

House Bill 2405 (Chapter 159) adds the city of Martinsville to the list of cities (Norfolk, Richmond, Hopewell, Newport News, Petersburg, Fredericksburg, and Hampton) that are authorized to have a special commissioner convey tax-delinquent real estate to the locality in lieu of a public sale at auction when the tax-delinquent property has an assessed value of \$100,000 or less. House Bill 2060 raises the threshold in all of these cities from \$100,000 or less to \$150,000 or less.

Effective: July 1, 2019

Amended: § 58.1-3970.1

\hypertarget{personal-property-tax}{%
\section{Personal Property Tax}\label{personal-property-tax}}

\hypertarget{constitutional-amendment-personal-property-tax-exemption-for-motor-vehicle-of-a-disabled-veteran}{%
\subsection{Constitutional Amendment: Personal Property Tax Exemption for Motor Vehicle of a Disabled Veteran}\label{constitutional-amendment-personal-property-tax-exemption-for-motor-vehicle-of-a-disabled-veteran}}

House Joint Resolution 676 (Chapter 822) is a fi rst resolu-tion proposing a constitutional amendment that permits the General Assembly to authorize the governing body of any county, city, or town to exempt from taxation one motor vehicle of a veteran who has a 100 percent service-connected, permanent, and total disability. The amendment provides that only automobiles and pickup trucks qualify for the exemption.

Additionally, the exemption will only be applicable on the date the motor vehicle is acquired or the effective date of the amendment, whichever is later, but will not be applicable for any period of time prior to the effective date of the amendment.

Effective: July 1, 2019

\hypertarget{personal-property-tax-exemption-for-agricultural-vehicles}{%
\subsection{Personal Property Tax Exemption for Agricultural Vehicles}\label{personal-property-tax-exemption-for-agricultural-vehicles}}

House Bill 2733 (Chapter 259) expands the definition of agricultural use motor vehicles for personal property taxation purposes. It changes the criteria from motor vehicles used ``exclusively'' for agricultural purposes to motor vehicles used ``primarily'' for agricultural purposes, and for which the owner is not required to obtain a registration certificate, license plate, and decal or pay a registration fee.

It also expands the definition of trucks or tractor trucks that are used by farmers in their farming operations for the transportation of farm animals or other farm products or for the transport of farm-related machinery. The criteria is changed from vehicles used ``exclusively'' by farmers in their farming operations to vehicles used ``primarily'' by farmers in their farming operations.

Further, this Act expands the classifi cation of farm machinery and equipment that a local governing body may exempt, to include equipment and machinery used by a nursery for the production of horticultural products, and any farm tractor, regardless of whether such farm tractor is used exclusively for agricultural purposes.

Local governing bodies have the option to exempt these classifi cations, in whole or in part, from taxation or to provide for a different rate of taxation thereon.

Effective: July 1, 2019

Amended: § 58.1-3505

\hypertarget{intangible-personal-property-tax-classifi-cation-of-certain-business-property}{%
\subsection{Intangible Personal Property Tax: Classifi cation of Certain Business Property}\label{intangible-personal-property-tax-classifi-cation-of-certain-business-property}}

House Bill 2440 (Chapter 255) classifies as intangible per-sonal property, tangible personal property: i) that is used in a trade or business; ii) with an original cost of less than \$25; and iii) that is not classified as machinery and tools, merchants' capital, or short-term rental property. It also exempts such property from taxation.

Intangible personal property is a separate class of prop-erty segregated for taxation by the Commonwealth. The Commonwealth does not currently tax intangible personal property. Localities are prohibited from taxing intangible personal property.

Certain personal property, while tangible in fact, has previously been designated as intangible and thus exempted from state and local taxation. For example, tangible personal property used in manufacturing, mining, water well drill-ing, radio or television broadcasting, dairy, dry cleaning, or laundry businesses has been designated as exempt intangible personal property.

Effective: July 1, 2019

Amended: §§ 58.1-1101 and 58.1-1103

\begin{center}\rule{0.5\linewidth}{0.5pt}\end{center}

\(^1\) Excerpted from the local tax legislation section of the Department of Taxation's 2019 Legislative Summary. Minor changes were made in format and punctuation. See \url{https://tax.virginia.gov/legislative-sum-mary-reports}

\hypertarget{real-property-tax-in-2019}{%
\chapter{Real Property Tax in 2019}\label{real-property-tax-in-2019}}

The real property tax is by far the most important source of tax revenue for localities. In fiscal year 2018, the most recent year available from the Auditor of Public Accounts, it accounted for 55.5 percent of tax revenue for cities, 64.6 percent for counties, and 29.1 percent for large towns. These are averages; the relative importance of taxes in individual cities, counties, and towns varies significantly. For information on individual localities, see Appendix C.
The \emph{Code of Virginia}, §§ 58.1-3200 through 58.1-3389, authorizes localities to levy taxes on real property (land, including the buildings and improvements on it). There is no restriction on the tax rate that may be imposed. Section 58.1-3201 provides that all general reassessments or annual assessments shall be at 100 percent of fair market value.

\hypertarget{public-service-corporations}{%
\section{PUBLIC SERVICE CORPORATIONS}\label{public-service-corporations}}

Property owned by so-called public service corporations is not assessed by localities. Instead, that task is delegated to the State Corporation Commission (SCC) and the Department of Taxation.The State Corporation Commission assesses electric utilities and cooperatives, gas pipeline distribution companies, public service water companies, and telephone and telegraph companies. The Department of Taxation assesses pipeline transmission companies and railroads.

In fiscal year 2018, the property tax on public service corporations accounted for 1.7 percent of tax revenue for
cities, 2.6 percent for counties, and 0.8 percent for large towns. These are averages; the relative importance of the tax
in individual cities, counties, and towns varies significantly. In two counties with large power generating facilities the
property tax on public service corporations accounts for a very large share of local tax revenue. In Bath County the share was 47.6 percent and in Surry County the share was 61.1 percent. For more information on individual localities, see Appendix C.

The commissioner of the revenue or another designated official in each city or county is required to provide by
January 1 of each year to any public service company with property in its area a copy of the property boundaries of
the locality in which any part of the company is located (§ 58.1-2601). The State Corporation Commission or the
Department of Taxation send out their assessments for the property based on these boundaries (§ 58.1-2602). Localities examine the assessments to determine their correctness. If correct, the locality determines the equalized assessed valuation of the corporate property by applying the local assessment ratio prevailing in the locality for other real estate (§ 58.1-2604). Local taxes are then assigned to real and tangible personal property at the real property tax rate current in the locality (§ 58.1-2606).

\hypertarget{tax-relief-programs}{%
\section{TAX RELIEF PROGRAMS}\label{tax-relief-programs}}

There are several types of locally financed tax relief programs available. Section 3 of this study contains information on so-called circuit breaker plans for the elderly and disabled. Section 4 covers land use assessments for agricultural, horticultural, forestal, and open space real estate. Section 5 contains information on preferential assessments for
agricultural and forestal districts. Finally, Section 6 covers property tax exemptions for certain rehabilitated real estate and other exemptions.

Only the city of Charlottesville, Loudoun County, and Arlington County reported providing tax relief for low-income
owners and renters who are not elderly. The city of Charlottesville administers the Charlottesville Housing Affordability Program (CHAP) to help low and middle income homeowners. The program awards grants up to \$1,000 to homeowners with houses assessed at less than \$375,000 and having an annual income less than \(55,000.\)\^{}1\$ Loudoun County administers the Affordable Dwelling Unit Program for renters and first-time buyers. Buyers need an income greater than 30 percent but less than 70 percent of the area median income to participate. Qualified renters are eligible to rent apartments at rates from \$630 to \$1,300. Arlington County's Housing Grants Program is available to working families with at least one child under age 18. Personal assets may not exceed \$35,000 and there is an income limit based on household size.

Localities are permitted to institute deferral for a portion of the real estate tax by § 58.1-3219 of the \emph{Code of Virginia}. Localities are permitted to grant deferrals from the full amount by which each taxpayer's real estate tax levy exceeds 105 percent of the previous year's tax, or such higher percentage adopted by the locality. Deferred taxes are subject to interest in an amount established by the governing body, not to exceed the rate published by the IRS code.\(^2\) The deferral potentially applies to every property owner, not just the elderly and disabled. (For deferrals limited to the elderly and disabled see Section 3 of this study.)

The deferral program is rarely used. Administrative problems appear to be the major reason for the unpopularity of deferral programs. Loudoun County had a deferral program in place in the 1990s but terminated it ``\ldots{} because the program was administratively complex, cumbersome and required staff time in disproportion to the benefit received by the taxpayer.''\(^3\) The cities of Alexandria, Falls Church, and Fairfax and the counties of Fairfax and Henrico considered deferral but
did not adopt it. According to Henrico staff, ``The administrative procedures for tracing the properties and recovering the
relevant taxes upon either the death of the owner or transfer of the property itself would be both cumbersome and time consuming and could not be accomplished with existing staffing levels or existing computer systems.''\(^4\) Another reason for the unpopularity of the programs may be that taxpayers only receive postponement, not removal, of the tax liability. The cities of Charlottesville and Richmond, the county of Middlesex, and the town of Amherst were the only localities reporting a deferral program in 2019.

\hypertarget{statutory-rates-special-taxes-due-dates-proration-and-billing-practices}{%
\section{STATUTORY RATES, SPECIAL TAXES, DUE DATES, PRORATION, AND BILLING PRACTICES}\label{statutory-rates-special-taxes-due-dates-proration-and-billing-practices}}

\textbf{Table 2.1} provides general information associated with real property taxes in Virginia's localities. The table provides
an estimate by locality of both the number of total taxable real estate parcels and the number of residential parcels.
Twenty-seven cities, 80 counties and 52 towns provided estimates of one or both types of parcels. The total number of parcels in cities ranged from a high of 158,431 (Virginia Beach) down to 2,456 (Lexington). Among counties, the number of parcels ranged from a high of 354,687 (Fairfax) down to 3,940 (Highland).

Table 2.1 also lists the statutory (nominal) tax rates. The statutory rate is the rate used by localities and is applied to the assessed value of a property. In the table, statutory rates are listed under calendar year (CY) or fiscal year (FY) columns depending on the locality's assessment cycle. In most cases the calendar year tax rate listed runs from January 1 to December 31 and the fiscal year rate runs from July 1 to June 30. The provisions explaining the assessment cycle requirements are found in § 58.1-3010 and § 58.1-3011 of the \emph{Code of Virginia}. However, some localities report a calendar year assessment schedule with a fiscal year valuation. Six cities (Chesapeake, Harrisonburg, Martinsville, Roanoke, Salem, and Suffolk) and one county (James City) report this practice. Otherwise, 15 cities and 88 counties reported using the calendar year cycle while 176 cities and 6 counties used fiscal year assessment cycles.

The statutory tax rates were reported to the Cooper Center by all cities and counties and 112 of the responding towns. The text table below lists the averages for the statutory rates from the localities.

\begin{Shaded}
\begin{Highlighting}[]
\CommentTok{\#table name: Statutory Real Estate Tax Rates per $100 of Assessed Taxable Value for Localities Reporting, CY 2019 and FY 2020}
\end{Highlighting}
\end{Shaded}

Statutory rates are generally higher for cities than counties. The rates are lowest in towns because they are subordinate to counties and have limited responsibilities.

Tax due dates vary among localities. Generally, if taxes are paid annually, they are due by December 5. If paid semiannually, they are due by June 5 and December 5. However, some localities have different due dates, as provided by § 58.1-3916 of the \emph{Code}.

Most cities have semiannual tax due dates with payments required in June and December. Of the 38 cities, 2 required taxes due annually, 31 semiannually, and 5 quarterly. Among the counties, 32 had annual tax due dates, while 63 had semiannual requirements. Of the towns responding to this question, 80 reported annual due dates, and 32 required semiannual payments.

A locality is permitted to prorate the taxable amount. Any county, city, or town electing to prorate new buildings which are substantially complete prior to November 1 must do so at the time the building is complete or fit to live in. Of the 38 cities, 33 reported prorating taxes while 5 reported not doing so. Among counties, 67 prorated their taxes while 28 did not. Reports from the towns that answered this question indicated that 47 prorated their taxes while 652 did not.

The final column of Table 2.1 pertains to town billing practices. Three possibilities exist: (1) a town sends out its own bills and collects its taxes (TT in the table), (2) a town collects its taxes but the county sends the bills (CT in the
table), or (3) a town has the county bill and collect the taxes (CC in the table). Of the towns that answered the question,
the overwhelming majority, 100, reported billing and collecting their own taxes. Four said they collected taxes, while in three the county both billed and collected town taxes.

\textbf{Table 2.2, Table 2.3,} and \textbf{Table 2.4} provide additional information concerning statutory real property tax rates. The \emph{Code} allows localities to add special purpose levies on top of the real property rate for various purposes. Table 2.2 deals with the category of special districts. A special district is organized to perform a single governmental function or a restricted number of related functions. Special districts usually have the power to incur debt and levy taxes to fund special activities such as capital improvements, emergency services, sewer and water services, or pest control within those districts. Thirteen cities, 14 counties, and 4 towns reported levying these taxes. The table includes the base (statutory) rate for the locality, the district in which the activity takes place, the purpose of the activity, and the special rate imposed for that activity. Most special activity taxes are in addition to the base rate, though some are simply a flat fee, and others are a percentage rate based on improvements to the property.

Another special district category is the community development authority (CDA). Such an authority is a district created by the locality based on a petition from the property owners to help develop and maintain desired public infrastructure improvements, such as roads and buildings. The CDA is usually associated with development interests, such as retail centers, industrial centers, or tourism centers. Generally the CDA pays for development by issuing bonds and then having the property owners pay special assessments based on the level of debt. Assessments are levied either by placing a tax, such as \$0.25 per \$100 of assessed value, on the property within the district or by a special assessment each year that determines the benefit from the improvements and allocates them by property value. Depending on how the bond agreement is structured, assessment payments may be made directly to bondholders or to the locality. Table 2.3 lists community development authorities by locality. The table includes the name of the project, the purpose, the size, the bond amount, and, where possible, the current value. Three cities and 8 counties reported having CDAs.

The final category of special districts is that of localities within the Northern Virginia Transportation Authority. Localities within this authority have the ability to tax real property associated with industrial and commercial use up
to \$0.125 per \$100 of assessed value to help fund transportation improvements. In 2009, an amendment to § 58.1-3221.3 specified that the revenues generated by the tax were to be used solely for (1) new road construction, design, and right-of-way acquisition, (2) new public transit construction, design, and right-of-way acquisition, (3) capital costs related to new transportation projects, or (4) the issuance costs and debt service on any bonds issued to support capital costs. There are 11 localities in the region of the authority: the cities of Alexandria, Fairfax, Falls Church, Fredericksburg, Manassas, and Manassas Park and the counties of Arlington, Fairfax, Loudoun, Prince William, and Stafford. Of those, one city (Fairfax) and two counties (Arlington and Fairfax) reported implementing the tax, as shown in Table 2.4.

\hypertarget{assessment-practices-reassessments-assessed-values}{%
\section{ASSESSMENT PRACTICES, REASSESSMENTS, ASSESSED VALUES}\label{assessment-practices-reassessments-assessed-values}}

\textbf{Table 2.5} details assessment practices among localities. The table includes cities and counties, but not towns, because only a small percentage of towns provided substantive answers. For those interested in the towns that responded, data are available from the Cooper Center upon request.

The second column lists whether a locality has a full-time assessor. Twenty-seven cities reported employing a full-time property tax assessor, while 11 did not. In contrast, only 36 counties had a full-time assessor while 59 did not. This reflects the fact that many counties reassess property less frequently than cities. No towns had assessors, since towns rely on assessed values established by their host counties.

Columns three, four, five, and six of Table 2.5 provide data on the conduct of general reassessments and cover four questions. (1) Are reassessments done by the locality or contracted out? (2) What is the reassessment frequency? (3) Is physical inspection part of the reassessment? (4) When was the reassessment last done? Regarding the conduct of the general reassessment, 28 cities reported conducting reassessments in-house while 10 reported contracting with outside assessors. Twenty-eight counties reported doing general reassessments in-house, while 67 reported contracting out for services. Section 58.1-3250 of the \emph{Code} requires cities to have a general reassessment of real estate every two years. However, any city with a total population of 30,000 or less may elect to conduct its general reassessments at four-year intervals.\(^5\) Counties are required to have a general reassessment every four years (§ 58.1-3252). There is an exception for counties with a total population of 50,000 or less. These counties may elect to reassess at either five-year or six-year intervals (§ 58.1-3252). However, nothing in these sections affects the power of cities and counties to reassess more frequently. A large majority of the cities (30) reassess at one or two year intervals. In contrast, less than three out of ten counties (27) reassess that frequently. Virtually all of the populous cities and counties reassess annually or biennially. Towns rely on their surrounding county to provide assessments, so a town's reassessment occurs with the same frequency as the county's. The reassessment periods are summarized in the table on the following page.

Column seven of Table 2.5 shows information about maintenance assessments. While general reassessments involve reassessing all parcels to reflect changes in market value, maintenance assessments involve adjusting assessed values between reassessments because of new construction, improvements, damages, demolitions, subdivisions, and consolidations. Thirty-three cities responded that they performed maintenance assessments using staff, while five reported contracting for the work. Among counties, 66 reported performing maintenance reassessments using staff, while 29 reported contracting the work to independent appraisers.

Columns eight and nine of Table 2.5 cover physical inspection. Physical inspection refers to the actual inspection of the property as opposed to computerized mass-appraisal of parcels. If a locality responded that it did not perform physical inspections during the general reassessment, two further questions were asked:

\begin{Shaded}
\begin{Highlighting}[]
\CommentTok{\#table name: Reassessment Periods for Real Estate, 2019}
\end{Highlighting}
\end{Shaded}

\begin{enumerate}
\def\labelenumi{(\arabic{enumi})}
\tightlist
\item
  Does the locality perform a physical inspection at all? (2) If so, what is the inspection cycle? Among cities that responded, 18 reportedly did not have a physical inspection separate from the general reassessment cycle. Twenty others reported having a physical inspection cycle, the periods ranging anywhere from two to six years. Among counties that responded, 70 indicated they performed physical inspections during general reassessment, while 25 reported having physical inspection cycles ranging anywhere from one to six years.
\end{enumerate}

\textbf{Table 2.6} provides unpublished Department of Taxation 2018 data on total taxable assessed value of real estate by category. Taxable assessed value shows property qualifying for use value at its use value, not its market value. The percentage distribution of taxable assessed value is shown for two types of residential property (single-family and multi-family) as well as commercial and industrial property and agricultural property.

The text table on the next page compares the taxable assessed value by category for cities and counties. The total assessed value for all cities amounted to \$277.4 billion. Single-family residential property averaged 64.9 percent of taxable assessed value. Multi-family residential property averaged 11.1 percent of taxable assessed value. Commercial/industrial properties averaged just over one-quarter of the total value at 23.9 percent, while agricultural property values amounted to only 0.1 percent.

The total assessed value of property by category for counties in 2018 amounted to \$854.5 billion. Of that amount, 72.0 percent of assessed value was associated with single-family residential property, 5.9 percent with multi-family residential property, 18.2 percent with commercial/industrial property, and 4.0 percent with agricultural property.

With the total amounts from cities and counties combined, the total assessed valuation amounted to \$1,131.9 billion. Of that, 70.2 percent applied to single-family residential property, 7.1 percent applied to multi-family residential property, 19.6 percent applied to commercial/industrial property, and 3.0 percent to agricultural property.

Looking at the percentage breakdown for each type of locality, in 2018 the share of taxable

\begin{Shaded}
\begin{Highlighting}[]
\CommentTok{\#table name: Taxable Assessed Value by Category for Cities and Counties, 2018}
\end{Highlighting}
\end{Shaded}

assessed value for cities in the single-family residential category was between 40 percent and 59.9 percent in 19 cities and 60 percent or more in 18 cities. All cities but two had multi-family residential values under 19.9 percent of the total assessed value. Commercial and industrial property was the second most common category with 21 of the cities having between 20 percent and 39.9 percent of their property valuations coming from this type of property. Finally, only the cities of Suffolk and Franklin had more than 2 percent of their property valuation associated with agriculture.

Among counties the breakdown was slightly different. As in cities, the single-family residential value dominated the percentage breakdown. The single-family residential assessment percentage amounted to 60 percent or more for 71 counties. Another 20 received between 40 percent and 59.9 percent of the valuation from single-family residential real estate, while in four counties residential valuations amounted to no more than 39.9 percent of the total taxable assessed value (Buchanan, Dickenson, Highland, and Sussex). In contrast, only in Arlington county did the multi-family residential average share of value exceed 19.9 percent.

The category with the second highest valuation in counties was commercial and industrial property. Eighty-two counties had such property valued no higher than 19.9 percent of the total assessed value of property within the locality. In general, the percentage of assessed value in counties for commercial and industrial properties was less than that for cities (though two counties, coal-rich Dickenson and Buchanan, had the highest percentage valuations of such property). Finally, agricultural property averaged the least total assessed valuation in counties, though the percentage varied greatly among the individual counties. In 30 counties, valuations associated with agricultural property made up 20 percent or more of the total assessed value within the locality. The percentage in one county (Sussex) was 82.0 percent. The taxable assessed values for agriculture were much lower than they would have been without the advantage of use value assessment, a program explained in Sections 4 and 5.

\hypertarget{effective-tax-rates}{%
\section{EFFECTIVE TAX RATES}\label{effective-tax-rates}}

Tax rates are generally discussed in terms of either statutory (nominal) rates or effective rates. The statutory rate is the rate used by localities and is applied to the assessed value of a property.The effective rate is published by the Virginia Department of Taxation in their annual assessment/sales ratio study. The department derives the effective rate by multiplying the statutory tax rate by the median assessment ratio. In normal times when property values are rising, the median assessment ratio is usually less than 100 percent

\begin{Shaded}
\begin{Highlighting}[]
\CommentTok{\#table name: Share of Assessed Value of Real Estate by Category, 2018}
\end{Highlighting}
\end{Shaded}

because reassessments lag market increases and tend to be conservative. Consequently, the statutory rate is generally higher than the effective rate. However, this may not be true in difficult real estate markets. A limitation of the effective rates published by the Virginia Department of Taxation is that they are not current. The most recent year available at the present time is 2017. Despite the time lag, effective rates are important because they give a more accurate reflection of the differences in real property tax rates across localities.

\textbf{Table 2.7} shows city and county average effective tax rates in the year 2017. The department makes its computation in order to control for the variance in localities' assessment procedures and timing. Therefore, when comparing tax rates among localities, the reader may wish to consult both Tables 2.1 and 2.7. Table 2.1 shows statutory rates in 2019. Table 2.7 shows statutory and effective rates in 2017. The following text table summarizes the effective tax rates for the localities shown in Table 2.7.

It should also be pointed out that the Virginia Department of Taxation does not use the locally reported statutory tax rate in its computations. Instead, it calculates the statutory rate by dividing the real estate levy by the local real

\begin{Shaded}
\begin{Highlighting}[]
\CommentTok{\#table name: Effective Real Estate Tax Rates, 2017}
\end{Highlighting}
\end{Shaded}

estate \emph{taxable assessed value},\(^6\) as reported in the local land book. This method of computing the statutory tax rate takes additional district levies into account.\(^7\)

In 2 cities and 10 counties the statutory rate was less than the effective rate. In two cities and seven counties statutory and effective rates were the same. Finally, in 34 cities and 78 counties statutory rates exceeded effective rates.

\begin{Shaded}
\begin{Highlighting}[]
\CommentTok{\#table name: Statutory and Effective Real Estate Tax Rates, 2017}
\end{Highlighting}
\end{Shaded}

The real property tax rates reported in Table 2.7 are a more accurate reflection of the differences among localities in tax rates on real property than those in Table 2.1 because they control for variations in assessment frequency and technique among localities. Table 2.7 also shows the latest reassessment in effect when the median ratio study was conducted, the number of sales used in the study, the median ratio, and the coefficient of dispersion.

The coefficient of dispersion measures how closely the individual ratios of each locality are arrayed around the median ratio. The formula for the coefficient of dispersion (CD) is:

where \[X_i\] represents the assessment/sales ratio for the \emph{i}th sale in a sample of size \emph{n}, and \[X_m\] represents the median ratio of the sample.\(^8\) If there were no dispersion, the CD would equal zero.

The text table below summarizes the coefficients of dispersion tabulated for the cities and counties. Eighteen of the cities had CDs of no more than 9.9 percent. Eight had CDs between 10 percent and 14.9 percent, 7 had CDs between 15

\begin{Shaded}
\begin{Highlighting}[]
\CommentTok{\#table name: Coefficient of Dispersion, 2017}
\end{Highlighting}
\end{Shaded}

and 19.9 percent, and 4 had CDs between 20 and 24.9 percent. Counties tended to vary more in the degree of dispersion. Thirteen had CDs between 5 and 9.9 percent, 18 had CDs between 10 and 14.9 percent, 25 had CDs between 15 and 19.9 percent, 26 had CDs between 20 and 24.9 percent, 11 had CDs between 25 and 29.9 percent, and 2 had CDs between 30 and 34.9 percent.

There is no upper limit for what is tolerable, but the International Association of Assessing Officers recommends an upper limit of 15 percent for residential properties.\(^9\) Twenty-eight cities and 34 counties met the 15 percent standard.\(^10\)

As one would expect, the quality of local assessments, as measured by the CD is generally better in those localities that reassess annually, biennially, or that have just conducted a general reassessment. In 2017, of the 57 localities with CDs under 15 percent, all but 12 reassessed annually (28), biennially (10), or had just completed general reassessments (7).

\hypertarget{miscellaneous-items}{%
\section{MISCELLANEOUS ITEMS}\label{miscellaneous-items}}

\textbf{Table 2.8} presents miscellaneous taxes and exemptions related to real property. The first is the recreation tax. The \emph{Code} in §15.2-1807 permits localities to collect a real estate tax for recreation areas and playgrounds that is not to exceed \$0.02/\$100 of the assessed value of a property. This tax was reported by Charlottesville City.

The second column refers to the tax deferral ordinance permitted by § 58.1-3219 regarding the deferral of a portion of real estate tax increases when the tax exceeds 105 percent of the real property tax on property owned by a taxpayer in the previous year. Four localities (Charlottesville City, Richmond City, Middlesex County, and Amherst Town) reported implementing this deferral.

The third column refers to the establishment of a tax increment financing fund used to encourage development in certain areas and permitted by § 58.1-3245 of the \emph{Code}. Six cities (Bristol, Charlottesville, Chesapeake, Emporia, Newport News, Virginia Beach, and Waynesboro), four counties (Arlington, Augusta, Fairfax, and Hanover), and one town (Front Royal) reported having implemented such a fund.

The fourth column refers to separate real property tax rates for energy-efficient buildings as permitted by § 58.1-3221.2 of the \emph{Code}. Three cities (Charlottesville, Roanoke, and Virginia Beach) reported having special rates for such real estate.

The fifth column lists localities that reported providing a separate real property classification for improvements to real property used in the manufacture of renewable energy. Only the cities of Charlottesville and Roanoke reported having this separate rate.

Finally, the last column refers to low-income grant programs, discussed earlier in this text under the subheading, ``Tax Relief Programs.'' Only the cities of Charlottesville and Norfolk, and the county of Arlington reported having these programs.

\begin{Shaded}
\begin{Highlighting}[]
\CommentTok{\#Table 2.1 "Real Property Statutory (Nominal) Tax Rates, CY 2019 and FY 2020"}


\CommentTok{\#Table 2.2 "Additional Real Property Special District Tax Levies for Special Purposes, 2019"}


\CommentTok{\#Table 2.3 "Community Development Authorities Requiring a Special Purpose Real Property Levy, 2019" }


\CommentTok{\#Table 2.4 "Special Purpose Real Property Tax Levies on Commercial Property in Northern Virginia Transportation Authority Region, 2019" }


\CommentTok{\#Table 2.5 "Real Property Assessment Procedures for Virginia Localities, 2019" }


\CommentTok{\#Table 2.6 "Assessed Value of Real Property by Category and by Locality, 2018*"}


\CommentTok{\#Table 2.7 "Real Property Effective True Tax Rates, 2017"}


\CommentTok{\#Table 2.8 "Real Property Miscellaneous Items, 2019"}
\end{Highlighting}
\end{Shaded}

\begin{center}\rule{0.5\linewidth}{0.5pt}\end{center}

\(^1\) Charlottesville Housing Affordability Program: \url{https://www}.
charlottesville.org/departments-and-services/departments-a-g/
commissioner-of-revenue/real-estate-tax-relief-for-the-elderlyand-disabled. Loudoun County Affordable Dwelling Unit Program: \url{http://www.loudoun.gov/adu}. Arlington County Housing
Grants Program: \url{http://housing.arlingtonva.us/get-help/rentalservices/local-housing-grants/}.

\(^2\) The statute allows the use of the Internal Revenue Service rate. Section 6621 of the Internal Revenue Code establishes a rate of 3 percent plus the federal short-term rate. In December 2019, when the short-term rate was 1.616 percent, the combined annual rate was 4.61 percent.

\(^3\) City of Alexandria, \emph{Budget Memo \#46: Review of Other Jurisdictions' Experience with a Real Estate Tax Deferral Program for the General Population} (Councilman Speck's Request), 4/25/2003.

\(^4\) Henrico County, \emph{Budget Memo \#46}.

\(^5\) The \emph{Code} does not specify which census is to be used.

\(^6\) Taxable assessed value treats property qualifying for use value
as taxable at its use value rather than at its full market value.

\(^7\) Virginia Department of Taxation, \emph{The 2017 Virginia Assessment/Sales Ratio Study} (Richmond, February 2019), p.~35. The study
can be found at \url{https://tax.virginia.gov/assessment-sales-ratiostudies}.

\(^8\) Virginia Department of Taxation, \emph{The 2017 Virginia Assessment/Sales Ratio Study}, p.~34.

\(^9\) International Association of Assessing Officers, \emph{Standard on Ratio Studies}, (approved April 2013), p.~17. \url{http://www.iaao}.
org/media/standards/Standard\_on\_Ratio\_Studies.pdf.

\(^10\) The Department of Taxation's study applies to all types of property, not just residential property. Nonetheless, the majority of
the measured sales are for single-family residential properties.

\hypertarget{real-property-tax-relief-plans-and-housing-grants-for-the-elderly-and-disabled-in-2019}{%
\chapter{Real Property Tax Relief Plans and Housing Grants for the Elderly and Disabled in 2019}\label{real-property-tax-relief-plans-and-housing-grants-for-the-elderly-and-disabled-in-2019}}

Sections 58.1-3210 through 58.1-3218 of the \emph{Code of Virginia} provides that localities may adopt an ordinance allowing property tax relief for elderly and disabled persons. The relief may be in the form of either deferral or exemption from taxes. The applicant for tax relief must be either disabled or not less than 65 years of age and must be the owner of the property for which relief is sought (§ 58.1-3210). The property must be the sole dwelling of the applicant. In addition, localities have the option of exempting or deferring the portion of a person's tax that represents the increase in tax liability since the year the taxpayer reached 65 years of age or became disabled.

Localities are allowed to establish by ordinance the net financial worth and annual income limitations pertaining to
owners, relatives and non-relatives living in the dwelling(§ 58.1-3212) of qualified elderly or handicapped persons.
Further, mobile homes that are owned by elderly and disabled persons are included in the allowable property tax
exemptions whether or not mobile homes are permanently affixed. Finally, local governments are authorized to extend
tax relief for the elderly and disabled to dwellings that are jointly owned by individuals, not all of whom are over 65
or totally disabled.

The text table below indicates the range and media nof the combined gross income allowance and combined
net worth limitations for those cities, counties, and towns responding to the survey.

\begin{Shaded}
\begin{Highlighting}[]
\CommentTok{\#table name: Relief Plan Statistics: Gross Income and Net Worth, 2019}
\end{Highlighting}
\end{Shaded}

The following text table indicates, for those localities responding, how many localities have a tax relief plan that
applies to both the elderly and the disabled, the elderly only,or the disabled only.

\begin{Shaded}
\begin{Highlighting}[]
\CommentTok{\# table name: Relief Plans for Elderly and Disabled, 2019}
\end{Highlighting}
\end{Shaded}

A majority of the localities exempt an owner from all or part of the taxes on the dwelling; usually the exemption is based on a sliding scale, with the percentage of the exemption decreasing as the income and/or net worth of the owner increases.

\textbf{Table 3.1} summarizes the various tax relief plans offered to elderly and disabled property owners in Virginia.
The figures under the combined gross income heading reflect, first, the maximum allowable income (including the
income of all relatives living with the owner) for an owner to be eligible for relief and, second, the amount of income
of each relative living in the household, except the spouse, who is exempted from this amount.

For example, if the table reads ``\$7,500; first \$1,500 exempt,'' this indicates that the combined income of the
owner and all relatives living with him/her may not exceed \$7,500, except that the first \$1,500 of income of each relative other than the spouse is excluded when computing this amount. The combined net worth amount listed usually
excludes the value of the dwelling and a given parcel of land upon which the dwelling is situated.

\textbf{Table 3.2} details relief plans for renters. As the table indicates, few localities offer such plans. Only five cities
(Alexandria, Charlottesville, Fairfax, Falls Church, and Hampton) and one county (Fairfax) reported having plans
for renters.

\textbf{Table 3.3} lists the combined elderly and disabled beneficiaries reported by each locality in 2018 or 2019 and
the amount of revenue foregone by each locality because of the homeowner exemptions. The amounts were reported
by 23 cities, 66 counties, and 31 towns that responded to the question. The amounts reported foregone totaled \$21,698,890
for cities, \$60,242,734 for counties and \$636,229 for the reporting towns. The grand total amount foregone by
reporting cities, counties, and towns was \$82,577,853. An estimate of the average revenue foregone per beneficiary
is also provided for localities reporting both number of beneficiaries and foregone revenue. For cities, the average
revenue foregone was \$1,518 per beneficiary. The amount for counties was \$1,581, and for towns it was \$360.

\begin{Shaded}
\begin{Highlighting}[]
\CommentTok{\#Table 3.1 Real Property Owner Tax Relief Plans for the Elderly and Disabled, 2019}

\CommentTok{\#Table 3.2 Real Property Renter Tax Relief Plans for the Elderly and Disabled, 2019}

\CommentTok{\#Table 3.3 Real Property Tax Relief Plans for the Elderly and Disabled Homeowners: Number of Beneficiaries and Foregone Tax Revenue, 2018 or 2019}
\end{Highlighting}
\end{Shaded}

\hypertarget{motor-vehicle-local-license-tax-2019}{%
\chapter{Motor Vehicle Local License Tax 2019}\label{motor-vehicle-local-license-tax-2019}}

In fiscal year 2018, the most recent year available from the Auditor of Public Accounts, the motor vehicle local license tax, popularly known as the local decal tax, even though many of the localities imposing the tax no longer use a decal as evidence of payment, accounted for 1.1 percent of the total tax revenue for cities, 1.1 percent for counties and 2.0 percent for large towns. These are averages; the relative importance of this tax in individual cities, counties and large towns varies significantly. For information on individual localities see Appendix C.
\textbar{}
\textbar{} Section 46.2-752 of the \emph{Code of Virginia} authorizes cities, counties, and towns to levy a license tax on motor vehicles, trailers, and semitrailers. The amount of the tax may not be greater than the tax imposed by the state. Currently, the base registration fees for non-commercial passenger vehicles are \$33 for vehicles under 4,000 pounds and \$38 for heavier vehicles (§ 46.2-694.2). Motorcycle fees are \$18 with a \$3 surcharge included {[}§ 46.2-694 (A) (10){]}. The \emph{Code} stipulates similar guidelines for commercial vehicles, buses, trailers, and other motor vehicles. The \emph{Code} also provides for additional fees for specified government services, such as \$6.25 for emergency medical service (EMS) programs {[}\emph{Code of Virginia} § 46.2-694 (A) (13) and \emph{2014 Appropriations Act} § 3-6.02{]} to be paid to the state treasury and provides for a \(1.50 addition for the official motor vehicle safety inspection program to be paid at registration (§ 46.2-1168). | | No locality may impose a license tax on any vehicle when the owner pays a similar tax to the locality in which the vehicle is normally stored. Furthermore, no locality may impose a local license tax on any vehicle that is owned by a nonresident of such locality and is used exclusively for pleasure or personal transportation (i.e., for non-business uses). For example, the tax would not apply to a personal vehicle owned by a nonresident college student and used only for pleasure or personal transportation. Vehicles used for state business by nonresident officials, dealer demonstration vehicles and the vehicles of common carriers are also exempt from local license taxes. | | The situs for the assessment of motor vehicles is clarified in § 58.1.3511. Business vehicles with a weight of 10,000 pounds or less are considered to be in the jurisdiction in which the owner of the business: (1) is required to file a tangible personal property tax return for any vehicle used in the business, and (2) has a definite place of business from which the use of the business vehicle is directed or controlled. | | If a town within a county levies a motor vehicle license tax, the county must credit the owner with the tax paid to the town. Also, if the town tax is equal to the maximum allowed by law, then the county may not impose any further tax. Likewise, no county license tax may be imposed on vehicles that are subject to license taxes imposed by a town constituting a separate school division (§46.2-752)\)\^{}1\$.
\textbar{}
\textbar{} \textbf{Table 15.1} presents the local motor vehicle license taxes on automobiles, motorcycles, and trucks. Column one indicates the date that the fee must be paid or a decal, if applicable, must be affixed to a motor vehicle to denote payment of license fees. Thirty-two cities and 83 counties reported imposing the tax. Of the reporting towns, 103 said they levied the tax. The second column gives the tax rate on private passenger vehicles. Most localities levy a fl at tax between \$15 and \$30 for passenger vehicles under 4,000 pounds. The table also shows the exemption status for elderly or disabled persons. Seven localities offer tax relief for the elderly, while 30 exempt the disabled from this tax. The final two columns give the tax rates on motorcycles and trucks. The tax ranges from \$3 to \$35 for motorcycles and from \$3 up to \$250 (depending on weight) for trucks.
\textbar{}
\textbar{} The following text table summarizes the range of tax charged for private passenger vehicles under 4,000 pounds.

\begin{Shaded}
\begin{Highlighting}[]
\CommentTok{\#Text table "License Tax for Private Passenger Vehicles Under 4,000 Pounds, 2019" goes here}
\end{Highlighting}
\end{Shaded}

\hfill\break
~~Cities had a median license tax of \$27.00; the median tax for both counties and towns was \$25. For cities the mean license tax for private passenger vehicles was \$28.09. The first quartile measure was \$25 while the third quartile was \$32.25. For counties, the mean was \$26.67. The first and third quartiles were \$23.00 and \$30.00, respectively. For towns, the mean was \$23.18. The first and third quartiles were \$20 and \$25 respectively.\\
~\\
\hspace*{0.333em}\hspace*{0.333em}\textbf{Table 15.2} lists whether localities require the display of decals and whether localities permit special exemptions from paying the motor vehicle license tax other than those for the elderly and disabled. Twenty-six cities, 78 counties, and 62 towns reported granting payment exemptions. The most popular category for exemption was for local fire and rescue department members.\\
~\\
\hspace*{0.333em}\hspace*{0.333em}In recent years, many localities have dispensed with the decal because new technology has allowed them to track payments without the use of the decal. Most now collect the motor vehicle license tax along with the personal property tax on motor vehicles. So far, 30 cities, 83 counties, and 81 towns reported they no longer required decal placement on automobile windshields.\\

\begin{Shaded}
\begin{Highlighting}[]
\CommentTok{\#Table 15.1 "Motor Vehicle Local License Tax, 2019" goes here}

\CommentTok{\#Table 15.2 "Motor Vehicle Local License Tax Decal Display Policy and Exemptions, 2019" goes here}
\end{Highlighting}
\end{Shaded}

\(^1\) The \emph{Code} refers to school district rather than school division. Colonial Beach and West Point are the only towns that have school divisions.

\hypertarget{meals-transient-occupancy-cigarettes-tobacco-and-admissions-excise-taxes}{%
\chapter{Meals, Transient Occupancy, Cigarettes, Tobacco, and Admissions Excise Taxes}\label{meals-transient-occupancy-cigarettes-tobacco-and-admissions-excise-taxes}}

Among the many local taxes levied by Virginia's localities are four excise taxes on meals, transient occupancy, cigarettes and admissions. \textbf{Table 16.1} provides a detailed list of rates for these taxes for the 38 cities, 82 counties, and 108 towns reporting at least one of these taxes.
\textbar{}
\#\#\# MEALS TAX
\textbar{} The meals tax is a flat percentage imposed on the price of a meal. In fiscal year 2018, the most recent year available from the Auditor of Public Accounts, the tax accounted for 7.5 percent of the total tax revenue for cities, 1.2 percent for counties, and 23.5 percent for large towns. The low percentage for counties is explained by the fact that slightly less than one-half of the counties employ the tax and those that have it cannot exceed a rate of 4 percent, whereas cities and towns are allowed to impose a higher tax rate. The authority to levy this tax varies greatly among jurisdictions, so the tax varies significantly among individual cities, counties, and towns. For information on tax receipts of individual localities, see Appendix C.
\textbar{}
\textbar{} Counties are restricted in their authority to levy the meals tax within the limits of an incorporated town unless the town grants the county authority to do so (§ 58.1-3711). Cities and towns are granted the authority to levy the tax under the ``general taxing powers'' found in their charters (§ 58.1-3840).
\textbar{}
\textbar{} Counties may levy a meals tax on food and beverages offered for human consumption if the tax is approved in a voter referendum (§ 58.1-3833). However, several counties have been exempted from the voter referendum requirement {[}see § 58.1-3833 (B) of the Code of Virginia{]}. Cities and towns do not need to have a referendum when deciding to impose the tax.
\textbar{}
\textbar{} There are certain restrictions in applying the meals tax. The tax cannot be imposed on food that meets the definition of food under the Federal Food Stamp Program, with the exception of sandwiches, salad bar items, certain prepackaged salads, and non-factory sealed beverages. It does not apply to certain volunteer and non-profit organizations that might sell food on an occasional basis nor does it apply to churches and their members. Also, the meals tax cannot exceed 4 percent in counties. Cities and towns may exceed that rate. Accordingly, 34 cities and 78 towns report charging a meals tax over 4 percent. In addition, the meals tax does not apply to gratuities, whether or not a restaurant makes them mandatory.
\textbar{}
\textbar{} The first column of \textbf{Table 16.1} lists the rates for the meals tax. All cities impose a meals tax. The median tax rate is 6 percent. The minimum rate, charged by four cities, is 4 percent, and the maximum, charged by Covington is 8 percent. The median meals tax rate is lower among the 50 counties that report having it. All counties that report having the meal tax have a rate of 4 percent. Among the 105 towns that report having a meals tax, the minimum rate is 2 percent, the maximum 8 percent, and the median rate is 5 percent.
\textbar{}
\textbar{} The text table summarizes the dispersion of the meal tax rates among cities, counties, and towns.

\begin{Shaded}
\begin{Highlighting}[]
\CommentTok{\#Text table "Meals Tax Rates, 2019" goes here}
\end{Highlighting}
\end{Shaded}

\hfill\break
~~The local meals tax is in addition to the state 4.3 percent sales tax (5 percent in localities constituting transportation districts in northern Virginia and Hampton Roads) and the 1 percent local option sales tax (see § 58.1-603). This means that the combined state and local tax rate on restaurant meals could be anywhere in the range of 7 to 14 percent for cities, counties, and towns that impose this tax. Such rates apply to all restaurant meals whether consumed at elegant dining establishments or fast food providers.\\

\hypertarget{transient-occupancy-tax}{%
\subsection{TRANSIENT OCCUPANCY TAX}\label{transient-occupancy-tax}}

~~The transient occupancy tax (lodging tax) is a flat percentage imposed on the charge for the occupancy of any room or space in hotels, motels, boarding houses, travel camp-grounds, and other facilities providing lodging for less than thirty days. The tax applies to rooms intended or suitable for dwelling and sleeping. Therefore, the tax does not apply to rooms used for alternative purposes, such as banquet rooms and meeting rooms.\\
~\\
\hspace*{0.333em}\hspace*{0.333em}In fiscal year 2018, the occupancy tax accounted for 2.2 percent of the total tax revenue for cities, 0.9 percent for counties, and 5.6 percent for large towns. These are averages; the relative importance of the tax varies significantly among individual cities, counties, and towns. For information on tax receipts of individual localities, see Appendix C.\\
~\\
\hspace*{0.333em}\hspace*{0.333em}According to § 58.1-3819, counties may levy a transient occupancy tax with a maximum tax rate of 2 percent. Counties specified in § 58.1-3819(A) may increase their transient occupancy tax to a maximum of 5 percent. The portion of the tax collections exceeding 2 percent must be used by the county for tourism and tourism related expenses. According to § 58.1-3819, the following counties are permitted to levy the 5 percent rate: Accomack, Albemarle, Alleghany, Amherst, Arlington, Augusta, Bedford, Bland, Botetourt, Brunswick, Campbell, Caroline, Carroll, Craig, Cumberland, Dickenson, Dinwiddie, Floyd, Franklin, Frederick, Giles, Gloucester, Goochland, Grayson, Greene, Greensville, Halifax, Highland, Isle of Wight, James City, King George, Loudoun, Madison, Mecklenburg, Montgomery, Nelson, Northampton, Page, Patrick, Powhatan, Prince Edward, Prince George, Prince William, Pulaski, Rockbridge, Rockingham, Russell, Smyth, Spotsylvania, Stafford, Tazewell, Warren, Washington, Wise, Wythe, and York.\\
~\\
\hspace*{0.333em}\hspace*{0.333em}Certain counties are permitted to levy higher rates. Roanoke County was given permission to levy a rate of 7 percent in 2012, with a portion of the revenue going to tourism advertisement. James City and York counties have 5 percent rates but are also allowed to charge an additional \$2 per room per night. The proceeds of these additional taxes go to tourism advertising (§ 58.1-3823(C)). Certain cities and towns also charge specific dollar amounts in addition to the percent rates; they are the cities of Alexandria, Lynchburg, Newport News, and Norfolk and the town of Dumfries. It is assumed, but not verified, that these policies are permitted by the localities' charters.\\
~\\
\hspace*{0.333em}\hspace*{0.333em}In 2018 the General Assembly authorized the replacement of a regional transient occupancy tax in the Northern Virginia Transportation District with a 2 percent transient occupancy tax to fund transportation in that area. This tax includes the counties of Arlington, Fairfax, and Loudoun, and the cities of Alexandria, Fairfax, and Falls Church. In addition, the assembly funded a 2 percent local transportation transient occupancy tax for the localities of Prince William County and Manassas City and Manassas Park City.\\
~\\
\hspace*{0.333em}\hspace*{0.333em}Counties are restricted in their authority to levy the lodging tax within the limits of an incorporated town unless the town grants the county authority to do so (§ 58.1-3711). Cities and towns are granted the authority to levy the lodging taxes under the ``general taxing powers'' found in their charters (§ 58.1-3840).\\
~\\
\hspace*{0.333em}\hspace*{0.333em}The median rate for the 37 cities that report using the transient occupancy tax is 8 percent, the minimum 2 percent, and the maximum is 11 percent (Emporia). Seventy-nine counties report imposing a transient occupancy tax. The extremes range from 2 to 8 percent with a median rate of 5 percent. The 77 towns that report having the tax show a median of 5 percent with a minimum rate of 1 percent and a maximum of 9 percent. The following text table summarizes the dispersion of the transient occupancy tax among cities, counties, and towns:

\begin{Shaded}
\begin{Highlighting}[]
\CommentTok{\#Text table "Transient Occupancy Taxes, 2019" goes here}
\end{Highlighting}
\end{Shaded}

\hfill\break
~~The local transient occupancy tax is in addition to the state 4.3 percent sales tax (5 percent in localities constituting transportation districts in Northern Virginia and Hampton Roads) and the 1 percent local option sales tax. This means that the combined state and local tax rate for hotel-motel stays can be very high. In a special district of Virginia Beach the combined rate is 16.5 percent (10.5 percent transient occupancy tax, 1 percent local option sales and use tax, and 5 percent state sales and use tax applicable for localities in Hampton Roads).\\

\hypertarget{cigarette-and-tobacco-taxes}{%
\subsection{CIGARETTE AND TOBACCO TAXES}\label{cigarette-and-tobacco-taxes}}

~~In fiscal year 2018, cigarette and tobacco taxes accounted for 0.9 percent of the total tax revenue collected by cities, 0.1 percent of that collected by counties, and 2.1 percent of that collected by large towns. The very low percentage for counties is attributable to the fact that few counties levy cigarette and tobacco taxes. These are averages; the relative importance of the tax varies significantly among individual cities and towns. For information on individual localities, see Appendix C.\\
~\\
\hspace*{0.333em}\hspace*{0.333em}The state is authorized by § 58.1-1001 of the Code to impose an excise tax of 1.5 cents on each cigarette sold or stored (30 cents on a pack of 20). Section 58.1-3830 allows for the local taxation of the sale or use of cigarettes. Cities and towns are granted the authority to levy the tax under the ``general taxing powers'' found in their charters (§ 58.1-3840). The right to levy the tax has been granted to only two counties by general law. Fairfax and Arlington counties may levy the cigarette tax with a maximum rate of 5 cents per pack or the amount levied under state law, whichever is greater (§ 58.1-3831). The two counties have followed the state's example and raised their taxes to 30 cents for a pack of 20. No county cigarette tax is applicable within town limits if the town's governing body does not authorize that county to levy the tax. This restriction applies to towns in Fairfax County, including Herndon, Vienna, and Occoquan.\\
~\\
\hspace*{0.333em}\hspace*{0.333em}Unlike the meals and transient occupancy taxes, which are added directly to the bill at the time of purchase, the cigarette tax is added onto the price per pack before the purchaser buys the cigarettes. The tobacco tax is levied either as a flat tax or as a portion of gross receipts. If no schedule is given in \textbf{Table 16.1}, then it should be read as a flat tax. A total of 31 cities levy some sort of tax on cigarettes, while 2 counties and 66 towns report doing so. The following text table, based on the tax of a pack of 20 cigarettes, summarizes the dispersion of cigarette taxes among cities, counties and towns.

\begin{Shaded}
\begin{Highlighting}[]
\CommentTok{\#Text table "Cigarette Tax on a Pack of 20 in 2019" goes here}
\end{Highlighting}
\end{Shaded}

\hfill\break
~~The cigarette tax is in addition to the state 4.3 percent sales tax (5 percent in localities constituting transportation districts in Northern Virginia and Hampton Roads) and the 1 percent local option sales tax.\\

\hypertarget{admissions-tax}{%
\subsection{ADMISSIONS TAX}\label{admissions-tax}}

~~In fiscal year 2018, the admissions tax accounted for 0.4 percent of the total tax revenue for cities. Receipts were negligible for counties and large towns. These are averages; the relative importance of the tax varies significantly among individual localities. For information on receipts by individual localities, see Appendix C.\\
~\\
\hspace*{0.333em}\hspace*{0.333em}Events to which admissions are charged are classified into five groups by § 58.1-3817 of the \emph{Code of Virginia}; they are: (1) those events from which the gross receipts go entirely to charitable purposes; (2) admissions charged for events sponsored by public and private educational institutions; (3) admissions charged for entry into museums, botanical or similar gardens and zoos; (4) admissions charged for sporting events; and (5) all other admissions.\\
~\\
\hspace*{0.333em}\hspace*{0.333em}In imposing the admissions tax, localities have the authority to tax each class of admissions with the same or with a different tax rate. A locality may impose admission taxes at lower rates for events held in privately-owned facilities than for events held in facilities owned by the locality. Section 58.1-3818 allows a locality to exempt certain qualified charitable events from admissions tax charges. Fifteen counties (Arlington, Brunswick, Charlotte, Clarke, Culpeper, Dinwiddie, Fairfax, Madison, Nelson, New Kent, Prince George, Scott, Stafford, Sussex, and Washington) have been granted permission to levy an admissions tax at a rate not to exceed 10 percent of the amount of charge for admissions (§§ 58.1-3818 and 58.1-3840). Only three counties, Dinwiddie, Roanoke, and Washington, report levying the tax.\\
~\\
\hspace*{0.333em}\hspace*{0.333em}Cities and towns are granted the authority to levy the admissions tax under the ``general taxing powers'' found in their charters (§ 58.1-3840). As shown in the text table, 18 cities and 3 towns (Cape Charles, Culpeper, and Vinton) reported levying the admissions tax. For cities, the levy ranged from 5 percent to the full 10 percent. The median rate was 7 percent.

\begin{Shaded}
\begin{Highlighting}[]
\CommentTok{\#Text table "Admissions Tax, 2019" goes here}

\CommentTok{\#Table 16.1 "Meals, Transient Occupancy, Cigarette, and Admissions Excise Taxes, 2019" goes here}
\end{Highlighting}
\end{Shaded}

\hypertarget{taxes-on-natural-resources}{%
\chapter{Taxes on Natural Resources}\label{taxes-on-natural-resources}}

~~Taxes on natural resources are rarely used by localities because many are not endowed with such resources. As a consequence, natural resources taxes accounted for less than 0.1 percent of total city tax revenue in fiscal year 2018, 0.2 percent of total county tax revenue, and less than 0.1 percent of total tax revenue of large towns, according to information from the Auditor of Public Accounts. These are averages; the vast majority of localities receive no revenue from this source. All the exceptions are located in Southwest Virginia. For information on individual localities, see Appendix C.\\
~\\
\hspace*{0.333em}\hspace*{0.333em}Localities are permitted to impose several taxes on natural resources. \textbf{Table 17.1} provides tax rates for the cities and counties having such natural resource-related taxes in effect during the 2019 tax year.\\

\hypertarget{taxation-of-mineral-lands}{%
\subsection{TAXATION OF MINERAL LANDS}\label{taxation-of-mineral-lands}}

~~Under § 58.1-3286 of the \emph{Code of Virginia}, localities are required to ``\ldots specially and separately assess at the fair market value all mineral lands and the improvements thereon\ldots{}'' and enter those assessments separately from assessments of other lands and improvements. Mineral lands are taxed at the same rate as other real estate in the locality. Localities may request technical assistance from the Virginia Department of Taxation in assessing mineral lands and minerals, provided money is available to the department to defray the cost of the assistance (§ 58.1-3287). Instead of employing the real property tax for mineral lands, localities are permitted to substitute a severance tax on mineral sales, not to exceed 1 percent.\\
~\\
\hspace*{0.333em}\hspace*{0.333em}In 2009, this section was amended to allow Buchanan County to reassess mineral lands on an annual basis for purposes of determining the real property tax on such land. Other real estate is still subject to assessment every six years. Currently, 2 cities and 23 counties report assessing taxes on minerals. Among the several that commented on their mineral tax, most stated they used the land assessment method. The city of Norton, however, stated that its tax was based on a loading tax of \$0.05/ton.\\

\hypertarget{severance-tax}{%
\subsection{SEVERANCE TAX}\label{severance-tax}}

~~Under § 58.1-3712, any city or county may levy a license tax on businesses engaged in severing coal and gases from the earth. The maximum rate permitted is 1 percent of the gross receipts from sales. A 2012 bill reduced the rates of the local coal severance tax for small mines from 1 percent to 0.75 percent of the gross receipts from the sale of coal. ``Small mine'' is defined here as a mine that sells less than 10,000 tons of coal per month.\\
~\\
\hspace*{0.333em}\hspace*{0.333em}Localities choosing to use § 58.1-3712 may not exercise the option to levy a 1 percent severance tax under § 58.1-3286. Under § 58.1-3712.1, the maximum rate permitted for severing oil is one-half of 1 percent from the sale of the extracted oil. Notwithstanding the rate limits established in § 58.1-3712, cities or counties may impose an additional license tax of 1 percent of the gross receipts from the sale of gas severed as authorized by § 58.1-3713.4. The funds from this additional levy are paid into the general fund of the localities except for members of the Virginia Coalfield Economic Development Fund, where one-half of the revenues must be paid to the fund. The members of the fund are the counties of Buchanan, Dickenson, Lee, Russell, Scott, Tazewell, and Wise and the city of Norton.\\

\hypertarget{coal-and-gas-road-improvement-tax}{%
\subsection{COAL AND GAS ROAD IMPROVEMENT TAX}\label{coal-and-gas-road-improvement-tax}}

~~Notwithstanding the rate limits described in the previous paragraph, localities are permitted by § 58.1-3713 to levy up to an additional 1 percent license tax on the gross receipts of coal and gas extracted from the ground. As with the severance tax on coal, the coal road improvement tax has been modified to reduce the tax from 1 percent to 0.75 percent for small mines. This tax was originally scheduled to end in 2007, but the General Assembly extended the sunset clause a number of times, most recently to December 31, 2017.\\
~\\
\hspace*{0.333em}\hspace*{0.333em}The amount collected under this tax must be paid into a special fund to be called the Coal and Gas Road Improvement Fund of the particular county or city where the tax is collected. In addition, ``the county may also, in its discretion, elect to improve city or town roads with its funds if consent of the city or town council is obtained.'' One-half of the revenue paid to this fund may be used for the purpose of funding the construction of new water systems and lines in areas of insufficient natural supply of water. Those same funds may also be used to improve existing water and sewer systems. Localities are required to develop and ratify an annual funding plan for such projects. Under § 58.1-3713.1, 20 percent of the funds collected in Wise County are distributed to the six incorporated towns within the county's boundaries (Appalachia, Big Stone Gap, Coeburn, Pound, Saint Paul, and Wise) and the city of Norton. The distribution is determined as follows: (a) 25 percent is divided among the incorporated towns and the city of Norton based on the number of registered motor vehicles in each town and the city of Norton, and (b) 75 percent is divided equally among the towns and the city of Norton. The Coal and Gas Road Improvement Advisory Committee in the city of Norton and county must develop a plan before July 1 of each year for road improvements for the following fiscal year. For localities in the Virginia Coalfield Economic Development Authority (Lee, Wise, Scott, Buchanan, Russell, Tazewell, and Dickenson counties and the city of Norton), the receipts from this tax are distributed as follows: three-fourths to the Coal and Gas Road Improvement Fund and one-fourth to the Virginia Coalfield Economic Development Fund. The purpose of this fund is to enhance the economic base for the seven counties and one city in the authority.

\begin{center}\rule{0.5\linewidth}{0.5pt}\end{center}

\label{tab:table17-1}Natural Resource Taxes, 2019

Locality

Coal \& Gas Severance Tax\\
(§ 58.1-3712)

Oil Severance Tax\\
(§ 58.1-3712.1)

Additional Gas Severance Tax\\
(§ 58.1-3713.4)

Coal \& Gas Road Improvement Tax\\
(§ 58.1-3713)

Tax on Mineral Land\\
(§ 58.1-3286)

Accomack County

--

--

--

--

No

Albemarle County

--

--

--

--

No

Alleghany County

--

--

--

--

No

Amelia County

--

--

--

--

No

Amherst County

--

--

--

--

Yes

Appomattox County

--

--

--

--

No

Arlington County

--

--

--

--

No

Augusta County

--

--

--

--

Yes

Bath County

--

--

--

--

No

Bedford County

--

--

--

--

Yes

Bland County

--

--

--

--

No

Botetourt County

--

--

--

--

No

Brunswick County

0.0

0.0

0

0.0

Yes

Buchanan County

1.0

0.5

1

1.0

No

Buckingham County

--

--

--

--

Yes

Campbell County

--

--

--

--

Yes

Caroline County

--

--

--

--

Yes

Carroll County

--

--

--

--

No

Charles City County

--

--

--

--

No

Charlotte County

--

--

--

--

No

Chesterfield County

--

0.5

--

--

No

Clarke County

--

--

--

--

No

Craig County

--

--

--

--

No

Culpeper County

--

--

--

--

Yes

Cumberland County

--

--

--

--

No

Dickenson County

1.0

0.5

1

1.0

Yes

Dinwiddie County

--

--

--

--

No

Essex County

--

--

--

--

No

Fairfax County

--

--

--

--

No

Fauquier County

--

--

--

--

No

Floyd County

--

--

--

--

No

Fluvanna County

--

--

--

--

No

Franklin County

--

--

--

--

No

Frederick County

--

--

--

--

No

Giles County

--

--

--

--

No

Gloucester County

--

--

--

--

No

Goochland County

--

--

--

--

Yes

Grayson County

--

--

--

--

Yes

Greene County

--

--

--

--

No

Greensville County

--

--

--

--

Yes

Halifax County

--

--

--

--

No

Hanover County

--

--

--

--

Yes

Henrico County

--

--

--

--

No

Henry County

--

--

--

--

No

Highland County

--

--

--

--

Yes

Isle of Wight County

--

--

--

--

No

James City County

--

--

--

--

No

King \& Queen County

--

--

--

--

No

King George County

--

--

--

--

No

King William County

--

--

--

--

Yes

Lancaster County

--

--

--

--

No

Lee County

2.0

0.5

2

1.0

Yes

Loudoun County

--

--

--

--

No

Louisa County

--

--

--

--

No

Lunenburg County

--

--

--

--

No

Madison County

--

--

--

--

No

Mathews County

--

--

--

--

No

Mecklenburg County

--

--

--

--

No

Middlesex County

--

--

--

--

No

Montgomery County

--

--

--

--

No

Nelson County

--

--

--

--

No

New Kent County

--

--

--

--

No

Northampton County

--

--

--

--

No

Northumberland County

--

--

--

--

No

Nottoway County

--

--

--

--

No

Orange County

--

--

--

--

No

Page County

--

--

--

--

No

Patrick County

--

--

--

--

No

Pittsylvania County

--

--

--

--

Yes

Powhatan County

--

--

--

--

Yes

Prince Edward County

--

--

--

--

No

Prince George County

--

--

--

--

No

Prince William County

--

--

--

--

No

Pulaski County

--

1.0

1

1.0

No

Rappahannock County

--

--

--

--

No

Richmond County

--

--

--

--

No

Roanoke County

--

--

--

--

No

Rockbridge County

--

--

--

--

No

Rockingham County

--

1.0

--

--

Yes

Russell County

1.0

0.5

--

1.0

Yes

Scott County

1.0

0.5

--

1.0

No

Shenandoah County

--

--

--

--

No

Smyth County

--

--

--

--

No

Southampton County

--

--

--

--

No

Spotsylvania County

--

--

--

--

No

Stafford County

--

--

--

--

No

Surry County

--

--

--

--

No

Sussex County

--

--

--

--

No

Tazewell County

1.5

--

1

0.5

Yes

Warren County

--

--

--

--

Yes

Washington County

0.0

0.0

0

0.0

Yes

Westmoreland County

--

--

--

--

No

Wise County

--

0.5

1

--

Yes

Wythe County

--

--

--

--

No

York County

--

--

--

--

No

Alexandria City

--

--

--

--

No

Bristol City

--

--

--

--

No

Buena Vista City

--

--

--

--

Yes

Charlottesville City

--

--

--

--

No

Chesapeake City

--

--

--

--

No

Colonial Heights City

--

--

--

--

No

Covington City

--

--

--

--

No

Danville City

--

--

--

--

No

Emporia City

--

--

--

--

No

Fairfax City

--

--

--

--

No

Falls Church City

--

--

--

--

No

Franklin City

--

--

--

--

No

Fredericksburg City

--

--

--

--

No

Galax City

--

--

--

--

No

Hampton City

--

--

--

--

No

Harrisonburg City

--

--

--

--

No

Hopewell City

--

--

--

--

No

Lexington City

--

--

--

--

No

Lynchburg City

--

--

--

--

No

Manassas City

--

--

--

--

No

Manassas Park City

--

--

--

--

No

Martinsville City

--

--

--

--

No

Newport News City

--

--

--

--

No

Norfolk City

--

--

--

--

No

Norton City

1.0

--

--

1.0

Yes

Petersburg City

--

--

--

--

No

Poquoson City

--

--

--

--

No

Portsmouth City

--

--

--

--

No

Radford City

--

--

--

--

No

Richmond City

--

--

--

--

No

Roanoke City

--

--

--

--

No

Salem City

--

--

--

--

No

Staunton City

--

--

--

--

No

Suffolk City

--

--

--

--

No

Virginia Beach City

--

--

--

--

No

Waynesboro City

--

--

--

--

No

Williamsburg City

--

--

--

--

No

Winchester City

--

--

--

--

No

Abingdon Town

--

--

--

--

No

Accomac Town

--

--

--

--

No

Alberta Town

--

--

--

--

No

Altavista Town

--

--

--

--

No

Amherst Town

--

--

--

--

No

Appalachia Town

--

--

--

--

No

Appomattox Town

--

--

--

--

No

Ashland Town

--

--

--

--

No

Bedford Town

--

--

--

--

No

Berryville Town

--

--

--

--

No

Big Stone Gap Town

--

--

--

--

No

Blacksburg Town

--

--

--

--

No

Blackstone Town

--

--

--

--

No

Bluefield Town

--

--

--

--

No

Boones Mill Town

--

--

--

--

No

Bowling Green Town

--

--

--

--

No

Boyce Town

--

--

--

--

No

Boydton Town

--

--

--

--

No

Bridgewater Town

--

--

--

--

No

Broadway Town

--

--

--

--

No

Brookneal Town

--

--

--

--

No

Buchanan Town

--

--

--

--

No

Cape Charles Town

--

--

--

--

No

Charlotte Court House Town

--

--

--

--

No

Chase City Town

--

--

--

--

No

Chatham Town

--

--

--

--

No

Chilhowie Town

--

--

--

--

No

Chincoteague Town

--

--

--

--

No

Christiansburg Town

--

--

--

--

No

Claremont Town

--

--

--

--

No

Clarksville Town

--

--

--

--

No

Clifton Forge Town

--

--

--

--

No

Clintwood Town

--

--

--

--

No

Coeburn Town

--

--

--

--

No

Colonial Beach Town

--

--

--

--

No

Courtland Town

--

--

--

--

No

Craigsville Town

--

--

--

--

No

Culpeper Town

--

--

--

--

No

Damascus Town

--

--

--

--

No

Dayton Town

--

--

--

--

No

Dillwyn Town

--

--

--

--

No

Drakes Branch Town

--

--

--

--

No

Dublin Town

--

--

--

--

No

Dumfries Town

--

--

--

--

No

Dungannon Town

--

--

--

--

No

Eastville Town

--

--

--

--

No

Edinburg Town

--

--

--

--

No

Elkton Town

--

--

--

--

No

Exmore Town

--

--

--

--

No

Farmville Town

--

--

--

--

No

Fincastle Town

--

--

--

--

No

Floyd Town

--

--

--

--

No

Front Royal Town

--

--

--

--

No

Gate City Town

--

--

--

--

No

Glade Spring Town

--

--

--

--

No

Glasgow Town

--

--

--

--

No

Gordonsville Town

--

--

--

--

No

Goshen Town

--

--

--

--

No

Gretna Town

--

--

--

--

No

Grottoes Town

--

--

--

--

No

Grundy Town

--

--

--

--

No

Hamilton Town

--

--

--

--

No

Haymarket Town

--

--

--

--

No

Haysi Town

--

--

--

--

No

Herndon Town

--

--

--

--

No

Hillsville Town

--

--

--

--

No

Honaker Town

--

--

--

--

No

Hurt Town

--

--

--

--

No

Independence Town

--

--

--

--

No

Ivor Town

--

--

--

--

No

Kenbridge Town

--

--

--

--

No

Keysville Town

--

--

--

--

No

Kilmarnock Town

--

--

--

--

No

La Crosse Town

--

--

--

--

No

Lawrenceville Town

--

--

--

--

No

Lebanon Town

--

--

--

--

No

Leesburg Town

--

--

--

--

No

Louisa Town

--

--

--

--

No

Lovettsville Town

--

--

--

--

No

Luray Town

--

--

--

--

No

Madison Town

--

--

--

--

No

Marion Town

--

--

--

--

No

Middleburg Town

--

--

--

--

No

Mineral Town

--

--

--

--

No

Montross Town

--

--

--

--

No

Mount Crawford Town

--

--

--

--

No

Mount Jackson Town

--

--

--

--

No

Narrows Town

--

--

--

--

No

New Market Town

--

--

--

--

No

Nickelsville Town

--

--

--

--

No

Occoquan Town

--

--

--

--

No

Onancock Town

--

--

--

--

No

Orange Town

--

--

--

--

No

Phenix Town

--

--

--

--

No

Pulaski Town

--

--

--

--

No

Purcellville Town

--

--

--

--

No

Remington Town

--

--

--

--

No

Richlands Town

--

--

--

--

No

Rocky Mount Town

--

--

--

--

No

Round Hill Town

--

--

--

--

No

Rural Retreat Town

--

--

--

--

No

Saint Paul Town

--

--

--

--

No

Saltville Town

--

--

--

--

No

Scottsville Town

--

--

--

--

No

Shenandoah Town

--

--

--

--

No

Smithfield Town

--

--

--

--

No

South Boston Town

--

--

--

--

No

South Hill Town

--

--

--

--

No

Stanley Town

--

--

--

--

No

Stony Creek Town

--

--

--

--

No

Strasburg Town

--

--

--

--

No

Surry Town

--

--

--

--

No

Tappahannock Town

--

--

--

--

No

Tazewell Town

--

--

--

--

No

Timberville Town

--

--

--

--

No

Toms Brook Town

--

--

--

--

No

Urbanna Town

--

--

--

--

No

Victoria Town

--

--

--

--

No

Vienna Town

--

--

--

--

No

Vinton Town

--

--

--

--

No

Virgilina Town

--

--

--

--

No

Wachapreague Town

--

--

--

--

No

Wakefield Town

--

--

--

--

No

Warrenton Town

--

--

--

--

No

Warsaw Town

--

--

--

--

No

West Point Town

--

--

--

--

No

Windsor Town

--

--

--

--

No

Wise Town

--

--

--

--

No

Woodstock Town

--

--

--

--

No

Wytheville Town

--

--

--

--

No

Belle Haven Town

--

--

--

--

No

Bloxom Town

--

--

--

--

No

Branchville Town

--

--

--

--

No

Brodnax Town

--

--

--

--

No

Burkeville Town

--

--

--

--

No

Capron Town

--

--

--

--

No

Cheriton Town

--

--

--

--

No

Clifton Town

--

--

--

--

No

Clinchco Town

--

--

--

--

No

Dendron Town

--

--

--

--

No

Fries Town

--

--

--

--

No

Halifax Town

--

--

--

--

No

Hillsboro Town

--

--

--

--

No

Iron Gate Town

--

--

--

--

No

Irvington Town

--

--

--

--

No

Jarratt Town

--

--

--

--

No

Keller Town

--

--

--

--

No

McKenney Town

--

--

--

--

No

Middletown Town

--

--

--

--

No

Newsoms Town

--

--

--

--

No

Onley Town

--

--

--

--

No

Pennington Gap Town

--

--

--

--

No

Pocahontas Town

--

--

--

--

No

Port Royal Town

--

--

--

--

No

Pound Town

--

--

--

--

No

Rich Creek Town

--

--

--

--

No

Ridgeway Town

--

--

--

--

No

Saxis Town

--

--

--

--

No

Scottsburg Town

--

--

--

--

No

Troutville Town

--

--

--

--

No

Washington Town

--

--

--

--

No

Waverly Town

--

--

--

--

No

Weber City Town

--

--

--

--

No

\hypertarget{legal-document-taxes}{%
\chapter{Legal Document Taxes}\label{legal-document-taxes}}

~~In fiscal year 2018, the most recent year available from the Auditor of Public Accounts, taxes on legal documents accounted for 0.5 percent of total tax revenue for cities and 0.8 percent for counties. Towns do not have this tax. These are averages; the relative importance of taxes in individual localities may vary significantly. For information on individual localities, see Appendix C.\\
~\\
\hspace*{0.333em}\hspace*{0.333em}Section 58.1-3800 of the Code of Virginia authorizes the governing body of any city or county to impose a recordation tax in an amount equal to one-third of the state recordation tax. The recordation tax generally applies to real and personal property in connection with deeds of trust, mortgages, and leases, and to contracts involving the sale of rolling stock or equipment (§§ 58.1-807 and 58.1-808).\\
~\\
\hspace*{0.333em}\hspace*{0.333em}Local governments are not permitted to impose a levy when the state recordation tax imposed is 50 cents or more (§ 58.1-3800). Consequently, local governments cannot levy a tax on such documents as certain corporate charter amendments (§ 58.1-801), deeds of release (§ 58.1-805), or deeds of partition (§ 58.1-806) as the state tax imposed is already 50 cents per \$100.\\
~\\
\hspace*{0.333em}\hspace*{0.333em}Sections 58.1-809 and 58.1-810 also specifically exempt certain types of deed modifications from being taxed. Deeds of confirmation or correction, deeds to which the only parties are husband and wife, and modifications or supplements to the original deeds are not taxed. Finally, § 58.1-811 lists a number of exemptions to the recordation tax.\\
~\\
\hspace*{0.333em}\hspace*{0.333em}Currently, the state recordation tax on the first \$10 million of value is 25 cents per \$100, so cities and counties can impose a maximum tax of 8.3 cents per \$100 on the first \$10 million, one-third of the 25 cent state rate. Above \$10 million there is a declining scale of charges applicable (§ 58.1-3803).\\
~\\
\hspace*{0.333em}\hspace*{0.333em}In addition to a tax on real and personal property, §§ 58.1-3805 and 58.1-1718 authorize cities and counties to impose a tax on the probate of every will or grant of administration equal to one-third of the state tax on such probate or grant of administration. Currently, the state tax on wills and grants of administration is 10 cents per \$100 or a fraction of \$100 for estates valued at greater than \$15,000 (§ 58.1-1712). Therefore, the maximum local rate is 3.3 cents.\\
~\\
\hspace*{0.333em}\hspace*{0.333em}A related \emph{state} tax is levied in localities associated with the Northern Virginia Transportation Authority. The tax is a grantor's fee of \$0.15 per \$100 on the value of real property property sold. This was created as part of the 2013 transportation bill.\\
~\\
\hspace*{0.333em}\hspace*{0.333em}\textbf{Table 18.1} provides information on the recordation tax and the wills and administration tax for the 35 cities and 89 counties that report imposing one or both of them. The following text table shows range of recordation taxes and taxes on wills and administration imposed by localities.

\begin{Shaded}
\begin{Highlighting}[]
\CommentTok{\#Text table "Recordation Tax and Tax on Wills and Administration, 2019" goes here}

\CommentTok{\#Table 18.1 "Legal Document Taxes, 2019" goes here}
\end{Highlighting}
\end{Shaded}

\hypertarget{miscellaneous-taxes-2018}{%
\chapter{Miscellaneous Taxes 2018}\label{miscellaneous-taxes-2018}}

This section includes a number of taxes and exemptions that are not covered in the previous sections: the local option sales and use tax, the bank franchise tax, the communication sales and use tax, the short-term (daily) rental tax, and other miscellaneous taxes. The local option sales tax has been adopted by every city and county and, by law, all use the same tax rate. Also, as explained below, counties must share a portion of sales tax collections with incorporated towns within their boundaries. Wherever the bank franchise tax is imposed, the rate is the same. In addition to those major taxes, this section covers the communications sales and use tax and other miscellaneous taxes for which information was provided on the survey form when local governments were asked to specify any miscellaneous taxes that fell outside the scope of the survey questions.

\hypertarget{local-option-and-state-sales-and-use-taxes}{%
\subsection{LOCAL OPTION AND STATE SALES AND USE TAXES}\label{local-option-and-state-sales-and-use-taxes}}

In fiscal year 2018, the most recent year available from the Auditor of Public Accounts, the local option sales and use tax accounted for 8.0 percent of local tax revenue for cities, 6.4 percent for counties and 9.2 percent for large towns. These are averages; the relative importance of taxes in individual cities, counties and towns may vary signifi cantly. For information on individual localities, see Appendix C.

Each city and county is permitted by § 58.1-605 to establish a general retail sales tax, ``at the rate of 1 percent to provide revenue for the general fund of such city or county.'' This tax applies to dealers with a retail presence in Virginia. Sales of any items from such operations incur the 1 percent sales tax. Sales tax monies are then collected by the Virginia Department of Taxation and sent to the Department of the Treasury. That agency credits the accounts of the localities where the sales occurred and disburses the monies to the localities on a monthly basis (§ 58.1-605.F).

Cities and counties are also permitted to establish a local use tax at the rate of 1 percent for the purpose of providing revenue to the general fund of the locality. The use tax is similar in purpose to the retail sales tax, but its aim is somewhat distinct: it applies to dealers that do not have a physical retail presence in Virginia. It is a tax levied on the use of tangible personal property within the state that has been stored or sold out-of-state.

Special distribution requirements apply to counties with incorporated towns (§ 58.1-605.G). Where the town constitutes a special school division and is operated as a separate school division under a town school board\(^1\), the county is required to pay to the town a proportionate share of the full amount of tax receipts based on the school age population within the town compared to the school age population in the entire county. If the town does not constitute a separate school division, then one-half of county collections is distributed to the town based on the proportion of the school age population within the town to the school age population of the entire county, provided the town complies with certain conditions.

Certain items are exempted from the state sales and use tax and may be exempted from the local option sales and use tax also. Each locality is permitted by § 58.1-609 to exempt fuels meant for domestic consumption from the 1 percent component of the tax. These fuels include artificial or propane gas, firewood, coal, or home heating oil. Only 11 localities answered that they exempted such fuels from the tax. The localities were the counties of Alleghany, Campbell, Madison, Patrick, Pittsylvania, Prince George and Washington and the cities of Chesapeake, Covington, Harrisonburg, and Portsmouth.

The state portion of the sales and use tax was raised from 4 percent to 4.3 percent effective July 1, 2013. House Bill 2313, Chapter 766, further increased the amount by an additional 0.7 amount for localities in the Northern Virginia and Hampton Roads planning districts. The additional taxes do not apply to food purchased for human consumption. The Northern Virginia Planning District consists of the counties of Arlington, Fairfax, Loudoun, and Prince William and the cities of Alexandria, Fairfax, Falls Church, Manassas, and Manassas Park. The Hampton Roads Planning District consists of the counties of Isle of Wight, James City, South-ampton, and York and the cities of Chesapeake, Franklin, Hampton, Newport News, Norfolk, Poquoson, Portsmouth, Suffolk, Virginia Beach, and Williamsburg. The purpose of this additional state tax is to fund the Northern Virginia Transportation Authority and the Hampton Roads Construction Fund, respectively. Consequently, the new sales and use rate is made up of a 1.0 percent local tax rate as well as a 4.3 state tax rate for most localities and a 5.0 percent state tax rate for localities associated with transportation commissions.

\hypertarget{state-motor-fuels-tax-on-distributors}{%
\subsection{STATE MOTOR FUELS TAX ON DISTRIBUTORS}\label{state-motor-fuels-tax-on-distributors}}

An additional state tax that applies only to specific localities is the fuel distribution license tax. It is a state tax on distributors of motor fuels to retailers in qualifying localities. Under § 58.1-2295 a state tax of 2.1 percent may be imposed on any distributor in a qualifying locality in the business of selling fuels at wholesale to retail dealers for retail sale within the qualifying locality. To be eligible a locality must be: (i) any county or city that is a member of a transportation district in which a rail commuter mass transport system and a bus commuter mass transport system are owned or operated by an agency as defined in § 15.2- 4502, or (ii) any county or city that is a member of a transportation district subject to § 15.2- 4515 and is contiguous to the Northern Virginia Transportation District. In addition, § 58.1-1722 excludes the amount of the tax imposed and collected by the distributor from the distributor's gross receipts for purposes of BPOL taxes imposed under Chapter 37 (§ 58.1-3700 et seq.).

The 2.1 percent state tax is imposed in 11 localities that belong to two transportation commissions. The Northern Virginia Transportation Commission (NVTC) consists of Fairfax, Loudoun, and Arlington counties and Alexandria, Fairfax, and Falls Church cities. The tax helps provide financial support for the activities of the Washington Metropolitan Area Transit Authority (WMATA), also known as Metro, and the Virginia Railway Express (VRE), the commuter line between Washington D.C. and Manassas and Fredericksburg. The other commission, the Potomac and Rappa-hannock Transportation Commission (PRTC), consists of three cities (Fredericksburg, Manassas, and Manassas Park), and two counties (Prince William and Stafford). It provides support to rail transport (VRE) in the affected counties and bus services originating in Prince William County through Omniride and Omnilink.

House Bill 2313, Chapter 766, authorized the state tax in certain localites in the Hampton Roads Planning District. These are the counties of Isle of Wight, James City, Southampton, and York, and the cities of Chesapeake, Hampton, Franklin, Newport News, Norfolk, Suffolk, Virginia Beach, Williamsburg, Poquoson, and Portsmouth. The tax began on July 1, 2013.

\hypertarget{bank-franchise-tax}{%
\subsection{BANK FRANCHISE TAX}\label{bank-franchise-tax}}

The bank franchise tax, also known as the bank stock tax, accounted for 0.7 percent of city tax revenue infiscal year 2018, 0.5 percent of county tax revenue, and 4.2 percent of the tax revenue of large towns. These are averages; the relative importance of taxes in individual cities, counties, and towns may vary significantly. For information on individual localities, see Appendix C.

The state of Virginia levies a bank franchise tax on all banks in Virginia at a rate of \$1 on each \$100 of net capital (§ 58.1-1204). Net capital is defi ned and its computation explained in § 58.1-1205. According to this section, net capital is determined by adding together a bank's capital, surplus, undivided profits, and one half of any reserve for loan losses net of applicable deferred tax to obtain gross capital and deducting therefrom (i) the assessed value of real estate as provided in § 58.1-1206, (ii) the book value of tangible personal property under § 58.1-1206, (iii) the pro rata share of government obligations as set forth in § 58.1-1206, (iv) the capital accounts of any bank subsidiaries under § 58.1-1206, (v) the amount of any reserve for marketable securities valuation which is included in capital, surplus and undivided profits as defined hereinabove to the extent that such reserve reflects the difference between the book value and the market value of such marketable securities on December 31 next preceding the date for filing the bank's return under § 58.1-1207, and (vi) the value of goodwill described under subdivision A 5 of § 58.1-1206.

Cities (§ 58.1-1208), counties (§ 58.1-1210), and incorporated towns (§ 58.1-1209) are permitted to charge an additional franchise tax of 80 percent of the state rate of taxation. If a locality imposes the local tax, then a bank is entitled to a credit against the state franchise tax equal to the total amount of local franchise tax paid (§ 58.1-1213). All localities that impose the bank franchise tax do so at the maximum rate allowed by statute.

If a bank has branches in more than one taxable subdivision (that is, city, county, or incorporated town), the tax imposed by the subdivision must be in the proportion of the taxable value of the net capital based on the total deposits of the bank or banks located inside the taxing subdivision to the total deposits in Virginia of the bank as of the end of the preceding year (§ 58.1-1211).

The survey asked whether a locality levied a bank tax. Of those localities that answered, all cities, 85 counties, and 108 towns answered affirmatively. The number of counties responding positively contrasts with the number of counties that reported receiving money from the tax in the Auditor of Public Accounts' Comparative Report. The reported disparity may be because a number of counties answered positively for having the tax when they actually only processed forms for towns having the tax. A list of localities that reported imposing the tax can be found in \textbf{Table 19.1}.

\hypertarget{communications-sales-and-use-tax}{%
\subsection{COMMUNICATIONS SALES AND USE TAX}\label{communications-sales-and-use-tax}}

In 2006, legislation enacted by the General Assembly, House Bill 568, replaced many state and local taxes and fees on communications services with a flat 5 percent rate. The tax is collected from consumers by their service providers and is then remitted to the Virginia Department of Taxation. The department then distributes the monies to the localities on a percentage basis derived from their participation in the local taxes which the new fl at tax superseded. The communication sales and use tax is a state tax not a local tax. Beginning in FY 2010 the Auditor of Public Accounts reported the proceeds as part of noncategorical state aid to localities.

The communications sales and use tax replaced a variety of local taxes: the consumer utility tax on land line and wireless telephone service, the local E-911 tax on land line telephone service, a portion of the BPOL tax assessed on public service companies by certain localities that impose the tax at a rate higher than 0.5 percent, the local video programming excise tax on cable television services, and the local consumer utility tax on cable television service.

The communications sales and use tax does not affect several related taxes: the state E-911 fee on wireless telephone service; the public rights-of-way use fee on land line telephone service; and the local tax of 0.5 percent on public service companies (also called the utility license tax).

\textbf{Table 19.2} presents a listing of the localities that received distributions from the communications sales and use tax in fiscal year 2018. The information was taken from Table 5.6 of the Virginia Department of Taxation's Annual Report, Fiscal Year 2018, the latest year available.

\hypertarget{short-term-daily-rental-tax}{%
\subsection{SHORT-TERM DAILY RENTAL TAX}\label{short-term-daily-rental-tax}}

In 2010 the General Assembly modified short-term rental property classifications. Short-term rental property can once again be included in merchants' capital as a separate classification. Consequently, localities may tax this property either as merchants' capital or short-term rental property, but not as both. Whether considered under the merchants' capital tax or the short-term property tax, the category of property shall not be considered tangible personal property for purposes of taxation.

The new law maintains the usual exclusions. Therefore, the category of short-term rental property still excludes ``(i) trailers as defined in § 46.2-100, and (ii) other tangible personal property required to be licensed or registered with the Department of Motor Vehicles, Department of Game and Inland Fisheries, or Department of Aviation (§ 58.1-3510.4).'' The most important exception listed is motor vehicles for rent. These fall under the merchants' capital tax as a separate classification, discussed in Section 8.

For purposes of taxation under the short-term rental tax, property is classified into two types: short-term rental property and heavy equipment short-term rental property (§ 58.1-3510.6). Short-term rental property may be taxed at 1 percent of gross receipts. Heavy equipment short-term rental property may be taxed up to 1.5 percent of gross receipts. \textbf{Table 19.3} lists the 20 cities, 19 counties, and 2 towns that reported having the short-term rental tax.

\begin{Shaded}
\begin{Highlighting}[]
\CommentTok{\#Table 19.1 "Localities Reporting That They Levy a Bank Franchise Tax, 2019", goes here}

\CommentTok{\#Table 19.2 "Localities Receiving Communications Sales and Use Tax Distributions, FY 2018"}

\CommentTok{\#Table 19.3 "Short{-}Term Daily Rental Tax, 2019*" goes here}
\end{Highlighting}
\end{Shaded}

\begin{center}\rule{0.5\linewidth}{0.5pt}\end{center}

\(^1\) The \emph{Code} refers to school districts. The Virginia Department of Education refers to school divisions. Colonial Beach and West Point are the only towns with school divisions. Obviously, the \emph{Code} is referring to those towns.

\begin{itemize}
\tightlist
\item
  As noted in the text for Section 19, the tax excludes motor vehicles for rent.
\end{itemize}

  \bibliography{book.bib}

\end{document}
