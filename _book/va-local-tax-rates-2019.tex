% Options for packages loaded elsewhere
\PassOptionsToPackage{unicode}{hyperref}
\PassOptionsToPackage{hyphens}{url}
%
\documentclass[
]{book}
\usepackage{amsmath,amssymb}
\usepackage{lmodern}
\usepackage{ifxetex,ifluatex}
\ifnum 0\ifxetex 1\fi\ifluatex 1\fi=0 % if pdftex
  \usepackage[T1]{fontenc}
  \usepackage[utf8]{inputenc}
  \usepackage{textcomp} % provide euro and other symbols
\else % if luatex or xetex
  \usepackage{unicode-math}
  \defaultfontfeatures{Scale=MatchLowercase}
  \defaultfontfeatures[\rmfamily]{Ligatures=TeX,Scale=1}
\fi
% Use upquote if available, for straight quotes in verbatim environments
\IfFileExists{upquote.sty}{\usepackage{upquote}}{}
\IfFileExists{microtype.sty}{% use microtype if available
  \usepackage[]{microtype}
  \UseMicrotypeSet[protrusion]{basicmath} % disable protrusion for tt fonts
}{}
\makeatletter
\@ifundefined{KOMAClassName}{% if non-KOMA class
  \IfFileExists{parskip.sty}{%
    \usepackage{parskip}
  }{% else
    \setlength{\parindent}{0pt}
    \setlength{\parskip}{6pt plus 2pt minus 1pt}}
}{% if KOMA class
  \KOMAoptions{parskip=half}}
\makeatother
\usepackage{xcolor}
\IfFileExists{xurl.sty}{\usepackage{xurl}}{} % add URL line breaks if available
\IfFileExists{bookmark.sty}{\usepackage{bookmark}}{\usepackage{hyperref}}
\hypersetup{
  pdftitle={Virginia Local Tax Rates, 2019},
  pdfauthor={Stephen C. Kulp, Weldon Cooper Center for Public Service, University of Virginia},
  hidelinks,
  pdfcreator={LaTeX via pandoc}}
\urlstyle{same} % disable monospaced font for URLs
\usepackage{color}
\usepackage{fancyvrb}
\newcommand{\VerbBar}{|}
\newcommand{\VERB}{\Verb[commandchars=\\\{\}]}
\DefineVerbatimEnvironment{Highlighting}{Verbatim}{commandchars=\\\{\}}
% Add ',fontsize=\small' for more characters per line
\usepackage{framed}
\definecolor{shadecolor}{RGB}{248,248,248}
\newenvironment{Shaded}{\begin{snugshade}}{\end{snugshade}}
\newcommand{\AlertTok}[1]{\textcolor[rgb]{0.94,0.16,0.16}{#1}}
\newcommand{\AnnotationTok}[1]{\textcolor[rgb]{0.56,0.35,0.01}{\textbf{\textit{#1}}}}
\newcommand{\AttributeTok}[1]{\textcolor[rgb]{0.77,0.63,0.00}{#1}}
\newcommand{\BaseNTok}[1]{\textcolor[rgb]{0.00,0.00,0.81}{#1}}
\newcommand{\BuiltInTok}[1]{#1}
\newcommand{\CharTok}[1]{\textcolor[rgb]{0.31,0.60,0.02}{#1}}
\newcommand{\CommentTok}[1]{\textcolor[rgb]{0.56,0.35,0.01}{\textit{#1}}}
\newcommand{\CommentVarTok}[1]{\textcolor[rgb]{0.56,0.35,0.01}{\textbf{\textit{#1}}}}
\newcommand{\ConstantTok}[1]{\textcolor[rgb]{0.00,0.00,0.00}{#1}}
\newcommand{\ControlFlowTok}[1]{\textcolor[rgb]{0.13,0.29,0.53}{\textbf{#1}}}
\newcommand{\DataTypeTok}[1]{\textcolor[rgb]{0.13,0.29,0.53}{#1}}
\newcommand{\DecValTok}[1]{\textcolor[rgb]{0.00,0.00,0.81}{#1}}
\newcommand{\DocumentationTok}[1]{\textcolor[rgb]{0.56,0.35,0.01}{\textbf{\textit{#1}}}}
\newcommand{\ErrorTok}[1]{\textcolor[rgb]{0.64,0.00,0.00}{\textbf{#1}}}
\newcommand{\ExtensionTok}[1]{#1}
\newcommand{\FloatTok}[1]{\textcolor[rgb]{0.00,0.00,0.81}{#1}}
\newcommand{\FunctionTok}[1]{\textcolor[rgb]{0.00,0.00,0.00}{#1}}
\newcommand{\ImportTok}[1]{#1}
\newcommand{\InformationTok}[1]{\textcolor[rgb]{0.56,0.35,0.01}{\textbf{\textit{#1}}}}
\newcommand{\KeywordTok}[1]{\textcolor[rgb]{0.13,0.29,0.53}{\textbf{#1}}}
\newcommand{\NormalTok}[1]{#1}
\newcommand{\OperatorTok}[1]{\textcolor[rgb]{0.81,0.36,0.00}{\textbf{#1}}}
\newcommand{\OtherTok}[1]{\textcolor[rgb]{0.56,0.35,0.01}{#1}}
\newcommand{\PreprocessorTok}[1]{\textcolor[rgb]{0.56,0.35,0.01}{\textit{#1}}}
\newcommand{\RegionMarkerTok}[1]{#1}
\newcommand{\SpecialCharTok}[1]{\textcolor[rgb]{0.00,0.00,0.00}{#1}}
\newcommand{\SpecialStringTok}[1]{\textcolor[rgb]{0.31,0.60,0.02}{#1}}
\newcommand{\StringTok}[1]{\textcolor[rgb]{0.31,0.60,0.02}{#1}}
\newcommand{\VariableTok}[1]{\textcolor[rgb]{0.00,0.00,0.00}{#1}}
\newcommand{\VerbatimStringTok}[1]{\textcolor[rgb]{0.31,0.60,0.02}{#1}}
\newcommand{\WarningTok}[1]{\textcolor[rgb]{0.56,0.35,0.01}{\textbf{\textit{#1}}}}
\usepackage{longtable,booktabs,array}
\usepackage{calc} % for calculating minipage widths
% Correct order of tables after \paragraph or \subparagraph
\usepackage{etoolbox}
\makeatletter
\patchcmd\longtable{\par}{\if@noskipsec\mbox{}\fi\par}{}{}
\makeatother
% Allow footnotes in longtable head/foot
\IfFileExists{footnotehyper.sty}{\usepackage{footnotehyper}}{\usepackage{footnote}}
\makesavenoteenv{longtable}
\usepackage{graphicx}
\makeatletter
\def\maxwidth{\ifdim\Gin@nat@width>\linewidth\linewidth\else\Gin@nat@width\fi}
\def\maxheight{\ifdim\Gin@nat@height>\textheight\textheight\else\Gin@nat@height\fi}
\makeatother
% Scale images if necessary, so that they will not overflow the page
% margins by default, and it is still possible to overwrite the defaults
% using explicit options in \includegraphics[width, height, ...]{}
\setkeys{Gin}{width=\maxwidth,height=\maxheight,keepaspectratio}
% Set default figure placement to htbp
\makeatletter
\def\fps@figure{htbp}
\makeatother
\setlength{\emergencystretch}{3em} % prevent overfull lines
\providecommand{\tightlist}{%
  \setlength{\itemsep}{0pt}\setlength{\parskip}{0pt}}
\setcounter{secnumdepth}{5}
\usepackage{booktabs}
\ifluatex
  \usepackage{selnolig}  % disable illegal ligatures
\fi
\usepackage[]{natbib}
\bibliographystyle{apalike}

\title{Virginia Local Tax Rates, 2019}
\usepackage{etoolbox}
\makeatletter
\providecommand{\subtitle}[1]{% add subtitle to \maketitle
  \apptocmd{\@title}{\par {\large #1 \par}}{}{}
}
\makeatother
\subtitle{38th Annual Edition: Information for All Cities and Counties and Selected Incorporated Towns}
\author{Stephen C. Kulp, Weldon Cooper Center for Public Service, University of Virginia}
\date{2021}

\begin{document}
\maketitle

{
\setcounter{tocdepth}{1}
\tableofcontents
}
\hypertarget{introduction}{%
\chapter*{Introduction}\label{introduction}}
\addcontentsline{toc}{chapter}{Introduction}

\hypertarget{foreward}{%
\section*{Foreward}\label{foreward}}
\addcontentsline{toc}{section}{Foreward}

This is the thirty-eighth edition of the Cooper Center's annual publication about the tax rates of Virginia's local governments. In addition to information about tax rates, the publication contains details about tax administration, valuation methods, and due dates. There is also information on water and sewer rates, waste disposal charges and numerous other aspects of local government finance. This comprehensive guide to local taxes is based on information gathered in the spring, summer, and early fall of 2019. The study includes all of Virginia's 38 independent cities and 95 counties and 118 of the 190 incorporated towns. The included towns account for 92 percent of the Commonwealth's population in towns\(^1\). The study also contains information from several outside sources, including two Department of Taxation studies, 2019 Legislative Summary and The 2017 Assessment/Sales Ratio Study, as well as Department of Taxation information on the assessed value of real estate by type of property. We also used the Auditor of Public Accounts' Comparative Report of Local Government Revenues and Expenditures, Year Ended June 30, 2018, the Commission on Local Governments' Report on Proffered Cash Payments and Expenditures by Virginia's Counties, Cities and Towns, 2017-2018, and the Depart-ment of Housing and Community Development's Virginia Enterprise Zone Program 2018 Grant Year Annual Report.

\hypertarget{organization-of-the-book}{%
\section*{Organization of the Book}\label{organization-of-the-book}}
\addcontentsline{toc}{section}{Organization of the Book}

The study is divided into 26 sections. Section 1 is a reprint of the ``Local Tax Legislation'' section of the Department of Taxation's 2019 Legislative Summary. The original Department of Taxation report is available at its website\(^2\). Sections 2 through 26 cover specific taxes, fees, service charges, cash proffers, enterprise zones, and financial documents on the web. Most of the data came from a detailed web-based ques-tionnaire sent to all cities, counties, and incorporated towns (see Appendix A for a printed version). Appendix B provides a listing of names, phone numbers, and email addresses, when available, of respondents and non-respondents to the questionnaire. Appendix C shows the percentage share of total local taxes represented by each specific tax for each locality based on data from the Auditor of Public Accounts for fiscal year 2018. Information is provided for each city and county and for 38 populous incorporated towns. Finally, Appendix D contains 2018 population estimates for cities, counties and towns from the Cooper Center's Demographics Research Group. The population information is provided to give readers some perspective on the relative size of localities.

Please note that the web addresses provided in this publication were good at the time this text was printed. However, some links are unstable and may not work with certain browsers or they may be modified or withdrawn subsequent to publication.

\hypertarget{about-the-survey}{%
\section*{About the Survey}\label{about-the-survey}}
\addcontentsline{toc}{section}{About the Survey}

In 2019, localities could choose between online or printed versions of the questionnaire. The Cooper Center has made its best efforts to accurately refl ect in this report the responses of localities based on the survey or follow-up queries.

In the tables three dots (\ldots) are used to show there was no response and ``N/A'' is used to indicate ``not applicable.'' Readers may use the telephone/email list in Appendix B to contact local offi cials in order to obtain clarification and additional detail.

\hypertarget{some-components-of-local-taxes}{%
\section*{Some Components of Local Taxes}\label{some-components-of-local-taxes}}
\addcontentsline{toc}{section}{Some Components of Local Taxes}

This book deals mainly with local sources of revenue for local governments. Though localities might also receive federal and state resources, an important part of local funding comes from local sources. The Auditor of Public Accounts, Comparative Report of Local Government Revenues and Expenditures provides data on these local sources. The fol-lowing analysis uses the data from their report for the year ended June 30, 2018.

A common misperception is that nearly all local tax revenue comes from the real property tax. True, the real property tax is the dominant source, accounting for 61.9 percent of city-county tax revenue in FY 2018, the latest year available (see text table below). But three other taxes---the personal property tax, the local option sales and use tax, and the business license tax---together accounted for 24.5 percent of total tax revenue. The remaining 14.6 percent of tax revenue came from more than a dozen other taxes.

\begin{table}

\caption{\label{tab:unnamed-chunk-2}Sources of Virginia Local Government Tax Revenue, FY 2018}
\centering
\begin{tabular}[t]{l|l|r}
\hline
Tax & Amount (\$) & \% of Total\\
\hline
Total taxes & \$17,967,385,766 & 100.00\\
\hline
Real property & \$10,946,877,675 & 60.93\\
\hline
Personal property & \$2,370,758,768 & 13.19\\
\hline
Local option sales and use & \$1,239,855,163 & 6.90\\
\hline
Business license & \$771,958,263 & 4.30\\
\hline
Restaurant meals & \$612,940,580 & 3.41\\
\hline
Public service corporation property & \$412,121,081 & 2.29\\
\hline
Consumer utility & \$327,627,947 & 1.82\\
\hline
Hotel and motel room & \$244,412,96 & 1.36\\
\hline
Machinery and tools & \$233,076,157 & 1.30\\
\hline
Motor vehicle license & \$197,705,384 & 1.10\\
\hline
Recordation and will & \$126,458,487 & 0.70\\
\hline
Bank stock & \$117,199,137 & 0.65\\
\hline
Other local taxes & \$92,124,397 & 0.51\\
\hline
Tobacco & \$65,150,996 & 0.36\\
\hline
Coal, oil, and gas & \$28,510,002 & 0.16\\
\hline
Admission & \$21,815,169 & 0.12\\
\hline
Franchise license & \$16,362,103 & 0.09\\
\hline
Merchants' Capital & \$14,301,188 & 0.08\\
\hline
Penalties and interest & \$128,130,305 & 0.71\\
\hline
\end{tabular}
\end{table}

There are six localities where the real property tax is not dominant. Bath and Surry counties have large power plants that pay public service corporation property taxes that overwhelm other sources. Buchanan County has rich mineral deposits subject to local severance taxes that exceed the real property tax. Covington City and Alleghany County receive large shares of revenue from machinery and tools taxes on MeadWestvaco's large paperboard manufacturing facility. Finally, the small city of Norton, the least populous independent city in Virginia\(^3\) (3,908 in 2018) received almost as much money from the local option sales and use tax as from the real property tax. In the remaining 127 cities and counties where the real property tax is dominant, its relative importance varies from 30.3 percent of total tax revenue in Galax City to 78.8 percent in Lancaster County (see Appendix C).

Thirty-six cities (two cities--Hopewell and Petersburg--did not provide information for the 2018 Comparative Report) and 95 counties imposed four of the taxes shown in the previous table---the real property tax, the personal property tax, the local option sales and use tax, and the public service corporation property tax. Most, but not all, localities imposed recordation and will taxes, consumer utility taxes, motor vehicle license taxes, and taxes on manufacturers' machinery and tools. Nonetheless, as shown in the next text table, there are a number of taxes, a few of them signifi cant sources of revenue, which are not levied by all localities. Also, some of the taxes are used so sparingly that their revenue yield is very low.

\begin{table}

\caption{\label{tab:unnamed-chunk-3}Number of Virginia Localities Imposing Taxes by Type, FY 2018}
\centering
\begin{tabular}[t]{l|r|r|r}
\hline
Tax & Cities & Counties & Total\\
\hline
Real property & 36 & 95 & 131\\
\hline
Personal property & 36 & 95 & 131\\
\hline
Local option sales and use & 36 & 95 & 131\\
\hline
Public service corporation property & 36 & 95 & 131\\
\hline
Consumer utility & 36 & 92 & 128\\
\hline
Recordation and wills & 32 & 93 & 125\\
\hline
Motor vehicle license & 32 & 86 & 118\\
\hline
Machinery and tools property & 31 & 85 & 116\\
\hline
Bank stock & 36 & 64 & 100\\
\hline
Hotel and motel room & 32 & 67 & 99\\
\hline
Business license & 36 & 52 & 88\\
\hline
Restaurant meals & 36 & 49 & 85\\
\hline
Franchise license & 11 & 37 & 48\\
\hline
Merchants’ capital & 1 & 43 & 44\\
\hline
Tobacco & 29 & 2 & 31\\
\hline
Admission & 18 & 3 & 21\\
\hline
Coal, oil, and gas & 1 & 6 & 7\\
\hline
Other local taxes & 23 & 49 & 72\\
\hline
\end{tabular}
\end{table}

There are three major reasons why local governments do not to impose some taxes: (1) The locality lacks a tax base for a particular tax (e.g., a locality must have a bank in order to apply a bank stock tax and a locality must have taxable mineral deposits to impose coal, oil, and gas taxes). (2) The locality is faced with state restrictions (e.g., county excise taxes on hotel and motel room rental have tax rate restrictions imposed by the state; county restaurant meals taxes must be approved in a voter referendum; tobacco taxes are permitted in only two counties; and county admissions taxes are subject to many restrictions). In regard to the busi-ness, professional, and occupational license tax (BPOL tax), counties must choose either the BPOL tax or the merchants' capital tax. Counties are not permitted to impose a business license tax within the boundaries of an incorporated town situated within the county without permission of the town. This means that counties with large shares of business activity within towns are motivated to impose a merchants' capital tax that can be applied countywide. (3) The locality chooses not to impose a permitted tax (e.g., Richmond City, a community with a large cigarette manufacturing plant, has not adopted a consumer tobacco tax even though all cities are granted the authority to levy such a tax).

\hypertarget{partnership-with-lexisnexis}{%
\section*{Partnership with LexisNexis}\label{partnership-with-lexisnexis}}
\addcontentsline{toc}{section}{Partnership with LexisNexis}

The Weldon Cooper Center for Public Service is partner-ing with the publisher LexisNexis to produce the annual Tax Rates books. The Cooper Center still prepares and distributes the survey and writes up the results. LexisNexis publishes the book and fulfi lls orders from interested parties. This arrangement allows us to concentrate on providing the most accurate and up-to-date information about Virginia tax rates and leverages LexisNexis' considerable expertise in production and distribution of the annual volume. We hope the arrangement will lead to continued improvements in our Virginia Local Tax Rates series.

\hypertarget{study-personnel}{%
\section*{Study Personnel}\label{study-personnel}}
\addcontentsline{toc}{section}{Study Personnel}

Stephen C. Kulp, Research Specialist at the Center for Economic and Policy Studies, was responsible for work on the project. He refined the new database, administered the survey, translated the results into tables, checked relevant code sections, assisted with the development of the web-based questionnaire, and made appropriate changes in the text. Jennifer Nelson, of the Cooper Center's Publications Section, designed the cover. Cooper Center employee Albert W. Spengler, who authored this study for a number of years prior to 1991, laid the foundation for the study when it was his responsibility.

The strong support for this publication by the Virginia Association of Counties and the Virginia Municipal League helps ensure our continued efforts to provide this resource as a basic reference on Virginia local taxes.

\hypertarget{final-comments}{%
\section*{Final Comments}\label{final-comments}}
\addcontentsline{toc}{section}{Final Comments}

The Cooper Center is grateful to the many local officials throughout the commonwealth who supplied the survey information presented in this study. Their willingness to provide information and their patience in answering follow-up questions is what makes this book successful. The high response rates could not have been achieved without their cooperation. Corrections to the text or suggestions for possible changes or additions to future editions can be made using the email address and phone number listed below.

Stephen C. Kulp

Research Specialist

Center for Economic and Policy Studies

Weldon Cooper Center for Public Service

University of Virginia

Charlottesville

February 2020

\(^1\) Locality population figures are based on estimates developed by the Demographics Research Group of the Weldon Cooper Center for Public Service. See Appendix D.

\(^2\) \url{https://tax.virginia.gov/legislative-summary-reports}

\(^3\) Weldon Cooper Center for Public Service, University of Virginia. \url{https://demographics.coopercenter.org/population-estimates-age-sex-race-hispanic-towns/}

\hypertarget{summary-of-legislative-changes-in-local-taxation-in-2019}{%
\chapter{Summary of Legislative Changes in Local Taxation in 2019}\label{summary-of-legislative-changes-in-local-taxation-in-2019}}

\hypertarget{general-provisions}{%
\section{General Provisions}\label{general-provisions}}

\hypertarget{local-license-tax-on-mobile-food-units}{%
\subsection{Local License Tax on Mobile Food Units}\label{local-license-tax-on-mobile-food-units}}

Senate Bill 1425 (Chapter 791) provides that when the owner of a new business that operates a mobile food unit has paid a license tax as required by the locality in which the mobile food unit is registered, the owner is not required to pay a license tax to any other locality for conducting business from such mobile food unit in such a locality.\(^1\)

This exemption from paying the license tax in other localities expires two years after the payment of the initial license tax in the locality in which the mobile food unit is registered. During the two year exemption period, the owner is entitled to exempt up to three mobile food units from license taxation in other localities. However, the owner of the mobile food unit is required to register with the Commissioner of the Revenue or Director of Finance in any locality in which he conducts business from such mobile food unit, regardless of whether the owner is exempt from paying license tax in the locality.

This Act defines ``mobile food unit'' as a restaurant that is mounted on wheels and readily moveable from place to place at all times during operation. It also defines ``new business'' as a business that locates for the first time to do business in a locality. A business will not be deemed a new business based on a merger, acquisition, similar business combination, name change, or a change to its business form.

Without the exemption provided in this Act, localities are authorized to impose business, professional and occupational license (BPOL) taxes upon local businesses. Generally, the BPOL tax is levied on the privilege of engaging in business at a definite place of business within a Virginia locality. Businesses that are mobile, however, can be subject to license taxes or fees in multiple localities in certain situations.

Effective: July 1, 2019

Added: § 58.1-3715.1

\hypertarget{local-gas-road-improvement-tax-extension-of-sunset-provision}{%
\subsection{Local Gas Road Improvement Tax; Extension of Sunset Provision}\label{local-gas-road-improvement-tax-extension-of-sunset-provision}}

House Bill 2555 (Chapter 24) and Senate Bill 1165 (Chapter 191) extend the sunset date for the local gas road improve-ment tax from January 1, 2020 to January 1, 2022. The authority to impose the local gas road improvement tax was previously scheduled to sunset on January 1, 2020.

The localities that comprise the Virginia Coalfield Economic Development Authority may impose a local gas road improvement tax that is capped at a rate of one percent of the gross receipts from the sale of gases severed within the locality. Under current law, the revenues generated from this tax are allocated as follows: 75\% are paid into a special fund in each locality called the Coal and Gas Road Improvement Fund, where at least 50\% are spent on road improvements and 25\% may be spent on new water and sewer systems or the construction, repair, or enhancement of natural gas systems and lines within the locality; and the remaining 25\% of the revenue is paid to the Virginia Coal-fi eld Economic Development Fund. The Virginia Coalfi eld Economic Development Authority is comprised of the City of Norton, and the Counties of Buchanan, Dickenson, Lee, Russell, Scott, Tazewell, and Wise.

Effective: July 1, 2019

Amended: § 58.1-3713

\hypertarget{private-collectors-authorized-for-use-by-localities-to-collect-delinquent-debts}{%
\subsection{Private Collectors Authorized for Use by Localities to Collect Delinquent Debts}\label{private-collectors-authorized-for-use-by-localities-to-collect-delinquent-debts}}

Senate Bill 1301 (Chapter 271) allows a local treasurer to employ private collection agents to assist with the collection of delinquent amounts due other than delinquent local taxes that have been delinquent for a period of three months or more and for which the appropriate statute of limitations has not run.

Effective: July 1, 2019

Amended: § 58.1-3919.1

\hypertarget{real-property-tax}{%
\section{Real Property Tax}\label{real-property-tax}}

\hypertarget{real-property-tax-exemptions-for-elderly-and-disabled-computation-of-income-limitation}{%
\subsection{Real Property Tax Exemptions for Elderly and Disabled: Computation of Income Limitation}\label{real-property-tax-exemptions-for-elderly-and-disabled-computation-of-income-limitation}}

House Bill 1937 (Chapter 16) provides that, if a locality has established a real estate tax exemption for the elderly and handicapped and enacted an income limitation related to the exemption, it may exclude, for purposes of calculating the income limitation, any disability income received by a family member or nonrelative who lives in the dwelling and who is permanently and totally disabled.

Under current law, if a locality's tax relief ordinance establishes an annual income limitation, the computation of annual income is calculated by adding together the income received during the preceding calendar year of the owners of the dwelling who use it as their principal residence; and the owners' relatives who live in the dwelling, except for those relatives living in the dwelling and providing bona fide caregiving services to the owner whether such relatives are compensated or not; and at the option of each locality, nonrelatives of the owner who live in the dwelling except for bona fide tenants or bona fide caregivers of the owner, whether compensated or not.

Effective: July 1, 2019

Amended: § 58.1-3212

\hypertarget{real-property-tax-exemption-for-elderly-and-disabled-improvements-to-a-dwelling}{%
\subsection{Real Property Tax Exemption for Elderly and Disabled: Improvements to a Dwelling}\label{real-property-tax-exemption-for-elderly-and-disabled-improvements-to-a-dwelling}}

House Bill 2150 (Chapter 736) and Senate Bill 1196 (Chapter 737) clarify the definition of ``dwelling,'' for purposes of the real property tax exemption for owners who are 65 years of age or older or permanently and totally disabled, to include certain improvements to the exempt land and the land on which the improvements are situated. These Acts define the term ``dwelling'' to include an improvement to the land that is not used for a business purpose but is used to house certain motor vehicles or household goods.

Under current law, in order to be granted real property tax relief, qualifying property must be owned by and occupied as the sole dwelling of a person who is at least 65 years of age, or, if the local ordinance provides, any person with a permanent disability. Dwellings jointly held by spouses, with no other joint owners, qualify if either spouse is 65 or over or permanently and totally disabled.

Effective: July 1, 2019

Amended: § 58.1-3210

\hypertarget{real-property-tax-partial-exemption-from-real-property-taxes-for-flood-mitigation-efforts}{%
\subsection{Real Property Tax: Partial Exemption from Real Property Taxes for Flood Mitigation Efforts}\label{real-property-tax-partial-exemption-from-real-property-taxes-for-flood-mitigation-efforts}}

Senate Bill 1588 (Chapter 754) enables a locality to provide by ordinance a partial exemption from real property taxes for flooding abatement, mitigation, or resiliency efforts for improved real estate that is subject to recurrent flooding, as authorized by an amendment to Article X, Section 6 of the Constitution of Virginia that was adopted by the voters on November 6, 2018.

This act provides that exemptions may only be granted for qualifying flood improvements that do not increase the size of any impervious area and are made to qualifying structures or to land. ``Qualifying structures'' are defined as structures that were completed prior to July 1, 2018 or were completed more than 10 years prior to the completion of the improvements. For improvements made to land, the improvements must be made primarily for the benefit of one or more qualifying structures. No exemption will be authorized for any improvements made prior to July 1, 2018.

A locality is granted the authority to (i) establish flood protection standards that qualifying flood improvements must meet in order to be eligible for the exemption; (ii) determine the amount of the exemption; (iii) set income or property value limitations on eligibility; (iv) provide that the exemption shall only last for a certain number of years; (v) determine, based upon flood risk, areas of the locality where the exemption may be claimed; and (vi) establish preferred actions for qualifying for the exemption, including living shorelines.

Effective: July 1, 2019

Amended: § 58.1-3228.1

\hypertarget{real-property-tax-exemption-for-certain-surviving-spouses}{%
\subsection{Real Property Tax: Exemption for Certain Surviving Spouses}\label{real-property-tax-exemption-for-certain-surviving-spouses}}

House Bill 1655 (Chapter 15) and Senate Bill 1270 (Chapter 801) allow surviving spouses of disabled veterans to continue to qualify for a real property tax exemption regardless of whether the surviving spouse moves to a different residence, as authorized by an amendment to subdivision (a) of Section 6-A of Article X of the Constitution of Virginia that was adopted by the voters on November 6, 2018. If a surviving spouse was eligible for the exemption but lost such eligibility due to a change in residence, then the surviving spouse is eligible for the exemption again, beginning January 1, 2019.

These Acts also clarify that the real property tax exemptions for spouses of service members killed in action and spouses of certain emergency service providers killed in the line of duty continue to apply regardless of the spouse's moving to a new principal residence.

Effective: Taxable years beginning on or after January 1, 2019

Amended: §§ 58.1-3219.5, 3219.9, and 3219.14

\hypertarget{land-preservation-special-assessment-optional-limit-on-annual-increase-in-assessed-value}{%
\subsection{Land Preservation; Special Assessment, Optional Limit on Annual Increase in Assessed Value}\label{land-preservation-special-assessment-optional-limit-on-annual-increase-in-assessed-value}}

House Bill 2365 (Chapter 22) authorizes localities that employ use value assessments for certain classes of real property to provide by ordinance that the annual increase in the assessed value of eligible property may not exceed a specified dollar amount per acre.

Effective: July 1, 2019

Amended: § 58.1-3231

\hypertarget{virginia-regional-industrial-act-revenue-sharing-composite-index}{%
\subsection{Virginia Regional Industrial Act: Revenue Sharing; Composite Index}\label{virginia-regional-industrial-act-revenue-sharing-composite-index}}

House Bill 1838 (Chapter 534) requires that the Department of Taxation's calculation of the true values of real estate and public service company property component of the Commonwealth's educational composite index of local ability-to-pay take into account arrangements by localities entered into pursuant to the Virginia Regional Industrial Facilities Act, whereby a portion of tax revenue is initially paid to one locality and redistributed to another locality. This Act requires such calculation to properly apportion the percentage of tax revenue ultimately received by each locality.

Effective: July 1, 2021

Amended: § 58.1-6407

\hypertarget{real-estate-with-delinquent-taxes-or-liens-appointment-of-special-commissioner-increase-required-value}{%
\subsection{Real Estate with Delinquent Taxes or Liens: Appointment of Special Commissioner; Increase Required Value}\label{real-estate-with-delinquent-taxes-or-liens-appointment-of-special-commissioner-increase-required-value}}

House Bill 2060 (Chapter 541) increases the assessed value of a parcel of land that could be subject to appointment of a special commissioner to convey the real estate to a locality as a result of unpaid real property taxes or liens from \$50,000or less to \$75,000 or less in most localities. In the Cities of Norfolk, Richmond, Hopewell, Newport News, Petersburg, Fredericksburg, and Hampton, this Act increases the threshold from \$100,000 or less to \$150,000 or less.

Effective: July 1, 2019

Amended: § 58.1-3970.1

\hypertarget{real-estate-with-delinquent-taxes-or-liens-appointment-of-special-commissioner-in-the-city-of-martinsville}{%
\subsection{Real Estate with Delinquent Taxes or Liens; Appointment of Special Commissioner in the City of Martinsville}\label{real-estate-with-delinquent-taxes-or-liens-appointment-of-special-commissioner-in-the-city-of-martinsville}}

House Bill 2405 (Chapter 159) adds the city of Martinsville to the list of cities (Norfolk, Richmond, Hopewell, Newport News, Petersburg, Fredericksburg, and Hampton) that are authorized to have a special commissioner convey tax-delinquent real estate to the locality in lieu of a public sale at auction when the tax-delinquent property has an assessed value of \$100,000 or less. House Bill 2060 raises the threshold in all of these cities from \$100,000 or less to \$150,000 or less.

Effective: July 1, 2019

Amended: § 58.1-3970.1

\hypertarget{personal-property-tax}{%
\section{Personal Property Tax}\label{personal-property-tax}}

\hypertarget{constitutional-amendment-personal-property-tax-exemption-for-motor-vehicle-of-a-disabled-veteran}{%
\subsection{Constitutional Amendment: Personal Property Tax Exemption for Motor Vehicle of a Disabled Veteran}\label{constitutional-amendment-personal-property-tax-exemption-for-motor-vehicle-of-a-disabled-veteran}}

House Joint Resolution 676 (Chapter 822) is a fi rst resolu-tion proposing a constitutional amendment that permits the General Assembly to authorize the governing body of any county, city, or town to exempt from taxation one motor vehicle of a veteran who has a 100 percent service-connected, permanent, and total disability. The amendment provides that only automobiles and pickup trucks qualify for the exemption.

Additionally, the exemption will only be applicable on the date the motor vehicle is acquired or the effective date of the amendment, whichever is later, but will not be applicable for any period of time prior to the effective date of the amendment.

Effective: July 1, 2019

\hypertarget{personal-property-tax-exemption-for-agricultural-vehicles}{%
\subsection{Personal Property Tax Exemption for Agricultural Vehicles}\label{personal-property-tax-exemption-for-agricultural-vehicles}}

House Bill 2733 (Chapter 259) expands the definition of agricultural use motor vehicles for personal property taxation purposes. It changes the criteria from motor vehicles used ``exclusively'' for agricultural purposes to motor vehicles used ``primarily'' for agricultural purposes, and for which the owner is not required to obtain a registration certificate, license plate, and decal or pay a registration fee.

It also expands the definition of trucks or tractor trucks that are used by farmers in their farming operations for the transportation of farm animals or other farm products or for the transport of farm-related machinery. The criteria is changed from vehicles used ``exclusively'' by farmers in their farming operations to vehicles used ``primarily'' by farmers in their farming operations.

Further, this Act expands the classifi cation of farm machinery and equipment that a local governing body may exempt, to include equipment and machinery used by a nursery for the production of horticultural products, and any farm tractor, regardless of whether such farm tractor is used exclusively for agricultural purposes.

Local governing bodies have the option to exempt these classifi cations, in whole or in part, from taxation or to provide for a different rate of taxation thereon.

Effective: July 1, 2019

Amended: § 58.1-3505

\hypertarget{intangible-personal-property-tax-classifi-cation-of-certain-business-property}{%
\subsection{Intangible Personal Property Tax: Classifi cation of Certain Business Property}\label{intangible-personal-property-tax-classifi-cation-of-certain-business-property}}

House Bill 2440 (Chapter 255) classifies as intangible per-sonal property, tangible personal property: i) that is used in a trade or business; ii) with an original cost of less than \$25; and iii) that is not classified as machinery and tools, merchants' capital, or short-term rental property. It also exempts such property from taxation.

Intangible personal property is a separate class of prop-erty segregated for taxation by the Commonwealth. The Commonwealth does not currently tax intangible personal property. Localities are prohibited from taxing intangible personal property.

Certain personal property, while tangible in fact, has previously been designated as intangible and thus exempted from state and local taxation. For example, tangible personal property used in manufacturing, mining, water well drill-ing, radio or television broadcasting, dairy, dry cleaning, or laundry businesses has been designated as exempt intangible personal property.

Effective: July 1, 2019

Amended: §§ 58.1-1101 and 58.1-1103

\begin{center}\rule{0.5\linewidth}{0.5pt}\end{center}

\(^1\) Excerpted from the local tax legislation section of the Department of Taxation's 2019 Legislative Summary. Minor changes were made in format and punctuation. See \url{https://tax.virginia.gov/legislative-sum-mary-reports}

\hypertarget{real-property-tax-in-2019}{%
\chapter{Real Property Tax in 2019}\label{real-property-tax-in-2019}}

The real property tax is by far the most important source of tax revenue for localities. In fiscal year 2018, the most recent year available from the Auditor of Public Accounts, it accounted for 55.5 percent of tax revenue for cities, 64.6 percent for counties, and 29.1 percent for large towns. These are averages; the relative importance of taxes in individual cities, counties, and towns varies significantly. For information on individual localities, see Appendix C.
The \emph{Code of Virginia}, §§ 58.1-3200 through 58.1-3389, authorizes localities to levy taxes on real property (land, including the buildings and improvements on it). There is no restriction on the tax rate that may be imposed. Section 58.1-3201 provides that all general reassessments or annual assessments shall be at 100 percent of fair market value.

\hypertarget{public-service-corporations}{%
\section{PUBLIC SERVICE CORPORATIONS}\label{public-service-corporations}}

Property owned by so-called public service corporations is not assessed by localities. Instead, that task is delegated to the State Corporation Commission (SCC) and the Department of Taxation.The State Corporation Commission assesses electric utilities and cooperatives, gas pipeline distribution companies, public service water companies, and telephone and telegraph companies. The Department of Taxation assesses pipeline transmission companies and railroads.

In fiscal year 2018, the property tax on public service corporations accounted for 1.7 percent of tax revenue for
cities, 2.6 percent for counties, and 0.8 percent for large towns. These are averages; the relative importance of the tax
in individual cities, counties, and towns varies significantly. In two counties with large power generating facilities the
property tax on public service corporations accounts for a very large share of local tax revenue. In Bath County the share was 47.6 percent and in Surry County the share was 61.1 percent. For more information on individual localities, see Appendix C.

The commissioner of the revenue or another designated official in each city or county is required to provide by
January 1 of each year to any public service company with property in its area a copy of the property boundaries of
the locality in which any part of the company is located (§ 58.1-2601). The State Corporation Commission or the
Department of Taxation send out their assessments for the property based on these boundaries (§ 58.1-2602). Localities examine the assessments to determine their correctness. If correct, the locality determines the equalized assessed valuation of the corporate property by applying the local assessment ratio prevailing in the locality for other real estate (§ 58.1-2604). Local taxes are then assigned to real and tangible personal property at the real property tax rate current in the locality (§ 58.1-2606).

\hypertarget{tax-relief-programs}{%
\section{TAX RELIEF PROGRAMS}\label{tax-relief-programs}}

There are several types of locally financed tax relief programs available. Section 3 of this study contains information on so-called circuit breaker plans for the elderly and disabled. Section 4 covers land use assessments for agricultural, horticultural, forestal, and open space real estate. Section 5 contains information on preferential assessments for
agricultural and forestal districts. Finally, Section 6 covers property tax exemptions for certain rehabilitated real estate and other exemptions.

Only the city of Charlottesville, Loudoun County, and Arlington County reported providing tax relief for low-income
owners and renters who are not elderly. The city of Charlottesville administers the Charlottesville Housing Affordability Program (CHAP) to help low and middle income homeowners. The program awards grants up to \$1,000 to homeowners with houses assessed at less than \$375,000 and having an annual income less than \(55,000.\)\^{}1\$ Loudoun County administers the Affordable Dwelling Unit Program for renters and first-time buyers. Buyers need an income greater than 30 percent but less than 70 percent of the area median income to participate. Qualified renters are eligible to rent apartments at rates from \$630 to \$1,300. Arlington County's Housing Grants Program is available to working families with at least one child under age 18. Personal assets may not exceed \$35,000 and there is an income limit based on household size.

Localities are permitted to institute deferral for a portion of the real estate tax by § 58.1-3219 of the \emph{Code of Virginia}. Localities are permitted to grant deferrals from the full amount by which each taxpayer's real estate tax levy exceeds 105 percent of the previous year's tax, or such higher percentage adopted by the locality. Deferred taxes are subject to interest in an amount established by the governing body, not to exceed the rate published by the IRS code.\(^2\) The deferral potentially applies to every property owner, not just the elderly and disabled. (For deferrals limited to the elderly and disabled see Section 3 of this study.)

The deferral program is rarely used. Administrative problems appear to be the major reason for the unpopularity of deferral programs. Loudoun County had a deferral program in place in the 1990s but terminated it ``\ldots{} because the program was administratively complex, cumbersome and required staff time in disproportion to the benefit received by the taxpayer.''\(^3\) The cities of Alexandria, Falls Church, and Fairfax and the counties of Fairfax and Henrico considered deferral but
did not adopt it. According to Henrico staff, ``The administrative procedures for tracing the properties and recovering the
relevant taxes upon either the death of the owner or transfer of the property itself would be both cumbersome and time consuming and could not be accomplished with existing staffing levels or existing computer systems.''\(^4\) Another reason for the unpopularity of the programs may be that taxpayers only receive postponement, not removal, of the tax liability. The cities of Charlottesville and Richmond, the county of Middlesex, and the town of Amherst were the only localities reporting a deferral program in 2019.

\hypertarget{statutory-rates-special-taxes-due-dates-proration-and-billing-practices}{%
\section{STATUTORY RATES, SPECIAL TAXES, DUE DATES, PRORATION, AND BILLING PRACTICES}\label{statutory-rates-special-taxes-due-dates-proration-and-billing-practices}}

\textbf{Table 2.1} provides general information associated with real property taxes in Virginia's localities. The table provides
an estimate by locality of both the number of total taxable real estate parcels and the number of residential parcels.
Twenty-seven cities, 80 counties and 52 towns provided estimates of one or both types of parcels. The total number of parcels in cities ranged from a high of 158,431 (Virginia Beach) down to 2,456 (Lexington). Among counties, the number of parcels ranged from a high of 354,687 (Fairfax) down to 3,940 (Highland).

Table 2.1 also lists the statutory (nominal) tax rates. The statutory rate is the rate used by localities and is applied to the assessed value of a property. In the table, statutory rates are listed under calendar year (CY) or fiscal year (FY) columns depending on the locality's assessment cycle. In most cases the calendar year tax rate listed runs from January 1 to December 31 and the fiscal year rate runs from July 1 to June 30. The provisions explaining the assessment cycle requirements are found in § 58.1-3010 and § 58.1-3011 of the \emph{Code of Virginia}. However, some localities report a calendar year assessment schedule with a fiscal year valuation. Six cities (Chesapeake, Harrisonburg, Martinsville, Roanoke, Salem, and Suffolk) and one county (James City) report this practice. Otherwise, 15 cities and 88 counties reported using the calendar year cycle while 176 cities and 6 counties used fiscal year assessment cycles.

The statutory tax rates were reported to the Cooper Center by all cities and counties and 112 of the responding towns. The text table below lists the averages for the statutory rates from the localities.

\begin{Shaded}
\begin{Highlighting}[]
\CommentTok{\#table name: Statutory Real Estate Tax Rates per $100 of Assessed Taxable Value for Localities Reporting, CY 2019 and FY 2020}
\end{Highlighting}
\end{Shaded}

Statutory rates are generally higher for cities than counties. The rates are lowest in towns because they are subordinate to counties and have limited responsibilities.

Tax due dates vary among localities. Generally, if taxes are paid annually, they are due by December 5. If paid semiannually, they are due by June 5 and December 5. However, some localities have different due dates, as provided by § 58.1-3916 of the \emph{Code}.

Most cities have semiannual tax due dates with payments required in June and December. Of the 38 cities, 2 required taxes due annually, 31 semiannually, and 5 quarterly. Among the counties, 32 had annual tax due dates, while 63 had semiannual requirements. Of the towns responding to this question, 80 reported annual due dates, and 32 required semiannual payments.

A locality is permitted to prorate the taxable amount. Any county, city, or town electing to prorate new buildings which are substantially complete prior to November 1 must do so at the time the building is complete or fit to live in. Of the 38 cities, 33 reported prorating taxes while 5 reported not doing so. Among counties, 67 prorated their taxes while 28 did not. Reports from the towns that answered this question indicated that 47 prorated their taxes while 652 did not.

The final column of Table 2.1 pertains to town billing practices. Three possibilities exist: (1) a town sends out its own bills and collects its taxes (TT in the table), (2) a town collects its taxes but the county sends the bills (CT in the
table), or (3) a town has the county bill and collect the taxes (CC in the table). Of the towns that answered the question,
the overwhelming majority, 100, reported billing and collecting their own taxes. Four said they collected taxes, while in three the county both billed and collected town taxes.

\textbf{Table 2.2, Table 2.3,} and \textbf{Table 2.4} provide additional information concerning statutory real property tax rates. The \emph{Code} allows localities to add special purpose levies on top of the real property rate for various purposes. Table 2.2 deals with the category of special districts. A special district is organized to perform a single governmental function or a restricted number of related functions. Special districts usually have the power to incur debt and levy taxes to fund special activities such as capital improvements, emergency services, sewer and water services, or pest control within those districts. Thirteen cities, 14 counties, and 4 towns reported levying these taxes. The table includes the base (statutory) rate for the locality, the district in which the activity takes place, the purpose of the activity, and the special rate imposed for that activity. Most special activity taxes are in addition to the base rate, though some are simply a flat fee, and others are a percentage rate based on improvements to the property.

Another special district category is the community development authority (CDA). Such an authority is a district created by the locality based on a petition from the property owners to help develop and maintain desired public infrastructure improvements, such as roads and buildings. The CDA is usually associated with development interests, such as retail centers, industrial centers, or tourism centers. Generally the CDA pays for development by issuing bonds and then having the property owners pay special assessments based on the level of debt. Assessments are levied either by placing a tax, such as \$0.25 per \$100 of assessed value, on the property within the district or by a special assessment each year that determines the benefit from the improvements and allocates them by property value. Depending on how the bond agreement is structured, assessment payments may be made directly to bondholders or to the locality. Table 2.3 lists community development authorities by locality. The table includes the name of the project, the purpose, the size, the bond amount, and, where possible, the current value. Three cities and 8 counties reported having CDAs.

The final category of special districts is that of localities within the Northern Virginia Transportation Authority. Localities within this authority have the ability to tax real property associated with industrial and commercial use up
to \$0.125 per \$100 of assessed value to help fund transportation improvements. In 2009, an amendment to § 58.1-3221.3 specified that the revenues generated by the tax were to be used solely for (1) new road construction, design, and right-of-way acquisition, (2) new public transit construction, design, and right-of-way acquisition, (3) capital costs related to new transportation projects, or (4) the issuance costs and debt service on any bonds issued to support capital costs. There are 11 localities in the region of the authority: the cities of Alexandria, Fairfax, Falls Church, Fredericksburg, Manassas, and Manassas Park and the counties of Arlington, Fairfax, Loudoun, Prince William, and Stafford. Of those, one city (Fairfax) and two counties (Arlington and Fairfax) reported implementing the tax, as shown in Table 2.4.

\hypertarget{assessment-practices-reassessments-assessed-values}{%
\section{ASSESSMENT PRACTICES, REASSESSMENTS, ASSESSED VALUES}\label{assessment-practices-reassessments-assessed-values}}

\textbf{Table 2.5} details assessment practices among localities. The table includes cities and counties, but not towns, because only a small percentage of towns provided substantive answers. For those interested in the towns that responded, data are available from the Cooper Center upon request.

The second column lists whether a locality has a full-time assessor. Twenty-seven cities reported employing a full-time property tax assessor, while 11 did not. In contrast, only 36 counties had a full-time assessor while 59 did not. This reflects the fact that many counties reassess property less frequently than cities. No towns had assessors, since towns rely on assessed values established by their host counties.

Columns three, four, five, and six of Table 2.5 provide data on the conduct of general reassessments and cover four questions. (1) Are reassessments done by the locality or contracted out? (2) What is the reassessment frequency? (3) Is physical inspection part of the reassessment? (4) When was the reassessment last done? Regarding the conduct of the general reassessment, 28 cities reported conducting reassessments in-house while 10 reported contracting with outside assessors. Twenty-eight counties reported doing general reassessments in-house, while 67 reported contracting out for services. Section 58.1-3250 of the \emph{Code} requires cities to have a general reassessment of real estate every two years. However, any city with a total population of 30,000 or less may elect to conduct its general reassessments at four-year intervals.\(^5\) Counties are required to have a general reassessment every four years (§ 58.1-3252). There is an exception for counties with a total population of 50,000 or less. These counties may elect to reassess at either five-year or six-year intervals (§ 58.1-3252). However, nothing in these sections affects the power of cities and counties to reassess more frequently. A large majority of the cities (30) reassess at one or two year intervals. In contrast, less than three out of ten counties (27) reassess that frequently. Virtually all of the populous cities and counties reassess annually or biennially. Towns rely on their surrounding county to provide assessments, so a town's reassessment occurs with the same frequency as the county's. The reassessment periods are summarized in the table on the following page.

Column seven of Table 2.5 shows information about maintenance assessments. While general reassessments involve reassessing all parcels to reflect changes in market value, maintenance assessments involve adjusting assessed values between reassessments because of new construction, improvements, damages, demolitions, subdivisions, and consolidations. Thirty-three cities responded that they performed maintenance assessments using staff, while five reported contracting for the work. Among counties, 66 reported performing maintenance reassessments using staff, while 29 reported contracting the work to independent appraisers.

Columns eight and nine of Table 2.5 cover physical inspection. Physical inspection refers to the actual inspection of the property as opposed to computerized mass-appraisal of parcels. If a locality responded that it did not perform physical inspections during the general reassessment, two further questions were asked:

\begin{Shaded}
\begin{Highlighting}[]
\CommentTok{\#table name: Reassessment Periods for Real Estate, 2019}
\end{Highlighting}
\end{Shaded}

\begin{enumerate}
\def\labelenumi{(\arabic{enumi})}
\tightlist
\item
  Does the locality perform a physical inspection at all? (2) If so, what is the inspection cycle? Among cities that responded, 18 reportedly did not have a physical inspection separate from the general reassessment cycle. Twenty others reported having a physical inspection cycle, the periods ranging anywhere from two to six years. Among counties that responded, 70 indicated they performed physical inspections during general reassessment, while 25 reported having physical inspection cycles ranging anywhere from one to six years.
\end{enumerate}

\textbf{Table 2.6} provides unpublished Department of Taxation 2018 data on total taxable assessed value of real estate by category. Taxable assessed value shows property qualifying for use value at its use value, not its market value. The percentage distribution of taxable assessed value is shown for two types of residential property (single-family and multi-family) as well as commercial and industrial property and agricultural property.

The text table on the next page compares the taxable assessed value by category for cities and counties. The total assessed value for all cities amounted to \$277.4 billion. Single-family residential property averaged 64.9 percent of taxable assessed value. Multi-family residential property averaged 11.1 percent of taxable assessed value. Commercial/industrial properties averaged just over one-quarter of the total value at 23.9 percent, while agricultural property values amounted to only 0.1 percent.

The total assessed value of property by category for counties in 2018 amounted to \$854.5 billion. Of that amount, 72.0 percent of assessed value was associated with single-family residential property, 5.9 percent with multi-family residential property, 18.2 percent with commercial/industrial property, and 4.0 percent with agricultural property.

With the total amounts from cities and counties combined, the total assessed valuation amounted to \$1,131.9 billion. Of that, 70.2 percent applied to single-family residential property, 7.1 percent applied to multi-family residential property, 19.6 percent applied to commercial/industrial property, and 3.0 percent to agricultural property.

Looking at the percentage breakdown for each type of locality, in 2018 the share of taxable

\begin{Shaded}
\begin{Highlighting}[]
\CommentTok{\#table name: Taxable Assessed Value by Category for Cities and Counties, 2018}
\end{Highlighting}
\end{Shaded}

assessed value for cities in the single-family residential category was between 40 percent and 59.9 percent in 19 cities and 60 percent or more in 18 cities. All cities but two had multi-family residential values under 19.9 percent of the total assessed value. Commercial and industrial property was the second most common category with 21 of the cities having between 20 percent and 39.9 percent of their property valuations coming from this type of property. Finally, only the cities of Suffolk and Franklin had more than 2 percent of their property valuation associated with agriculture.

Among counties the breakdown was slightly different. As in cities, the single-family residential value dominated the percentage breakdown. The single-family residential assessment percentage amounted to 60 percent or more for 71 counties. Another 20 received between 40 percent and 59.9 percent of the valuation from single-family residential real estate, while in four counties residential valuations amounted to no more than 39.9 percent of the total taxable assessed value (Buchanan, Dickenson, Highland, and Sussex). In contrast, only in Arlington county did the multi-family residential average share of value exceed 19.9 percent.

The category with the second highest valuation in counties was commercial and industrial property. Eighty-two counties had such property valued no higher than 19.9 percent of the total assessed value of property within the locality. In general, the percentage of assessed value in counties for commercial and industrial properties was less than that for cities (though two counties, coal-rich Dickenson and Buchanan, had the highest percentage valuations of such property). Finally, agricultural property averaged the least total assessed valuation in counties, though the percentage varied greatly among the individual counties. In 30 counties, valuations associated with agricultural property made up 20 percent or more of the total assessed value within the locality. The percentage in one county (Sussex) was 82.0 percent. The taxable assessed values for agriculture were much lower than they would have been without the advantage of use value assessment, a program explained in Sections 4 and 5.

\hypertarget{effective-tax-rates}{%
\section{EFFECTIVE TAX RATES}\label{effective-tax-rates}}

Tax rates are generally discussed in terms of either statutory (nominal) rates or effective rates. The statutory rate is the rate used by localities and is applied to the assessed value of a property.The effective rate is published by the Virginia Department of Taxation in their annual assessment/sales ratio study. The department derives the effective rate by multiplying the statutory tax rate by the median assessment ratio. In normal times when property values are rising, the median assessment ratio is usually less than 100 percent

\begin{Shaded}
\begin{Highlighting}[]
\CommentTok{\#table name: Share of Assessed Value of Real Estate by Category, 2018}
\end{Highlighting}
\end{Shaded}

because reassessments lag market increases and tend to be conservative. Consequently, the statutory rate is generally higher than the effective rate. However, this may not be true in difficult real estate markets. A limitation of the effective rates published by the Virginia Department of Taxation is that they are not current. The most recent year available at the present time is 2017. Despite the time lag, effective rates are important because they give a more accurate reflection of the differences in real property tax rates across localities.

\textbf{Table 2.7} shows city and county average effective tax rates in the year 2017. The department makes its computation in order to control for the variance in localities' assessment procedures and timing. Therefore, when comparing tax rates among localities, the reader may wish to consult both Tables 2.1 and 2.7. Table 2.1 shows statutory rates in 2019. Table 2.7 shows statutory and effective rates in 2017. The following text table summarizes the effective tax rates for the localities shown in Table 2.7.

It should also be pointed out that the Virginia Department of Taxation does not use the locally reported statutory tax rate in its computations. Instead, it calculates the statutory rate by dividing the real estate levy by the local real

\begin{Shaded}
\begin{Highlighting}[]
\CommentTok{\#table name: Effective Real Estate Tax Rates, 2017}
\end{Highlighting}
\end{Shaded}

estate \emph{taxable assessed value},\(^6\) as reported in the local land book. This method of computing the statutory tax rate takes additional district levies into account.\(^7\)

In 2 cities and 10 counties the statutory rate was less than the effective rate. In two cities and seven counties statutory and effective rates were the same. Finally, in 34 cities and 78 counties statutory rates exceeded effective rates.

\begin{Shaded}
\begin{Highlighting}[]
\CommentTok{\#table name: Statutory and Effective Real Estate Tax Rates, 2017}
\end{Highlighting}
\end{Shaded}

The real property tax rates reported in Table 2.7 are a more accurate reflection of the differences among localities in tax rates on real property than those in Table 2.1 because they control for variations in assessment frequency and technique among localities. Table 2.7 also shows the latest reassessment in effect when the median ratio study was conducted, the number of sales used in the study, the median ratio, and the coefficient of dispersion.

The coefficient of dispersion measures how closely the individual ratios of each locality are arrayed around the median ratio. The formula for the coefficient of dispersion (CD) is:

where \[X_i\] represents the assessment/sales ratio for the \emph{i}th sale in a sample of size \emph{n}, and \[X_m\] represents the median ratio of the sample.\(^8\) If there were no dispersion, the CD would equal zero.

The text table below summarizes the coefficients of dispersion tabulated for the cities and counties. Eighteen of the cities had CDs of no more than 9.9 percent. Eight had CDs between 10 percent and 14.9 percent, 7 had CDs between 15

\begin{Shaded}
\begin{Highlighting}[]
\CommentTok{\#table name: Coefficient of Dispersion, 2017}
\end{Highlighting}
\end{Shaded}

and 19.9 percent, and 4 had CDs between 20 and 24.9 percent. Counties tended to vary more in the degree of dispersion. Thirteen had CDs between 5 and 9.9 percent, 18 had CDs between 10 and 14.9 percent, 25 had CDs between 15 and 19.9 percent, 26 had CDs between 20 and 24.9 percent, 11 had CDs between 25 and 29.9 percent, and 2 had CDs between 30 and 34.9 percent.

There is no upper limit for what is tolerable, but the International Association of Assessing Officers recommends an upper limit of 15 percent for residential properties.\(^9\) Twenty-eight cities and 34 counties met the 15 percent standard.\(^10\)

As one would expect, the quality of local assessments, as measured by the CD is generally better in those localities that reassess annually, biennially, or that have just conducted a general reassessment. In 2017, of the 57 localities with CDs under 15 percent, all but 12 reassessed annually (28), biennially (10), or had just completed general reassessments (7).

\hypertarget{miscellaneous-items}{%
\section{MISCELLANEOUS ITEMS}\label{miscellaneous-items}}

\textbf{Table 2.8} presents miscellaneous taxes and exemptions related to real property. The first is the recreation tax. The \emph{Code} in §15.2-1807 permits localities to collect a real estate tax for recreation areas and playgrounds that is not to exceed \$0.02/\$100 of the assessed value of a property. This tax was reported by Charlottesville City.

The second column refers to the tax deferral ordinance permitted by § 58.1-3219 regarding the deferral of a portion of real estate tax increases when the tax exceeds 105 percent of the real property tax on property owned by a taxpayer in the previous year. Four localities (Charlottesville City, Richmond City, Middlesex County, and Amherst Town) reported implementing this deferral.

The third column refers to the establishment of a tax increment financing fund used to encourage development in certain areas and permitted by § 58.1-3245 of the \emph{Code}. Six cities (Bristol, Charlottesville, Chesapeake, Emporia, Newport News, Virginia Beach, and Waynesboro), four counties (Arlington, Augusta, Fairfax, and Hanover), and one town (Front Royal) reported having implemented such a fund.

The fourth column refers to separate real property tax rates for energy-efficient buildings as permitted by § 58.1-3221.2 of the \emph{Code}. Three cities (Charlottesville, Roanoke, and Virginia Beach) reported having special rates for such real estate.

The fifth column lists localities that reported providing a separate real property classification for improvements to real property used in the manufacture of renewable energy. Only the cities of Charlottesville and Roanoke reported having this separate rate.

Finally, the last column refers to low-income grant programs, discussed earlier in this text under the subheading, ``Tax Relief Programs.'' Only the cities of Charlottesville and Norfolk, and the county of Arlington reported having these programs.

\begin{Shaded}
\begin{Highlighting}[]
\CommentTok{\#Table 2.1 "Real Property Statutory (Nominal) Tax Rates, CY 2019 and FY 2020"}


\CommentTok{\#Table 2.2 "Additional Real Property Special District Tax Levies for Special Purposes, 2019"}


\CommentTok{\#Table 2.3 "Community Development Authorities Requiring a Special Purpose Real Property Levy, 2019" }


\CommentTok{\#Table 2.4 "Special Purpose Real Property Tax Levies on Commercial Property in Northern Virginia Transportation Authority Region, 2019" }


\CommentTok{\#Table 2.5 "Real Property Assessment Procedures for Virginia Localities, 2019" }


\CommentTok{\#Table 2.6 "Assessed Value of Real Property by Category and by Locality, 2018*"}


\CommentTok{\#Table 2.7 "Real Property Effective True Tax Rates, 2017"}


\CommentTok{\#Table 2.8 "Real Property Miscellaneous Items, 2019"}
\end{Highlighting}
\end{Shaded}

\begin{center}\rule{0.5\linewidth}{0.5pt}\end{center}

\(^1\) Charlottesville Housing Affordability Program: \url{https://www}.
charlottesville.org/departments-and-services/departments-a-g/
commissioner-of-revenue/real-estate-tax-relief-for-the-elderlyand-disabled. Loudoun County Affordable Dwelling Unit Program: \url{http://www.loudoun.gov/adu}. Arlington County Housing
Grants Program: \url{http://housing.arlingtonva.us/get-help/rentalservices/local-housing-grants/}.

\(^2\) The statute allows the use of the Internal Revenue Service rate. Section 6621 of the Internal Revenue Code establishes a rate of 3 percent plus the federal short-term rate. In December 2019, when the short-term rate was 1.616 percent, the combined annual rate was 4.61 percent.

\(^3\) City of Alexandria, \emph{Budget Memo \#46: Review of Other Jurisdictions' Experience with a Real Estate Tax Deferral Program for the General Population} (Councilman Speck's Request), 4/25/2003.

\(^4\) Henrico County, \emph{Budget Memo \#46}.

\(^5\) The \emph{Code} does not specify which census is to be used.

\(^6\) Taxable assessed value treats property qualifying for use value
as taxable at its use value rather than at its full market value.

\(^7\) Virginia Department of Taxation, \emph{The 2017 Virginia Assessment/Sales Ratio Study} (Richmond, February 2019), p.~35. The study
can be found at \url{https://tax.virginia.gov/assessment-sales-ratiostudies}.

\(^8\) Virginia Department of Taxation, \emph{The 2017 Virginia Assessment/Sales Ratio Study}, p.~34.

\(^9\) International Association of Assessing Officers, \emph{Standard on Ratio Studies}, (approved April 2013), p.~17. \url{http://www.iaao}.
org/media/standards/Standard\_on\_Ratio\_Studies.pdf.

\(^10\) The Department of Taxation's study applies to all types of property, not just residential property. Nonetheless, the majority of
the measured sales are for single-family residential properties.

\#Section 3
\#\# Real Property Tax Relief Plans and Housing Grants for the Elderly and Disabled, 2019

Sections 58.1-3210 through 58.1-3218 of the \emph{Code of Virginia} provides that localities may adopt an ordinance allowing property tax relief for elderly and disabled persons. The relief may be in the form of either deferral or exemption from taxes. The applicant for tax relief must be either disabled or not less than 65 years of age and must be the owner of the property for which relief is sought (§ 58.1-3210). The property must be the sole dwelling of the applicant. In addition, localities have the option of exempting or deferring the portion of a person's tax that represents the increase in tax liability since the year the taxpayer reached 65 years of age or became disabled.

Localities are allowed to establish by ordinance the net financial worth and annual income limitations pertaining to
owners, relatives and non-relatives living in the dwelling(§ 58.1-3212) of qualified elderly or handicapped persons.
Further, mobile homes that are owned by elderly and disabled persons are included in the allowable property tax
exemptions whether or not mobile homes are permanently affixed. Finally, local governments are authorized to extend
tax relief for the elderly and disabled to dwellings that are jointly owned by individuals, not all of whom are over 65
or totally disabled.

The text table below indicates the range and media nof the combined gross income allowance and combined
net worth limitations for those cities, counties, and towns responding to the survey.

\begin{Shaded}
\begin{Highlighting}[]
\CommentTok{\#table name: Relief Plan Statistics: Gross Income and Net Worth, 2019}
\end{Highlighting}
\end{Shaded}

The following text table indicates, for those localities responding, how many localities have a tax relief plan that
applies to both the elderly and the disabled, the elderly only,or the disabled only.

\begin{Shaded}
\begin{Highlighting}[]
\CommentTok{\# table name: Relief Plans for Elderly and Disabled, 2019}
\end{Highlighting}
\end{Shaded}

A majority of the localities exempt an owner from all or part of the taxes on the dwelling; usually the exemption is based on a sliding scale, with the percentage of the exemption decreasing as the income and/or net worth of the owner increases.

\textbf{Table 3.1} summarizes the various tax relief plans offered to elderly and disabled property owners in Virginia.
The figures under the combined gross income heading reflect, first, the maximum allowable income (including the
income of all relatives living with the owner) for an owner to be eligible for relief and, second, the amount of income
of each relative living in the household, except the spouse, who is exempted from this amount.

For example, if the table reads ``\$7,500; first \$1,500 exempt,'' this indicates that the combined income of the
owner and all relatives living with him/her may not exceed \$7,500, except that the first \$1,500 of income of each relative other than the spouse is excluded when computing this amount. The combined net worth amount listed usually
excludes the value of the dwelling and a given parcel of land upon which the dwelling is situated.

\textbf{Table 3.2} details relief plans for renters. As the table indicates, few localities offer such plans. Only five cities
(Alexandria, Charlottesville, Fairfax, Falls Church, and Hampton) and one county (Fairfax) reported having plans
for renters.

\textbf{Table 3.3} lists the combined elderly and disabled beneficiaries reported by each locality in 2018 or 2019 and
the amount of revenue foregone by each locality because of the homeowner exemptions. The amounts were reported
by 23 cities, 66 counties, and 31 towns that responded to the question. The amounts reported foregone totaled \$21,698,890
for cities, \$60,242,734 for counties and \$636,229 for the reporting towns. The grand total amount foregone by
reporting cities, counties, and towns was \$82,577,853. An estimate of the average revenue foregone per beneficiary
is also provided for localities reporting both number of beneficiaries and foregone revenue. For cities, the average
revenue foregone was \$1,518 per beneficiary. The amount for counties was \$1,581, and for towns it was \$360.

\begin{Shaded}
\begin{Highlighting}[]
\CommentTok{\#Table 3.1 Real Property Owner Tax Relief Plans for the Elderly and Disabled, 2019}

\CommentTok{\#Table 3.2 Real Property Renter Tax Relief Plans for the Elderly and Disabled, 2019}

\CommentTok{\#Table 3.3 Real Property Tax Relief Plans for the Elderly and Disabled Homeowners: Number of Beneficiaries and Foregone Tax Revenue, 2018 or 2019}
\end{Highlighting}
\end{Shaded}


  \bibliography{book.bib}

\end{document}
