% Options for packages loaded elsewhere
\PassOptionsToPackage{unicode}{hyperref}
\PassOptionsToPackage{hyphens}{url}
%
\documentclass[
]{book}
\usepackage{amsmath,amssymb}
\usepackage{lmodern}
\usepackage{ifxetex,ifluatex}
\ifnum 0\ifxetex 1\fi\ifluatex 1\fi=0 % if pdftex
  \usepackage[T1]{fontenc}
  \usepackage[utf8]{inputenc}
  \usepackage{textcomp} % provide euro and other symbols
\else % if luatex or xetex
  \usepackage{unicode-math}
  \defaultfontfeatures{Scale=MatchLowercase}
  \defaultfontfeatures[\rmfamily]{Ligatures=TeX,Scale=1}
\fi
% Use upquote if available, for straight quotes in verbatim environments
\IfFileExists{upquote.sty}{\usepackage{upquote}}{}
\IfFileExists{microtype.sty}{% use microtype if available
  \usepackage[]{microtype}
  \UseMicrotypeSet[protrusion]{basicmath} % disable protrusion for tt fonts
}{}
\makeatletter
\@ifundefined{KOMAClassName}{% if non-KOMA class
  \IfFileExists{parskip.sty}{%
    \usepackage{parskip}
  }{% else
    \setlength{\parindent}{0pt}
    \setlength{\parskip}{6pt plus 2pt minus 1pt}}
}{% if KOMA class
  \KOMAoptions{parskip=half}}
\makeatother
\usepackage{xcolor}
\IfFileExists{xurl.sty}{\usepackage{xurl}}{} % add URL line breaks if available
\IfFileExists{bookmark.sty}{\usepackage{bookmark}}{\usepackage{hyperref}}
\hypersetup{
  pdftitle={Virginia Local Tax Rates, 2019},
  pdfauthor={Stephen C. Kulp, Weldon Cooper Center for Public Service, University of Virginia},
  hidelinks,
  pdfcreator={LaTeX via pandoc}}
\urlstyle{same} % disable monospaced font for URLs
\usepackage{color}
\usepackage{fancyvrb}
\newcommand{\VerbBar}{|}
\newcommand{\VERB}{\Verb[commandchars=\\\{\}]}
\DefineVerbatimEnvironment{Highlighting}{Verbatim}{commandchars=\\\{\}}
% Add ',fontsize=\small' for more characters per line
\usepackage{framed}
\definecolor{shadecolor}{RGB}{248,248,248}
\newenvironment{Shaded}{\begin{snugshade}}{\end{snugshade}}
\newcommand{\AlertTok}[1]{\textcolor[rgb]{0.94,0.16,0.16}{#1}}
\newcommand{\AnnotationTok}[1]{\textcolor[rgb]{0.56,0.35,0.01}{\textbf{\textit{#1}}}}
\newcommand{\AttributeTok}[1]{\textcolor[rgb]{0.77,0.63,0.00}{#1}}
\newcommand{\BaseNTok}[1]{\textcolor[rgb]{0.00,0.00,0.81}{#1}}
\newcommand{\BuiltInTok}[1]{#1}
\newcommand{\CharTok}[1]{\textcolor[rgb]{0.31,0.60,0.02}{#1}}
\newcommand{\CommentTok}[1]{\textcolor[rgb]{0.56,0.35,0.01}{\textit{#1}}}
\newcommand{\CommentVarTok}[1]{\textcolor[rgb]{0.56,0.35,0.01}{\textbf{\textit{#1}}}}
\newcommand{\ConstantTok}[1]{\textcolor[rgb]{0.00,0.00,0.00}{#1}}
\newcommand{\ControlFlowTok}[1]{\textcolor[rgb]{0.13,0.29,0.53}{\textbf{#1}}}
\newcommand{\DataTypeTok}[1]{\textcolor[rgb]{0.13,0.29,0.53}{#1}}
\newcommand{\DecValTok}[1]{\textcolor[rgb]{0.00,0.00,0.81}{#1}}
\newcommand{\DocumentationTok}[1]{\textcolor[rgb]{0.56,0.35,0.01}{\textbf{\textit{#1}}}}
\newcommand{\ErrorTok}[1]{\textcolor[rgb]{0.64,0.00,0.00}{\textbf{#1}}}
\newcommand{\ExtensionTok}[1]{#1}
\newcommand{\FloatTok}[1]{\textcolor[rgb]{0.00,0.00,0.81}{#1}}
\newcommand{\FunctionTok}[1]{\textcolor[rgb]{0.00,0.00,0.00}{#1}}
\newcommand{\ImportTok}[1]{#1}
\newcommand{\InformationTok}[1]{\textcolor[rgb]{0.56,0.35,0.01}{\textbf{\textit{#1}}}}
\newcommand{\KeywordTok}[1]{\textcolor[rgb]{0.13,0.29,0.53}{\textbf{#1}}}
\newcommand{\NormalTok}[1]{#1}
\newcommand{\OperatorTok}[1]{\textcolor[rgb]{0.81,0.36,0.00}{\textbf{#1}}}
\newcommand{\OtherTok}[1]{\textcolor[rgb]{0.56,0.35,0.01}{#1}}
\newcommand{\PreprocessorTok}[1]{\textcolor[rgb]{0.56,0.35,0.01}{\textit{#1}}}
\newcommand{\RegionMarkerTok}[1]{#1}
\newcommand{\SpecialCharTok}[1]{\textcolor[rgb]{0.00,0.00,0.00}{#1}}
\newcommand{\SpecialStringTok}[1]{\textcolor[rgb]{0.31,0.60,0.02}{#1}}
\newcommand{\StringTok}[1]{\textcolor[rgb]{0.31,0.60,0.02}{#1}}
\newcommand{\VariableTok}[1]{\textcolor[rgb]{0.00,0.00,0.00}{#1}}
\newcommand{\VerbatimStringTok}[1]{\textcolor[rgb]{0.31,0.60,0.02}{#1}}
\newcommand{\WarningTok}[1]{\textcolor[rgb]{0.56,0.35,0.01}{\textbf{\textit{#1}}}}
\usepackage{longtable,booktabs,array}
\usepackage{calc} % for calculating minipage widths
% Correct order of tables after \paragraph or \subparagraph
\usepackage{etoolbox}
\makeatletter
\patchcmd\longtable{\par}{\if@noskipsec\mbox{}\fi\par}{}{}
\makeatother
% Allow footnotes in longtable head/foot
\IfFileExists{footnotehyper.sty}{\usepackage{footnotehyper}}{\usepackage{footnote}}
\makesavenoteenv{longtable}
\usepackage{graphicx}
\makeatletter
\def\maxwidth{\ifdim\Gin@nat@width>\linewidth\linewidth\else\Gin@nat@width\fi}
\def\maxheight{\ifdim\Gin@nat@height>\textheight\textheight\else\Gin@nat@height\fi}
\makeatother
% Scale images if necessary, so that they will not overflow the page
% margins by default, and it is still possible to overwrite the defaults
% using explicit options in \includegraphics[width, height, ...]{}
\setkeys{Gin}{width=\maxwidth,height=\maxheight,keepaspectratio}
% Set default figure placement to htbp
\makeatletter
\def\fps@figure{htbp}
\makeatother
\setlength{\emergencystretch}{3em} % prevent overfull lines
\providecommand{\tightlist}{%
  \setlength{\itemsep}{0pt}\setlength{\parskip}{0pt}}
\setcounter{secnumdepth}{5}
\usepackage{booktabs}
\ifluatex
  \usepackage{selnolig}  % disable illegal ligatures
\fi
\usepackage[]{natbib}
\bibliographystyle{apalike}
\newlength{\cslhangindent}
\setlength{\cslhangindent}{1.5em}
\newlength{\csllabelwidth}
\setlength{\csllabelwidth}{3em}
\newenvironment{CSLReferences}[2] % #1 hanging-ident, #2 entry spacing
 {% don't indent paragraphs
  \setlength{\parindent}{0pt}
  % turn on hanging indent if param 1 is 1
  \ifodd #1 \everypar{\setlength{\hangindent}{\cslhangindent}}\ignorespaces\fi
  % set entry spacing
  \ifnum #2 > 0
  \setlength{\parskip}{#2\baselineskip}
  \fi
 }%
 {}
\usepackage{calc}
\newcommand{\CSLBlock}[1]{#1\hfill\break}
\newcommand{\CSLLeftMargin}[1]{\parbox[t]{\csllabelwidth}{#1}}
\newcommand{\CSLRightInline}[1]{\parbox[t]{\linewidth - \csllabelwidth}{#1}\break}
\newcommand{\CSLIndent}[1]{\hspace{\cslhangindent}#1}

\title{Virginia Local Tax Rates, 2019}
\usepackage{etoolbox}
\makeatletter
\providecommand{\subtitle}[1]{% add subtitle to \maketitle
  \apptocmd{\@title}{\par {\large #1 \par}}{}{}
}
\makeatother
\subtitle{38th Annual Edition: Information for All Cities and Counties and Selected Incorporated Towns}
\author{Stephen C. Kulp, Weldon Cooper Center for Public Service, University of Virginia}
\date{2021}

\begin{document}
\maketitle

{
\setcounter{tocdepth}{1}
\tableofcontents
}
\hypertarget{introduction}{%
\chapter*{Introduction}\label{introduction}}
\addcontentsline{toc}{chapter}{Introduction}

\hypertarget{foreward}{%
\section*{Foreward}\label{foreward}}
\addcontentsline{toc}{section}{Foreward}

This is the thirty-eighth edition of the Cooper Center's annual publication about the tax rates of Virginia's local governments. In addition to information about tax rates, the publication contains details about tax administration, valuation methods, and due dates. There is also information on water and sewer rates, waste disposal charges and numerous other aspects of local government finance. This comprehensive guide to local taxes is based on information gathered in the spring, summer, and early fall of 2019. The study includes all of Virginia's 38 independent cities and 95 counties and 118 of the 190 incorporated towns. The included towns account for 92 percent of the Commonwealth's population in towns.\footnote{Locality population figures are based on estimates developed by the \href{https://demographics.coopercenter.org}{Demographics Research Group of the Weldon Cooper Center for Public Service}. See Appendix D.} The study also contains information from several outside sources, including two Department of Taxation studies, 2019 Legislative Summary and The 2017 Assessment/Sales Ratio Study, as well as Department of Taxation information on the assessed value of real estate by type of property. We also used the Auditor of Public Accounts' Comparative Report of Local Government Revenues and Expenditures, Year Ended June 30, 2018, the Commission on Local Governments' Report on Proffered Cash Payments and Expenditures by Virginia's Counties, Cities and Towns, 2017-2018, and the Department of Housing and Community Development's Virginia Enterprise Zone Program 2018 Grant Year Annual Report.

\hypertarget{organization-of-the-book}{%
\section*{Organization of the Book}\label{organization-of-the-book}}
\addcontentsline{toc}{section}{Organization of the Book}

The study is divided into 26 sections. Section 1 is a reprint of the ``Local Tax Legislation'' section of the Department of Taxation's 2019 Legislative Summary. The original Department of Taxation report is available at \href{https://tax.virginia.gov/legislative-summary-reports}{its website}. Sections 2 through 26 cover specific taxes, fees, service charges, cash proffers, enterprise zones, and financial documents on the web. Most of the data came from a detailed web-based questionnaire sent to all cities, counties, and incorporated towns (see Appendix A for a printed version). Appendix B provides a listing of names, phone numbers, and email addresses, when available, of respondents and non-respondents to the questionnaire. Appendix C shows the percentage share of total local taxes represented by each specific tax for each locality based on data from the Auditor of Public Accounts for fiscal year 2018. Information is provided for each city and county and for 38 populous incorporated towns. Finally, Appendix D contains 2018 population estimates for cities, counties and towns from the Cooper Center's Demographics Research Group. The population information is provided to give readers some perspective on the relative size of localities.

Please note that the web addresses provided in this publication were good at the time this text was printed. However, some links are unstable and may not work with certain browsers or they may be modified or withdrawn subsequent to publication.

\hypertarget{about-the-survey}{%
\section*{About the Survey}\label{about-the-survey}}
\addcontentsline{toc}{section}{About the Survey}

In 2019, localities could choose between online or printed versions of the questionnaire. The Cooper Center has made its best efforts to accurately reflect in this report the responses of localities based on the survey or follow-up queries.

In the tables three dots (\ldots) are used to show there was no response and ``N/A'' is used to indicate ``not applicable.'' Readers may use the telephone/email list in Appendix B to contact local officials in order to obtain clarification and additional detail.

\hypertarget{some-components-of-local-taxes}{%
\section*{Some Components of Local Taxes}\label{some-components-of-local-taxes}}
\addcontentsline{toc}{section}{Some Components of Local Taxes}

This book deals mainly with local sources of revenue for local governments. Though localities might also receive federal and state resources, an important part of local funding comes from local sources. The Auditor of Public Accounts, Comparative Report of Local Government Revenues and Expenditures provides data on these local sources. The following analysis uses the data from their report for the year ended June 30, 2018.

A common misperception is that nearly all local tax revenue comes from the real property tax. True, the real property tax is the dominant source, accounting for 61.9 percent of city-county tax revenue in FY 2018, the latest year available (see text table below). But three other taxes-----the personal property tax, the local option sales and use tax, and the business license tax-----together accounted for 24.5 percent of total tax revenue. The remaining 14.6 percent of tax revenue came from more than a dozen other taxes.

\begin{table}

\caption{\label{tab:unnamed-chunk-2}Sources of Virginia Local Government Tax Revenue, FY 2018}
\centering
\begin{tabular}[t]{l|r|r}
\hline
Tax & Amount (\$) & \% of Total\\
\hline
Total taxes & \$17,967,385,766 & 100.00\\
\hline
Real property & \$10,946,877,675 & 60.93\\
\hline
Personal property & \$2,370,758,768 & 13.19\\
\hline
Local option sales and use & \$1,239,855,163 & 6.90\\
\hline
Business license & \$771,958,263 & 4.30\\
\hline
Restaurant meals & \$612,940,580 & 3.41\\
\hline
Public service corporation property & \$412,121,081 & 2.29\\
\hline
Consumer utility & \$327,627,947 & 1.82\\
\hline
Hotel and motel room & \$244,412,96 & 1.36\\
\hline
Machinery and tools & \$233,076,157 & 1.30\\
\hline
Motor vehicle license & \$197,705,384 & 1.10\\
\hline
Recordation and will & \$126,458,487 & 0.70\\
\hline
Bank stock & \$117,199,137 & 0.65\\
\hline
Other local taxes & \$92,124,397 & 0.51\\
\hline
Tobacco & \$65,150,996 & 0.36\\
\hline
Coal, oil, and gas & \$28,510,002 & 0.16\\
\hline
Admission & \$21,815,169 & 0.12\\
\hline
Franchise license & \$16,362,103 & 0.09\\
\hline
Merchants' Capital & \$14,301,188 & 0.08\\
\hline
Penalties and interest & \$128,130,305 & 0.71\\
\hline
\end{tabular}
\end{table}

There are six localities where the real property tax is not dominant. Bath and Surry counties have large power plants that pay public service corporation property taxes that overwhelm other sources. Buchanan County has rich mineral deposits subject to local severance taxes that exceed the real property tax. Covington City and Alleghany County receive large shares of revenue from machinery and tools taxes on MeadWestvaco's large paperboard manufacturing facility. Finally, the small city of Norton, the \href{https://demographics.coopercenter.org/population-estimates-age-sex-race-hispanic-towns/}{least populous independent city in Virginia} (3,908 in 2018) received almost as much money from the local option sales and use tax as from the real property tax. In the remaining 127 cities and counties where the real property tax is dominant, its relative importance varies from 30.3 percent of total tax revenue in Galax City to 78.8 percent in Lancaster County (see Appendix C).

Thirty-six cities (two cities--Hopewell and Petersburg--did not provide information for the 2018 Comparative Report) and 95 counties imposed four of the taxes shown in the previous table---the real property tax, the personal property tax, the local option sales and use tax, and the public service corporation property tax. Most, but not all, localities imposed recordation and will taxes, consumer utility taxes, motor vehicle license taxes, and taxes on manufacturers' machinery and tools. Nonetheless, as shown in the next text table, there are a number of taxes, a few of them signifi cant sources of revenue, which are not levied by all localities. Also, some of the taxes are used so sparingly that their revenue yield is very low.

\begin{table}

\caption{\label{tab:unnamed-chunk-3}Number of Virginia Localities Imposing Taxes by Type, FY 2018}
\centering
\begin{tabular}[t]{l|r|r|r}
\hline
Tax & Cities & Counties & Total\\
\hline
Real property & 36 & 95 & 131\\
\hline
Personal property & 36 & 95 & 131\\
\hline
Local option sales and use & 36 & 95 & 131\\
\hline
Public service corporation property & 36 & 95 & 131\\
\hline
Consumer utility & 36 & 92 & 128\\
\hline
Recordation and wills & 32 & 93 & 125\\
\hline
Motor vehicle license & 32 & 86 & 118\\
\hline
Machinery and tools property & 31 & 85 & 116\\
\hline
Bank stock & 36 & 64 & 100\\
\hline
Hotel and motel room & 32 & 67 & 99\\
\hline
Business license & 36 & 52 & 88\\
\hline
Restaurant meals & 36 & 49 & 85\\
\hline
Franchise license & 11 & 37 & 48\\
\hline
Merchants’ capital & 1 & 43 & 44\\
\hline
Tobacco & 29 & 2 & 31\\
\hline
Admission & 18 & 3 & 21\\
\hline
Coal, oil, and gas & 1 & 6 & 7\\
\hline
Other local taxes & 23 & 49 & 72\\
\hline
\end{tabular}
\end{table}

There are three major reasons why local governments do not to impose some taxes: (1) The locality lacks a tax base for a particular tax (e.g., a locality must have a bank in order to apply a bank stock tax and a locality must have taxable mineral deposits to impose coal, oil, and gas taxes). (2) The locality is faced with state restrictions (e.g., county excise taxes on hotel and motel room rental have tax rate restrictions imposed by the state; county restaurant meals taxes must be approved in a voter referendum; tobacco taxes are permitted in only two counties; and county admissions taxes are subject to many restrictions). In regard to the busi-ness, professional, and occupational license tax (BPOL tax), counties must choose either the BPOL tax or the merchants' capital tax. Counties are not permitted to impose a business license tax within the boundaries of an incorporated town situated within the county without permission of the town. This means that counties with large shares of business activity within towns are motivated to impose a merchants' capital tax that can be applied countywide. (3) The locality chooses not to impose a permitted tax (e.g., Richmond City, a community with a large cigarette manufacturing plant, has not adopted a consumer tobacco tax even though all cities are granted the authority to levy such a tax).

\hypertarget{partnership-with-lexisnexis}{%
\section*{Partnership with LexisNexis}\label{partnership-with-lexisnexis}}
\addcontentsline{toc}{section}{Partnership with LexisNexis}

The Weldon Cooper Center for Public Service is partner-ing with the publisher LexisNexis to produce the annual Tax Rates books. The Cooper Center still prepares and distributes the survey and writes up the results. LexisNexis publishes the book and fulfi lls orders from interested parties. This arrangement allows us to concentrate on providing the most accurate and up-to-date information about Virginia tax rates and leverages LexisNexis' considerable expertise in production and distribution of the annual volume. We hope the arrangement will lead to continued improvements in our Virginia Local Tax Rates series.

\hypertarget{study-personnel}{%
\section*{Study Personnel}\label{study-personnel}}
\addcontentsline{toc}{section}{Study Personnel}

Stephen C. Kulp, Research Specialist at the Center for Economic and Policy Studies, was responsible for work on the project. He refined the new database, administered the survey, translated the results into tables, checked relevant code sections, assisted with the development of the web-based questionnaire, and made appropriate changes in the text. Jennifer Nelson, of the Cooper Center's Publications Section, designed the cover. Cooper Center employee Albert W. Spengler, who authored this study for a number of years prior to 1991, laid the foundation for the study when it was his responsibility.

The strong support for this publication by the Virginia Association of Counties and the Virginia Municipal League helps ensure our continued efforts to provide this resource as a basic reference on Virginia local taxes.

\hypertarget{final-comments}{%
\section*{Final Comments}\label{final-comments}}
\addcontentsline{toc}{section}{Final Comments}

The Cooper Center is grateful to the many local officials throughout the Commonwealth who supplied the survey information presented in this study. Their willingness to provide information and their patience in answering follow-up questions is what makes this book successful. The high response rates could not have been achieved without their cooperation.

Stephen C. Kulp

Research Specialist

Center for Economic and Policy Studies

Weldon Cooper Center for Public Service

University of Virginia

Charlottesville

February 2020

\hypertarget{summary-of-legislative-changes-in-local-taxation-in-2019}{%
\chapter{Summary of Legislative Changes in Local Taxation in 2019}\label{summary-of-legislative-changes-in-local-taxation-in-2019}}

\hypertarget{general-provisions}{%
\section{General Provisions}\label{general-provisions}}

\hypertarget{local-license-tax-on-mobile-food-units}{%
\subsection{Local License Tax on Mobile Food Units}\label{local-license-tax-on-mobile-food-units}}

Senate Bill 1425 (Chapter 791) provides that when the owner of a new business that operates a mobile food unit has paid a license tax as required by the locality in which the mobile food unit is registered, the owner is not required to pay a license tax to any other locality for conducting business from such mobile food unit in such a locality.\footnote{Excerpted from the local tax legislation section of the Department of Taxation's 2019 Legislative Summary. Minor changes were made in format and punctuation. See \url{https://tax.virginia.gov/legislative-sum-mary-reports}}

This exemption from paying the license tax in other localities expires two years after the payment of the initial license tax in the locality in which the mobile food unit is registered. During the two year exemption period, the owner is entitled to exempt up to three mobile food units from license taxation in other localities. However, the owner of the mobile food unit is required to register with the Commissioner of the Revenue or Director of Finance in any locality in which he conducts business from such mobile food unit, regardless of whether the owner is exempt from paying license tax in the locality.

This Act defines ``mobile food unit'' as a restaurant that is mounted on wheels and readily moveable from place to place at all times during operation. It also defines ``new business'' as a business that locates for the first time to do business in a locality. A business will not be deemed a new business based on a merger, acquisition, similar business combination, name change, or a change to its business form.

Without the exemption provided in this Act, localities are authorized to impose business, professional and occupational license (BPOL) taxes upon local businesses. Generally, the BPOL tax is levied on the privilege of engaging in business at a definite place of business within a Virginia locality. Businesses that are mobile, however, can be subject to license taxes or fees in multiple localities in certain situations.

Effective: July 1, 2019

Added: § 58.1-3715.1

\hypertarget{local-gas-road-improvement-tax-extension-of-sunset-provision}{%
\subsection{Local Gas Road Improvement Tax; Extension of Sunset Provision}\label{local-gas-road-improvement-tax-extension-of-sunset-provision}}

House Bill 2555 (Chapter 24) and Senate Bill 1165 (Chapter 191) extend the sunset date for the local gas road improve-ment tax from January 1, 2020 to January 1, 2022. The authority to impose the local gas road improvement tax was previously scheduled to sunset on January 1, 2020.

The localities that comprise the Virginia Coalfield Economic Development Authority may impose a local gas road improvement tax that is capped at a rate of one percent of the gross receipts from the sale of gases severed within the locality. Under current law, the revenues generated from this tax are allocated as follows: 75\% are paid into a special fund in each locality called the Coal and Gas Road Improvement Fund, where at least 50\% are spent on road improvements and 25\% may be spent on new water and sewer systems or the construction, repair, or enhancement of natural gas systems and lines within the locality; and the remaining 25\% of the revenue is paid to the Virginia Coal-fi eld Economic Development Fund. The Virginia Coalfi eld Economic Development Authority is comprised of the City of Norton, and the Counties of Buchanan, Dickenson, Lee, Russell, Scott, Tazewell, and Wise.

Effective: July 1, 2019

Amended: § 58.1-3713

\hypertarget{private-collectors-authorized-for-use-by-localities-to-collect-delinquent-debts}{%
\subsection{Private Collectors Authorized for Use by Localities to Collect Delinquent Debts}\label{private-collectors-authorized-for-use-by-localities-to-collect-delinquent-debts}}

Senate Bill 1301 (Chapter 271) allows a local treasurer to employ private collection agents to assist with the collection of delinquent amounts due other than delinquent local taxes that have been delinquent for a period of three months or more and for which the appropriate statute of limitations has not run.

Effective: July 1, 2019

Amended: § 58.1-3919.1

\hypertarget{real-property-tax}{%
\section{Real Property Tax}\label{real-property-tax}}

\hypertarget{real-property-tax-exemptions-for-elderly-and-disabled-computation-of-income-limitation}{%
\subsection{Real Property Tax Exemptions for Elderly and Disabled: Computation of Income Limitation}\label{real-property-tax-exemptions-for-elderly-and-disabled-computation-of-income-limitation}}

House Bill 1937 (Chapter 16) provides that, if a locality has established a real estate tax exemption for the elderly and handicapped and enacted an income limitation related to the exemption, it may exclude, for purposes of calculating the income limitation, any disability income received by a family member or nonrelative who lives in the dwelling and who is permanently and totally disabled.

Under current law, if a locality's tax relief ordinance establishes an annual income limitation, the computation of annual income is calculated by adding together the income received during the preceding calendar year of the owners of the dwelling who use it as their principal residence; and the owners' relatives who live in the dwelling, except for those relatives living in the dwelling and providing bona fide caregiving services to the owner whether such relatives are compensated or not; and at the option of each locality, nonrelatives of the owner who live in the dwelling except for bona fide tenants or bona fide caregivers of the owner, whether compensated or not.

Effective: July 1, 2019

Amended: § 58.1-3212

\hypertarget{real-property-tax-exemption-for-elderly-and-disabled-improvements-to-a-dwelling}{%
\subsection{Real Property Tax Exemption for Elderly and Disabled: Improvements to a Dwelling}\label{real-property-tax-exemption-for-elderly-and-disabled-improvements-to-a-dwelling}}

House Bill 2150 (Chapter 736) and Senate Bill 1196 (Chapter 737) clarify the definition of ``dwelling,'' for purposes of the real property tax exemption for owners who are 65 years of age or older or permanently and totally disabled, to include certain improvements to the exempt land and the land on which the improvements are situated. These Acts define the term ``dwelling'' to include an improvement to the land that is not used for a business purpose but is used to house certain motor vehicles or household goods.

Under current law, in order to be granted real property tax relief, qualifying property must be owned by and occupied as the sole dwelling of a person who is at least 65 years of age, or, if the local ordinance provides, any person with a permanent disability. Dwellings jointly held by spouses, with no other joint owners, qualify if either spouse is 65 or over or permanently and totally disabled.

Effective: July 1, 2019

Amended: § 58.1-3210

\hypertarget{real-property-tax-partial-exemption-from-real-property-taxes-for-flood-mitigation-efforts}{%
\subsection{Real Property Tax: Partial Exemption from Real Property Taxes for Flood Mitigation Efforts}\label{real-property-tax-partial-exemption-from-real-property-taxes-for-flood-mitigation-efforts}}

Senate Bill 1588 (Chapter 754) enables a locality to provide by ordinance a partial exemption from real property taxes for flooding abatement, mitigation, or resiliency efforts for improved real estate that is subject to recurrent flooding, as authorized by an amendment to Article X, Section 6 of the Constitution of Virginia that was adopted by the voters on November 6, 2018.

This act provides that exemptions may only be granted for qualifying flood improvements that do not increase the size of any impervious area and are made to qualifying structures or to land. ``Qualifying structures'' are defined as structures that were completed prior to July 1, 2018 or were completed more than 10 years prior to the completion of the improvements. For improvements made to land, the improvements must be made primarily for the benefit of one or more qualifying structures. No exemption will be authorized for any improvements made prior to July 1, 2018.

A locality is granted the authority to (i) establish flood protection standards that qualifying flood improvements must meet in order to be eligible for the exemption; (ii) determine the amount of the exemption; (iii) set income or property value limitations on eligibility; (iv) provide that the exemption shall only last for a certain number of years; (v) determine, based upon flood risk, areas of the locality where the exemption may be claimed; and (vi) establish preferred actions for qualifying for the exemption, including living shorelines.

Effective: July 1, 2019

Amended: § 58.1-3228.1

\hypertarget{real-property-tax-exemption-for-certain-surviving-spouses}{%
\subsection{Real Property Tax: Exemption for Certain Surviving Spouses}\label{real-property-tax-exemption-for-certain-surviving-spouses}}

House Bill 1655 (Chapter 15) and Senate Bill 1270 (Chapter 801) allow surviving spouses of disabled veterans to continue to qualify for a real property tax exemption regardless of whether the surviving spouse moves to a different residence, as authorized by an amendment to subdivision (a) of Section 6-A of Article X of the Constitution of Virginia that was adopted by the voters on November 6, 2018. If a surviving spouse was eligible for the exemption but lost such eligibility due to a change in residence, then the surviving spouse is eligible for the exemption again, beginning January 1, 2019.

These Acts also clarify that the real property tax exemptions for spouses of service members killed in action and spouses of certain emergency service providers killed in the line of duty continue to apply regardless of the spouse's moving to a new principal residence.

Effective: Taxable years beginning on or after January 1, 2019

Amended: §§ 58.1-3219.5, 3219.9, and 3219.14

\hypertarget{land-preservation-special-assessment-optional-limit-on-annual-increase-in-assessed-value}{%
\subsection{Land Preservation; Special Assessment, Optional Limit on Annual Increase in Assessed Value}\label{land-preservation-special-assessment-optional-limit-on-annual-increase-in-assessed-value}}

House Bill 2365 (Chapter 22) authorizes localities that employ use value assessments for certain classes of real property to provide by ordinance that the annual increase in the assessed value of eligible property may not exceed a specified dollar amount per acre.

Effective: July 1, 2019

Amended: § 58.1-3231

\hypertarget{virginia-regional-industrial-act-revenue-sharing-composite-index}{%
\subsection{Virginia Regional Industrial Act: Revenue Sharing; Composite Index}\label{virginia-regional-industrial-act-revenue-sharing-composite-index}}

House Bill 1838 (Chapter 534) requires that the Department of Taxation's calculation of the true values of real estate and public service company property component of the Commonwealth's educational composite index of local ability-to-pay take into account arrangements by localities entered into pursuant to the Virginia Regional Industrial Facilities Act, whereby a portion of tax revenue is initially paid to one locality and redistributed to another locality. This Act requires such calculation to properly apportion the percentage of tax revenue ultimately received by each locality.

Effective: July 1, 2021

Amended: § 58.1-6407

\hypertarget{real-estate-with-delinquent-taxes-or-liens-appointment-of-special-commissioner-increase-required-value}{%
\subsection{Real Estate with Delinquent Taxes or Liens: Appointment of Special Commissioner; Increase Required Value}\label{real-estate-with-delinquent-taxes-or-liens-appointment-of-special-commissioner-increase-required-value}}

House Bill 2060 (Chapter 541) increases the assessed value of a parcel of land that could be subject to appointment of a special commissioner to convey the real estate to a locality as a result of unpaid real property taxes or liens from \$50,000or less to \$75,000 or less in most localities. In the Cities of Norfolk, Richmond, Hopewell, Newport News, Petersburg, Fredericksburg, and Hampton, this Act increases the threshold from \$100,000 or less to \$150,000 or less.

Effective: July 1, 2019

Amended: § 58.1-3970.1

\hypertarget{real-estate-with-delinquent-taxes-or-liens-appointment-of-special-commissioner-in-the-city-of-martinsville}{%
\subsection{Real Estate with Delinquent Taxes or Liens; Appointment of Special Commissioner in the City of Martinsville}\label{real-estate-with-delinquent-taxes-or-liens-appointment-of-special-commissioner-in-the-city-of-martinsville}}

House Bill 2405 (Chapter 159) adds the city of Martinsville to the list of cities (Norfolk, Richmond, Hopewell, Newport News, Petersburg, Fredericksburg, and Hampton) that are authorized to have a special commissioner convey tax-delinquent real estate to the locality in lieu of a public sale at auction when the tax-delinquent property has an assessed value of \$100,000 or less. House Bill 2060 raises the threshold in all of these cities from \$100,000 or less to \$150,000 or less.

Effective: July 1, 2019

Amended: § 58.1-3970.1

\hypertarget{personal-property-tax}{%
\section{Personal Property Tax}\label{personal-property-tax}}

\hypertarget{constitutional-amendment-personal-property-tax-exemption-for-motor-vehicle-of-a-disabled-veteran}{%
\subsection{Constitutional Amendment: Personal Property Tax Exemption for Motor Vehicle of a Disabled Veteran}\label{constitutional-amendment-personal-property-tax-exemption-for-motor-vehicle-of-a-disabled-veteran}}

House Joint Resolution 676 (Chapter 822) is a fi rst resolu-tion proposing a constitutional amendment that permits the General Assembly to authorize the governing body of any county, city, or town to exempt from taxation one motor vehicle of a veteran who has a 100 percent service-connected, permanent, and total disability. The amendment provides that only automobiles and pickup trucks qualify for the exemption.

Additionally, the exemption will only be applicable on the date the motor vehicle is acquired or the effective date of the amendment, whichever is later, but will not be applicable for any period of time prior to the effective date of the amendment.

Effective: July 1, 2019

\hypertarget{personal-property-tax-exemption-for-agricultural-vehicles}{%
\subsection{Personal Property Tax Exemption for Agricultural Vehicles}\label{personal-property-tax-exemption-for-agricultural-vehicles}}

House Bill 2733 (Chapter 259) expands the definition of agricultural use motor vehicles for personal property taxation purposes. It changes the criteria from motor vehicles used ``exclusively'' for agricultural purposes to motor vehicles used ``primarily'' for agricultural purposes, and for which the owner is not required to obtain a registration certificate, license plate, and decal or pay a registration fee.

It also expands the definition of trucks or tractor trucks that are used by farmers in their farming operations for the transportation of farm animals or other farm products or for the transport of farm-related machinery. The criteria is changed from vehicles used ``exclusively'' by farmers in their farming operations to vehicles used ``primarily'' by farmers in their farming operations.

Further, this Act expands the classifi cation of farm machinery and equipment that a local governing body may exempt, to include equipment and machinery used by a nursery for the production of horticultural products, and any farm tractor, regardless of whether such farm tractor is used exclusively for agricultural purposes.

Local governing bodies have the option to exempt these classifi cations, in whole or in part, from taxation or to provide for a different rate of taxation thereon.

Effective: July 1, 2019

Amended: § 58.1-3505

\hypertarget{intangible-personal-property-tax-classifi-cation-of-certain-business-property}{%
\subsection{Intangible Personal Property Tax: Classifi cation of Certain Business Property}\label{intangible-personal-property-tax-classifi-cation-of-certain-business-property}}

House Bill 2440 (Chapter 255) classifies as intangible per-sonal property, tangible personal property: i) that is used in a trade or business; ii) with an original cost of less than \$25; and iii) that is not classified as machinery and tools, merchants' capital, or short-term rental property. It also exempts such property from taxation.

Intangible personal property is a separate class of prop-erty segregated for taxation by the Commonwealth. The Commonwealth does not currently tax intangible personal property. Localities are prohibited from taxing intangible personal property.

Certain personal property, while tangible in fact, has previously been designated as intangible and thus exempted from state and local taxation. For example, tangible personal property used in manufacturing, mining, water well drill-ing, radio or television broadcasting, dairy, dry cleaning, or laundry businesses has been designated as exempt intangible personal property.

Effective: July 1, 2019

Amended: §§ 58.1-1101 and 58.1-1103

\hypertarget{real-property-tax-in-2019}{%
\chapter{Real Property Tax in 2019}\label{real-property-tax-in-2019}}

The real property tax is by far the most important source of tax revenue for localities. In fiscal year 2018, the most recent year available from the Auditor of Public Accounts, it accounted for 55.5 percent of tax revenue for cities, 64.6 percent for counties, and 29.1 percent for large towns. These are averages; the relative importance of taxes in individual cities, counties, and towns varies significantly. For information on individual localities, see Appendix C.

The \emph{Code of Virginia}, §§ 58.1-3200 through 58.1-3389, authorizes localities to levy taxes on real property (land, including the buildings and improvements on it). There is no restriction on the tax rate that may be imposed. Section 58.1-3201 provides that all general reassessments or annual assessments shall be at 100 percent of fair market value.

\hypertarget{public-service-corporations}{%
\section{PUBLIC SERVICE CORPORATIONS}\label{public-service-corporations}}

Property owned by so-called public service corporations is not assessed by localities. Instead, that task is delegated to the State Corporation Commission (SCC) and the Department of Taxation.The State Corporation Commission assesses electric utilities and cooperatives, gas pipeline distribution companies, public service water companies, and telephone and telegraph companies. The Department of Taxation assesses pipeline transmission companies and railroads.

In fiscal year 2018, the property tax on public service corporations accounted for 1.7 percent of tax revenue for cities, 2.6 percent for counties, and 0.8 percent for large towns. These are averages; the relative importance of the tax in individual cities, counties, and towns varies significantly. In two counties with large power generating facilities the property tax on public service corporations accounts for a very large share of local tax revenue. In Bath County the share was 47.6 percent and in Surry County the share was 61.1 percent. For more information on individual localities, see Appendix C.

The commissioner of the revenue or another designated official in each city or county is required to provide by January 1 of each year to any public service company with property in its area a copy of the property boundaries of the locality in which any part of the company is located (§ 58.1-2601). The State Corporation Commission or the Department of Taxation send out their assessments for the property based on these boundaries (§ 58.1-2602). Localities examine the assessments to determine their correctness. If correct, the locality determines the equalized assessed valuation of the corporate property by applying the local assessment ratio prevailing in the locality for other real estate (§ 58.1-2604). Local taxes are then assigned to real and tangible personal property at the real property tax rate current in the locality (§ 58.1-2606).

\hypertarget{tax-relief-programs}{%
\section{TAX RELIEF PROGRAMS}\label{tax-relief-programs}}

There are several types of locally financed tax relief programs available. Section 3 of this study contains information on so-called circuit breaker plans for the elderly and disabled. Section 4 covers land use assessments for agricultural, horticultural, forestal, and open space real estate. Section 5 contains information on preferential assessments for agricultural and forestal districts. Finally, Section 6 covers property tax exemptions for certain rehabilitated real estate and other exemptions.

Only the city of Charlottesville, Loudoun County, and Arlington County reported providing tax relief for low-income owners and renters who are not elderly. The city of Charlottesville administers the Charlottesville Housing Affordability Program (CHAP) to help low and middle income homeowners. The program awards grants up to \$1,000 to homeowners with houses assessed at less than \$375,000 and having an annual income less than \$55,000.\footnote{Commission on Local Government, Report on Proffered Cash Payments and Expenditures by Virginia's Counties, Cities and Towns, 2017-2018. \url{https://www.dhcd.virginia.gov/cash-proffers}.} Loudoun County administers the Affordable Dwelling Unit Program for renters and first-time buyers. Buyers need an income greater than 30 percent but less than 70 percent of the area median income to participate. Qualified renters are eligible to rent apartments at rates from \$630 to \$1,300. Arlington County's Housing Grants Program is available to working families with at least one child under age 18. Personal assets may not exceed \$35,000 and there is an income limit based on household size.

Localities are permitted to institute deferral for a portion of the real estate tax by § 58.1-3219 of the \emph{Code of Virginia}. Localities are permitted to grant deferrals from the full amount by which each taxpayer's real estate tax levy exceeds 105 percent of the previous year's tax, or such higher percentage adopted by the locality. Deferred taxes are subject to interest in an amount established by the governing body, not to exceed the rate published by the IRS code.\footnote{\url{http://www.tax.virginia.gov/content/local-tax-rates}} The deferral potentially applies to every property owner, not just the elderly and disabled. (For deferrals limited to the elderly and disabled see Section 3 of this study.)

The deferral program is rarely used. Administrative problems appear to be the major reason for the unpopularity of deferral programs. Loudoun County had a deferral program in place in the 1990s but terminated it ``\ldots{} because the program was administratively complex, cumbersome and required staff time in disproportion to the benefit received by the taxpayer.''\footnote{See ``What Will Become of the Car Tax?'' by John L. Knapp in \emph{Virginia Issues and Answers}. (Winter 2006), Vol. 13, No.~1, pp.~27-31.
  \url{http://www.via.vt.edu/winter06/index.html}} The cities of Alexandria, Falls Church, and Fairfax and the counties of Fairfax and Henrico considered deferral but did not adopt it. According to Henrico staff, ``The administrative procedures for tracing the properties and recovering the relevant taxes upon either the death of the owner or transfer of the property itself would be both cumbersome and time consuming and could not be accomplished with existing staffing levels or existing computer systems.''\footnote{Virginia Cooperative Extension, ``A Citizens' Guide to The Use Value Taxation Program in Virginia.'' \url{https://pubs.ext.vt.edu/448/448-037/448-037.html}} Another reason for the unpopularity of the programs may be that taxpayers only receive postponement, not removal, of the tax liability. The cities of Charlottesville and Richmond, the county of Middlesex, and the town of Amherst were the only localities reporting a deferral program in 2019.

\hypertarget{statutory-rates-special-taxes-due-dates-proration-and-billing-practices}{%
\section{STATUTORY RATES, SPECIAL TAXES, DUE DATES, PRORATION, AND BILLING PRACTICES}\label{statutory-rates-special-taxes-due-dates-proration-and-billing-practices}}

\textbf{Table 2.1} provides general information associated with real property taxes in Virginia's localities. The table provides an estimate by locality of both the number of total taxable real estate parcels and the number of residential parcels. Twenty-seven cities, 80 counties and 52 towns provided estimates of one or both types of parcels. The total number of parcels in cities ranged from a high of 158,431 (Virginia Beach) down to 2,456 (Lexington). Among counties, the number of parcels ranged from a high of 354,687 (Fairfax) down to 3,940 (Highland).

Table 2.1 also lists the statutory (nominal) tax rates. The statutory rate is the rate used by localities and is applied to the assessed value of a property. In the table, statutory rates are listed under calendar year (CY) or fiscal year (FY) columns depending on the locality's assessment cycle. In most cases the calendar year tax rate listed runs from January 1 to December 31 and the fiscal year rate runs from July 1 to June 30. The provisions explaining the assessment cycle requirements are found in § 58.1-3010 and § 58.1-3011 of the \emph{Code of Virginia}. However, some localities report a calendar year assessment schedule with a fiscal year valuation. Six cities (Chesapeake, Harrisonburg, Martinsville, Roanoke, Salem, and Suffolk) and one county (James City) report this practice. Otherwise, 15 cities and 88 counties reported using the calendar year cycle while 176 cities and 6 counties used fiscal year assessment cycles.

The statutory tax rates were reported to the Cooper Center by all cities and counties and 112 of the responding towns. The text table below lists the averages for the statutory rates from the localities.

\begin{Shaded}
\begin{Highlighting}[]
\CommentTok{\#table name: Statutory Real Estate Tax Rates per $100 of Assessed Taxable Value for Localities Reporting, CY 2019 and FY 2020}
\end{Highlighting}
\end{Shaded}

Statutory rates are generally higher for cities than counties. The rates are lowest in towns because they are subordinate to counties and have limited responsibilities.

Tax due dates vary among localities. Generally, if taxes are paid annually, they are due by December 5. If paid semiannually, they are due by June 5 and December 5. However, some localities have different due dates, as provided by § 58.1-3916 of the \emph{Code}.

Most cities have semiannual tax due dates with payments required in June and December. Of the 38 cities, 2 required taxes due annually, 31 semiannually, and 5 quarterly. Among the counties, 32 had annual tax due dates, while 63 had semiannual requirements. Of the towns responding to this question, 80 reported annual due dates, and 32 required semiannual payments.

A locality is permitted to prorate the taxable amount. Any county, city, or town electing to prorate new buildings which are substantially complete prior to November 1 must do so at the time the building is complete or fit to live in. Of the 38 cities, 33 reported prorating taxes while 5 reported not doing so. Among counties, 67 prorated their taxes while 28 did not. Reports from the towns that answered this question indicated that 47 prorated their taxes while 652 did not.

The final column of Table 2.1 pertains to town billing practices. Three possibilities exist: (1) a town sends out its own bills and collects its taxes (TT in the table), (2) a town collects its taxes but the county sends the bills (CT in the table), or (3) a town has the county bill and collect the taxes (CC in the table). Of the towns that answered the question, the overwhelming majority, 100, reported billing and collecting their own taxes. Four said they collected taxes, while in three the county both billed and collected town taxes.

\textbf{Table 2.2, Table 2.3,} and \textbf{Table 2.4} provide additional information concerning statutory real property tax rates. The \emph{Code} allows localities to add special purpose levies on top of the real property rate for various purposes. Table 2.2 deals with the category of special districts. A special district is organized to perform a single governmental function or a restricted number of related functions. Special districts usually have the power to incur debt and levy taxes to fund special activities such as capital improvements, emergency services, sewer and water services, or pest control within those districts. Thirteen cities, 14 counties, and 4 towns reported levying these taxes. The table includes the base (statutory) rate for the locality, the district in which the activity takes place, the purpose of the activity, and the special rate imposed for that activity. Most special activity taxes are in addition to the base rate, though some are simply a flat fee, and others are a percentage rate based on improvements to the property.

Another special district category is the community development authority (CDA). Such an authority is a district created by the locality based on a petition from the property owners to help develop and maintain desired public infrastructure improvements, such as roads and buildings. The CDA is usually associated with development interests, such as retail centers, industrial centers, or tourism centers. Generally the CDA pays for development by issuing bonds and then having the property owners pay special assessments based on the level of debt. Assessments are levied either by placing a tax, such as \$0.25 per \$100 of assessed value, on the property within the district or by a special assessment each year that determines the benefit from the improvements and allocates them by property value. Depending on how the bond agreement is structured, assessment payments may be made directly to bondholders or to the locality. Table 2.3 lists community development authorities by locality. The table includes the name of the project, the purpose, the size, the bond amount, and, where possible, the current value. Three cities and 8 counties reported having CDAs.

The final category of special districts is that of localities within the Northern Virginia Transportation Authority. Localities within this authority have the ability to tax real property associated with industrial and commercial use up to \$0.125 per \$100 of assessed value to help fund transportation improvements. In 2009, an amendment to § 58.1-3221.3 specified that the revenues generated by the tax were to be used solely for (1) new road construction, design, and right-of-way acquisition, (2) new public transit construction, design, and right-of-way acquisition, (3) capital costs related to new transportation projects, or (4) the issuance costs and debt service on any bonds issued to support capital costs. There are 11 localities in the region of the authority: the cities of Alexandria, Fairfax, Falls Church, Fredericksburg, Manassas, and Manassas Park and the counties of Arlington, Fairfax, Loudoun, Prince William, and Stafford. Of those, one city (Fairfax) and two counties (Arlington and Fairfax) reported implementing the tax, as shown in Table 2.4.

\hypertarget{assessment-practices-reassessments-assessed-values}{%
\section{ASSESSMENT PRACTICES, REASSESSMENTS, ASSESSED VALUES}\label{assessment-practices-reassessments-assessed-values}}

\textbf{Table 2.5} details assessment practices among localities. The table includes cities and counties, but not towns, because only a small percentage of towns provided substantive answers. For those interested in the towns that responded, data are available from the Cooper Center upon request.

The second column lists whether a locality has a full-time assessor. Twenty-seven cities reported employing a full-time property tax assessor, while 11 did not. In contrast, only 36 counties had a full-time assessor while 59 did not. This reflects the fact that many counties reassess property less frequently than cities. No towns had assessors, since towns rely on assessed values established by their host counties.

Columns three, four, five, and six of Table 2.5 provide data on the conduct of general reassessments and cover four questions. (1) Are reassessments done by the locality or contracted out? (2) What is the reassessment frequency? (3) Is physical inspection part of the reassessment? (4) When was the reassessment last done? Regarding the conduct of the general reassessment, 28 cities reported conducting reassessments in-house while 10 reported contracting with outside assessors. Twenty-eight counties reported doing general reassessments in-house, while 67 reported contracting out for services. Section 58.1-3250 of the \emph{Code} requires cities to have a general reassessment of real estate every two years. However, any city with a total population of 30,000 or less may elect to conduct its general reassessments at four-year intervals.\footnote{The \emph{Code} does not specify which census is to be used.} Counties are required to have a general reassessment every four years (§ 58.1-3252). There is an exception for counties with a total population of 50,000 or less. These counties may elect to reassess at either five-year or six-year intervals (§ 58.1-3252). However, nothing in these sections affects the power of cities and counties to reassess more frequently. A large majority of the cities (30) reassess at one or two year intervals. In contrast, less than three out of ten counties (27) reassess that frequently. Virtually all of the populous cities and counties reassess annually or biennially. Towns rely on their surrounding county to provide assessments, so a town's reassessment occurs with the same frequency as the county's. The reassessment periods are summarized in the table on the following page.

Column seven of Table 2.5 shows information about maintenance assessments. While general reassessments involve reassessing all parcels to reflect changes in market value, maintenance assessments involve adjusting assessed values between reassessments because of new construction, improvements, damages, demolitions, subdivisions, and consolidations. Thirty-three cities responded that they performed maintenance assessments using staff, while five reported contracting for the work. Among counties, 66 reported performing maintenance reassessments using staff, while 29 reported contracting the work to independent appraisers.

Columns eight and nine of Table 2.5 cover physical inspection. Physical inspection refers to the actual inspection of the property as opposed to computerized mass-appraisal of parcels. If a locality responded that it did not perform physical inspections during the general reassessment, two further questions were asked:

\begin{Shaded}
\begin{Highlighting}[]
\CommentTok{\#table name: Reassessment Periods for Real Estate, 2019}
\end{Highlighting}
\end{Shaded}

\begin{enumerate}
\def\labelenumi{(\arabic{enumi})}
\tightlist
\item
  Does the locality perform a physical inspection at all? (2) If so, what is the inspection cycle? Among cities that responded, 18 reportedly did not have a physical inspection separate from the general reassessment cycle. Twenty others reported having a physical inspection cycle, the periods ranging anywhere from two to six years. Among counties that responded, 70 indicated they performed physical inspections during general reassessment, while 25 reported having physical inspection cycles ranging anywhere from one to six years.
\end{enumerate}

\textbf{Table 2.6} provides unpublished Department of Taxation 2018 data on total taxable assessed value of real estate by category. Taxable assessed value shows property qualifying for use value at its use value, not its market value. The percentage distribution of taxable assessed value is shown for two types of residential property (single-family and multi-family) as well as commercial and industrial property and agricultural property.

The text table on the next page compares the taxable assessed value by category for cities and counties. The total assessed value for all cities amounted to \$277.4 billion. Single-family residential property averaged 64.9 percent of taxable assessed value. Multi-family residential property averaged 11.1 percent of taxable assessed value. Commercial/industrial properties averaged just over one-quarter of the total value at 23.9 percent, while agricultural property values amounted to only 0.1 percent.

The total assessed value of property by category for counties in 2018 amounted to \$854.5 billion. Of that amount, 72.0 percent of assessed value was associated with single-family residential property, 5.9 percent with multi-family residential property, 18.2 percent with commercial/industrial property, and 4.0 percent with agricultural property.

With the total amounts from cities and counties combined, the total assessed valuation amounted to \$1,131.9 billion. Of that, 70.2 percent applied to single-family residential property, 7.1 percent applied to multi-family residential property, 19.6 percent applied to commercial/industrial property, and 3.0 percent to agricultural property.

Looking at the percentage breakdown for each type of locality, in 2018 the share of taxable

\begin{Shaded}
\begin{Highlighting}[]
\CommentTok{\#table name: Taxable Assessed Value by Category for Cities and Counties, 2018}
\end{Highlighting}
\end{Shaded}

assessed value for cities in the single-family residential category was between 40 percent and 59.9 percent in 19 cities and 60 percent or more in 18 cities. All cities but two had multi-family residential values under 19.9 percent of the total assessed value. Commercial and industrial property was the second most common category with 21 of the cities having between 20 percent and 39.9 percent of their property valuations coming from this type of property. Finally, only the cities of Suffolk and Franklin had more than 2 percent of their property valuation associated with agriculture.

Among counties the breakdown was slightly different. As in cities, the single-family residential value dominated the percentage breakdown. The single-family residential assessment percentage amounted to 60 percent or more for 71 counties. Another 20 received between 40 percent and 59.9 percent of the valuation from single-family residential real estate, while in four counties residential valuations amounted to no more than 39.9 percent of the total taxable assessed value (Buchanan, Dickenson, Highland, and Sussex). In contrast, only in Arlington county did the multi-family residential average share of value exceed 19.9 percent.

The category with the second highest valuation in counties was commercial and industrial property. Eighty-two counties had such property valued no higher than 19.9 percent of the total assessed value of property within the locality. In general, the percentage of assessed value in counties for commercial and industrial properties was less than that for cities (though two counties, coal-rich Dickenson and Buchanan, had the highest percentage valuations of such property). Finally, agricultural property averaged the least total assessed valuation in counties, though the percentage varied greatly among the individual counties. In 30 counties, valuations associated with agricultural property made up 20 percent or more of the total assessed value within the locality. The percentage in one county (Sussex) was 82.0 percent. The taxable assessed values for agriculture were much lower than they would have been without the advantage of use value assessment, a program explained in Sections 4 and 5.

\hypertarget{effective-tax-rates}{%
\section{EFFECTIVE TAX RATES}\label{effective-tax-rates}}

Tax rates are generally discussed in terms of either statutory (nominal) rates or effective rates. The statutory rate is the rate used by localities and is applied to the assessed value of a property.The effective rate is published by the Virginia Department of Taxation in their annual assessment/sales ratio study. The department derives the effective rate by multiplying the statutory tax rate by the median assessment ratio. In normal times when property values are rising, the median assessment ratio is usually less than 100 percent

\begin{Shaded}
\begin{Highlighting}[]
\CommentTok{\#table name: Share of Assessed Value of Real Estate by Category, 2018}
\end{Highlighting}
\end{Shaded}

because reassessments lag market increases and tend to be conservative. Consequently, the statutory rate is generally higher than the effective rate. However, this may not be true in difficult real estate markets. A limitation of the effective rates published by the Virginia Department of Taxation is that they are not current. The most recent year available at the present time is 2017. Despite the time lag, effective rates are important because they give a more accurate reflection of the differences in real property tax rates across localities.

\textbf{Table 2.7} shows city and county average effective tax rates in the year 2017. The department makes its computation in order to control for the variance in localities' assessment procedures and timing. Therefore, when comparing tax rates among localities, the reader may wish to consult both Tables 2.1 and 2.7. Table 2.1 shows statutory rates in 2019. Table 2.7 shows statutory and effective rates in 2017. The following text table summarizes the effective tax rates for the localities shown in Table 2.7.

It should also be pointed out that the Virginia Department of Taxation does not use the locally reported statutory tax rate in its computations. Instead, it calculates the statutory rate by dividing the real estate levy by the local real

\begin{Shaded}
\begin{Highlighting}[]
\CommentTok{\#table name: Effective Real Estate Tax Rates, 2017}
\end{Highlighting}
\end{Shaded}

estate \emph{taxable assessed value},\footnote{Taxable assessed value treats property qualifying for use value
  as taxable at its use value rather than at its full market value.} as reported in the local land book. This method of computing the statutory tax rate takes additional district levies into account.\footnote{Virginia Department of Taxation, \emph{The 2017 Virginia Assessment/Sales Ratio Study} (Richmond, February 2019), p.~35. The study
  can be found at \url{https://tax.virginia.gov/assessment-sales-ratiostudies}.}

In 2 cities and 10 counties the statutory rate was less than the effective rate. In two cities and seven counties statutory and effective rates were the same. Finally, in 34 cities and 78 counties statutory rates exceeded effective rates.

\begin{Shaded}
\begin{Highlighting}[]
\CommentTok{\#table name: Statutory and Effective Real Estate Tax Rates, 2017}
\end{Highlighting}
\end{Shaded}

The real property tax rates reported in Table 2.7 are a more accurate reflection of the differences among localities in tax rates on real property than those in Table 2.1 because they control for variations in assessment frequency and technique among localities. Table 2.7 also shows the latest reassessment in effect when the median ratio study was conducted, the number of sales used in the study, the median ratio, and the coefficient of dispersion.

The coefficient of dispersion measures how closely the individual ratios of each locality are arrayed around the median ratio. The formula for the coefficient of dispersion (CD) is:

where \[X_i\] represents the assessment/sales ratio for the \emph{i}th sale in a sample of size \emph{n}, and \[X_m\] represents the median ratio of the sample.\footnote{Virginia Department of Taxation, \emph{The 2017 Virginia Assessment/Sales Ratio Study}, p.~34.} If there were no dispersion, the CD would equal zero.

The text table below summarizes the coefficients of dispersion tabulated for the cities and counties. Eighteen of the cities had CDs of no more than 9.9 percent. Eight had CDs between 10 percent and 14.9 percent, 7 had CDs between 15

\begin{Shaded}
\begin{Highlighting}[]
\CommentTok{\#table name: Coefficient of Dispersion, 2017}
\end{Highlighting}
\end{Shaded}

and 19.9 percent, and 4 had CDs between 20 and 24.9 percent. Counties tended to vary more in the degree of dispersion. Thirteen had CDs between 5 and 9.9 percent, 18 had CDs between 10 and 14.9 percent, 25 had CDs between 15 and 19.9 percent, 26 had CDs between 20 and 24.9 percent, 11 had CDs between 25 and 29.9 percent, and 2 had CDs between 30 and 34.9 percent.

There is no upper limit for what is tolerable, but the International Association of Assessing Officers recommends an upper limit of 15 percent for residential properties.\footnote{International Association of Assessing Officers, \emph{Standard on Ratio Studies}, (approved April 2013), p.~17. \url{http://www.iaao}.
  org/media/standards/Standard\_on\_Ratio\_Studies.pdf.} Twenty-eight cities and 34 counties met the 15 percent standard.\footnote{The Department of Taxation's study applies to all types of property, not just residential property. Nonetheless, the majority of
  the measured sales are for single-family residential properties.}

As one would expect, the quality of local assessments, as measured by the CD is generally better in those localities that reassess annually, biennially, or that have just conducted a general reassessment. In 2017, of the 57 localities with CDs under 15 percent, all but 12 reassessed annually (28), biennially (10), or had just completed general reassessments (7).

\hypertarget{miscellaneous-items}{%
\section{MISCELLANEOUS ITEMS}\label{miscellaneous-items}}

\textbf{Table 2.8} presents miscellaneous taxes and exemptions related to real property. The first is the recreation tax. The \emph{Code} in §15.2-1807 permits localities to collect a real estate tax for recreation areas and playgrounds that is not to exceed \$0.02/\$100 of the assessed value of a property. This tax was reported by Charlottesville City.

The second column refers to the tax deferral ordinance permitted by § 58.1-3219 regarding the deferral of a portion of real estate tax increases when the tax exceeds 105 percent of the real property tax on property owned by a taxpayer in the previous year. Four localities (Charlottesville City, Richmond City, Middlesex County, and Amherst Town) reported implementing this deferral.

The third column refers to the establishment of a tax increment financing fund used to encourage development in certain areas and permitted by § 58.1-3245 of the \emph{Code}. Six cities (Bristol, Charlottesville, Chesapeake, Emporia, Newport News, Virginia Beach, and Waynesboro), four counties (Arlington, Augusta, Fairfax, and Hanover), and one town (Front Royal) reported having implemented such a fund.

The fourth column refers to separate real property tax rates for energy-efficient buildings as permitted by § 58.1-3221.2 of the \emph{Code}. Three cities (Charlottesville, Roanoke, and Virginia Beach) reported having special rates for such real estate.

The fifth column lists localities that reported providing a separate real property classification for improvements to real property used in the manufacture of renewable energy. Only the cities of Charlottesville and Roanoke reported having this separate rate.

Finally, the last column refers to low-income grant programs, discussed earlier in this text under the subheading, ``Tax Relief Programs.'' Only the cities of Charlottesville and Norfolk, and the county of Arlington reported having these programs.

\begin{Shaded}
\begin{Highlighting}[]
\CommentTok{\#Table 2.1 "Real Property Statutory (Nominal) Tax Rates, CY 2019 and FY 2020"}


\CommentTok{\#Table 2.2 "Additional Real Property Special District Tax Levies for Special Purposes, 2019"}


\CommentTok{\#Table 2.3 "Community Development Authorities Requiring a Special Purpose Real Property Levy, 2019" }


\CommentTok{\#Table 2.4 "Special Purpose Real Property Tax Levies on Commercial Property in Northern Virginia Transportation Authority Region, 2019" }


\CommentTok{\#Table 2.5 "Real Property Assessment Procedures for Virginia Localities, 2019" }


\CommentTok{\#Table 2.6 "Assessed Value of Real Property by Category and by Locality, 2018*"}


\CommentTok{\#Table 2.7 "Real Property Effective True Tax Rates, 2017"}


\CommentTok{\#Table 2.8 "Real Property Miscellaneous Items, 2019"}
\end{Highlighting}
\end{Shaded}

\hypertarget{real-property-tax-relief-plans-and-housing-grants-for-the-elderly-and-disabled-in-2019}{%
\chapter{Real Property Tax Relief Plans and Housing Grants for the Elderly and Disabled in 2019}\label{real-property-tax-relief-plans-and-housing-grants-for-the-elderly-and-disabled-in-2019}}

Sections 58.1-3210 through 58.1-3218 of the \emph{Code of Virginia} provides that localities may adopt an ordinance allowing property tax relief for elderly and disabled persons. The relief may be in the form of either deferral or exemption from taxes. The applicant for tax relief must be either disabled or not less than 65 years of age and must be the owner of the property for which relief is sought (§ 58.1-3210). The property must be the sole dwelling of the applicant. In addition, localities have the option of exempting or deferring the portion of a person's tax that represents the increase in tax liability since the year the taxpayer reached 65 years of age or became disabled.

Localities are allowed to establish by ordinance the net financial worth and annual income limitations pertaining to owners, relatives and non-relatives living in the dwelling(§ 58.1-3212) of qualified elderly or handicapped persons. Further, mobile homes that are owned by elderly and disabled persons are included in the allowable property tax exemptions whether or not mobile homes are permanently affixed. Finally, local governments are authorized to extend tax relief for the elderly and disabled to dwellings that are jointly owned by individuals, not all of whom are over 65 or totally disabled.

The text table below indicates the range and media of the combined gross income allowance and combined net worth limitations for those cities, counties, and towns responding to the survey.

\begin{Shaded}
\begin{Highlighting}[]
\CommentTok{\#table name: Relief Plan Statistics: Gross Income and Net Worth, 2019}
\end{Highlighting}
\end{Shaded}

The following text table indicates, for those localities responding, how many localities have a tax relief plan that applies to both the elderly and the disabled, the elderly only,or the disabled only.

\begin{Shaded}
\begin{Highlighting}[]
\CommentTok{\# table name: Relief Plans for Elderly and Disabled, 2019}
\end{Highlighting}
\end{Shaded}

A majority of the localities exempt an owner from all or part of the taxes on the dwelling; usually the exemption is based on a sliding scale, with the percentage of the exemption decreasing as the income and/or net worth of the owner increases.

\textbf{Table 3.1} summarizes the various tax relief plans offered to elderly and disabled property owners in Virginia. The figures under the combined gross income heading reflect, first, the maximum allowable income (including the income of all relatives living with the owner) for an owner to be eligible for relief and, second, the amount of income of each relative living in the household, except the spouse, who is exempted from this amount.

For example, if the table reads ``\$7,500; first \$1,500 exempt,'' this indicates that the combined income of the owner and all relatives living with him/her may not exceed \$7,500, except that the first \$1,500 of income of each relative other than the spouse is excluded when computing this amount. The combined net worth amount listed usually excludes the value of the dwelling and a given parcel of land upon which the dwelling is situated.

\textbf{Table 3.2} details relief plans for renters. As the table indicates, few localities offer such plans. Only five cities (Alexandria, Charlottesville, Fairfax, Falls Church, and Hampton) and one county (Fairfax) reported having plans for renters.

\textbf{Table 3.3} lists the combined elderly and disabled beneficiaries reported by each locality in 2018 or 2019 and the amount of revenue foregone by each locality because of the homeowner exemptions. The amounts were reported by 23 cities, 66 counties, and 31 towns that responded to the question. The amounts reported foregone totaled \$21,698,890 for cities, \$60,242,734 for counties and \$636,229 for the reporting towns. The grand total amount foregone by reporting cities, counties, and towns was \$82,577,853. An estimate of the average revenue foregone per beneficiary is also provided for localities reporting both number of beneficiaries and foregone revenue. For cities, the average revenue foregone was \$1,518 per beneficiary. The amount for counties was \$1,581, and for towns it was \$360.

\begin{Shaded}
\begin{Highlighting}[]
\CommentTok{\#Table 3.1 Real Property Owner Tax Relief Plans for the Elderly and Disabled, 2019}

\CommentTok{\#Table 3.2 Real Property Renter Tax Relief Plans for the Elderly and Disabled, 2019}

\CommentTok{\#Table 3.3 Real Property Tax Relief Plans for the Elderly and Disabled Homeowners: Number of Beneficiaries and Foregone Tax Revenue, 2018 or 2019}
\end{Highlighting}
\end{Shaded}

\hypertarget{land-use-value-assessments-for-agricultural-horticultural-forestal-and-open-space-real-estate}{%
\chapter{Land Use Value Assessments for Agricultural, Horticultural, Forestal, and Open Space Real Estate}\label{land-use-value-assessments-for-agricultural-horticultural-forestal-and-open-space-real-estate}}

The \emph{Code of Virginia}, §§ 58.1-3230 through 58.1-3244, allows any locality that has adopted a comprehensive land use plan to enact a local ordinance providing for special assessments of agricultural, horticultural, forestal, and open space real estate. (Also see Article 10, Section 2, of the \emph{Constitution of Virginia}.) Before implementing such an ordinance, the locality's land use plan must have been adopted by June 30 of the year preceding the one in which taxes are first assessed and levied under the special assessment provision. (For localities that have adopted a fiscal year assessment date of July 1, the plan must have been adopted by December 31 of the preceding year).

Since 1957, every state has enacted legislation allowing some type of preferential treatment of farmland and in some states, like Virginia, open space land has also been included. There is an ongoing debate among tax specialists about how effectively preferential assessment slows conversion of land to more intensive uses. If the difference in returns from farming and development is high enough, development will occur even if farmland is taxed at a zero rate. A 1998 study showed that preferential assessment of farmland ``\ldots{} produced a gradual but significant difference in the loss of farmland that after a 20-year period amounted to about 10 percent more of land in a county being retained in farming than would have otherwise been the case.''\footnote{Commission on Local Government, Report on Proffered Cash Payments and Expenditures by Virginia's Counties, Cities and Towns, 2017-2018. \url{https://www.dhcd.virginia.gov/cash-proffers}.} This was a general result and the effectiveness of the policy would depend on local circumstances with greater success associated with modest development pressure. Additional information on use value assessment as well as other land preservation techniques is contained in a \emph{Virginia News Letter} article by Terance Rephann.\footnote{\url{http://www.tax.virginia.gov/content/local-tax-rates}}

\hypertarget{agricultural-horitcultural-forestal-and-open-space-real-estate}{%
\section{AGRICULTURAL, HORITCULTURAL, FORESTAL, AND OPEN SPACE REAL ESTATE}\label{agricultural-horitcultural-forestal-and-open-space-real-estate}}

The authorizing statute sets forth certain definitions for qualifying property. Real estate devoted to agricultural use includes either land devoted to the bona fide production for sale of plants and animals useful to man or land that meets the requirements for payments or other compensation pursuant to a soil conservation program. Real estate devoted to horticultural use is either land devoted to the bona fide production for sale of fruits, vegetables, and nursery and floral products, or land that meets the requirements for payments from a soil conservation program. Real estate devoted to forestal use in land devoted to tree growth in such quantity and so spaced as to constitute a forest area. And finally, real estate devoted to open space is real property used to preserve park and recreational areas, conserve land or other natural resources, or preserve floodways and land of historic or scenic value. Under this definition, golf courses can be considered open space property.

Agricultural and horticultural land must consist of a minimum of five acres, unless the acreage is used for aquaculture or raising specialty crops, in which case it may be less than five acres. Forestal land must consist of a minimum of 20 acres. Open space land must consist of a minimum of five acres. Exceptions include land adjacent to a scenic river, a scenic highway, a Virginia Byway, or public property in the Virginia Outdoors Plan as well as property in any city, county, or town having population density greater than 5,000 per square mile; in those localities the governing body may adopt a two acre minimum.

\textbf{Table 4.1} presents the information for the 114 localities reporting a land use assessment ordinance in effect for the 2019 tax year. The table includes the effective date of the ordinance, the types of real estate included, the cost of the application fee, the use value per acre valuation used by the locality, and the comparable State Land Evaluation Advisory Council (SLEAC) use value estimate. Section 5 provides details on the related program of agricultural and forestal districts.

\hypertarget{local-authority-in-land-use-assessments}{%
\section{LOCAL AUTHORITY IN LAND USE ASSESSMENTS}\label{local-authority-in-land-use-assessments}}

Nineteen cities, 75 counties, and 20 towns reported having some type of real estate subject to land use assessment in 2019. A locality is not required to include each of the four classifications of property in its local ordinance. It may choose which classifications are subject to land use assessment. Twelve cities, 36 counties, and 13 towns reported excluding one or more types of property. Upon the adoption of a land use assessment ordinance, the locality is authorized to direct a general reassessment in the following year.

In order to have their land assessed on the basis of use, property owners must apply to the local assessing officer at least 60 days preceding the tax year for which the special assessment is sought.\footnote{See ``What Will Become of the Car Tax?'' by John L. Knapp in \emph{Virginia Issues and Answers}. (Winter 2006), Vol. 13, No.~1, pp.~27-31.
  \url{http://www.via.vt.edu/winter06/index.html}} Localities may also require the owner to pay an application fee.

Localities may also have a minimum prior use requirement. However, prior use requirements can be waived for real estate devoted to the production of agricultural and horticultural crops that require more than two years from initial planting until commercially feasible harvesting.

Finally, property that would otherwise qualify for land use assessment is not disqualified because a portion of the property is being used for a different purpose, if it is authorized by a special use permit or allowed by zoning. However, that portion being used for a different purpose is deemed a separate piece of property for assessment purposes.

\hypertarget{the-use-of-special-assessments}{%
\section{THE USE OF SPECIAL ASSESSMENTS}\label{the-use-of-special-assessments}}

In 1973, the first year in which local ordinances could take effect, only four localities - the counties of Fauquier, Loudoun, Prince William, and the city of Virginia Beach - adopted special assessment ordinances. Currently, 114 localities report land use assessment ordinances in effect for at least one type of property. The total acreage reported covered is 247,180 for cities and 7,157,010 for counties. Nine towns reported 76,473 acres; this number is presumably already included in the county counts.

Localities reporting agricultural assessment ordinances numbered 110, while localities with forestal assessment ordinances numbered 88, and those with horticultural special assessments numbered 88. Open space special assessments are less common; 60 localities reported them. The breakdown of types of special assessment is shown in the text table.

\begin{Shaded}
\begin{Highlighting}[]
\CommentTok{\#table name: Types of Special Assessments, 2019}
\end{Highlighting}
\end{Shaded}

\hypertarget{application-fees-for-agricultural-land}{%
\section{APPLICATION FEES FOR AGRICULTURAL LAND}\label{application-fees-for-agricultural-land}}

More localities charge special assessment application fees for agricultural land than for any other type. Application fees for agricultural land vary widely by locality. They can be one-time charges or may have to be revalidated after several years. Among the cities, six reported charging no fee, two reported charging a one-time fee, eight reported a flat fee for each application, and three reported a base fee plus an additional amount per acre. Reported fees were as high as \$300 in the city of Staunton for its one-time fee.

Among the 75 counties reporting having land use assessments, one (Stafford) reported charging no fees, 29 reported a flat fee, one (York) reported a flat fee revalidated every sixth year, and 44 reported charging some variant of a base fee plus an additional amount per acre or per parcel.

Twenty-three towns reported having land use assessments. Eleven reported using the same method for determining application fees as used by the county in which the town is located. Five reported imposing no fees, four charged a base fee plus an additional amount per acre, and three charged a flat rate. The highest application fee reported was for the town of Blacksburg which has a flat fee of \$150.

\hypertarget{valuing-real-estate-for-land-use-assessment}{%
\section{VALUING REAL ESTATE FOR LAND USE ASSESSMENT}\label{valuing-real-estate-for-land-use-assessment}}

The local assessing officer has responsibility for determining which real estate meets the state-imposed criteria for land use assessment. This officer may request an opinion, depending on the type of property, from the Director of the Department of Conservation and Recreation, the State Forester, or the Commissioner of Agriculture and Consumer Services. These agency heads are also authorized to provide, either to the commissioner of revenue or to the assessing officer of each locality that has adopted a land use assessment ordinance, a statement of uniform statewide standards to be used in determining the qualifications for each type of property. Further, the State Land Evaluation Advisory Council is required to provided each locality using special assessments with a recommended range of suggested values for each type of property, based on the productive earning power of that particular type of land. SLEAC provides estimates based on two methods - an income method and a cash rental rate method.\footnote{Virginia Cooperative Extension, ``A Citizens' Guide to The Use Value Taxation Program in Virginia.'' \url{https://pubs.ext.vt.edu/448/448-037/448-037.html}} The income method capitalizes the average net income for agricultural properties in different categories (cropland 1, cropland 2, pastureland 1, pastureland 2, etc.). The method also provides a downward adjustment for land at risk of flooding. The rental rate method capitalizes average rents on agricultural properties in a locality or in the region if the sample for a locality is too small. The two methods do not have to provide similar figures. For the SLEAC estimates by locality for the two methods see \emph{Virginia's Use Value Assessment Program}, ``Agricultural and Horticultural Estimates,'' at \url{https://aaec.vt.edu/extension/use-value.html}

Only those indices of value that relate to agricultural, horticultural, forestal or open space use may be considered in determining the assessment. Any structure not related to such special use and the real estate upon which the structure is located cannot be included in the special assessment but must be taxed on the basis used for assessing other real property in the locality.

In our survey we included a question about the use value per acre used by the locality to determine the taxable value of Class I agricultural land, one of several classifications of agricultural land estimated by SLEAC. Seventy-eight localities (14 cities and 64 counties) provided information. We have also listed the SLEAC values for both the income and cash rental methods for comparison. In some cases, the local estimate seems to mirror either the income or rental method. In other cases, the locality seems to have chosen its own method. These differences in the valuations between SLEAC and the locality may be caused by a number of factors: (1) the locality may have better information on local conditions than SLEAC; (2) the locality may use different assessment procedures; or (3) the locality may have made an administrative decision to assess use value at a higher or lower value than SLEAC. A 2008 article by two Virginia Tech economists, \emph{Why Use-value Estimates Can Differ Between Counties}, by Franklin Bruce Jr.~and Gordon E. Graham, explains why variation exists in use-value estimates for neighboring localities. See \url{https://pubs.ext.vt.edu/446/446-013/446-013.html}.

For general information on use values and other aspects of the program, see the home page for Virginia's use value assessment program at \url{https://aaec.vt.edu/extension/use-value.html}.

\hypertarget{changes-in-use}{%
\section{CHANGES IN USE}\label{changes-in-use}}

Land use assessment can remain in effect only as long as the property is used for the purpose for which the special assessment was granted. A change from use value assessment will be based only upon a change in the use of the land. A change in ownership does not bring about a change in assessment unless the new owner changes the use of the land from a qualifying use to a non-qualifying use.

If the qualifying land reverts to a non-qualifying use, the property is subject to rollback taxes. These taxes are equal to the amount by which the tax on the property, had it been assessed on the same basis as other non-qualifying property in the locality, exceeds the tax that was paid on the property under special assessment. This provision is applicable to the five most recent complete tax years prior to the change. Property owners are also held responsible for a 5 percent payment penalty and for an interest penalty established by each locality, pursuant to §§ 58.1-3915 and 58.1-3916. Any change in use must be reported to the commissioner of revenue or other assessing officer within 60 days. Failure to comply subjects the owner to the amount of tax due plus interest and penalties, to be determined by the local ordinance.

There is also a penalty for any misstatement made in the application for special assessment. In such a case, the owner is liable for all taxes that would have been incurred had the real estate not been subject to special assessment, together with penalties due on such sum. If the misstatement was made with the intent to defraud the locality, the owner is assessed an additional penalty of 100 percent of the unpaid taxes.

\begin{Shaded}
\begin{Highlighting}[]
\CommentTok{\# Table 4.1 Land Use Value Assessments for Agricultural, Horticultural, Forestal, and Open Space Real Estate, 2019}
\end{Highlighting}
\end{Shaded}

\hypertarget{agricultural-and-forestal-districts}{%
\chapter{Agricultural and Forestal Districts}\label{agricultural-and-forestal-districts}}

Local governments are permitted to enact an ordinance providing for the creation of agricultural and forestal districts. Such districts are intended, as the \emph{Code} states, ``\ldots{} to conserve and to encourage the development and improvement of the commonwealth's agricultural and forestal lands for the production of food and other agricultural and forestal products.'' According to the \emph{Code}, the districts also ``\ldots conserve and protect agricultural and forestal lands as valued natural and ecological resources which provide essential open spaces for clean air sheds, watershed protection, wildlife habitat, as well as for aesthetic purposes.'' The authority for such districts is provided by the \emph{Code of Virginia}, §§ 15.2-4300 through 15.2-4314 (Agricultural and Forestal Districts Act) and §§ 15.2-4400 through 15.2-4407 (Local Agricultural and Forestal Districts Act).

In accordance with the Agricultural and Forestal Districts Act, each district must have a core of no less than 200 acres in one parcel or in contiguous parcels; however, districts of local significance created under the act may be as small as 20 acres.\footnote{Commission on Local Government, Report on Proffered Cash Payments and Expenditures by Virginia's Counties, Cities and Towns, 2017-2018. \url{https://www.dhcd.virginia.gov/cash-proffers}.} Further, the local governing body must review each district within four to ten years after its creation and every four to ten years thereafter. For additional information relating to the creation of the districts, see § 15.2-4305.

Land devoted to agricultural and forestal production within an agricultural and forestal district qualifies for special assessment for land use whether or not a local land use plan or special assessments ordinance has been adopted, provided that the land meets the criteria set forth in §58.1-3230 et seq. of the \emph{Code} (see also § 15.2-4312).\footnote{\url{http://www.tax.virginia.gov/content/local-tax-rates}}

Three cities and 28 counties reported having a total of 372 agricultural and forestal districts. In addition, two towns, Blacksburg and Louisa, reported a total of two districts. In terms of acreage, Cities reported a total of 2,530 acres and the two towns reported a total of 1,422 acres - 1,360 acres and 62 acres, respectively. These numbers were negligible compared to the 736,140 acres reported by counties. Of the counties, those reporting the greatest number of acres within agricultural and forestal districts were Fauquier (78,755 acres), Accomack (74,093 acres), Albemarle (72,665 acres), Culpeper (46,487 acres), and Isle of Wight (41,317 acres).

The following text table shows by year when the existing city and county districts came into existence. Four new districts were reported in 2019.

\begin{Shaded}
\begin{Highlighting}[]
\CommentTok{\# table name: Agricultural and Forestal Districts by Year of Creation for Cities and Counties, 1978 and 2019}
\end{Highlighting}
\end{Shaded}

\textbf{Table 5.1} presents information for all cities, counties, and towns which reported agricultural and forestal districts. The table includes the district creation date, acreage, and the review period for each district. Three cities, 28 counties and two towns reported having an agricultural and forestal district ordinance in effect for the 2019 tax year.

Section 4 of this publication provides details on the related program of land use value assessments and cites literature that questions the effectiveness of special assessments in slowing the conversion of participating land to other uses.

\begin{Shaded}
\begin{Highlighting}[]
\CommentTok{\# Table 5.1 Agricultural and Forestal Districts, 2019}
\end{Highlighting}
\end{Shaded}

\hypertarget{property-tax-exemptions-for-certain-rehabilitated-real-estate-and-miscellaneous-property-exemptions}{%
\chapter{Property Tax Exemptions for Certain Rehabilitated Real Estate and Miscellaneous Property Exemptions}\label{property-tax-exemptions-for-certain-rehabilitated-real-estate-and-miscellaneous-property-exemptions}}

\hypertarget{general-provisions-1}{%
\section{General provisions}\label{general-provisions-1}}

The \emph{Code of Virginia} provides that localities may adopt an ordinance allowing property tax exemption for certain rehabilitated commercial and industrial real estate (§ 58.1-3221) and residential real estate (§§ 58.1-3220 and 58.1-3220.1). To qualify for the exemption, the rehabilitated structure must be at least 15 years old for residential property or 20 years old for commercial or industrial property and must meet other restrictions that the locality may require. Exceptions exist for commercial and industrial property in state enterprise zones or local technology zones. In such instances, the minimum age may be 15 years. Real estate containing a hotel or motel no less than 35 years of age that has been substantially renovated may qualify for a partial exemption. The ordinance, in addition to any other restrictions, may restrict exemptions to real property located within described districts whose boundaries are determined by the governing body. Further, if rehabilitation is achieved through demolition and replacement of the structure, and the structure demolished is a registered Virginia landmark or is determined by the Department of Conservation and Recreation to contribute to the significance of a registered historic district, then the exemption does not apply (§ 58.1-3220).

A locality may impose a fee for applications for real property tax exemptions and credits for rehabilitated structures. Under §§ 58.1-3220, 58.1-3220.1, and 58.1-3221 a fee of not more than \$125 for residential properties and not more than \$250 for commercial, industrial, and/or apartment properties of six units or more may be applied.

The partial exemption from property taxation may be an amount equal to a percentage of the increase in assessed value resulting from the renovation or to an amount up to 50 percent of the cost of the renovation. The commissioner of the revenue or another local assessing officer determines the assessed value of the structure. The exemption begins on January 1 of the year following completion of the rehabilitation, with maximum exemption periods of 10 years for residential real estate and 15 years for commercial and industrial real estate. Localities may opt to shorten the time span or to reduce the amount of exemption in annual steps over the entire period or a portion of the time limitation, or both.

\textbf{Table 6.1} contains information about the 32 cities, 21 counties, and 9 responding towns that have adopted a rehabilitation ordinance. The table also includes the minimum age requirement, the exemption schedule, and the percentage increase in assessed value required for exemption.

\hypertarget{miscellaneous-property-exemptions}{%
\section{MISCELLANEOUS PROPERTY EXEMPTIONS}\label{miscellaneous-property-exemptions}}

Certain miscellaneous property tax exemptions are authorized in the \emph{Code} from § 58.1-3660 and § 58.1-3666. Most exemptions pertain to real property, but several include both real and personal property items as part of their categories. Few localities reported authorizing these exemptions. For instance, in the latest survey no locality reportedly allowed exemptions for erosion control improvements (§ 58.1-3665).

However, a small number of localities did report exempting property such as (1) environmental restoration sites (§ 58.1-3664); (2) recycling equipment and facilities, and solar energy equipment, devices and facilities (§ 58.1- 3661); (3) generating and co-generating equipment used for energy conservation (§ 58.1-3662); (4) certified stormwater management developments (§ 58.1-3660.1); and (5) wetlands and riparian buffers (§ 58.1-3666).

Survey information for miscellaneous property exemptions is shown in \textbf{Table 6.2}. The contents of the table are summarized following this text discussion of the various exemptions.

\hypertarget{environmental-restoration-site}{%
\section{Environmental Restoration Site}\label{environmental-restoration-site}}

Any county, city or town may grant exemption or partial exemption from local taxation on certified environmental restoration sites. Section 58.1-3664 lists the requirements to qualify for this exemption as: ``\ldots real estate which contains or did contain environmental contamination from the release of hazardous substances, hazardous wastes, solid waste or petroleum, the restoration of which would abate or prevent pollution to the atmosphere or waters of the Commonwealth and which (i) is subject to voluntary remediation pursuant to § 10.1-1232 and (ii) receives a certificate of continued eligibility from the Virginia Waste Management Board during each year which it qualifies for the tax treatment described in this section.''

\hypertarget{recycling-and-solar-energy-equipment}{%
\section{Recycling and Solar Energy Equipment}\label{recycling-and-solar-energy-equipment}}

A similar exemption or partial exemption is authorized by § 58.1-3661 for certified recycling equipment, facilities or devices and certified solar energy equipment, facilities or devices. Certified recycling items are defined as machinery and equipment certified by the Department of Waste Management as integral to the recycling process and for use primarily for the purpose of abating and/or preventing pollution of the atmosphere or water.

Certified solar energy items are defined as any property, including real and personal property, equipment, facilities or devices which collect or use solar energy for water heating, space heating or cooling, or other applications which would otherwise require a conventional source of energy such as petroleum products, natural gas, or electricity.

\hypertarget{generating-equipment}{%
\section{Generating Equipment}\label{generating-equipment}}

Generating equipment installed after 1974 for the purpose of converting from oil or natural gas to coal or to wood, wood bark, wood residue, or to any other alternate energy source for manufacturing and any co-generating equipment installed since that date to be used in manufacturing may be classified separately for property taxation. According to § 58.1-3662, localities may adopt an ordinance authorizing exemption or partial exemption for generating and co-generating equipment used for energy conservation. The ordinance becomes effective on January 1 of the year following the year of adoption.

\hypertarget{stormwater-management-developments}{%
\section{Stormwater Management Developments}\label{stormwater-management-developments}}

According to § 58.1-3660.1, certified stormwater management developments may be classified separately for property tax purposes. Such property is classified as ``any real estate improvements constructed from permeable material, such as, but not limited to, roads, parking lots, patios, and driveways, which are otherwise constructed of impermeable materials, and which the Department of Conservation and Recreation has certified to be designed, constructed, or reconstructed for the primary purpose of abating or preventing pollution of the atmosphere or waters \ldots{} by minimizing stormwater runoff.''

\hypertarget{wetlands-and-riparian-buffers}{%
\section{Wetlands and Riparian Buffers}\label{wetlands-and-riparian-buffers}}

Wetlands and riparian buffers are considered a separate classification of property subject to perpetual easement according to requirements established in § 58.1-3666. A wetland is defined as an area ``\ldots{} inundated or saturated by surface or ground water at a frequency or duration sufficient to support, and that under normal conditions does support, a prevalence of vegetation typically adapted for life in saturated soil conditions, and that is subject to a perpetual easement permitting inundation by water.'' A riparian buffer is an area ``\ldots{} of trees, shrubs or other vegetation, subject to a perpetual easement permitting inundation by water, that is (i) at least thirty-five feet in width, (ii) adjacent to a body of water, and (iii) managed to maintain the integrity of stream channels and shorelines and reduce the effects of upland sources of pollution by trapping filtering, and converting sediments, nutrients, and other chemicals.''

\hypertarget{summary-of-miscellaneous-exemptions}{%
\section{Summary of Miscellaneous Exemptions}\label{summary-of-miscellaneous-exemptions}}

The exemptions applying to property used for environmental restoration, recycling, solar energy, energy conservation, stormwater development, and wetlands and riparian buffers are summarized in Table 6.2. One town and 1 city reported an exemption for an environmental site. Eight cities and 7 counties reported exempting recycling equipment and facilities. Eleven cities and 17 counties reported exempting solar energy equipment and facilities. One city and 2 counties reported exempting generating equipment used for energy conservation purposes. Two cities, 3 counties and 2 towns reported exempting certified stormwater development property. Finally, 2 cities, 1 county, and 1 town reported an exemption for wetlands and riparian buffers.

\begin{Shaded}
\begin{Highlighting}[]
\CommentTok{\# Table 6.1 Property Tax Exemptions for Certain Rehabilitated Real Estate, 2019}

\CommentTok{\# Table 6.2 Property Tax Exemptions for Restoration Sites, Recycling, Solar Energy, Generators, Stormwater Developments, and Wetlands, 2019}
\end{Highlighting}
\end{Shaded}

\hypertarget{service-charges-on-tax-exempt-property}{%
\chapter{Service Charges on Tax-Exempt Property}\label{service-charges-on-tax-exempt-property}}

Sections 58.1-3400 through 58.1-3407 of the \emph{Code of Virginia} authorize localities to impose service charges on otherwise tax-exempt property. Several types of property are excluded from this provision, including the land and buildings of churches used exclusively for worship and property used exclusively for nonprofit private educational or charitable purposes.

In 1981, the Virginia General Assembly amended the \emph{Code} to restrict the use of the service charge on the value of real estate owned by the commonwealth to those localities where such property---excluding hospitals, educational institutions, roadway property, or property held for future construction---exceeds 3 percent of the value of all real estate located within the jurisdiction's boundaries. However, the service charge may still be levied on faculty and staff housing owned by state educational institutions and on property of the Virginia Port Authority, regardless of the portion of state-owned property located within the locality.

The service charge is based on the assessed value of the state- or privately-owned real estate and the amount the locality has expended in furnishing police and fire protection, refuse collection and disposal, and the cost of public school education (applicable only in the case of faculty and staff housing of an educational institution). These expenditures must exclude any federal or state grants specifically designated for these purposes and any assistance provided to localities under Title 14.1, Article 10, Law-Enforcement Expenditures, of the \emph{Code of Virginia}. If such services are not provided to the tax-exempt real estate or are funded by another service charge, the expenditures may not be included in calculations.

For (1) properties owned by religious organizations and used for religious purposes or (2) properties used for private, nonprofit educational or charitable purposes, the service charge may not exceed 20 percent of the real estate tax rate (or 50 percent in the case of faculty and staff housing at private educational institutions). The charge is determined by dividing the expenditures, as defined in the previous paragraph, by the assessed fair market value of all the real estate within the locality, except real estate owned by the United States government or by any of its instrumentalities.

The city of Richmond, as the seat of state government, clearly satisfies the 3 percent requirement. In addition, a number of other localities impose service charges either because they have faculty and staff housing or because they claim to exceed the 3 percent rule. The primary reason for the claim is the presence of a state institution of higher education or of a state correctional institution. However, in instances where the 3 percent requirement may not have been reached, an affected state agency may voluntarily have agreed to pay a service charge.

Based on the survey and some follow-up conversations, it was found that localities that have state educational institutions and which also charge the service fee include the cities of Charlottesville (University of Virginia), Fredericksburg (University of Mary Washington), Harrisonburg (James Madison University), Lexington (Virginia Military Institute), and Wise County (University of Virginia at Wise). Counties that impose service charges based on the presence of correctional institutions include Brunswick (Brunswick Correctional Center and Brunswick Work Center for Women), Buckingham (Buckingham Correctional Center and Dillwyn Correctional Center), Greensville (Greensville Correctional Center and Greensville Work Center), Fluvanna (Fluvanna Correctional Center for Women), Southampton (Southampton Correctional Center, Southampton Work Center for Men, Southampton Pre-Release and Work Center for Women, and Deerfield Correctional Center), and Wise (Red Onion State Prison, Wallens Ridge State Prison, and Wise Correctional Unit \#18). \textbf{Table 7.1} shows that 12 cities, 7 counties, and 1 town report imposing a service charge of some sort on state-owned or privately-owned property.

\begin{Shaded}
\begin{Highlighting}[]
\CommentTok{\# Table 7.1 Service Charges on Tax{-}exempt Property, 2019}
\end{Highlighting}
\end{Shaded}

\hypertarget{merchants-capital-tax}{%
\chapter{Merchants' Capital Tax}\label{merchants-capital-tax}}

The merchants' capital tax accounted for 0.1 percent of tax revenue for counties and less than 0.1 percent for towns in fiscal year 2018, the most recent year available from the Auditor of Public Accounts. No cities employ the tax and only 41 of the 95 counties use it exclusively. Four counties use the tax in conjunction with the business, professional, and occupational license (BPOL) tax. The other counties rely solely on the BPOL tax.The relative importance of the merchants' capital tax varies in the localities that collect it. For information on individual localities, see Appendix C.

The \emph{Code of Virginia}, §§ 58.1-3509 and 58.1-3510, provides that localities may impose a local tax on merchants' capital. Localities also have the option to exempt specific types of merchants from part or all of the tax. Merchants' capital is defined as the inventory of stock on hand, daily rental motor vehicles as defined in § 58.1-2401, and all other taxable personal property (except tangible personal property not for sale as merchandise, which is taxed as tangible personal property), excluding money on hand and on deposit.

Property held for rental in a short-term rental business could be subject to the merchants' capital tax. However, such property may also be classified under § 58.1-3510.4 making it subject to a separate freestanding tax. Consequently, daily rental property is discussed in this section and in Section 19, Miscellaneous Taxes.

In 2018, a separate classification was created for merchant' capital of wholesalers inventory normally contained in a structure of 100,000 square feet, with at least 100,000 square feet used to contain the inventory.

According to § 58.1-3704 of the \emph{Code}, no locality may impose a merchants' capital tax if it also imposes a BPOL tax on retail merchants. A number of localities impose both of these taxes, but they do not use the BPOL tax for retail sales.

In 1978, the General Assembly enacted legislation (§ 58.1-3509 of the \emph{Code}) that froze merchants' capital tax rates at their January 1, 1978 level. Localities that had raised their rates and/or assessment ratios after February 1, 1977 were required to roll back their rates on July 1, 1978 to the February 1, 1977 rate and refund any amount in excess. (See \emph{Virginia, Acts of Assembly}, 1978, c.~817, cl. 2, p.~1407.) While the enabling legislation prohibits localities from raising the merchants' capital tax rates, it does not prohibit localities from lowering the rates if they choose to do so. Thus, a locality may still lower the tax liability of a merchant by reducing the statutory rate, the assessment ratio, or both.

As previously noted, the merchants' capital tax is used exclusively by 41 counties. It is also imposed by nine towns responding to the survey. In contrast, 44 counties and all of the cities report using the BPOL tax for retail merchants in lieu of the merchants' capital tax. Four counties (Amherst, Hanover, Louisa, and Southampton) use both the BPOL tax and the merchants' capital tax, maintaining the latter tax on retailers. Seven counties (Bath, Culpeper, Fluvanna, Northampton, Patrick, Rappahannock, and Washington) reported having neither tax.

Those counties employing the merchants' capital tax generally have one or more incorporated towns that are business centers and that impose the BPOL tax. This precludes the counties from using the BPOL tax within the town boundaries. In contrast, counties can impose the merchants' capital tax within town boundaries even if a town has a BPOL tax. Most of the towns that tax business use the BPOL tax.

\textbf{Table 8.1} shows the statutory (nominal) tax rates per \$100 for the counties and towns, the value used for assessment, and the percentage of value. As shown in the text table, the unweighted mean of the statutory tax rate for counties was \$1.93 per \$100 of assessed value. The median was \$1.00 and the first and third quartiles were \$0.69 and \$2.85, respectively. The unweighted mean of the statutory tax rate for towns was \$0.49 per \$100 of assessed value. The median was \$0.46, and the first and third quartiles were \$0.20 and \$0.72, respectively.

\begin{Shaded}
\begin{Highlighting}[]
\CommentTok{\# Table name: Merchants\textquotesingle{} Capital Statutory Tax Rate, 2019}
\end{Highlighting}
\end{Shaded}

A majority of the localities that impose the merchants' capital tax compute the assessment of capital on a percentage of the original cost. Of the 45 counties and 10 towns listed in the table, 43 counties and 5 towns reported using original cost as a basis for assessment. Information on statutory tax rates of towns that did not respond to the survey can be found in the Virginia Department of Taxation's local tax rates survey for tax year 2016, available on the Virginia Department of Taxation's website, \url{http://www.tax.virginia.gov/}. Please note that the rates in the department's survey are for the 2014 tax year; it is the most recent information available for towns that did not respond to the Cooper Center survey.

\textbf{Table 8.2} lists the components of the merchants' capital tax imposed by the localities. Of the 45 counties that impose the tax, all reported imposing the inventory tax component of the tax. Twenty-one impose the rental vehicle tax. Finally, 22 counties reported imposing the short-term rental tax.

All reporting towns used the inventory tax component. None reported imposing a short-term rental tax. Amherst, Timberville and Pembroke reported imposing the rental vehicle tax.

\begin{Shaded}
\begin{Highlighting}[]
\CommentTok{\# Table 8.1 Merchants\textquotesingle{} Capital Tax, Basic Features, 2019}

\CommentTok{\# Table 8.2 Merchants\textquotesingle{} Capital Tax Provisions Concerning Taxation of Inventories, Rental Vehicles, and Short{-}Term Rentals, 2019}
\end{Highlighting}
\end{Shaded}

\hypertarget{tangible-personal-property-tax}{%
\chapter{Tangible Personal Property Tax}\label{tangible-personal-property-tax}}

The personal property tax is the second most important source of tax revenue for cities and counties, though it is not as important for towns. In fiscal year 2018, the most recent year available from the Auditor of Public Accounts, the personal property tax accounted for 10.7 percent of tax revenue for cities, 14.6 percent for counties, and 4.5 percent for large towns. These are averages; the relative importance of taxes in individual cities, counties, and towns may vary significantly. For information on individual localities, see Appendix C.

Cities, counties, and towns are permitted to tax the tangible personal property of businesses and individuals pursuant to the \emph{Code of Virginia}, §§ 58.1-3500 through 58.1-3521. Included in this category are such items as motor vehicles, business furniture and fixtures, farming equipment, trailers, boats, recreational vehicles, and campers.

Localities may elect to prorate the taxes on motor vehicles, trailers, and boats which have acquired a situs within a locality after the tax day for the balance of the tax year. The proration must be on a monthly basis with a period of more than a half a month counted as a full month and a period of less than half a month not counted (§ 58.1-3516).

Under § 58.1-3504, localities may elect to exempt household goods and personal effects from taxation; these effects may now include personal electronic and communication devices such as cell phones, tablets, and personal home computers. Under § 58.1-3505, localities may also exempt certain farm animals, products, and machinery. In addition, according to § 58.1-3506, the following categories are segregated as separate classes of tangible personal property under the condition that the tax rate on these items may not exceed that levied on other classifications of tangible personal property: boats or watercraft weighing five tons or more; privately owned pleasure boats and watercraft used for recreational purposes only; certain aircraft; antique automobiles; certain heavy construction machinery; certain computer hardware; motor vehicles specially equipped to provide transportation for physically handicapped individuals; privately owned vans with a seating capacity for twelve or more used exclusively for a ride-sharing arrangement; motor vehicles owned by a nonprofit organization and used to deliver meals to homebound persons or to provide transportation for senior or handicapped citizens; privately owned camping trailers and motor homes, as defined in § 46.2-100, which are used for recreational purposes only; and motor vehicles owned by members or auxiliary members of a volunteer rescue squad or volunteer fire department. Section 58.1-3506 provides for the segregation of motor vehicles owned or leased by a motor carrier into a separate classification of personal property. In addition, vehicles that use clean special fuels as authorized by § 46.2-749.3, which include hydrogen, natural gas, and electricity, are also treated as a separate tangible personal property category. In 2014, a separate classification was added for new business property for businesses qualifying as new businesses under the local business incentive program.

The \emph{Code of Virginia} provides that all vehicles without motor power that are used or designed to be used as mobile homes are segregated as separate classes of tangible personal property. This is conditional upon the assessment ratio and the tax being the same as those applicable to real property {[}§ 58.1-3506, Subsection A.8., and § 58.1-3506, Clause (iii), Subsection B{]}.

In addition, tangible personal property used in research and development of businesses and certain energy conversion equipment used in manufacturing are segregated as separate classes of tangible personal property. This is conditional upon the assessment ratio and the tax not exceeding that applicable to machinery and tools {[}§ 58.1-3506 Clause (ii), Subsection B{]}. For more on the machinery and tools tax, see Section 10.

In addition to the property discussed in this section, the \emph{Code} lists several special categories of property which are exempt from real \emph{and} personal property taxes (see § 58.1- 3660 through § 58.1-3666). These categories are discussed in Section 6 under the heading, ``Miscellaneous Property Exemptions,'' and are listed in Table 6.2.

\hypertarget{information-on-personal-property-tax}{%
\section{INFORMATION ON PERSONAL PROPERTY TAX}\label{information-on-personal-property-tax}}

\textbf{Table 9.1} provides information related to the personal property tax, including the number of personal property accounts within a locality, the personal property tax rate, whether localities have special levies, property tax due dates, effective dates of assessment, options for payment of the personal property tax, and categories of vehicles for which proration is offered. In the survey, one city (Chesapeake) and one county (Accomack) reported some kind of special district levy. Regarding collections, 23 cities, 64 counties, and 88 towns reported collecting the tax once a year, while 15 cities, 31 counties, and 11 towns reported collecting it at least semi-annually. The most common due dates for payment of the tax are June 5 and December 5. Also, localities predominantly indicated January 1 as the effective date of assessment. Of the localities that reported imposing a personal property tax, 18 cities, 57 counties, and 19 towns offered options for the payment of the tax. The most common payment alternative provided by local governments is the option for taxpayers to prepay their balance at any time during the calendar year before the due date. The due date terms apply to all types of vehicles for all but 10 localities that answered the question.

Finally, 24 cities, 37 counties and 16 towns reported offering proration of the personal property tax on specific or all categories of motor vehicles. Though the term ``motor vehicle'' applies to all automotive vehicles with rubber tires for use on roadways, many localities use different definitions. For more detailed definitions of the categories for which proration is offered, please use the telephone/email listings in Appendix B to contact individual localities.

\textbf{Table 9.2} contains information on personal property tax exemptions for the elderly and disabled. The survey indicated that 13 cities, 45 counties, and 4 towns permitted some sort of exemption for the elderly or the disabled constrained by specific income and net worth limits.

\hypertarget{motor-vehicle-tax}{%
\section{MOTOR VEHICLE TAX}\label{motor-vehicle-tax}}

Historically, the most important tangible personal property category has been motor vehicles. This tax is often called the ``car tax'' even though it covers sport utility vehicles (SUVs), pickup and panel trucks, large trucks, minivans, and motorcycles as well. In the survey, localities were asked to provide the percentage of personal property taxes coming from motor vehicles in fiscal year 2019. The unweighted average percentages for cities, counties and towns were 68 percent, 66 percent, and 74 percent, respectively. It is possible that some localities misunderstood the question about this topic and incorrectly counted state Personal Property Tax Relief Act (PPTRA) reimbursements as part of a local tax instead of as non-categorical state aid.

The Personal Property Tax Relief Act of 1998 (§ 58.1-3524) established a system by which the state would reimburse localities for relief on the tangible personal property tax.\footnote{Commission on Local Government, Report on Proffered Cash Payments and Expenditures by Virginia's Counties, Cities and Towns, 2017-2018. \url{https://www.dhcd.virginia.gov/cash-proffers}.} Passenger cars, pickup or panel trucks, and motorcycles owned or leased by natural persons and used for non-business purposes were to have the tax eliminated on the first \$20,000 of value over a five year period. Twelve and one-half percent of the tax was to be eliminated in tax year 1998, 27.5 percent in tax year 1999, 47.5 percent in tax year 2000 and 70 percent in tax year 2001. One hundred percent was slated to be eliminated in tax year 2002 and thereafter, but this final step was not implemented due to Virginia's budget crisis in that period. Instead, in 2002, the General Assembly froze the reimbursement rate at 70 percent. Then, a special session of the General Assembly determined that the state would freeze what it was giving to localities at \$950 million annually. Beginning tax year 2006, each locality's percentage share from the state distribution is based upon its actual share of the state reimbursements from tax year 2005. Each locality receiving a state reimbursement must reduce its rate on the first \$20,000 value so that the sum of local tax revenue and state reimbursement to the locality approximates what the locality would have received based on the local valuation method and the local tax rate before the car tax rebate became law.

Vehicle assessed values are based on published market guides. For valuation of automobiles, all localities use the National Automobile Dealers' Association's \emph{Official Used Car Guide} (NADA) as their \emph{primary} valuation guide for cars and sport utility vehicles. When a vehicle is not listed in the primary guide, the locality obtains values from some other source. All cities and counties in Virginia levy this tax on motor vehicles.

Any comparison of personal property tax rates across localities is misleading if differences in the source of assessment value are not considered. Thus, the effective tax rates must be standardized by the assessment valuation method employed by a locality. To do this, an adjusted effective tax rate was calculated for each locality based on the NADA retail value of a 2018 Toyota Camry LE four-door sedan with a four-cylinder engine. In recent years, the Camry has been the best selling car in the U.S. The base data, summarized in the text table below, were obtained from NADA's \emph{Official Used Car Guide}.

\begin{Shaded}
\begin{Highlighting}[]
\CommentTok{\# table name: NADA Value, 2018 Toyota Camry, January 2019}
\end{Highlighting}
\end{Shaded}

The adjusted effective tax rate is found by multiplying the statutory tax rate by the percent of retail value and the assessment ratio. For those localities using the retail value and assessing at 100 percent, the statutory and effective tax rates are the same. The text table below summarizes the dispersion of the effective tax rates among localities.

In regard to individual localities, the adjusted effective rate for cities ranged from \$1.76 (Galax) to \$4.64 (Alexandria, Falls Church). The adjusted effective rate for counties ranged from \$0.30 (Bath) to \$4.35 (Greensville) and, in towns, ranged from \$0.04 (Eastville) to \$3.53 (Chatham). The much lower town rates reflect their limited fiscal responsibilities as subordinate units of government within counties. The town tax is in addition to the county tax.

\begin{Shaded}
\begin{Highlighting}[]
\CommentTok{\# table name: Adjusted Effective Tax Rates Among Localities, 2019}
\end{Highlighting}
\end{Shaded}

Besides the adjusted effective tax rate, \textbf{Table 9.3} also provides data on the tax rate, assessment value concept, the percent of retail value, the assessment ratio, percentage of personal property tax receipts from automobiles and light trucks, and the number of automobiles and light trucks within a locality. Among the cities that answered the question, the number of vehicles ranged from 429,645 in Virginia Beach to 3,040 in Norton. Among counties, the number ranged from 994,469 in Fairfax to 3,050 in Highland.

The assessment value is important because it provides an estimate of the percent of retail value the locality will assign to the vehicle when determining the effective tax rate. The assessment value used varies among localities. Care must be taken when evaluating the data based on the three valuation methods listed because a valuation method may have subcategories. The latest NADA book, for instance, lists three categories for trade-in value based on condition: rough, average, and clean. Other valuation guides may use some variant of this breakdown for the retail and loan value categories. This year and in past years our example listed the percentages based on clean retail, clean loan value, and clean trade-in.

The following text table shows the frequency of each valuation method among localities. Since many towns use the same concept as their respective counties, a tally is not shown for them.

\begin{Shaded}
\begin{Highlighting}[]
\CommentTok{\# table name: Frequency of Valuation Methods, 2019}
\end{Highlighting}
\end{Shaded}

Localities incorporate an assessment ratio in the valuation process. Most cities and counties use a 100 percent ratio of whatever value concept they adopt. The following text table summarizes the dispersion of assessment ratios.

Information on tax rates of towns that did not respond to the survey can be found in the Virginia

\begin{Shaded}
\begin{Highlighting}[]
\CommentTok{\# table name: Dispersion of Assessment Ratios, 2019}
\end{Highlighting}
\end{Shaded}

Department of Taxation's local tax rates survey for tax year 2018.\footnote{\url{http://www.tax.virginia.gov/content/local-tax-rates}} The rates shown are the most recent information available for towns that did not respond to the Cooper Center survey.

\textbf{Table 9.4} continues with data related to the PPTRA for motor vehicles for tax years 2018 and 2019. The second column lists whether the locality offers exemptions for low-value automobiles and light trucks. Twenty-one cities, 49 counties and 25 towns reported offering an exemption of some sort to low-value vehicles. The third column refers to methods for applying PPTRA tax relief. A locality can use one of three methods: a reduced rate method (RR), a specific relief method that provides the same percentage of relief for all qualifying vehicles (SRSP), and a specific relief method that provides a declining percentage of relief as the vehicle's value rises (SRDP). The text table below summarizes the choices by all cities, 93 counties and the 66 towns that answered the question.

\begin{Shaded}
\begin{Highlighting}[]
\CommentTok{\# table name: Frequency of PPTRA Methods of Relief, 2019}
\end{Highlighting}
\end{Shaded}

Localities overwhelmingly use the specific relief method that provides the same percentage of relief for all qualifying vehicles. We assume the reporting towns use the same method as is used by the counties in which they are located.

The final set of columns provides data on the taxpayer liability for a vehicle assessed at \$20,000. What constitutes a \$20,000 vehicle in one locality may not match what constitutes a \$20,000 vehicle in another locality because of the differing valuation methods and assessment ratios used by the localities. Tax year 2019 is featured in the text table. The columns in Table 9.4 provide the locality's total car tax, the amount of the state credit, and the resulting taxpayer liability for 2018 and 2019. In some cases we were not given the tax on a vehicle, but were provided the percentage share covered by the tax, the credit, and the taxpayer liability. In such cases only the percentage is listed. The text table below summarizes the percentage of state aid reported by cities and counties.

\begin{Shaded}
\begin{Highlighting}[]
\CommentTok{\# table name: Dispersion of State{-}Aid Assessment Ratios, 2019}
\end{Highlighting}
\end{Shaded}

For the \$20,000 vehicle example, a lower percentage implies a higher resulting taxpayer liability relative to the total tax levied by a locality. Most cities provided a state credit between 50 percent and 60 percent of their total tax levied. The median state credit among cities in 2019 was 51.0 percent of the total tax, while the first quartile was 46.8 percent and the third quartile was 53.9 percent. Among counties the largest group reported the credit as a percentage of the total tax as between 20 percent and 49.9 percent. The median percentage of the taxpayer credit was 39.2 percent, with the first and third quartiles being 34.6 percent and 50.0 percent, respectively.

While the state credit for many localities usually diminishes each year, it is possible to have a greater state credit percentage for a current survey than for a previous one. Because the state payout to each locality is fixed, and the number and value of vehicles normally rise, it is generally assumed that as time passes the funding will decrease for each automobile. That expectation, however, does not account for either a possible disinflationary trend in the automobile market during a recession or a possible fall in the number of motor vehicles in the locality. In either of these cases a locality may be able to increase its payout percentage for each automobile within the locality.

The next text table summarizes the range of actual taxes for cities and counties based on the information from 2019. It summarizes the total tax, state credit and resulting taxpayer liability for those localities that provided dollar amounts. The measures of central tendency (the median and quartiles) do not include localities that did not answer.

As shown in the text table, 29 cities reported levying a tax between \$501 and \$1,000 before the PPTRA credit was factored in, while 1 reported levying taxes of \$1,001 or more and 4 reported levying taxes of \$500 or less. The median tax levied for all cities was \$808. Most PPTRA credits, 22 of the 34 reported, were between \$251 and \$500. The median credit was \$389. Most of the resulting taxpayer liabilities in cities were also between \$251 and \$500, with the median at \$399.

Among counties, original tax liabilities ranged from \$251 to over \$1,000. The median of the tax was \$720. Most counties gave credits in the \$251 to \$500 range, though about one-third provided a credit in the \$0 to \$250 range. The median credit among counties was \$274. Thirty-nine counties collected between \$251 and \$500 after the PPTRA

\begin{Shaded}
\begin{Highlighting}[]
\CommentTok{\# table name: Total Tax, State Credit and Tax Liability for a $20,000 Vehicle in Cities and Counties, 2019}
\end{Highlighting}
\end{Shaded}

tax credit was figured in. For counties, the median taxpayer liability after allowing for the credit was \$437.

\textbf{Table 9.5} lists localities that report giving a reduction in the personal property tax for high-mileage vehicles. This is permitted by § 58.1-3503.3, which states that the commissioner of the revenue, using an automobile pricing guide, may ``use all applicable adjustments in such guide to determine the value of each individual automobile.'' Many guides allow for adjustments in value for high- or low-mileage vehicles. Thirty-four cities, 73 counties, and 23 towns reported reduced valuations for high-mileage vehicles. Certain localities that reported giving such reductions also told us they couldn't really ascertain the number of beneficiaries or foregone revenue because the software they used to determine valuation didn't break down adjustments for them. Therefore, for some localities, though they responded that they had the reduction, they could not provide information about beneficiaries or foregone revenue.

Based on localities that did respond for both questions on beneficiaries and foregone revenues, there were a total of 9,605 beneficiaries of the high-mileage adjustment in cities, with the amount of revenue foregone totaling \$563,465. Among localities that provided both number of beneficiaries and revenues foregone, this amounted to an average reduction per beneficiary of \$58.66. In counties, the number of beneficiaries of the adjustment reported was 30,341. The amount of foregone revenue reported was \$1,313,558. The average reduction per beneficiary for those reporting both figures was \$43.29.

\textbf{Table 9.6} compares the tax rates and assessment components of the car tax between 1997, the year before \$\^{}the PPTRA went into effect, and 2019. The table provides information on localities that have raised their personal property taxes on motor vehicles since the beginning of the PPTRA.

When the PPTRA became law, some saw it as the beginning of the end of the ``car tax.'' However, as reimbursements rose and the state's fiscal condition worsened, the commonwealth decided to limit the rollback. As previously noted, now each locality is annually given a lump sum by the state that is applied to each resident's total property tax. The state reimbursements are based on 1997 effective rates as provided by the PPTRA. Any increase in the effective rate consequent to the 1997 rate is not covered by the PPTRA reimbursement from the state.\footnote{See ``What Will Become of the Car Tax?'' by John L. Knapp in \emph{Virginia Issues and Answers}. (Winter 2006), Vol. 13, No.~1, pp.~27-31.
  \url{http://www.via.vt.edu/winter06/index.html}}

Making certain assumptions about the assessment value concept (which will be discussed below), it appears that large majorities of cities and counties have increased their effective rates since 1997. Twenty-eight cities and 78 counties increased them. The assumption made here is that the value assessment concepts follow a clear path of valuation. In NADA's \emph{Official Used Car Guide}, for instance, the lowest valuation is applied to loan value, a higher valuation is applied to trade-in value, and the highest valuation is applied to retail value. This is the hierarchy one would expect to see when comparing average measures of loan, trade-in, and retail value, or clean measures of loan, trade-in, and retail value. A problem arises, however, with those valuations that maintain subcategories. NADA's multiple trade-in values, based on condition of vehicles, as discussed earlier, have not been tracked as separate categories. Therefore, we can't be sure whether certain localities have changed subcategories. Consequently, historical adjustments within this valuation cannot be determined from the table.

\textbf{Table 9.7} gives the pricing guide, the value used, the tax rate, and the depreciation schedule, if any, for large trucks, two tons and over. Answers were provided by all cities and counties and 75 of the responding towns.

\hypertarget{other-personal-property-taxes}{%
\section{OTHER PERSONAL PROPERTY TAXES}\label{other-personal-property-taxes}}

As previously noted, tangible personal property taxes are not limited to motor vehicles. There are about 20 categories in addition to motor vehicles, ranging from farm equipment to recreational vehicles and mobile homes (the general categories can be found from § 58.1-3504 through § 58.1- 3506). Household goods are a legal category, but no locality reports taxing them.

Localities exhibit a wide variation in their choices of valuation methods, pricing guides, and depreciation methods. Consequently, great care must be exercised when comparing taxes in different jurisdictions. Unless otherwise stated, the valuation method for the depreciation schedules is original cost.

A further problem pertains to towns. Certain towns provided a tax rate without showing a basis or depreciation schedule. In a follow-up for a previous survey, we called several towns in an attempt to elicit more information. Generally, a town representative confirmed the rate existed, but told us the county determined the actual depreciation schedule. The county representative confirmed that the county determined the town's depreciation schedule but added that if the county did not tax a particular item, there was no schedule. Therefore the town could not collect any taxes for that item.

\textbf{Table 9.8} displays tangible personal property taxes on heavy tools and machinery, computers, and generating equipment for business use for cities, counties and 48 reporting towns. The text table below summarizes how many localities report a tax rate for each category.

\begin{Shaded}
\begin{Highlighting}[]
\CommentTok{\# table name: Taxes on Heavy Tools and Machinery, Computer Hardware, and Generating Equipment, 2019}
\end{Highlighting}
\end{Shaded}

\textbf{Table 9.9} displays tax rates on research and development, business furniture and fixtures, and biotechnology equipment for cities, counties and 46 respondent towns. The text table below shows how many localities report a tax rate for each category.

\begin{Shaded}
\begin{Highlighting}[]
\CommentTok{\# table name: Taxes on Research and Development, Furniture and Fixtures, and Biotechnology, 2019}
\end{Highlighting}
\end{Shaded}

\textbf{Table 9.10} displays tax rates on computer hardware in data centers, farm equipment, and livestock for cities, counties and 17 respondent towns. The text table below shows how many localities report a tax rate for each category.

\begin{Shaded}
\begin{Highlighting}[]
\CommentTok{\# table name: Taxes on Computer Hardware in Data Centers, Livestock, and Farm Equipment, 2019}
\end{Highlighting}
\end{Shaded}

\textbf{Table 9.11} displays tax rates on boats and aircraft for cities, counties, and 53 respondent towns. The text table below shows how many localities report a tax rate for each category.

\begin{Shaded}
\begin{Highlighting}[]
\CommentTok{\# table name: Taxes on Boats Over Five Tons, Pleasure Boats, and Aircraft, 2019}
\end{Highlighting}
\end{Shaded}

\textbf{Table 9.12} displays tax rates on antique motor vehicles, recreational vehicles, and mobile homes for cities, counties, and 66 respondent towns. The text table below shows how many localities report a tax rate in each category.

\begin{Shaded}
\begin{Highlighting}[]
\CommentTok{\# table name: Taxes on Antique Motor Vehicles, Recreational Vehicles, and Mobile Homes, 2019}
\end{Highlighting}
\end{Shaded}

\textbf{Table 9.13} displays tax rates on horse trailers, motor vehicles powered solely by an electric motor, and special clean fuel vehicles (hydrogen, natural gas, electric) used for driving for cities, counties, and 29 respondent towns. The text table below shows how many localities reported a tax rate in each category.

\begin{Shaded}
\begin{Highlighting}[]
\CommentTok{\# table name: Taxes on Horse Trailers, Special Fuel Vehicles, and Electric Vehicles, 2019}
\end{Highlighting}
\end{Shaded}

\begin{Shaded}
\begin{Highlighting}[]
\CommentTok{\# Table 9.1 Tangible Personal Property Tax General Information, 2019}

\CommentTok{\# Table 9.2 Tangible Personal Property Tax Relief for Elderly and Disabled, 2019}

\CommentTok{\# Table 9.3 Tangible Personal Property Tax for Automobiles and Trucks of Less than Two Tons, 2019}

\CommentTok{\# Table 9.4 Personal Property Tax Relief Act for Motor Vehicles State Credit for $20,000 Vehicle, 2018 and 2019}

\CommentTok{\# Table 9.5 Localities That Report Having a Personal Property Tax Reduction for High Mileage Vehicles, 2018 or 2019}

\CommentTok{\# Table 9.6 Assessment Component Changes in Cities and Counties from 1997, When PPTRA Went into Effect, to 2019}

\CommentTok{\# Table 9.7 Tangible Personal Property Tax Rates for Large Trucks Two Tons and Over, 2019}

\CommentTok{\# Table 9.8 Tangible Personal Property Taxes Related to Business Use for Heavy Tools and Machinery, Computer Hardware, and Generating Equipment, 2019}

\CommentTok{\# Table 9.9 Tangible Personal Property Taxes Related to Business Use for Research and Development, Furniture and Fixtures, and Biotechnology Equipment, 2019}

\CommentTok{\# Table 9.10 Tangible Personal Property Taxes for Computer Hardware in Data Centers, Livestock, and Farm Equipment, 2019}

\CommentTok{\# Table 9.11 Tangible Personal Property Taxes for Boats and Aircraft, 2019}

\CommentTok{\# Table 9.12 Tangible Personal Property Taxes for Antique Motor Vehicles, Recreational Vehicles, and Mobile Homes, 2019}

\CommentTok{\# Table 9.13 Tangible Personal Property Taxes Related to Horse Trailers, Special Fuel Vehicles, and Electric Vehicles, 2019}
\end{Highlighting}
\end{Shaded}

\hypertarget{machinery-and-tools-property-tax}{%
\chapter{Machinery and Tools Property Tax}\label{machinery-and-tools-property-tax}}

In fiscal year 2018, the most recent year available from the Auditor of Public Accounts, the machinery and tools property tax accounted for 1.6 percent of total tax revenue for cities, 1.2 percent for counties, and 2.0 percent for large towns. These are averages; the relative importance of taxes in individual cities, counties, and towns may vary significantly. For information on individual localities, see Appendix C.

Under § 58.1-3507 of the \emph{Code of Virginia}, certain machinery and tools are segregated as tangible personal property for local taxation. According to the \emph{Code}, the classes of machinery and tools that are segregated are those that are used for ``manufacturing, mining, processing and reprocessing (excluding food processing), radio or television broadcasting, dairy, and laundry or dry cleaning.'' The tax rate on machinery and tools may not be higher than that imposed on other classes of tangible personal property.

Section 58.1-3507 provides a uniform classification for idle machinery. Idle machinery and tools are to be classified as intangible personal property no longer subject to local taxation. Items are defined to be idle if they have not been used for at least one year prior to the given tax day and no one can reasonably suppose that the machinery or tool will be returned to use in the given tax year.

Section 58.1-3980 provides an appeal procedure for the machinery and tools tax. The \emph{Code} states, ``\ldots{} any person, firm, or corporation assessed by a commissioner of the revenue \ldots{} aggrieved by any such assessment, may, within three years from the last day of the tax year for which such assessment is made, or within one year from the date of the assessment, whichever is later, apply to the commissioner of the revenue or such other official who made the assessment for a correction thereof.''

\textbf{Table 10.1} presents the 2018 tax rates on machinery and tools for the 37 cities, 91 counties, and 79 towns that reported imposing the tax. The machinery and tools tax is shown in the table according to the following categories: the basis of assessment, assessment type, the statutory (nominal) tax rate per \$100, the assessment ratio, and the effective tax rate (computed by multiplying the statutory tax rate by the assessment ratio). \emph{Effective tax rates among localities are only comparable if they use the same basis of assessment and apply it to the same age of property}. Most localities assess machinery and tools on the basis of original cost, fair market value, or book value. Frequently, a sliding scale is used, with the effective tax rate varying according to the age of the property.

Thirty-six cities reported using original cost as the basis of assessment. Eighty-eight counties imposing the tax used original cost. Finally, 69 of the towns reported basing their assessments on original cost. The remainder used fair market value or depreciated cost. In many cases it is accurate to say that towns followed the same method as the county in which they are located. However, some exceptions exist.

The following text table, using unweighted averages, compares localities using original cost as their basis. The table is based on machinery and equipment one year old. The medians for cities, counties and towns were \$1.05, \$0.90, and \$0.39, respectively. Town rates were in addition to rates imposed by their host counties.

\begin{Shaded}
\begin{Highlighting}[]
\CommentTok{\# table name: Machinery and Tools: Effective 1st Year Tax Rate per $100 for Localities Using Original Cost, 2019}
\end{Highlighting}
\end{Shaded}

\textbf{Table 10.2} presents the 2019 tax rates in industries which the \emph{Code} permits specific types of equipment to be categorized as machinery and tools. The separate classification is permitted by § 58.1-3508 and § 58.1-3508.1. Currently, 13 localities report having a separate tax on equipment in the semiconductor industry; 47 report having a machinery and tools tax in the forest harvesting industry; 67 localities report so in the vehicle cleaning industry; while only 3 localities reports having it as a separate category in the castings industry. Meanwhile, 7 localities report having the tax for equipment in the defense industry, and 2 localities report having the category in other businesses.

\textbf{Table 10.3} presents the number of machinery and tool accounts each locality reported for 2019. Twenty-eight cities reported their number of accounts, as did 69 counties and 27 towns. When we asked the question, we assumed localities organized their accounts by business entity (i.e., each business had an account and within that account resided any number of tools). However, based on responses from some localities, this might not always be the case. Some localities responded that the machinery or tool item, not the business entity, was the basis of the account. Others informed us that their machinery and tools accounts included items we did not expect, such as company work trucks. Localities which reported having such systems tended to report a higher number of accounts.

\begin{Shaded}
\begin{Highlighting}[]
\CommentTok{\# Table 10.1 Machinery and Tools Property Tax, General Information, 2019}

\CommentTok{\# Table 10.2 Machinery and Tools Tax on Specific Types of Equipment, 2019}

\CommentTok{\# Table 10.3 Machinery and Tools Tax, Number of Accounts, 2019}
\end{Highlighting}
\end{Shaded}

\hypertarget{utility-license-tax}{%
\chapter{Utility License Tax}\label{utility-license-tax}}

In fiscal year 2018, the most recent year available from the Auditor of Public Accounts, the utility license tax accounted for 0.1 percent of the total tax revenue for cities, 0.1 percent for counties, and 0.6 percent for large towns. These percentages are based on the franchise license tax reported in Appendix C. The franchise license tax includes not only the license fees of electric and water utilities, which are discussed in this section, but also cable television utilities, discussed in Section 12. These are averages; the relative importance of this tax in individual cities, counties, and towns varies significantly. For information on individual localities, see Appendix C.

\ldots{} see \ref(\citet{secn:Appendix-C}).

Localities in Virginia may impose a local license tax on certain types of public service corporations. As authorized by § 58.1-3731 of the \emph{Code}, localities may levy a license tax on telephone and water companies not to exceed one-half of 1 percent of the gross receipts of such company accruing from sales to the ultimate consumer in the locality. For telephone companies, long-distance calls are not taxable under this provision. County utility license taxes do not apply within the limits of an incorporated town if the town also imposes the tax.

Prior to 2006, any locality that had in effect before January 1, 1972 a tax rate exceeding the statutory ceiling could continue to tax at the previous level but could not raise the rate (see \emph{Virginia, Acts of Assembly, 1972}, c.~858). This provision changed in 2006 under the Virginia Communication Sales and Use Tax when the General Assembly eliminated the business license tax in excess of 0.5 percent.

In the latest survey 130 localities responded that they had a utility license tax on telephone service and 36 had a tax on water service. The table below summarizes the numbers of positive respondents by type of service and locality.

\begin{table}

\caption{\label{tab:unnamed-chunk-1}Localities Reporting the Utility License Tax, 2019}
\centering
\begin{tabular}[t]{l|r|r|r|r}
\hline
Utility & Cities & Counties & Towns & Total\\
\hline
Telephone & 30 & 43 & 57 & 130\\
\hline
Water & 8 & 21 & 7 & 36\\
\hline
\end{tabular}
\end{table}

Nearly all localities reported charging the maximum 0.5 percent (1/2 of 1 percent) permitted by the law. None reported charging a greater amount. A few localities reported charging less for the telephone utility tax, including the counties of Fairfax (0.24 percent), New Kent (0.42 percent) and Prince William (0.29 percent), and the towns of Haymarket (0.1 percent), Pembroke (0.3 percent), and Urbanna (0.23 percent).

\label{tab:Table11-1}Utility License Tax Table, 2019

Locality

Telephone

Water

Accomack County

0.50

0.50

Alleghany County

0.50

0.50

Amelia County

0.05

--

Arlington County

0.50

0.50

Augusta County

0.05

--

Bedford County

0.50

--

Campbell County

0.50

--

Caroline County

0.50

0.50

Carroll County

0.50

--

Charles City County

0.50

0.50

Clarke County

0.50

--

Craig County

0.50

--

Fairfax County

0.24

--

Fauquier County

0.50

0.50

Fluvanna County

0.50

--

Franklin County

0.50

--

Frederick County

0.50

--

Gloucester County

0.50

0.50

Goochland County

0.05

0.05

Hanover County

0.50

0.50

Henrico County

0.50

0.50

Isle of Wight County

0.50

0.50

James City County

0.50

0.50

King \& Queen County

0.50

--

King George County

0.50

--

King William County

0.50

--

Lunenburg County

0.50

--

Mathews County

0.50

--

New Kent County

0.42

0.42

Page County

--

0.50

Pittsylvania County

0.50

0.50

Prince Edward County

0.50

--

Prince George County

0.50

0.50

Prince William County

0.29

--

Rappahannock County

0.50

--

Roanoke County

0.50

0.50

Rockingham County

0.50

--

Southampton County

0.50

0.50

Stafford County

0.50

--

Surry County

0.50

0.50

Warren County

0.50

--

Washington County

0.50

0.50

Wise County

0.50

--

York County

0.50

0.50

Alexandria City

0.50

0.50

Buena Vista City

0.50

--

Charlottesville City

0.50

--

Chesapeake City

0.50

--

Covington City

0.50

--

Emporia City

20.00

--

Fairfax City

0.50

--

Franklin City

0.50

--

Fredericksburg City

0.50

--

Galax City

0.50

--

Hampton City

0.50

0.50

Harrisonburg City

0.50

--

Hopewell City

0.50

0.50

Lexington City

0.50

--

Lynchburg City

0.50

--

Manassas Park City

0.05

--

Martinsville City

0.50

--

Newport News City

0.50

--

Norfolk City

0.50

--

Norton City

0.50

--

Poquoson City

0.50

0.50

Portsmouth City

0.05

--

Richmond City

0.50

0.50

Roanoke City

0.50

0.50

Salem City

0.05

0.05

Staunton City

0.05

--

Suffolk City

0.50

--

Virginia Beach City

0.50

0.50

Waynesboro City

0.50

--

Winchester City

0.50

--

Abingdon Town

0.50

--

Amherst Town

0.50

--

Appomattox Town

0.50

--

Ashland Town

0.50

--

Big Stone Gap Town

0.50

--

Blacksburg Town

0.50

--

Blackstone Town

0.50

--

Boydton Town

0.50

--

Bridgewater Town

0.50

--

Brookneal Town

0.05

--

Cape Charles Town

0.50

--

Charlotte Court House Town

0.05

--

Chase City Town

0.50

--

Clarksville Town

0.50

--

Clifton Forge Town

0.50

--

Clintwood Town

0.50

--

Courtland Town

0.50

--

Damascus Town

0.50

--

Dayton Town

0.50

--

Dillwyn Town

0.50

--

Farmville Town

0.50

--

Front Royal Town

0.50

--

Gate City Town

0.50

0.50

Gordonsville Town

0.50

0.50

Goshen Town

0.50

--

Gretna Town

0.50

--

Grottoes Town

0.50

0.50

Haymarket Town

0.10

--

Haysi Town

0.50

--

Hillsville Town

0.50

--

Honaker Town

0.05

--

Kilmarnock Town

0.50

--

Lebanon Town

0.50

--

Lovettsville Town

0.50

--

Luray Town

0.50

--

Marion Town

0.50

--

Middleburg Town

0.50

0.50

New Market Town

0.05

--

Nickelsville Town

0.50

--

Orange Town

0.50

--

Purcellville Town

0.50

--

Rocky Mount Town

0.50

--

Round Hill Town

0.09

--

Rural Retreat Town

0.05

--

Saint Paul Town

0.05

--

Scottsville Town

0.50

--

Shenandoah Town

0.50

--

South Boston Town

0.50

--

Strasburg Town

0.50

--

Tappahannock Town

0.50

--

Urbanna Town

0.23

--

Vienna Town

0.50

--

Vinton Town

0.50

0.50

Warsaw Town

0.05

--

Windsor Town

0.50

0.50

Wise Town

0.50

0.50

Wytheville Town

0.50

--

\hypertarget{cable-television-system-franchise-tax}{%
\chapter{Cable Television System Franchise Tax}\label{cable-television-system-franchise-tax}}

On January 2007 the Virginia Communications Sales and Use Tax Act eliminated several local taxes, including the cable television system franchise tax (§ 15.2.2108), the local E-911 fees on land-line phone service, the business license taxes in excess of 0.5 percent gross revenues collected by several localities, the local consumer utility taxes on land line and wireless phones, the video programming excise tax (§ 58.1.3818.1), and the local consumer utility tax on cable television service which had been ``grandfathered'' in a few localities. These local taxes were replaced by a new state tax of 5 percent of the sales price of the service, which is collected by the Virginia Department of Taxation and remitted to localities as non-categorical state aid based on a percentage developed by the Auditor of Public Accounts in its report, \emph{Report of State and Local Communication Service Taxes and Fees: Report on Audit for the Year Ended June 30, 2006}, and available on the web at \url{http://www.apa.virginia.gov/APA_Reports/Reports.aspx}. Refer to Section 19, ``Miscellaneous Taxes,'' for more on the communications sales and use tax.

The cable television system franchise tax still exists in those localities with current contracts with cable operators. When those contracts expire, the localities will revert to the requirements of the state tax.

\textbf{Table 12.1} presents the localities with franchise fee contracts that extend to the end of 2019 and beyond. It includes the current franchise fee charged by the locality, whether the locality has multiple cable providers, and whether the locality authorizes a BPOL tax on the cable franchisee. Seven cities reported having contract clauses that extended to 2019 or beyond, as did 6 counties and 9 towns. The median of the fees for all localities was 5 percent. Thirty-four localities indicated that they had multiple cable providers.

\begin{Shaded}
\begin{Highlighting}[]
\CommentTok{\# Table 12.1 Cable Television System Tax, 2019}
\end{Highlighting}
\end{Shaded}

\hypertarget{consumers-utility-tax}{%
\chapter{Consumers' Utility Tax}\label{consumers-utility-tax}}

In fiscal year 2018, the most recent year available from the Auditor of Public Accounts, the consumers' utility tax accounted for 2.9 percent of the tax revenue collected by cities, 1.3 percent by counties and 3.6 percent by large towns. These are averages; the relative importance of the tax in individual cities, counties, and towns varies significantly. For information on individual localities, see Appendix C.

The \emph{Code of Virginia}, § 58.1-3814, authorizes localities to impose a tax on the consumers of public utilities. (This tax should not be confused with the utility license tax, a tax levied on utility providers, which is discussed in Section 11.) Residential customers of gas, water, and electric services are not to be taxed at a rate higher than 20 percent of the first \$15 of the monthly bill. Any locality that had in effect before July 1, 1972, a tax rate exceeding the current statutory ceiling may continue to tax at the previous level. There is no statutory ceiling on the tax on commercial or industrial consumers, and localities generally levy higher rates on those entities.

Counties are restricted in their authority to levy a consumers' utility tax within the limits of an incorporated town if the town itself also levies such a tax, provided the town maintains certain services. If localities wish to change rates for local consumer utility taxes, they must give 120 days notice to providers for such a rate change.

In 2001, the General Assembly repealed the utility license tax on providers of gas (any type used in residences, but not if sold in portable containers) and electric power and rearranged the rate structure of the consumers' utility tax for electricity and natural gas consumption (see § 58.1-3814). The taxes are now per kilowatt hour (kwh) of electricity used by consumers and per hundred cubic feet (ccf) of gas delivered monthly to consumers. The tax schedules and services of the provider are explained in § 58.1-2901 for electricity and § 58.1-2905 for natural gas. The maximum amount of tax that can be imposed on residential consumers as a result of either conversion is limited to \$3.00 per month, except where a higher limit already existed. According to § 58.1-3816.2 churches and religious bodies may be exempted from any or all the consumer utility taxes at the discretion of the locality.

In January 2007 the Virginia communications sales and use tax was implemented and several local taxes were eliminated, including the cable television system franchise tax, the local E-911 fees on land line phone service, the business license taxes in excess of 0.5 percent gross revenues collected by several localities, the local consumer utility taxes on land line and wireless phones, and the local consumer utility tax on cable television service except where it was ``grandfathered'' in a few localities. These local taxes were replaced by a new \emph{state} tax of 5 percent of the bill for the service, which is collected by the Virginia Department of Taxation and remitted to localities as non-categorical aid based on a percentage developed by the Auditor of Public Accounts from its report, \emph{Report of State and Local Communication Service Taxes and Fees: Report on Audit for the Year Ended June 30, 2006}, and available on the web at \url{http://www.apa.virginia.gov/APA_Reports/Reports.aspx}. Refer to Section 19, ``Miscellaneous Taxes,'' for more on the communications sales and use tax.

\textbf{Table 13.1} shows the monthly tax on electricity for residential, commercial, and industrial users. Thirty-six cities, 86 counties, and 85 towns reported having a tax on electricity in 2019. The format of charges in terms of kilowatt hours reflects the changes made in the 2001 law though some localities still use the older tax terminology. Consequently, a locality's rate might be described in terms of dollars per kilowatt hour (e.g., \$0.005/kwh) plus some minimum price or it might be described in the older manner (e.g., 10 percent on the first \$30 of the tax bill).

The consumers' tax on gas is listed in \textbf{Table 13.2}. As with the tax on electricity, the tax on gas has been changed to reflect the elimination of the utility license tax on gas companies and the subsequent incorporation of that tax into the consumers' utility tax. The usual format for the tax is now a given minimum, with a given tax per additional ccf (hundred cubic feet) of gas used by the consumer, up to a certain maximum amount charged. In 2019, 32 cities, 51 counties, and 44 towns reported imposing the tax on residential, commercial and industrial users.

Finally, \textbf{Table 13.3} lists localities with a monthly tax on water. Sixteen cities, 2 counties, and 3 towns reported having the tax. The water tax imposes a certain percentage tax on the first given dollar amount of usage, such as 10 percent on the first \$15 of usage.

The following text table summarizes the number of localities reporting these taxes.

\begin{Shaded}
\begin{Highlighting}[]
\CommentTok{\# table name: Consumers\textquotesingle{} Utility Tax in Localities, 2019}
\end{Highlighting}
\end{Shaded}

\begin{Shaded}
\begin{Highlighting}[]
\CommentTok{\# Table 13.1 Utility Consumers\textquotesingle{} Monthly Tax on Electricity, 2019}

\CommentTok{\# Table 13.2 Utility Consumers\textquotesingle{} Monthly Tax on Gas, 2019}

\CommentTok{\# Table 13.3 Utility Consumers\textquotesingle{} Monthly Tax on Water, 2019}
\end{Highlighting}
\end{Shaded}

\hypertarget{business-professional-and-occupational-license-tax}{%
\chapter{Business, Professional, and Occupational License Tax}\label{business-professional-and-occupational-license-tax}}

In fiscal year 2018, the most recent year available from the Auditor of Public Accounts, business license taxes, of which the business, professional, and occupational license tax (commonly referred to as the BPOL tax) makes up the largest part, accounted for 6.0 percent of tax revenue for cities, 3.4 percent for counties, and 11.9 percent for large towns. These are averages; the relative importance of the tax varies for individual cities, counties and towns. In fact, only slightly over half of the counties employ the tax. Others use the merchants' capital tax instead. Four counties (Amherst, Hanover, Louisa, and Southampton) reported using both taxes, maintaining the merchants' capital tax for retailers and the BPOL tax for other types of businesses. For information on individual localities, see Appendix C.

Localities are authorized to impose a local license tax on businesses, professions, and occupations operating within their jurisdictions unless they already levy a tax on merchants' capital.The BPOL tax is sanctioned by §§ 58.1-3700 through 58.1-3735 of the \emph{Code of Virginia}. The \emph{Code} establishes the dates between March 1 and May 1 as the time by which businesses must apply for a license. County BPOL taxes do not apply within the limits of an incorporated town unless the town grants the county authority to do so (§ 58.1-3711). Localities may charge a fee to each business for the issuance of a license. Each business classification such as retail or contracting, has a specific tax rate which cannot exceed maximums set by the state guidelines. Businesses pay the tax rate for the amount of gross receipts within each classification.

Although revised guidelines in January 1997 made administration of the BPOL tax more uniform in terms of due dates, penalties, interest, appeals, and definitions of situs, localities retained some flexibility. In 2000, the 1997 guidelines were updated. They are viewable on the internet site, \url{http://townhall.virginia.gov/L/GetFile.cfm?File=C:/TownHall/docroot/GuidanceDocs/161/GDoc_TAX_2537_v1.pdf}.

In 2011 the General Assembly passed a law allowing localities the option of imposing the tax on either gross receipts or the Virginia taxable income of the business. This option did not apply to certain public service corporations required to pay the 1/2 of 1 percent utility tax, which is considered a form of BPOL (see Section 11). The legislature also permitted relief from the BPOL tax, allowing localities to exempt new business from the tax for up to two years and second. allowing localities to exempt unprofitable businesses from the tax.

Localities may still determine how many separate licenses they issue to a business and whether to charge a fee for each business location or only one fee for a business with multiple locations. Some localities charge no fee or charge different fees depending on a firm's gross receipts. Some localities charge a minimum tax instead of a fee. For example, if a locality had a minimum license tax of \$30 then businesses with gross receipts below the threshold would pay \$30 instead of a smaller amount based on gross receipts. In addition, there are some localities that impose \emph{both} a license fee and a tax rate on businesses with gross receipts above the threshold.

The BPOL tax is collected by all cities and 51 of the 95 counties. The tax is also widely used by incorporated towns; 105 towns reported using the BPOL tax. The specific localities that impose the tax are listed in \textbf{Table 14.1} along with information regarding due dates, license fees, and thresholds.

For most localities, the filing and payment dates are March 1st, though there is quite a bit of variance from that date. Of the cities, 18 reported requiring a license fee, either by business or by location. Twenty-eight counties and 57 towns also reported requiring license fees of some sort. Finally, 20 cities, 33 counties, and 17 towns reported having a tax threshold requirement based on gross receipts.

\textbf{Table 14.2} lists the fees, minimum tax, and an explanation of the fee structures provided by the localities in the survey. Thirty-two cities reported having either a fee or a minimum tax, as did 41 counties and 98 towns.

\textbf{Table 14.3} shows specific tax rates by business classification for each locality. All 38 cities, 45 counties, and 98 towns reported having a tax on at least one business classification. An overview of the general practices of Virginia localities is shown in the text table below. Combining data from tables 14.2 and 14.3, it lists the median license fee and median gross receipts tax rate for cities, counties, and towns. If a locality reported different fees due to differences in total gross receipts, the median figures were calculated using the highest fee amount given because that provides an estimate of the greatest impact on the payer.

Only the localities that reported a fee or a tax rate in a particular category were included in the calculation of the medians in the following text table.

\begin{Shaded}
\begin{Highlighting}[]
\CommentTok{\# table name: BPOL License Fee and Tax Rate Per $100 in 2019}
\end{Highlighting}
\end{Shaded}

The median tax rates for the cities matched the maximum rates permitted by the state---\$0.16 per \$100 of gross receipts for contracting; \$0.20 for retail; \$0.36 for repair, personal, and business services; and \$0.58 for financial, real estate, and professional services. The median figures for counties and towns were less than those of the cities, indicating that counties and towns did not generally apply the maximum rates permitted by Virginia law.

The median rate of \$0.11 on wholesalers for cities was well above the state maximum of \$0.05 per \$100 of purchases. Cities are presumed to operate under grandfather clauses that allow them to impose higher rates. The median rate on wholesalers for counties and towns was \$0.05 per \$100.

The median license fee, which is generally imposed only upon businesses below the gross receipts tax threshold, was \$50 for the cities, \$40 for counties, and \$30 for towns.

One business classification not presented in Table 14.3 is that of rental property due to the small number of localities reporting it. Localities are permitted to charge a license fee, or levy a BPOL tax, on businesses renting real property. In 2019, only 24 localities reported taxing such businesses. They were the cities of Alexandria, Bristol, Fairfax, Falls Church, Fredericksburg, and Portsmouth; the counties of Albemarle, Arlington, Augusta, Fairfax, King George, Loudoun, Nelson, Pulaski, and Wythe; and the towns of Bridgewater, Chatham, Goshen, Haymarket, Narrows, Purcellville, Round Hill, Saint Paul, and Vienna.

\textbf{Table 14.4} lists taxes and fees on peddlers and itinerant merchants. All of the cities, 50 counties, and 93 towns reported some form of tax on peddlers. Annual fees charged by cities for retail peddling ranged anywhere from \$30 to \$500. Taxes on retail itinerant merchants and wholesale peddlers also ranged from \$30 to \$500, with some cities charging according to gross receipts and other cities according to gross purchases. Annual charges by counties ranged from a \$1 minimum fee to \$500, while towns charged anywhere from \$10 to \$500 per year.

\begin{Shaded}
\begin{Highlighting}[]
\CommentTok{\# Table 14.1 BPOL Due Dates and Other Provisions, 2019}

\CommentTok{\# Table 14.2 Specific BPOL Fees and Minimum Taxes, 2019}

\CommentTok{\# Table 14.3 Specific BPOL Tax Rates per $100 by Business Category, 2019}

\CommentTok{\# Table 14.4 Taxes and Fees on Peddlers and Itinerant Merchants, 2019}
\end{Highlighting}
\end{Shaded}

\hypertarget{motor-vehicle-local-license-tax-2019}{%
\chapter{Motor Vehicle Local License Tax 2019}\label{motor-vehicle-local-license-tax-2019}}

In fiscal year 2018, the most recent year available from the Auditor of Public Accounts, the motor vehicle local license tax, popularly known as the local decal tax, even though many of the localities imposing the tax no longer use a decal as evidence of payment, accounted for 1.1 percent of the total tax revenue for cities, 1.1 percent for counties and 2.0 percent for large towns. These are averages; the relative importance of this tax in individual cities, counties and large towns varies significantly. For information on individual localities see Appendix C.
\textbar{}
\textbar{} Section 46.2-752 of the \emph{Code of Virginia} authorizes cities, counties, and towns to levy a license tax on motor vehicles, trailers, and semitrailers. The amount of the tax may not be greater than the tax imposed by the state. Currently, the base registration fees for non-commercial passenger vehicles are \$33 for vehicles under 4,000 pounds and \$38 for heavier vehicles (§ 46.2-694.2). Motorcycle fees are \$18 with a \$3 surcharge included {[}§ 46.2-694 (A) (10){]}. The \emph{Code} stipulates similar guidelines for commercial vehicles, buses, trailers, and other motor vehicles. The \emph{Code} also provides for additional fees for specified government services, such as \$6.25 for emergency medical service (EMS) programs {[}\emph{Code of Virginia} § 46.2-694 (A) (13) and \emph{2014 Appropriations Act} § 3-6.02{]} to be paid to the state treasury and provides for a \$1.50 addition for the official motor vehicle safety inspection program to be paid at registration (§ 46.2-1168).
\textbar{}
\textbar{} No locality may impose a license tax on any vehicle when the owner pays a similar tax to the locality in which the vehicle is normally stored. Furthermore, no locality may impose a local license tax on any vehicle that is owned by a nonresident of such locality and is used exclusively for pleasure or personal transportation (i.e., for non-business uses). For example, the tax would not apply to a personal vehicle owned by a nonresident college student and used only for pleasure or personal transportation. Vehicles used for state business by nonresident officials, dealer demonstration vehicles and the vehicles of common carriers are also exempt from local license taxes.
\textbar{}
\textbar{} The situs for the assessment of motor vehicles is clarified in § 58.1.3511. Business vehicles with a weight of 10,000 pounds or less are considered to be in the jurisdiction in which the owner of the business: (1) is required to file a tangible personal property tax return for any vehicle used in the business, and (2) has a definite place of business from which the use of the business vehicle is directed or controlled.
\textbar{}
\textbar{} If a town within a county levies a motor vehicle license tax, the county must credit the owner with the tax paid to the town. Also, if the town tax is equal to the maximum allowed by law, then the county may not impose any further tax. Likewise, no county license tax may be imposed on vehicles that are subject to license taxes imposed by a town constituting a separate school division (§46.2-752)\footnote{Commission on Local Government, Report on Proffered Cash Payments and Expenditures by Virginia's Counties, Cities and Towns, 2017-2018. \url{https://www.dhcd.virginia.gov/cash-proffers}.}.
\textbar{}
\textbar{} \textbf{Table 15.1} presents the local motor vehicle license taxes on automobiles, motorcycles, and trucks. Column one indicates the date that the fee must be paid or a decal, if applicable, must be affixed to a motor vehicle to denote payment of license fees. Thirty-two cities and 83 counties reported imposing the tax. Of the reporting towns, 103 said they levied the tax. The second column gives the tax rate on private passenger vehicles. Most localities levy a fl at tax between \$15 and \$30 for passenger vehicles under 4,000 pounds. The table also shows the exemption status for elderly or disabled persons. Seven localities offer tax relief for the elderly, while 30 exempt the disabled from this tax. The final two columns give the tax rates on motorcycles and trucks. The tax ranges from \$3 to \$35 for motorcycles and from \$3 up to \$250 (depending on weight) for trucks.
\textbar{}
\textbar{} The following text table summarizes the range of tax charged for private passenger vehicles under 4,000 pounds.

\begin{Shaded}
\begin{Highlighting}[]
\CommentTok{\#Text table "License Tax for Private Passenger Vehicles Under 4,000 Pounds, 2019" goes here}
\end{Highlighting}
\end{Shaded}

\hfill\break
~~Cities had a median license tax of \$27.00; the median tax for both counties and towns was \$25. For cities the mean license tax for private passenger vehicles was \$28.09. The first quartile measure was \$25 while the third quartile was \$32.25. For counties, the mean was \$26.67. The first and third quartiles were \$23.00 and \$30.00, respectively. For towns, the mean was \$23.18. The first and third quartiles were \$20 and \$25 respectively.\\
~\\
\hspace*{0.333em}\hspace*{0.333em}\textbf{Table 15.2} lists whether localities require the display of decals and whether localities permit special exemptions from paying the motor vehicle license tax other than those for the elderly and disabled. Twenty-six cities, 78 counties, and 62 towns reported granting payment exemptions. The most popular category for exemption was for local fire and rescue department members.\\
~\\
\hspace*{0.333em}\hspace*{0.333em}In recent years, many localities have dispensed with the decal because new technology has allowed them to track payments without the use of the decal. Most now collect the motor vehicle license tax along with the personal property tax on motor vehicles. So far, 30 cities, 83 counties, and 81 towns reported they no longer required decal placement on automobile windshields.\\

\begin{Shaded}
\begin{Highlighting}[]
\CommentTok{\#Table 15.1 "Motor Vehicle Local License Tax, 2019" goes here}

\CommentTok{\#Table 15.2 "Motor Vehicle Local License Tax Decal Display Policy and Exemptions, 2019" goes here}
\end{Highlighting}
\end{Shaded}

\hypertarget{meals-transient-occupancy-cigarettes-tobacco-and-admissions-excise-taxes}{%
\chapter{Meals, Transient Occupancy, Cigarettes, Tobacco, and Admissions Excise Taxes}\label{meals-transient-occupancy-cigarettes-tobacco-and-admissions-excise-taxes}}

Among the many local taxes levied by Virginia's localities are four excise taxes on meals, transient occupancy, cigarettes and admissions. \textbf{Table 16.1} provides a detailed list of rates for these taxes for the 38 cities, 82 counties, and 108 towns reporting at least one of these taxes.
\textbar{}
\#\# MEALS TAX
\textbar{} The meals tax is a flat percentage imposed on the price of a meal. In fiscal year 2018, the most recent year available from the Auditor of Public Accounts, the tax accounted for 7.5 percent of the total tax revenue for cities, 1.2 percent for counties, and 23.5 percent for large towns. The low percentage for counties is explained by the fact that slightly less than one-half of the counties employ the tax and those that have it cannot exceed a rate of 4 percent, whereas cities and towns are allowed to impose a higher tax rate. The authority to levy this tax varies greatly among jurisdictions, so the tax varies significantly among individual cities, counties, and towns. For information on tax receipts of individual localities, see Appendix C.
\textbar{}
\textbar{} Counties are restricted in their authority to levy the meals tax within the limits of an incorporated town unless the town grants the county authority to do so (§ 58.1-3711). Cities and towns are granted the authority to levy the tax under the ``general taxing powers'' found in their charters (§ 58.1-3840).
\textbar{}
\textbar{} Counties may levy a meals tax on food and beverages offered for human consumption if the tax is approved in a voter referendum (§ 58.1-3833). However, several counties have been exempted from the voter referendum requirement {[}see § 58.1-3833 (B) of the Code of Virginia{]}. Cities and towns do not need to have a referendum when deciding to impose the tax.
\textbar{}
\textbar{} There are certain restrictions in applying the meals tax. The tax cannot be imposed on food that meets the definition of food under the Federal Food Stamp Program, with the exception of sandwiches, salad bar items, certain prepackaged salads, and non-factory sealed beverages. It does not apply to certain volunteer and non-profit organizations that might sell food on an occasional basis nor does it apply to churches and their members. Also, the meals tax cannot exceed 4 percent in counties. Cities and towns may exceed that rate. Accordingly, 34 cities and 78 towns report charging a meals tax over 4 percent. In addition, the meals tax does not apply to gratuities, whether or not a restaurant makes them mandatory.
\textbar{}
\textbar{} The first column of \textbf{Table 16.1} lists the rates for the meals tax. All cities impose a meals tax. The median tax rate is 6 percent. The minimum rate, charged by four cities, is 4 percent, and the maximum, charged by Covington is 8 percent. The median meals tax rate is lower among the 50 counties that report having it. All counties that report having the meal tax have a rate of 4 percent. Among the 105 towns that report having a meals tax, the minimum rate is 2 percent, the maximum 8 percent, and the median rate is 5 percent.
\textbar{}
\textbar{} The text table summarizes the dispersion of the meal tax rates among cities, counties, and towns.

\begin{Shaded}
\begin{Highlighting}[]
\CommentTok{\#Text table "Meals Tax Rates, 2019" goes here}
\end{Highlighting}
\end{Shaded}

\hfill\break
~~The local meals tax is in addition to the state 4.3 percent sales tax (5 percent in localities constituting transportation districts in northern Virginia and Hampton Roads) and the 1 percent local option sales tax (see § 58.1-603). This means that the combined state and local tax rate on restaurant meals could be anywhere in the range of 7 to 14 percent for cities, counties, and towns that impose this tax. Such rates apply to all restaurant meals whether consumed at elegant dining establishments or fast food providers.\\

\hypertarget{transient-occupancy-tax}{%
\section{TRANSIENT OCCUPANCY TAX}\label{transient-occupancy-tax}}

~~The transient occupancy tax (lodging tax) is a flat percentage imposed on the charge for the occupancy of any room or space in hotels, motels, boarding houses, travel camp-grounds, and other facilities providing lodging for less than thirty days. The tax applies to rooms intended or suitable for dwelling and sleeping. Therefore, the tax does not apply to rooms used for alternative purposes, such as banquet rooms and meeting rooms.\\
~\\
\hspace*{0.333em}\hspace*{0.333em}In fiscal year 2018, the occupancy tax accounted for 2.2 percent of the total tax revenue for cities, 0.9 percent for counties, and 5.6 percent for large towns. These are averages; the relative importance of the tax varies significantly among individual cities, counties, and towns. For information on tax receipts of individual localities, see Appendix C.\\
~\\
\hspace*{0.333em}\hspace*{0.333em}According to § 58.1-3819, counties may levy a transient occupancy tax with a maximum tax rate of 2 percent. Counties specified in § 58.1-3819(A) may increase their transient occupancy tax to a maximum of 5 percent. The portion of the tax collections exceeding 2 percent must be used by the county for tourism and tourism related expenses. According to § 58.1-3819, the following counties are permitted to levy the 5 percent rate: Accomack, Albemarle, Alleghany, Amherst, Arlington, Augusta, Bedford, Bland, Botetourt, Brunswick, Campbell, Caroline, Carroll, Craig, Cumberland, Dickenson, Dinwiddie, Floyd, Franklin, Frederick, Giles, Gloucester, Goochland, Grayson, Greene, Greensville, Halifax, Highland, Isle of Wight, James City, King George, Loudoun, Madison, Mecklenburg, Montgomery, Nelson, Northampton, Page, Patrick, Powhatan, Prince Edward, Prince George, Prince William, Pulaski, Rockbridge, Rockingham, Russell, Smyth, Spotsylvania, Stafford, Tazewell, Warren, Washington, Wise, Wythe, and York.\\
~\\
\hspace*{0.333em}\hspace*{0.333em}Certain counties are permitted to levy higher rates. Roanoke County was given permission to levy a rate of 7 percent in 2012, with a portion of the revenue going to tourism advertisement. James City and York counties have 5 percent rates but are also allowed to charge an additional \$2 per room per night. The proceeds of these additional taxes go to tourism advertising (§ 58.1-3823(C)). Certain cities and towns also charge specific dollar amounts in addition to the percent rates; they are the cities of Alexandria, Lynchburg, Newport News, and Norfolk and the town of Dumfries. It is assumed, but not verified, that these policies are permitted by the localities' charters.\\
~\\
\hspace*{0.333em}\hspace*{0.333em}In 2018 the General Assembly authorized the replacement of a regional transient occupancy tax in the Northern Virginia Transportation District with a 2 percent transient occupancy tax to fund transportation in that area. This tax includes the counties of Arlington, Fairfax, and Loudoun, and the cities of Alexandria, Fairfax, and Falls Church. In addition, the assembly funded a 2 percent local transportation transient occupancy tax for the localities of Prince William County and Manassas City and Manassas Park City.\\
~\\
\hspace*{0.333em}\hspace*{0.333em}Counties are restricted in their authority to levy the lodging tax within the limits of an incorporated town unless the town grants the county authority to do so (§ 58.1-3711). Cities and towns are granted the authority to levy the lodging taxes under the ``general taxing powers'' found in their charters (§ 58.1-3840).\\
~\\
\hspace*{0.333em}\hspace*{0.333em}The median rate for the 37 cities that report using the transient occupancy tax is 8 percent, the minimum 2 percent, and the maximum is 11 percent (Emporia). Seventy-nine counties report imposing a transient occupancy tax. The extremes range from 2 to 8 percent with a median rate of 5 percent. The 77 towns that report having the tax show a median of 5 percent with a minimum rate of 1 percent and a maximum of 9 percent. The following text table summarizes the dispersion of the transient occupancy tax among cities, counties, and towns:

\begin{Shaded}
\begin{Highlighting}[]
\CommentTok{\#Text table "Transient Occupancy Taxes, 2019" goes here}
\end{Highlighting}
\end{Shaded}

\hfill\break
~~The local transient occupancy tax is in addition to the state 4.3 percent sales tax (5 percent in localities constituting transportation districts in Northern Virginia and Hampton Roads) and the 1 percent local option sales tax. This means that the combined state and local tax rate for hotel-motel stays can be very high. In a special district of Virginia Beach the combined rate is 16.5 percent (10.5 percent transient occupancy tax, 1 percent local option sales and use tax, and 5 percent state sales and use tax applicable for localities in Hampton Roads).\\

\hypertarget{cigarette-and-tobacco-taxes}{%
\section{CIGARETTE AND TOBACCO TAXES}\label{cigarette-and-tobacco-taxes}}

~~In fiscal year 2018, cigarette and tobacco taxes accounted for 0.9 percent of the total tax revenue collected by cities, 0.1 percent of that collected by counties, and 2.1 percent of that collected by large towns. The very low percentage for counties is attributable to the fact that few counties levy cigarette and tobacco taxes. These are averages; the relative importance of the tax varies significantly among individual cities and towns. For information on individual localities, see Appendix C.\\
~\\
\hspace*{0.333em}\hspace*{0.333em}The state is authorized by § 58.1-1001 of the Code to impose an excise tax of 1.5 cents on each cigarette sold or stored (30 cents on a pack of 20). Section 58.1-3830 allows for the local taxation of the sale or use of cigarettes. Cities and towns are granted the authority to levy the tax under the ``general taxing powers'' found in their charters (§ 58.1-3840). The right to levy the tax has been granted to only two counties by general law. Fairfax and Arlington counties may levy the cigarette tax with a maximum rate of 5 cents per pack or the amount levied under state law, whichever is greater (§ 58.1-3831). The two counties have followed the state's example and raised their taxes to 30 cents for a pack of 20. No county cigarette tax is applicable within town limits if the town's governing body does not authorize that county to levy the tax. This restriction applies to towns in Fairfax County, including Herndon, Vienna, and Occoquan.\\
~\\
\hspace*{0.333em}\hspace*{0.333em}Unlike the meals and transient occupancy taxes, which are added directly to the bill at the time of purchase, the cigarette tax is added onto the price per pack before the purchaser buys the cigarettes. The tobacco tax is levied either as a flat tax or as a portion of gross receipts. If no schedule is given in \textbf{Table 16.1}, then it should be read as a flat tax. A total of 31 cities levy some sort of tax on cigarettes, while 2 counties and 66 towns report doing so. The following text table, based on the tax of a pack of 20 cigarettes, summarizes the dispersion of cigarette taxes among cities, counties and towns.

\begin{Shaded}
\begin{Highlighting}[]
\CommentTok{\#Text table "Cigarette Tax on a Pack of 20 in 2019" goes here}
\end{Highlighting}
\end{Shaded}

\hfill\break
~~The cigarette tax is in addition to the state 4.3 percent sales tax (5 percent in localities constituting transportation districts in Northern Virginia and Hampton Roads) and the 1 percent local option sales tax.\\

\hypertarget{admissions-tax}{%
\section{ADMISSIONS TAX}\label{admissions-tax}}

~~In fiscal year 2018, the admissions tax accounted for 0.4 percent of the total tax revenue for cities. Receipts were negligible for counties and large towns. These are averages; the relative importance of the tax varies significantly among individual localities. For information on receipts by individual localities, see Appendix C.\\
~\\
\hspace*{0.333em}\hspace*{0.333em}Events to which admissions are charged are classified into five groups by § 58.1-3817 of the \emph{Code of Virginia}; they are: (1) those events from which the gross receipts go entirely to charitable purposes; (2) admissions charged for events sponsored by public and private educational institutions; (3) admissions charged for entry into museums, botanical or similar gardens and zoos; (4) admissions charged for sporting events; and (5) all other admissions.\\
~\\
\hspace*{0.333em}\hspace*{0.333em}In imposing the admissions tax, localities have the authority to tax each class of admissions with the same or with a different tax rate. A locality may impose admission taxes at lower rates for events held in privately-owned facilities than for events held in facilities owned by the locality. Section 58.1-3818 allows a locality to exempt certain qualified charitable events from admissions tax charges. Fifteen counties (Arlington, Brunswick, Charlotte, Clarke, Culpeper, Dinwiddie, Fairfax, Madison, Nelson, New Kent, Prince George, Scott, Stafford, Sussex, and Washington) have been granted permission to levy an admissions tax at a rate not to exceed 10 percent of the amount of charge for admissions (§§ 58.1-3818 and 58.1-3840). Only three counties, Dinwiddie, Roanoke, and Washington, report levying the tax.\\
~\\
\hspace*{0.333em}\hspace*{0.333em}Cities and towns are granted the authority to levy the admissions tax under the ``general taxing powers'' found in their charters (§ 58.1-3840). As shown in the text table, 18 cities and 3 towns (Cape Charles, Culpeper, and Vinton) reported levying the admissions tax. For cities, the levy ranged from 5 percent to the full 10 percent. The median rate was 7 percent.

\begin{Shaded}
\begin{Highlighting}[]
\CommentTok{\#Text table "Admissions Tax, 2019" goes here}

\CommentTok{\#Table 16.1 "Meals, Transient Occupancy, Cigarette, and Admissions Excise Taxes, 2019" goes here}
\end{Highlighting}
\end{Shaded}

\hypertarget{taxes-on-natural-resources}{%
\chapter{Taxes on Natural Resources}\label{taxes-on-natural-resources}}

~~Taxes on natural resources are rarely used by localities because many are not endowed with such resources. As a consequence, natural resources taxes accounted for less than 0.1 percent of total city tax revenue in fiscal year 2018, 0.2 percent of total county tax revenue, and less than 0.1 percent of total tax revenue of large towns, according to information from the Auditor of Public Accounts. These are averages; the vast majority of localities receive no revenue from this source. All the exceptions are located in Southwest Virginia. For information on individual localities, see Appendix C.\\
~\\
\hspace*{0.333em}\hspace*{0.333em}Localities are permitted to impose several taxes on natural resources. \textbf{Table 17.1} provides tax rates for the cities and counties having such natural resource-related taxes in effect during the 2019 tax year.\\

\hypertarget{taxation-of-mineral-lands}{%
\section{TAXATION OF MINERAL LANDS}\label{taxation-of-mineral-lands}}

~~Under § 58.1-3286 of the \emph{Code of Virginia}, localities are required to ``\ldots specially and separately assess at the fair market value all mineral lands and the improvements thereon\ldots{}'' and enter those assessments separately from assessments of other lands and improvements. Mineral lands are taxed at the same rate as other real estate in the locality. Localities may request technical assistance from the Virginia Department of Taxation in assessing mineral lands and minerals, provided money is available to the department to defray the cost of the assistance (§ 58.1-3287). Instead of employing the real property tax for mineral lands, localities are permitted to substitute a severance tax on mineral sales, not to exceed 1 percent.\\
~\\
\hspace*{0.333em}\hspace*{0.333em}In 2009, this section was amended to allow Buchanan County to reassess mineral lands on an annual basis for purposes of determining the real property tax on such land. Other real estate is still subject to assessment every six years. Currently, 2 cities and 23 counties report assessing taxes on minerals. Among the several that commented on their mineral tax, most stated they used the land assessment method. The city of Norton, however, stated that its tax was based on a loading tax of \$0.05/ton.\\

\hypertarget{severance-tax}{%
\section{SEVERANCE TAX}\label{severance-tax}}

~~Under § 58.1-3712, any city or county may levy a license tax on businesses engaged in severing coal and gases from the earth. The maximum rate permitted is 1 percent of the gross receipts from sales. A 2012 bill reduced the rates of the local coal severance tax for small mines from 1 percent to 0.75 percent of the gross receipts from the sale of coal. ``Small mine'' is defined here as a mine that sells less than 10,000 tons of coal per month.\\
~\\
\hspace*{0.333em}\hspace*{0.333em}Localities choosing to use § 58.1-3712 may not exercise the option to levy a 1 percent severance tax under § 58.1-3286. Under § 58.1-3712.1, the maximum rate permitted for severing oil is one-half of 1 percent from the sale of the extracted oil. Notwithstanding the rate limits established in § 58.1-3712, cities or counties may impose an additional license tax of 1 percent of the gross receipts from the sale of gas severed as authorized by § 58.1-3713.4. The funds from this additional levy are paid into the general fund of the localities except for members of the Virginia Coalfield Economic Development Fund, where one-half of the revenues must be paid to the fund. The members of the fund are the counties of Buchanan, Dickenson, Lee, Russell, Scott, Tazewell, and Wise and the city of Norton.\\

\hypertarget{coal-and-gas-road-improvement-tax}{%
\section{COAL AND GAS ROAD IMPROVEMENT TAX}\label{coal-and-gas-road-improvement-tax}}

~~Notwithstanding the rate limits described in the previous paragraph, localities are permitted by § 58.1-3713 to levy up to an additional 1 percent license tax on the gross receipts of coal and gas extracted from the ground. As with the severance tax on coal, the coal road improvement tax has been modified to reduce the tax from 1 percent to 0.75 percent for small mines. This tax was originally scheduled to end in 2007, but the General Assembly extended the sunset clause a number of times, most recently to December 31, 2017.\\
~\\
\hspace*{0.333em}\hspace*{0.333em}The amount collected under this tax must be paid into a special fund to be called the Coal and Gas Road Improvement Fund of the particular county or city where the tax is collected. In addition, ``the county may also, in its discretion, elect to improve city or town roads with its funds if consent of the city or town council is obtained.'' One-half of the revenue paid to this fund may be used for the purpose of funding the construction of new water systems and lines in areas of insufficient natural supply of water. Those same funds may also be used to improve existing water and sewer systems. Localities are required to develop and ratify an annual funding plan for such projects. Under § 58.1-3713.1, 20 percent of the funds collected in Wise County are distributed to the six incorporated towns within the county's boundaries (Appalachia, Big Stone Gap, Coeburn, Pound, Saint Paul, and Wise) and the city of Norton. The distribution is determined as follows: (a) 25 percent is divided among the incorporated towns and the city of Norton based on the number of registered motor vehicles in each town and the city of Norton, and (b) 75 percent is divided equally among the towns and the city of Norton. The Coal and Gas Road Improvement Advisory Committee in the city of Norton and county must develop a plan before July 1 of each year for road improvements for the following fiscal year. For localities in the Virginia Coalfield Economic Development Authority (Lee, Wise, Scott, Buchanan, Russell, Tazewell, and Dickenson counties and the city of Norton), the receipts from this tax are distributed as follows: three-fourths to the Coal and Gas Road Improvement Fund and one-fourth to the Virginia Coalfield Economic Development Fund. The purpose of this fund is to enhance the economic base for the seven counties and one city in the authority.

\label{tab:table17-1}Natural Resource Taxes, 2019

Locality

Coal \& Gas Severance Tax\\
(§ 58.1-3712)

Oil Severance Tax\\
(§ 58.1-3712.1)

Additional Gas Severance Tax\\
(§ 58.1-3713.4)

Coal \& Gas Road Improvement Tax\\
(§ 58.1-3713)

Tax on Mineral Land\\
(§ 58.1-3286)

Accomack County

--

--

--

--

No

Albemarle County

--

--

--

--

No

Alleghany County

--

--

--

--

No

Amelia County

--

--

--

--

No

Amherst County

--

--

--

--

Yes

Appomattox County

--

--

--

--

No

Arlington County

--

--

--

--

No

Augusta County

--

--

--

--

Yes

Bath County

--

--

--

--

No

Bedford County

--

--

--

--

Yes

Bland County

--

--

--

--

No

Botetourt County

--

--

--

--

No

Brunswick County

0.0

0.0

0

0.0

Yes

Buchanan County

1.0

0.5

1

1.0

No

Buckingham County

--

--

--

--

Yes

Campbell County

--

--

--

--

Yes

Caroline County

--

--

--

--

Yes

Carroll County

--

--

--

--

No

Charles City County

--

--

--

--

No

Charlotte County

--

--

--

--

No

Chesterfield County

--

0.5

--

--

No

Clarke County

--

--

--

--

No

Craig County

--

--

--

--

No

Culpeper County

--

--

--

--

Yes

Cumberland County

--

--

--

--

No

Dickenson County

1.0

0.5

1

1.0

Yes

Dinwiddie County

--

--

--

--

No

Essex County

--

--

--

--

No

Fairfax County

--

--

--

--

No

Fauquier County

--

--

--

--

No

Floyd County

--

--

--

--

No

Fluvanna County

--

--

--

--

No

Franklin County

--

--

--

--

No

Frederick County

--

--

--

--

No

Giles County

--

--

--

--

No

Gloucester County

--

--

--

--

No

Goochland County

--

--

--

--

Yes

Grayson County

--

--

--

--

Yes

Greene County

--

--

--

--

No

Greensville County

--

--

--

--

Yes

Halifax County

--

--

--

--

No

Hanover County

--

--

--

--

Yes

Henrico County

--

--

--

--

No

Henry County

--

--

--

--

No

Highland County

--

--

--

--

Yes

Isle of Wight County

--

--

--

--

No

James City County

--

--

--

--

No

King \& Queen County

--

--

--

--

No

King George County

--

--

--

--

No

King William County

--

--

--

--

Yes

Lancaster County

--

--

--

--

No

Lee County

2.0

0.5

2

1.0

Yes

Loudoun County

--

--

--

--

No

Louisa County

--

--

--

--

No

Lunenburg County

--

--

--

--

No

Madison County

--

--

--

--

No

Mathews County

--

--

--

--

No

Mecklenburg County

--

--

--

--

No

Middlesex County

--

--

--

--

No

Montgomery County

--

--

--

--

No

Nelson County

--

--

--

--

No

New Kent County

--

--

--

--

No

Northampton County

--

--

--

--

No

Northumberland County

--

--

--

--

No

Nottoway County

--

--

--

--

No

Orange County

--

--

--

--

No

Page County

--

--

--

--

No

Patrick County

--

--

--

--

No

Pittsylvania County

--

--

--

--

Yes

Powhatan County

--

--

--

--

Yes

Prince Edward County

--

--

--

--

No

Prince George County

--

--

--

--

No

Prince William County

--

--

--

--

No

Pulaski County

--

1.0

1

1.0

No

Rappahannock County

--

--

--

--

No

Richmond County

--

--

--

--

No

Roanoke County

--

--

--

--

No

Rockbridge County

--

--

--

--

No

Rockingham County

--

1.0

--

--

Yes

Russell County

1.0

0.5

--

1.0

Yes

Scott County

1.0

0.5

--

1.0

No

Shenandoah County

--

--

--

--

No

Smyth County

--

--

--

--

No

Southampton County

--

--

--

--

No

Spotsylvania County

--

--

--

--

No

Stafford County

--

--

--

--

No

Surry County

--

--

--

--

No

Sussex County

--

--

--

--

No

Tazewell County

1.5

--

1

0.5

Yes

Warren County

--

--

--

--

Yes

Washington County

0.0

0.0

0

0.0

Yes

Westmoreland County

--

--

--

--

No

Wise County

--

0.5

1

--

Yes

Wythe County

--

--

--

--

No

York County

--

--

--

--

No

Alexandria City

--

--

--

--

No

Bristol City

--

--

--

--

No

Buena Vista City

--

--

--

--

Yes

Charlottesville City

--

--

--

--

No

Chesapeake City

--

--

--

--

No

Colonial Heights City

--

--

--

--

No

Covington City

--

--

--

--

No

Danville City

--

--

--

--

No

Emporia City

--

--

--

--

No

Fairfax City

--

--

--

--

No

Falls Church City

--

--

--

--

No

Franklin City

--

--

--

--

No

Fredericksburg City

--

--

--

--

No

Galax City

--

--

--

--

No

Hampton City

--

--

--

--

No

Harrisonburg City

--

--

--

--

No

Hopewell City

--

--

--

--

No

Lexington City

--

--

--

--

No

Lynchburg City

--

--

--

--

No

Manassas City

--

--

--

--

No

Manassas Park City

--

--

--

--

No

Martinsville City

--

--

--

--

No

Newport News City

--

--

--

--

No

Norfolk City

--

--

--

--

No

Norton City

1.0

--

--

1.0

Yes

Petersburg City

--

--

--

--

No

Poquoson City

--

--

--

--

No

Portsmouth City

--

--

--

--

No

Radford City

--

--

--

--

No

Richmond City

--

--

--

--

No

Roanoke City

--

--

--

--

No

Salem City

--

--

--

--

No

Staunton City

--

--

--

--

No

Suffolk City

--

--

--

--

No

Virginia Beach City

--

--

--

--

No

Waynesboro City

--

--

--

--

No

Williamsburg City

--

--

--

--

No

Winchester City

--

--

--

--

No

Abingdon Town

--

--

--

--

No

Accomac Town

--

--

--

--

No

Alberta Town

--

--

--

--

No

Altavista Town

--

--

--

--

No

Amherst Town

--

--

--

--

No

Appalachia Town

--

--

--

--

No

Appomattox Town

--

--

--

--

No

Ashland Town

--

--

--

--

No

Bedford Town

--

--

--

--

No

Belle Haven Town

--

--

--

--

No

Berryville Town

--

--

--

--

No

Big Stone Gap Town

--

--

--

--

No

Blacksburg Town

--

--

--

--

No

Blackstone Town

--

--

--

--

No

Bloxom Town

--

--

--

--

No

Bluefield Town

--

--

--

--

No

Boones Mill Town

--

--

--

--

No

Bowling Green Town

--

--

--

--

No

Boyce Town

--

--

--

--

No

Boydton Town

--

--

--

--

No

Branchville Town

--

--

--

--

No

Bridgewater Town

--

--

--

--

No

Broadway Town

--

--

--

--

No

Brodnax Town

--

--

--

--

No

Brookneal Town

--

--

--

--

No

Buchanan Town

--

--

--

--

No

Burkeville Town

--

--

--

--

No

Cape Charles Town

--

--

--

--

No

Capron Town

--

--

--

--

No

Charlotte Court House Town

--

--

--

--

No

Chase City Town

--

--

--

--

No

Chatham Town

--

--

--

--

No

Cheriton Town

--

--

--

--

No

Chilhowie Town

--

--

--

--

No

Chincoteague Town

--

--

--

--

No

Christiansburg Town

--

--

--

--

No

Claremont Town

--

--

--

--

No

Clarksville Town

--

--

--

--

No

Clifton Town

--

--

--

--

No

Clifton Forge Town

--

--

--

--

No

Clinchco Town

--

--

--

--

No

Clintwood Town

--

--

--

--

No

Coeburn Town

--

--

--

--

No

Colonial Beach Town

--

--

--

--

No

Courtland Town

--

--

--

--

No

Craigsville Town

--

--

--

--

No

Culpeper Town

--

--

--

--

No

Damascus Town

--

--

--

--

No

Dayton Town

--

--

--

--

No

Dendron Town

--

--

--

--

No

Dillwyn Town

--

--

--

--

No

Drakes Branch Town

--

--

--

--

No

Dublin Town

--

--

--

--

No

Dumfries Town

--

--

--

--

No

Dungannon Town

--

--

--

--

No

Eastville Town

--

--

--

--

No

Edinburg Town

--

--

--

--

No

Elkton Town

--

--

--

--

No

Exmore Town

--

--

--

--

No

Farmville Town

--

--

--

--

No

Fincastle Town

--

--

--

--

No

Floyd Town

--

--

--

--

No

Fries Town

--

--

--

--

No

Front Royal Town

--

--

--

--

No

Gate City Town

--

--

--

--

No

Glade Spring Town

--

--

--

--

No

Glasgow Town

--

--

--

--

No

Gordonsville Town

--

--

--

--

No

Goshen Town

--

--

--

--

No

Gretna Town

--

--

--

--

No

Grottoes Town

--

--

--

--

No

Grundy Town

--

--

--

--

No

Halifax Town

--

--

--

--

No

Hamilton Town

--

--

--

--

No

Haymarket Town

--

--

--

--

No

Haysi Town

--

--

--

--

No

Herndon Town

--

--

--

--

No

Hillsboro Town

--

--

--

--

No

Hillsville Town

--

--

--

--

No

Honaker Town

--

--

--

--

No

Hurt Town

--

--

--

--

No

Independence Town

--

--

--

--

No

Iron Gate Town

--

--

--

--

No

Irvington Town

--

--

--

--

No

Ivor Town

--

--

--

--

No

Jarratt Town

--

--

--

--

No

Keller Town

--

--

--

--

No

Kenbridge Town

--

--

--

--

No

Keysville Town

--

--

--

--

No

Kilmarnock Town

--

--

--

--

No

La Crosse Town

--

--

--

--

No

Lawrenceville Town

--

--

--

--

No

Lebanon Town

--

--

--

--

No

Leesburg Town

--

--

--

--

No

Louisa Town

--

--

--

--

No

Lovettsville Town

--

--

--

--

No

Luray Town

--

--

--

--

No

Madison Town

--

--

--

--

No

Marion Town

--

--

--

--

No

McKenney Town

--

--

--

--

No

Middleburg Town

--

--

--

--

No

Middletown Town

--

--

--

--

No

Mineral Town

--

--

--

--

No

Montross Town

--

--

--

--

No

Mount Crawford Town

--

--

--

--

No

Mount Jackson Town

--

--

--

--

No

Narrows Town

--

--

--

--

No

New Market Town

--

--

--

--

No

Newsoms Town

--

--

--

--

No

Nickelsville Town

--

--

--

--

No

Occoquan Town

--

--

--

--

No

Onancock Town

--

--

--

--

No

Onley Town

--

--

--

--

No

Orange Town

--

--

--

--

No

Pennington Gap Town

--

--

--

--

No

Phenix Town

--

--

--

--

No

Pocahontas Town

--

--

--

--

No

Port Royal Town

--

--

--

--

No

Pound Town

--

--

--

--

No

Pulaski Town

--

--

--

--

No

Purcellville Town

--

--

--

--

No

Remington Town

--

--

--

--

No

Rich Creek Town

--

--

--

--

No

Richlands Town

--

--

--

--

No

Ridgeway Town

--

--

--

--

No

Rocky Mount Town

--

--

--

--

No

Round Hill Town

--

--

--

--

No

Rural Retreat Town

--

--

--

--

No

Saint Paul Town

--

--

--

--

No

Saltville Town

--

--

--

--

No

Saxis Town

--

--

--

--

No

Scottsburg Town

--

--

--

--

No

Scottsville Town

--

--

--

--

No

Shenandoah Town

--

--

--

--

No

Smithfield Town

--

--

--

--

No

South Boston Town

--

--

--

--

No

South Hill Town

--

--

--

--

No

Stanley Town

--

--

--

--

No

Stony Creek Town

--

--

--

--

No

Strasburg Town

--

--

--

--

No

Surry Town

--

--

--

--

No

Tappahannock Town

--

--

--

--

No

Tazewell Town

--

--

--

--

No

Timberville Town

--

--

--

--

No

Toms Brook Town

--

--

--

--

No

Troutville Town

--

--

--

--

No

Urbanna Town

--

--

--

--

No

Victoria Town

--

--

--

--

No

Vienna Town

--

--

--

--

No

Vinton Town

--

--

--

--

No

Virgilina Town

--

--

--

--

No

Wachapreague Town

--

--

--

--

No

Wakefield Town

--

--

--

--

No

Warrenton Town

--

--

--

--

No

Warsaw Town

--

--

--

--

No

Washington Town

--

--

--

--

No

Waverly Town

--

--

--

--

No

Weber City Town

--

--

--

--

No

West Point Town

--

--

--

--

No

Windsor Town

--

--

--

--

No

Wise Town

--

--

--

--

No

Woodstock Town

--

--

--

--

No

Wytheville Town

--

--

--

--

No

\hypertarget{legal-document-taxes}{%
\chapter{Legal Document Taxes}\label{legal-document-taxes}}

~~In fiscal year 2018, the most recent year available from the Auditor of Public Accounts, taxes on legal documents accounted for 0.5 percent of total tax revenue for cities and 0.8 percent for counties. Towns do not have this tax. These are averages; the relative importance of taxes in individual localities may vary significantly. For information on individual localities, see Appendix C.\\
~\\
\hspace*{0.333em}\hspace*{0.333em}Section 58.1-3800 of the Code of Virginia authorizes the governing body of any city or county to impose a recordation tax in an amount equal to one-third of the state recordation tax. The recordation tax generally applies to real and personal property in connection with deeds of trust, mortgages, and leases, and to contracts involving the sale of rolling stock or equipment (§§ 58.1-807 and 58.1-808).\\
~\\
\hspace*{0.333em}\hspace*{0.333em}Local governments are not permitted to impose a levy when the state recordation tax imposed is 50 cents or more (§ 58.1-3800). Consequently, local governments cannot levy a tax on such documents as certain corporate charter amendments (§ 58.1-801), deeds of release (§ 58.1-805), or deeds of partition (§ 58.1-806) as the state tax imposed is already 50 cents per \$100.\\
~\\
\hspace*{0.333em}\hspace*{0.333em}Sections 58.1-809 and 58.1-810 also specifically exempt certain types of deed modifications from being taxed. Deeds of confirmation or correction, deeds to which the only parties are husband and wife, and modifications or supplements to the original deeds are not taxed. Finally, § 58.1-811 lists a number of exemptions to the recordation tax.\\
~\\
\hspace*{0.333em}\hspace*{0.333em}Currently, the state recordation tax on the first \$10 million of value is 25 cents per \$100, so cities and counties can impose a maximum tax of 8.3 cents per \$100 on the first \$10 million, one-third of the 25 cent state rate. Above \$10 million there is a declining scale of charges applicable (§ 58.1-3803).\\
~\\
\hspace*{0.333em}\hspace*{0.333em}In addition to a tax on real and personal property, §§ 58.1-3805 and 58.1-1718 authorize cities and counties to impose a tax on the probate of every will or grant of administration equal to one-third of the state tax on such probate or grant of administration. Currently, the state tax on wills and grants of administration is 10 cents per \$100 or a fraction of \$100 for estates valued at greater than \$15,000 (§ 58.1-1712). Therefore, the maximum local rate is 3.3 cents.\\
~\\
\hspace*{0.333em}\hspace*{0.333em}A related \emph{state} tax is levied in localities associated with the Northern Virginia Transportation Authority. The tax is a grantor's fee of \$0.15 per \$100 on the value of real property property sold. This was created as part of the 2013 transportation bill.\\
~\\
\hspace*{0.333em}\hspace*{0.333em}\textbf{Table 18.1} provides information on the recordation tax and the wills and administration tax for the 35 cities and 89 counties that report imposing one or both of them. The following text table shows range of recordation taxes and taxes on wills and administration imposed by localities.

\begin{Shaded}
\begin{Highlighting}[]
\CommentTok{\#Text table "Recordation Tax and Tax on Wills and Administration, 2019" goes here}

\CommentTok{\#Table 18.1 "Legal Document Taxes, 2019" goes here}
\end{Highlighting}
\end{Shaded}

\hypertarget{miscellaneous-taxes-2018}{%
\chapter{Miscellaneous Taxes 2018}\label{miscellaneous-taxes-2018}}

This section includes a number of taxes and exemptions that are not covered in the previous sections: the local option sales and use tax, the bank franchise tax, the communication sales and use tax, the short-term (daily) rental tax, and other miscellaneous taxes. The local option sales tax has been adopted by every city and county and, by law, all use the same tax rate. Also, as explained below, counties must share a portion of sales tax collections with incorporated towns within their boundaries. Wherever the bank franchise tax is imposed, the rate is the same. In addition to those major taxes, this section covers the communications sales and use tax and other miscellaneous taxes for which information was provided on the survey form when local governments were asked to specify any miscellaneous taxes that fell outside the scope of the survey questions.

\hypertarget{local-option-and-state-sales-and-use-taxes}{%
\section{LOCAL OPTION AND STATE SALES AND USE TAXES}\label{local-option-and-state-sales-and-use-taxes}}

In fiscal year 2018, the most recent year available from the Auditor of Public Accounts, the local option sales and use tax accounted for 8.0 percent of local tax revenue for cities, 6.4 percent for counties and 9.2 percent for large towns. These are averages; the relative importance of taxes in individual cities, counties and towns may vary significantly. For information on individual localities, see Appendix C.

Each city and county is permitted by § 58.1-605 to establish a general retail sales tax, ``at the rate of 1 percent to provide revenue for the general fund of such city or county.'' This tax applies to dealers with a retail presence in Virginia. Sales of any items from such operations incur the 1 percent sales tax. Sales tax monies are then collected by the Virginia Department of Taxation and sent to the Department of the Treasury. That agency credits the accounts of the localities where the sales occurred and disburses the monies to the localities on a monthly basis (§ 58.1-605.F).

Cities and counties are also permitted to establish a local use tax at the rate of 1 percent for the purpose of providing revenue to the general fund of the locality. The use tax is similar in purpose to the retail sales tax, but its aim is somewhat distinct: it applies to dealers that do not have a physical retail presence in Virginia. It is a tax levied on the use of tangible personal property within the state that has been stored or sold out-of-state.

Special distribution requirements apply to counties with incorporated towns (§ 58.1-605.G). Where the town constitutes a special school division and is operated as a separate school division under a town school board\footnote{Commission on Local Government, Report on Proffered Cash Payments and Expenditures by Virginia's Counties, Cities and Towns, 2017-2018. \url{https://www.dhcd.virginia.gov/cash-proffers}.}, the county is required to pay to the town a proportionate share of the full amount of tax receipts based on the school age population within the town compared to the school age population in the entire county. If the town does not constitute a separate school division, then one-half of county collections is distributed to the town based on the proportion of the school age population within the town to the school age population of the entire county, provided the town complies with certain conditions.

Certain items are exempted from the state sales and use tax and may be exempted from the local option sales and use tax also. Each locality is permitted by § 58.1-609 to exempt fuels meant for domestic consumption from the 1 percent component of the tax. These fuels include artificial or propane gas, firewood, coal, or home heating oil. Only 11 localities answered that they exempted such fuels from the tax. The localities were the counties of Alleghany, Campbell, Madison, Patrick, Pittsylvania, Prince George and Washington and the cities of Chesapeake, Covington, Harrisonburg, and Portsmouth.

The state portion of the sales and use tax was raised from 4 percent to 4.3 percent effective July 1, 2013. House Bill 2313, Chapter 766, further increased the amount by an additional 0.7 amount for localities in the Northern Virginia and Hampton Roads planning districts. The additional taxes do not apply to food purchased for human consumption. The Northern Virginia Planning District consists of the counties of Arlington, Fairfax, Loudoun, and Prince William and the cities of Alexandria, Fairfax, Falls Church, Manassas, and Manassas Park. The Hampton Roads Planning District consists of the counties of Isle of Wight, James City, South-ampton, and York and the cities of Chesapeake, Franklin, Hampton, Newport News, Norfolk, Poquoson, Portsmouth, Suffolk, Virginia Beach, and Williamsburg. The purpose of this additional state tax is to fund the Northern Virginia Transportation Authority and the Hampton Roads Construction Fund, respectively. Consequently, the new sales and use rate is made up of a 1.0 percent local tax rate as well as a 4.3 state tax rate for most localities and a 5.0 percent state tax rate for localities associated with transportation commissions.

\hypertarget{state-motor-fuels-tax-on-distributors}{%
\section{STATE MOTOR FUELS TAX ON DISTRIBUTORS}\label{state-motor-fuels-tax-on-distributors}}

An additional state tax that applies only to specific localities is the fuel distribution license tax. It is a state tax on distributors of motor fuels to retailers in qualifying localities. Under § 58.1-2295 a state tax of 2.1 percent may be imposed on any distributor in a qualifying locality in the business of selling fuels at wholesale to retail dealers for retail sale within the qualifying locality. To be eligible a locality must be: (i) any county or city that is a member of a transportation district in which a rail commuter mass transport system and a bus commuter mass transport system are owned or operated by an agency as defined in § 15.2- 4502, or (ii) any county or city that is a member of a transportation district subject to § 15.2- 4515 and is contiguous to the Northern Virginia Transportation District. In addition, § 58.1-1722 excludes the amount of the tax imposed and collected by the distributor from the distributor's gross receipts for purposes of BPOL taxes imposed under Chapter 37 (§ 58.1-3700 et seq.).

The 2.1 percent state tax is imposed in 11 localities that belong to two transportation commissions. The Northern Virginia Transportation Commission (NVTC) consists of Fairfax, Loudoun, and Arlington counties and Alexandria, Fairfax, and Falls Church cities. The tax helps provide financial support for the activities of the Washington Metropolitan Area Transit Authority (WMATA), also known as Metro, and the Virginia Railway Express (VRE), the commuter line between Washington D.C. and Manassas and Fredericksburg. The other commission, the Potomac and Rappa-hannock Transportation Commission (PRTC), consists of three cities (Fredericksburg, Manassas, and Manassas Park), and two counties (Prince William and Stafford). It provides support to rail transport (VRE) in the affected counties and bus services originating in Prince William County through Omniride and Omnilink.

House Bill 2313, Chapter 766, authorized the state tax in certain localites in the Hampton Roads Planning District. These are the counties of Isle of Wight, James City, Southampton, and York, and the cities of Chesapeake, Hampton, Franklin, Newport News, Norfolk, Suffolk, Virginia Beach, Williamsburg, Poquoson, and Portsmouth. The tax began on July 1, 2013.

\hypertarget{bank-franchise-tax}{%
\section{BANK FRANCHISE TAX}\label{bank-franchise-tax}}

The bank franchise tax, also known as the bank stock tax, accounted for 0.7 percent of city tax revenue in fiscal year 2018, 0.5 percent of county tax revenue, and 4.2 percent of the tax revenue of large towns. These are averages; the relative importance of taxes in individual cities, counties, and towns may vary significantly. For information on individual localities, see Appendix C.

The state of Virginia levies a bank franchise tax on all banks in Virginia at a rate of \$1 on each \$100 of net capital (§ 58.1-1204). Net capital is defined and its computation explained in § 58.1-1205. According to this section, net capital is determined by adding together a bank's capital, surplus, undivided profits, and one half of any reserve for loan losses net of applicable deferred tax to obtain gross capital and deducting therefrom (i) the assessed value of real estate as provided in § 58.1-1206, (ii) the book value of tangible personal property under § 58.1-1206, (iii) the pro rata share of government obligations as set forth in § 58.1-1206, (iv) the capital accounts of any bank subsidiaries under § 58.1-1206, (v) the amount of any reserve for marketable securities valuation which is included in capital, surplus and undivided profits as defined hereinabove to the extent that such reserve reflects the difference between the book value and the market value of such marketable securities on December 31 next preceding the date for filing the bank's return under § 58.1-1207, and (vi) the value of goodwill described under subdivision A 5 of § 58.1-1206.

Cities (§ 58.1-1208), counties (§ 58.1-1210), and incorporated towns (§ 58.1-1209) are permitted to charge an additional franchise tax of 80 percent of the state rate of taxation. If a locality imposes the local tax, then a bank is entitled to a credit against the state franchise tax equal to the total amount of local franchise tax paid (§ 58.1-1213). All localities that impose the bank franchise tax do so at the maximum rate allowed by statute.

If a bank has branches in more than one taxable subdivision (that is, city, county, or incorporated town), the tax imposed by the subdivision must be in the proportion of the taxable value of the net capital based on the total deposits of the bank or banks located inside the taxing subdivision to the total deposits in Virginia of the bank as of the end of the preceding year (§ 58.1-1211).

The survey asked whether a locality levied a bank tax. Of those localities that answered, all cities, 85 counties, and 108 towns answered affirmatively. The number of counties responding positively contrasts with the number of counties that reported receiving money from the tax in the Auditor of Public Accounts' Comparative Report. The reported disparity may be because a number of counties answered positively for having the tax when they actually only processed forms for towns having the tax. A list of localities that reported imposing the tax can be found in \textbf{Table 19.1}.

\hypertarget{communications-sales-and-use-tax}{%
\section{COMMUNICATIONS SALES AND USE TAX}\label{communications-sales-and-use-tax}}

In 2006, legislation enacted by the General Assembly, House Bill 568, replaced many state and local taxes and fees on communications services with a flat 5 percent rate. The tax is collected from consumers by their service providers and is then remitted to the Virginia Department of Taxation. The department then distributes the monies to the localities on a percentage basis derived from their participation in the local taxes which the new fl at tax superseded. The communication sales and use tax is a state tax not a local tax. Beginning in FY 2010 the Auditor of Public Accounts reported the proceeds as part of noncategorical state aid to localities.

The communications sales and use tax replaced a variety of local taxes: the consumer utility tax on land line and wireless telephone service, the local E-911 tax on land line telephone service, a portion of the BPOL tax assessed on public service companies by certain localities that impose the tax at a rate higher than 0.5 percent, the local video programming excise tax on cable television services, and the local consumer utility tax on cable television service.

The communications sales and use tax does not affect several related taxes: the state E-911 fee on wireless telephone service; the public rights-of-way use fee on land line telephone service; and the local tax of 0.5 percent on public service companies (also called the utility license tax).

\textbf{Table 19.2} presents a listing of the localities that received distributions from the communications sales and use tax in fiscal year 2018. The information was taken from Table 5.6 of the Virginia Department of Taxation's Annual Report, Fiscal Year 2018, the latest year available.

\hypertarget{short-term-daily-rental-tax}{%
\section{SHORT-TERM DAILY RENTAL TAX}\label{short-term-daily-rental-tax}}

In 2010 the General Assembly modified short-term rental property classifications. Short-term rental property can once again be included in merchants' capital as a separate classification. Consequently, localities may tax this property either as merchants' capital or short-term rental property, but not as both. Whether considered under the merchants' capital tax or the short-term property tax, the category of property shall not be considered tangible personal property for purposes of taxation.

The new law maintains the usual exclusions. Therefore, the category of short-term rental property still excludes ``(i) trailers as defined in § 46.2-100, and (ii) other tangible personal property required to be licensed or registered with the Department of Motor Vehicles, Department of Game and Inland Fisheries, or Department of Aviation (§ 58.1-3510.4).'' The most important exception listed is motor vehicles for rent. These fall under the merchants' capital tax as a separate classification, discussed in Section 8.

For purposes of taxation under the short-term rental tax, property is classified into two types: short-term rental property and heavy equipment short-term rental property (§ 58.1-3510.6). Short-term rental property may be taxed at 1 percent of gross receipts. Heavy equipment short-term rental property may be taxed up to 1.5 percent of gross receipts. \textbf{Table 19.3} lists the 20 cities, 19 counties, and 2 towns that reported having the short-term rental tax.

\begin{Shaded}
\begin{Highlighting}[]
\CommentTok{\#Table 19.1 "Localities Reporting That They Levy a Bank Franchise Tax, 2019", goes here}

\CommentTok{\#Table 19.2 "Localities Receiving Communications Sales and Use Tax Distributions, FY 2018"}

\CommentTok{\#Table 19.3 "Short{-}Term Daily Rental Tax, 2019*" goes here}
\end{Highlighting}
\end{Shaded}

\begin{itemize}
\tightlist
\item
  As noted in the text for Section 19, the tax excludes motor vehicles for rent.
\end{itemize}

\hypertarget{refuse-and-recycling-collection-fees}{%
\chapter{Refuse and Recycling Collection Fees}\label{refuse-and-recycling-collection-fees}}

Many Virginia localities collect, or authorize to have collected, refuse and recycled materials. In its survey, the Cooper Center inquired into the methods and fees for the collection of refuse and recycled materials. The answers are provided in four tables covering regular refuse pick up, tipping fees, recycling, and pickup of miscellaneous refuse items.

\hypertarget{refuse-collection}{%
\section{REFUSE COLLECTION}\label{refuse-collection}}

\textbf{Table 20.1} shows information reported on refuse collection by all 38 cities, and by 24 counties and 101 towns. The table contains information on frequency of collection, collection fees and private contracting. There are three methods of operation. Some Virginia localities levy a specific refuse collection service fee for the costs of collection. Others pay for collection costs with general tax revenues. Finally, some localities provide no service; instead, they leave refuse collection to private contractors.

A majority of cities and counties provide basic residential services on a weekly basis. Only the counties of Arlington, Chesterfield, and Halifax offer regular collections more frequently.

Regarding fees, 32 cities, 13 counties, and 62 towns reported imposing a residential refuse collection service fee. Eleven cities, 7 counties, and 42 towns contracted with private firms for refuse collection. The text table below shows this breakdown.

\begin{Shaded}
\begin{Highlighting}[]
\CommentTok{\#text table "Residential Refuse Collection, 2019" goes here}
\end{Highlighting}
\end{Shaded}

\textbf{Table 20.2} shows tipping fees charged by various localities to dump trash at landfills and waste transfer stations. Localities reporting imposing such fees included 9 cities, 34 counties, and 7 towns.

\hypertarget{recycling-programs}{%
\section{RECYCLING PROGRAMS}\label{recycling-programs}}

\textbf{Table 20.3} provides data on localities that have instituted recycling programs. As with refuse collection, these programs may be financed in a variety of ways. Many localities pick up recyclables and then finance the collection with a service charge. Other localities contract with a private firm. \textbf{Table 20.3} shows which localities offer collection of recyclables and which contract for collection with a private firm. It also shows the monthly fees associated with collecting recyclables.

Of the total survey respondents, 38 cities, 83 counties, and 67 towns reported having some form of recycling activity. Seventeen cities provided recycling collection directly, and 21 contracted it out. Thirty-seven counties provided services directly, while 46 contracted them out. Of the towns, 8 had their services provided by their host county, 25 provided direct services, and 34 contracted for services. The text table below shows this breakdown.

\begin{Shaded}
\begin{Highlighting}[]
\CommentTok{\#Text table "Residential Recycling Programs, 2019" goes here (uncertain if these text tables are queryable from the database, or if they were created out of calculations from te the data, or something else entirely)}
\end{Highlighting}
\end{Shaded}

For localities that charged a service fee, the amount ranged anywhere from \$1.33 to \$16.50 per month.

\begin{Shaded}
\begin{Highlighting}[]
\CommentTok{\#Table 20.1 "Refuse Collection Fees, 2019" goes here, formatting }\AlertTok{TBD}
\end{Highlighting}
\end{Shaded}

\label{tab:table20-2}Refuse Collection Tipping Fees Table, 2019

Locality

Tipping Fee

Accomack County

80.00

per ton

Albemarle County

--

Alleghany County

--

Amelia County

--

Amherst County

--

Appomattox County

--

Arlington County

--

Augusta County

45/ton industrial commercial

15/ton clean wood

10/ton mulch

Bath County

large firm=40/ton; small=13.36 fee; med=25

Bedford County

41 per ton commercial. Residents 1000 lbs. free per month per household then 57 per ton

Bland County

--

Botetourt County

--

Brunswick County

19.00 per ton

Buchanan County

--

Buckingham County

--

Campbell County

28.75 per ton for Region 2000 service Auth. members

38.75 per ton for non-members

Caroline County

--

Carroll County

--

Charles City County

--

Charlotte County

--

Chesterfield County

--

Clarke County

--

Craig County

--

Culpeper County

50.32/ton

Cumberland County

--

Dickenson County

60.00 per ton

Dinwiddie County

35/ton over 500 lbs

Essex County

--

Fairfax County

68/ton

Fauquier County

25.00 permit fee per vehicle

15.00 renewal fee per vehicle

5.00 replacement fee per vehicle

Floyd County

municip.comm.indust.only-ton-40; pick up 3 55/ton

Fluvanna County

--

Franklin County

Effective July 1 2013 43.00 per ton. TIRES: CAR LIGHT TRUCKS 2.00 EACH;SEMI TRUCK 4.00 EACH;TRACTOR/HEAVY EQUIP 6.00 EACH;ATV LAWN MOWER \& M/C 1.00 EACH; single wide mobile homes 400. each; double wide mobile homes 800. each.

Frederick County

Comm 50/ton; Concrete 15/ton

Giles County

--

Gloucester County

--

Goochland County

--

Grayson County

--

Greene County

1.00 per bag

Greensville County

55.00/ton Municipal solid waste

10.00 ton Burnable vegetative

100.00 ton Tires

Halifax County

--

Hanover County

50/ton

Henrico County

Henrico no longer accepts refuse from commercial collectors. A Public Use Area is available for Henrico residents at a cost of 3 per trip.

Henry County

--

Highland County

--

Isle of Wight County

--

James City County

--

King \& Queen County

--

King George County

N/A

King William County

--

Lancaster County

--

Lee County

Business 37.50/ton; tires 60.00/ton

check with County Administrators Office 276-346-7714

Loudoun County

5 flat rate to 65/ton; see www.loudoun.gov/Default.aspx?tabid=744\#unit

Louisa County

Please Contact Facilities Management/Public Works

540-967-3462.

Lunenburg County

Private firm/operator sets rate(s)

Madison County

65/ton for rolloffs; other size vehicles vary

Mathews County

--

Mecklenburg County

40.00 per ton

Middlesex County

--

Montgomery County

--

Nelson County

55 per ton

New Kent County

--

Northampton County

72.00 PER TON

Northumberland County

0

Nottoway County

--

Orange County

Contact the Orange County Landfill

540-672-9315

Page County

--

Patrick County

55.00 per ton

Pittsylvania County

41.00 ton

Powhatan County

--

Prince Edward County

35/ton for commercial and institutional

Prince George County

no

Prince William County

--

Pulaski County

34.50 (call Public Svc: 540-674-8720)

Rappahannock County

50.00 per ton for one non-commercial user. No others accepted

Richmond County

--

Roanoke County

--

Rockbridge County

54.50 ton

Rockingham County

58/ton

Russell County

31.56 per ton

Scott County

--

Shenandoah County

45/ton Commercial waste

36/ton Wood waste

52.50/ton Special/Rough waste

Smyth County

56.00 per ton

Southampton County

--

Spotsylvania County

--

Stafford County

--

Surry County

--

Sussex County

--

Tazewell County

--

Warren County

See charges below

Washington County

--

Westmoreland County

Construction debris 49.77/ton

Wise County

--

Wythe County

\begin{enumerate}
\def\labelenumi{\arabic{enumi}.}
\setcounter{enumi}{51}
\tightlist
\item
  per ton at transfer station- min. Anything under 1060 lbs is a minimum of 50.

  York County

  No tipping fee for solid waste customers. For others see below.

  Alexandria City

  --

  Bristol City

  30.00 per ton with a 15.00 minimum charge

  Buena Vista City

  --

  Charlottesville City

  Charge per Collection
\end{enumerate}

Container Size

(cubic yards) Compacted Uncompacted

2 6.00 25.00

4 12.00 50.00

6 19.00 75.00

8 25.00 100.00

Chesapeake City

--

Colonial Heights City

800 tractor-trailer load; 600 tandem-axle truck

Covington City

88.13 per Ton

Danville City

--

Emporia City

--

Fairfax City

--

Falls Church City

--

Franklin City

included in the rate

Fredericksburg City

--

Galax City

--

Hampton City

38.00 a ton

Harrisonburg City

--

Hopewell City

--

Lexington City

--

Lynchburg City

35 per ton for commercial.

25 per ton city residents.

Manassas City

--

Manassas Park City

--

Martinsville City

Done privately

Newport News City

--

Norfolk City

--

Norton City

--

Petersburg City

--

Poquoson City

--

Portsmouth City

--

Radford City

--

Richmond City

27.38

Roanoke City

45 per ton

Salem City

--

Staunton City

45/ton

Suffolk City

--

Virginia Beach City

--

Waynesboro City

44.00/ton

Williamsburg City

--

Winchester City

--

Abingdon Town

--

Accomac Town

--

Alberta Town

--

Altavista Town

--

Amherst Town

VARIES BY DUMPSTER SIZE AND COLLECTION FREQUENCY

Appalachia Town

--

Appomattox Town

--

Ashland Town

--

Bedford Town

60.00/ton

Belle Haven Town

--

Berryville Town

--

Big Stone Gap Town

--

Blacksburg Town

--

Blackstone Town

no

Bloxom Town

--

Bluefield Town

--

Boones Mill Town

--

Bowling Green Town

--

Boyce Town

No

Boydton Town

60

Branchville Town

--

Bridgewater Town

--

Broadway Town

--

Brodnax Town

--

Brookneal Town

--

Buchanan Town

--

Burkeville Town

--

Cape Charles Town

Collection rates vary based on item location and frequency.

Capron Town

--

Charlotte Court House Town

--

Chase City Town

1.84 per cubic yd/commercial dumpster service

Chatham Town

--

Cheriton Town

--

Chilhowie Town

--

Chincoteague Town

--

Christiansburg Town

Montgomery Regional Solid Waste Authority

44.50 per ton

Claremont Town

--

Clarksville Town

40.00 per ton

Clifton Town

--

Clifton Forge Town

--

Clinchco Town

--

Clintwood Town

--

Coeburn Town

--

Colonial Beach Town

--

Courtland Town

--

Craigsville Town

--

Culpeper Town

--

Damascus Town

--

Dayton Town

--

Dendron Town

--

Dillwyn Town

--

Drakes Branch Town

--

Dublin Town

--

Dumfries Town

--

Dungannon Town

--

Eastville Town

--

Edinburg Town

--

Elkton Town

--

Exmore Town

--

Farmville Town

--

Fincastle Town

--

Floyd Town

--

Fries Town

--

Front Royal Town

58.00 monthly for once weekly pickup and 116.00 for twice weekly pickup

Gate City Town

--

Glade Spring Town

In Town Residential - 15.00/mo

Out of Town Residential - 25.00/mo

Business - 25.00/mo

Glasgow Town

--

Gordonsville Town

--

Goshen Town

--

Gretna Town

--

Grottoes Town

--

Grundy Town

--

Halifax Town

--

Hamilton Town

--

Haymarket Town

--

Haysi Town

--

Herndon Town

--

Hillsboro Town

--

Hillsville Town

--

Honaker Town

--

Hurt Town

--

Independence Town

--

Iron Gate Town

--

Irvington Town

--

Ivor Town

--

Jarratt Town

--

Keller Town

--

Kenbridge Town

--

Keysville Town

--

Kilmarnock Town

--

La Crosse Town

1.10 per cu. yd. per pickup

Lawrenceville Town

--

Lebanon Town

--

Leesburg Town

--

Louisa Town

--

Lovettsville Town

--

Luray Town

--

Madison Town

--

Marion Town

--

McKenney Town

--

Middleburg Town

--

Middletown Town

--

Mineral Town

--

Montross Town

--

Mount Crawford Town

--

Mount Jackson Town

--

Narrows Town

--

New Market Town

--

Newsoms Town

--

Nickelsville Town

--

Occoquan Town

--

Onancock Town

--

Onley Town

--

Orange Town

--

Pennington Gap Town

--

Phenix Town

--

Pocahontas Town

--

Port Royal Town

--

Pound Town

--

Pulaski Town

--

Purcellville Town

--

Remington Town

0

Rich Creek Town

--

Richlands Town

--

Ridgeway Town

--

Rocky Mount Town

--

Round Hill Town

--

Rural Retreat Town

--

Saint Paul Town

--

Saltville Town

--

Saxis Town

--

Scottsburg Town

--

Scottsville Town

--

Shenandoah Town

--

Smithfield Town

--

South Boston Town

Landfill closed

South Hill Town

Varies by \# of dumpsters \& pickups: Commercial Only

Stanley Town

--

Stony Creek Town

--

Strasburg Town

--

Surry Town

--

Tappahannock Town

--

Tazewell Town

--

Timberville Town

--

Toms Brook Town

--

Troutville Town

--

Urbanna Town

--

Victoria Town

--

Vienna Town

--

Vinton Town

--

Virgilina Town

--

Wachapreague Town

--

Wakefield Town

--

Warrenton Town

--

Warsaw Town

--

Washington Town

--

Waverly Town

--

Weber City Town

--

West Point Town

--

Windsor Town

--

Wise Town

--

Woodstock Town

--

Wytheville Town

--

\label{tab:table20-3}Recycling Collection Fees Table, 2019

Locality

Provided Directly or Contracted

Service Fee

Accomack County

Contracted

--

Albemarle County

Contracted

--

Alleghany County

Not Applicable

--

Amelia County

Contracted

--

Amherst County

Contracted

--

Appomattox County

Directly

--

Arlington County

Contracted

44.85 /yr included in residential refuse coll fee

Augusta County

Directly

No

Bath County

Contracted

--

Bedford County

Directly

--

Bland County

Contracted

--

Botetourt County

Contracted

--

Brunswick County

Directly

--

Buchanan County

Not Applicable

--

Buckingham County

Directly

--

Campbell County

Directly

--

Caroline County

Directly

--

Carroll County

Contracted

--

Charles City County

Not Applicable

--

Charlotte County

Directly

--

Chesterfield County

Contracted

--

Clarke County

Directly

--

Craig County

Contracted

--

Culpeper County

Contracted

--

Cumberland County

Contracted

--

Dickenson County

Not Applicable

--

Dinwiddie County

Directly

--

Essex County

Contracted

--

Fairfax County

Contracted

--

Fauquier County

Contracted

--

Floyd County

Directly

--

Fluvanna County

Contracted

--

Franklin County

Directly

--

Frederick County

Contracted

--

Giles County

Directly

--

Gloucester County

Not Applicable

--

Goochland County

Contracted

25 annual fee per household

Grayson County

Directly

--

Greene County

Directly

--

Greensville County

Directly

--

Halifax County

Contracted

--

Hanover County

Contracted

28.00/year

Henrico County

Contracted

--

Henry County

Directly

--

Highland County

Directly

--

Isle of Wight County

Contracted

--

James City County

Contracted

--

King \& Queen County

Not Applicable

--

King George County

Contracted

--

King William County

Contracted

--

Lancaster County

Contracted

--

Lee County

Directly

--

Loudoun County

Directly

--

Louisa County

Not Applicable

--

Lunenburg County

Contracted

--

Madison County

Contracted

--

Mathews County

Not Applicable

--

Mecklenburg County

Directly

--

Middlesex County

Contracted

--

Montgomery County

Directly

--

Nelson County

Directly

--

New Kent County

Contracted

--

Northampton County

Contracted

--

Northumberland County

Contracted

--

Nottoway County

Not Applicable

--

Orange County

Contracted

contact the Landfill at 540-672-9315

Page County

Directly

--

Patrick County

Directly

--

Pittsylvania County

Directly

--

Powhatan County

Contracted

--

Prince Edward County

Directly

--

Prince George County

Contracted

--

Prince William County

Contracted

--

Pulaski County

Contracted

--

Rappahannock County

Not Applicable

--

Richmond County

Contracted

--

Roanoke County

Directly

--

Rockbridge County

Directly

--

Rockingham County

Directly

--

Russell County

Contracted

--

Scott County

Directly

--

Shenandoah County

Directly

--

Smyth County

Directly

--

Southampton County

Contracted

no charge to residents; county pays contractor 2.66 per household for bi-weekly service

Spotsylvania County

Directly

--

Stafford County

Contracted

42/ton for commercial users only

Surry County

Directly

--

Sussex County

Not Applicable

--

Tazewell County

Not Applicable

--

Warren County

Directly

--

Washington County

Directly

--

Westmoreland County

Contracted

--

Wise County

Contracted

--

Wythe County

Contracted

--

York County

Contracted

24.50 combined fee for garbage \& recycling collection

Alexandria City

Directly

Included in refuse fee

Bristol City

Directly

--

Buena Vista City

Directly

--

Charlottesville City

Contracted

--

Chesapeake City

Contracted

--

Colonial Heights City

Contracted

--

Covington City

Contracted

--

Danville City

Directly

110.00 Annual Fee

Emporia City

Directly

--

Fairfax City

Directly

--

Falls Church City

Contracted

--

Franklin City

Contracted

--

Fredericksburg City

Directly

--

Galax City

Contracted

--

Hampton City

Directly

7.25/wk recyclers/13.00 wk non-recyclers

Harrisonburg City

Directly

--

Hopewell City

Directly

--

Lexington City

Directly

--

Lynchburg City

Directly

--

Manassas City

Contracted

included in monthly fee

Manassas Park City

Contracted

Included in refuse fee

Martinsville City

Contracted

--

Newport News City

Contracted

included in solid waste user fee.

Norfolk City

Contracted

--

Norton City

Contracted

--

Petersburg City

Contracted

Included in trash pickup fee

Poquoson City

Contracted

Charge is built in with the refuse collection charge

Portsmouth City

Directly

--

Radford City

Contracted

--

Richmond City

Contracted

1.942 service fee monthly

Roanoke City

Directly

--

Salem City

Directly

--

Staunton City

Directly

For recycle only 16.50 per month.

Suffolk City

Contracted

included in refuse collection fee of 17.50 per month

Virginia Beach City

Contracted

--

Waynesboro City

Contracted

--

Williamsburg City

Contracted

--

Winchester City

Directly

--

Abingdon Town

Directly

--

Accomac Town

Not Applicable

--

Alberta Town

Not Applicable

--

Altavista Town

Contracted

15.00 per month

Amherst Town

Not Applicable

--

Appalachia Town

Not Applicable

--

Appomattox Town

Directly

--

Ashland Town

Contracted

--

Bedford Town

Directly

4.00

Belle Haven Town

Not Applicable

--

Berryville Town

Contracted

--

Big Stone Gap Town

Directly

--

Blacksburg Town

Contracted

included in the refuse fee

Blackstone Town

Directly

--

Bloxom Town

Not Applicable

--

Bluefield Town

Directly

--

Boones Mill Town

Not Applicable

--

Bowling Green Town

Not Applicable

--

Boyce Town

Contracted

--

Boydton Town

Not Applicable

--

Branchville Town

Not Applicable

--

Bridgewater Town

Directly

4.82/mo.

Broadway Town

Done by County

--

Brodnax Town

Not Applicable

--

Brookneal Town

Not Applicable

--

Buchanan Town

Contracted

--

Burkeville Town

Not Applicable

--

Cape Charles Town

Not Applicable

--

Capron Town

Not Applicable

--

Charlotte Court House Town

Not Applicable

--

Chase City Town

Not Applicable

--

Chatham Town

Not Applicable

--

Cheriton Town

Not Applicable

--

Chilhowie Town

Not Applicable

--

Chincoteague Town

Not Applicable

--

Christiansburg Town

Contracted

included in garbage service for residential only

Claremont Town

Not Applicable

--

Clarksville Town

Done by County

--

Clifton Town

Not Applicable

--

Clifton Forge Town

Contracted

--

Clinchco Town

Not Applicable

--

Clintwood Town

Not Applicable

--

Coeburn Town

Not Applicable

--

Colonial Beach Town

Contracted

--

Courtland Town

Not Applicable

--

Craigsville Town

Not Applicable

--

Culpeper Town

Directly

--

Damascus Town

Not Applicable

--

Dayton Town

Contracted

Included in monthly refuse fee

Dendron Town

Not Applicable

--

Dillwyn Town

Not Applicable

--

Drakes Branch Town

Directly

--

Dublin Town

Contracted

--

Dumfries Town

Not Applicable

--

Dungannon Town

Not Applicable

--

Eastville Town

Not Applicable

--

Edinburg Town

Directly

--

Elkton Town

Directly

--

Exmore Town

Not Applicable

--

Farmville Town

Directly

--

Fincastle Town

Not Applicable

--

Floyd Town

Directly

--

Fries Town

Not Applicable

--

Front Royal Town

Directly

included in refuse collection charge

Gate City Town

Not Applicable

--

Glade Spring Town

Not Applicable

--

Glasgow Town

Not Applicable

--

Gordonsville Town

Done by County

--

Goshen Town

Not Applicable

--

Gretna Town

Contracted

--

Grottoes Town

Not Applicable

--

Grundy Town

Not Applicable

--

Halifax Town

Not Applicable

--

Hamilton Town

Contracted

--

Haymarket Town

Contracted

--

Haysi Town

Not Applicable

--

Herndon Town

Directly

16 annually through 6/30/19

Hillsboro Town

Not Applicable

--

Hillsville Town

Contracted

--

Honaker Town

Not Applicable

--

Hurt Town

Not Applicable

--

Independence Town

Directly

--

Iron Gate Town

Not Applicable

--

Irvington Town

Not Applicable

--

Ivor Town

Done by County

--

Jarratt Town

Not Applicable

--

Keller Town

Not Applicable

--

Kenbridge Town

Contracted

--

Keysville Town

Done by County

--

Kilmarnock Town

Not Applicable

--

La Crosse Town

Not Applicable

--

Lawrenceville Town

Done by County

--

Lebanon Town

Not Applicable

--

Leesburg Town

Contracted

--

Louisa Town

Contracted

--

Lovettsville Town

Contracted

--

Luray Town

Directly

--

Madison Town

Not Applicable

--

Marion Town

Directly

--

McKenney Town

Not Applicable

--

Middleburg Town

Contracted

--

Middletown Town

Not Applicable

--

Mineral Town

Directly

--

Montross Town

Done by County

--

Mount Crawford Town

Not Applicable

--

Mount Jackson Town

Contracted

--

Narrows Town

Not Applicable

--

New Market Town

Contracted

--

Newsoms Town

Not Applicable

--

Nickelsville Town

Not Applicable

--

Occoquan Town

Contracted

--

Onancock Town

Not Applicable

--

Onley Town

Not Applicable

--

Orange Town

Contracted

--

Pennington Gap Town

Not Applicable

--

Phenix Town

Not Applicable

--

Pocahontas Town

Not Applicable

--

Port Royal Town

Not Applicable

--

Pound Town

Not Applicable

--

Pulaski Town

Not Applicable

--

Purcellville Town

Contracted

--

Remington Town

Done by County

--

Rich Creek Town

Not Applicable

--

Richlands Town

Not Applicable

--

Ridgeway Town

Not Applicable

--

Rocky Mount Town

Not Applicable

--

Round Hill Town

Contracted

--

Rural Retreat Town

Not Applicable

--

Saint Paul Town

Done by County

--

Saltville Town

Not Applicable

--

Saxis Town

Not Applicable

--

Scottsburg Town

Not Applicable

--

Scottsville Town

Not Applicable

--

Shenandoah Town

Not Applicable

--

Smithfield Town

Contracted

--

South Boston Town

Directly

--

South Hill Town

Contracted

Included in residential refuse collection fee - not available for commercial customers

Stanley Town

Not Applicable

--

Stony Creek Town

Not Applicable

--

Strasburg Town

Contracted

2.05 per month bi-weekly service

Surry Town

Not Applicable

--

Tappahannock Town

Directly

5.00 monthly

Tazewell Town

Not Applicable

--

Timberville Town

Contracted

--

Toms Brook Town

Not Applicable

--

Troutville Town

Not Applicable

--

Urbanna Town

Contracted

--

Victoria Town

Contracted

--

Vienna Town

Contracted

--

Vinton Town

Directly

--

Virgilina Town

Not Applicable

--

Wachapreague Town

Not Applicable

--

Wakefield Town

Not Applicable

--

Warrenton Town

Directly

--

Warsaw Town

Done by County

--

Washington Town

Not Applicable

--

Waverly Town

Not Applicable

--

Weber City Town

Not Applicable

--

West Point Town

Directly

--

Windsor Town

Not Applicable

--

Wise Town

Directly

--

Woodstock Town

Contracted

--

Wytheville Town

Directly

--

\hypertarget{residential-water-and-sewer-connection-and-usage-fees}{%
\chapter{Residential Water and Sewer Connection and Usage Fees}\label{residential-water-and-sewer-connection-and-usage-fees}}

The Code of Virginia § 15.2-2122 authorizes sewer connection fees to finance changes in a sewer system that improve public health. Localities may establish, construct, improve, enlarge, operate, and maintain a sewage disposal system with all that is necessary for the operation of such system. The terms under which the locality can charge a fee are defined in § 15.2-2119. In most cases, the information in this section does not include fees of service districts that are separate from local governments. For further information about these fees, refer to the Draper Aden Associates report, The 31st Annual Virginia Water and Wastewater Rate Report, 2019, found at \url{http://www.daa.com/resources/}

\hypertarget{connection-fees}{%
\section{CONNECTION FEES}\label{connection-fees}}

In this survey, we asked for the standard charges to connect a locality's pipelines to a residence. The question applies only to residential buildings, including single-family homes, townhouses, apartment buildings, and mobile homes. We asked for the combined fees, so the amount should include connection fees, availability fees, service charges, and any other fee charged by a locality.Connection fees for nonresidential structures were not surveyed because of their complexity.

\textbf{Table 21.1} provides the water and sewer connection fees for the 25 cities, 48 counties, and 90 towns that reported imposing them. Fee schedules used by localities differ, but in general, charges apply to mains, valves, and meters that are installed by the locality. When an owner or developer installs all of the necessary equipment, the charge is generally waived. The following text table lists the unweighted mean, median, and first and third quartiles for connection fees for single-family housing for cities and counties.

\begin{Shaded}
\begin{Highlighting}[]
\CommentTok{\#Text table "Residential Water and Sewer Combined Connection Fees for Cities and Counties, 2019" goes here }
\CommentTok{\#Will probably want to split it up by cities and counties as the original has done}
\end{Highlighting}
\end{Shaded}

\hypertarget{usage-fees}{%
\section{USAGE FEES}\label{usage-fees}}

\textbf{Table 21.2} lists water and sewer usage fees for 36 cities, 54 counties, and 98 towns. The fees are often multitiered with the first several thousand gallons charged at a higher unit rate and the remaining amount at a lower basis. However, the opposite charging method, a multi-tiered system with the first usage charged at a lower rate than later usage, is also used.

For localities that responded with a single fee and not a schedule, it is assumed that the fee listed applies to the standard residential connection, even though no information on meter size was available. If you have questions concerning responses given in this table, please contact the appropriate water and sewer department or authority in the locality or visit their web site if applicable.

\begin{Shaded}
\begin{Highlighting}[]
\CommentTok{\#Table 21.1 "Residential Water and Sewer Connection Fees, 2019" goes here}

\CommentTok{\#Table 21.2 "User Fees for Residential Water and Sewer, 2019"}
\end{Highlighting}
\end{Shaded}

\hypertarget{impact-fees-for-roads}{%
\chapter{Impact Fees for Roads}\label{impact-fees-for-roads}}

The Code of Virginia § 15.2-2319 authorizes localities identified by population or adjacency to certain localities (see § 15.2-2317) to assess and impose impact fees on new developments to pay all or part of the cost of reasonable road improvements attributable in substantial part to such development. Costs include, in addition to all labor, materials, machinery, and equipment for construction, (i) acquisition of land, rights-of-way, property rights, easements, and interests, including the costs of moving or relocating utilities; (ii) demolition or removal of any structure on land so acquired, including acquisition of land to which such structure may be moved; (iii) survey, engineering, and architectural expenses; (iv) legal, administrative, and other related expenses; and (v) interest charges and other financing costs if impact fees are used for the payment of principal and interest on bonds, notes, or other obligations issued by the county, city, or town to finance the road improvements (§ 15.2-2318).

Before it can adopt an enabling ordinance, the locality must establish an impact fee advisory committee (§ 15.2-2319). The locality may then delineate one or more impact fee service areas. Any impact fees collected from new development within an impact fee service area must be expended for road improvements in that impact fee service area (§ 15.2-2320).

Prior to adopting a system of impact fees, localities must conduct an assessment of road improvement needs benefitting an impact fee service area. From this needs assessment, a road improvement plan must be developed to improve existing roads and construct new roads within the impact fee service area. The improvement plan will then be incorporated into the locality's capital improvements program after a duly advertised public hearing (§ 15.2-2321).

After the adoption of the improvement program, the locality may adopt an ordinance establishing a system of impact fees to fund or recapture the cost of providing road improvements within the impact fee service areas. The ordinance will list a schedule of the impact fees for each service area (§ 15.2-2322).

Section 15.2-2323 specifies that the impact fee for a specific development or subdivision must be determined prior to or at the time when the site is approved. The ordinance must specify that the payment of fees be in one lump sum or through installments at a reasonable rate of interest for a fixed number of years.

The 2007 transportation funding legislation {[}House Bill 3202 (Chapter 896){]} authorized localities with established urban transportation service districts to impose additional impact fees subject to certain restrictions (§ 15.2-2320). Service districts are districts created within a locality ``to provide additional, more complete or more timely services of government than are desired in the locality or localities as a whole'' (§ 15.2-2400). The urban transportation service district had to be established in accordance with § 15.2-2403.1 in those counties which met the definition of urban county -- ``any county with a population of greater than 90,000, according to the United States Census of 2000, that did not maintain its roads as of January 1, 2007'' (§ 15.2-2403.1). The counties have to maintain the roads within the district.

The 2007 law applied only to counties with urban transportation service districts and had to be exercised in areas of the county outside of already established urban transportation service districts in parcels zoned agricultural that were being subdivided for by-right residential development. Also, the authority for the article expired on December 31, 2008 for any locality that had not established an urban transportation service district and adopted an impact fee ordinance in the new area by that date.

The law permits urban counties with existing urban transportation service districts to create new impact fee service areas. The locality must include within its capital improvements plan estimates of costs for public facilities necessary to serve residential uses. Such public facilities include but are not limited to: (i) roads, bridges, and signals; (ii) storm water and flood control facilities; (iii) parks, open space, and recreation areas; (iv) public safety facilities; (v) primary and secondary schools; (vi) libraries and related facilities (§ 15.2-2320). Only Stafford County reports having used this authority to impose new fees. \textbf{Table 22.1} lists four counties and one city that reported using impact fees.

\begin{Shaded}
\begin{Highlighting}[]
\CommentTok{\#Table 22.1 "Impact Fees For Road Improvement, 2019" goes here}
\end{Highlighting}
\end{Shaded}

\hypertarget{public-rights-of-way-use-fees}{%
\chapter{Public Rights-of-Way Use Fees}\label{public-rights-of-way-use-fees}}

The Code of Virginia § 56-468.1 authorizes certain localities to charge rights-of-way use fees for the use of publicly owned roads and property by certified telecommunication firms. Cities and towns whose public streets are not maintained by the Virginia Department of Transportation (VDOT), as well as any county that has chosen to withdraw from the secondary system of state highways (currently only Arlington and Henrico counties), may impose a public rights-of-way use fee by local ordinance. This fee is in exchange for the use of the locality's lands for electric poles or electric conduits by certified providers of telecommunications services.

The provider collects the use fee on a per access line basis by adding the fee to each end-user's monthly bill for local exchange telephone service (§ 56-468-1.G). The fee must be stated separately on the phone bill.

The fee is calculated each year by VDOT based on information about the number of access lines and footage of new installation that have occurred in the reporting localities. Based on this information, VDOT uses a formula to calculate the monthly fee per access line for participating localities. Starting July 1, 2019, the fee was \$1.20 per access line. Information about the rights-of-way use fee can be obtained from VDOT at: \url{http://www.virginiadot.org/business/row-usefee.asp}. The Code(§ 56-468.1.I) also permits any locality which had a franchise agreement or ordinance prior to July 1, 1998 to ``grandfather'' in the prior agreement provided that the county, city, or town does not discriminate among telecommunications providers and does not adopt any additional rights-of-way practices that do not comply with current laws.

\textbf{\emph{Table 23.1}} lists the localities that report having a rights-of-way agreement or a prior agreement that has been grand-fathered. The information is based on the Cooper Center's 2019 survey. The text table below summarizes the results:

\begin{Shaded}
\begin{Highlighting}[]
\CommentTok{\#Text table "Public Rights{-}of{-}Way Use Fees, 2019" goes here}

\CommentTok{\#Table 23.1 "Localities Imposing Public Rights{-}of{-}Way Use Fees, 2019*" goes here}
\end{Highlighting}
\end{Shaded}

\begin{itemize}
\tightlist
\item
  In years prior to 2009 this table was based on information provided by the Virginia Department of Transportation. The current table uses data based on responses to the Cooper Center's survey. To compare survey responses with VDOT information, refer to \url{http://virginiadot.org/business/row-usefee.asp}
\end{itemize}

\hypertarget{cash-proffers-fy-2018}{%
\chapter{Cash Proffers FY 2018}\label{cash-proffers-fy-2018}}

In Virginia proffers are permitted for conditional zoning, ``whereby a zoning reclassification may be allowed subject to certain conditions proffered by the zoning applicant for the protection of the community that are not generally applicable to land similarly zoned.'' (Code of Virginia, §§ 15.2-2296 through 15.2-2302). The Code § 15.2-2297 authorizes zoning ordinances to include voluntary proffers ``in writing, by the owner, of reasonable conditions, prior to a public hearing before the governing body, in addition to the regulations provided for the zoning district or zone by the ordinance, as a part of a rezoning or amendment to a zoning map'' provided that the rezoning itself gives rise to the needed conditions.

Eligibility requirements are listed in § 15.2-2298 and § 15.2-2303. Section 15.2-2298 gives localities the authority to accept proffers if: (1) the locality's growth rate met or exceeded 10 percent in the last decennial census (2010); (2) the locality is a city which adjoins another city or county that had a growth rate that met or exceeded 10 percent in the last decennial census; (3) any towns located within counties that had a growth rate that met or exceeded 10 percent in the last decennial census; and (4) any county contiguous with at least three counties that had a growth rate that met or exceeded 10 percent in the last decennial census.

Further eligibility requirements listed in § 15.2-2303 permit proffers for (1) any county with an urban county executive form of government; (2) any city next to or surrounded by a county with an urban county executive form of government; (3) any county next to a county with an urban county executive form of government; (4) any city next to or surrounded by a county contiguous to a county with an urban county executive form of government; (5) any town within a county contiguous to a county with an urban county executive form of government; and (6) any county east of the Chesapeake Bay (i.e., Accomack and Northampton counties). Finally, § 15.2-2303.1 permits proffers for any county with a 1990 census population between 10,300 and 11,000 through which an interstate highway passes. This section was meant to include New Kent County.

Proffers may entail the giving of property, property improvements, or cash. Proffers of cash payments are required to be disclosed to the Commission on Local Government in accordance with § 15.2-2303.2.There is no requirement for reporting non-cash proffers, a category that may be significant. Cash proffers are reported in an annual commission publication\footnote{Commission on Local Government, Report on Proffered Cash Payments and Expenditures by Virginia's Counties, Cities and Towns, 2017-2018. \url{https://www.dhcd.virginia.gov/cash-proffers}.}. The study presented here covers fiscal year 2018. In that period, the commission shows a total of 298 localities eligible to receive cash proffers (36 cities, 89 counties, and 177 towns). Of those, 36 reported cash proffer activity.

The following text table shows the total cash proffer revenue expended annually from 2011 through 2018.

\begin{Shaded}
\begin{Highlighting}[]
\CommentTok{\#Text table "Total Cash Proffer Revenue Expended, Fiscal Years 2010 to 2017" goes here.}
\end{Highlighting}
\end{Shaded}

The following text table shows the relative importance of the various types of cash proffer revenue expended in fiscal year 2018. Road improvements accounted for the most important use (38.1 percent). Other important uses schools (30.0), and fire and rescue/public safety (14.4). \textbf{\emph{Table 24.1}} lists fiscal year 2018 cash proffer revenue collected and expended by locality and purpose.

\begin{Shaded}
\begin{Highlighting}[]
\CommentTok{\#Text table "Relative Importance of Various Types of Cash Proffers Expended in FY 2018"}
\CommentTok{\#Pull the relevant data here}
\CommentTok{\#Will probably need to calculate the percent of total for the type of proffer here}
\CommentTok{\#Turn it into a table here}
\end{Highlighting}
\end{Shaded}

\begin{Shaded}
\begin{Highlighting}[]
\CommentTok{\#Recreation of "Table 24.1 Total Cash Proffer Revenue Collected and Expended by Purpose, by Locality, FY 2018" goes here}
\end{Highlighting}
\end{Shaded}

\hypertarget{virginia-enterprise-zone-program-2018}{%
\chapter{Virginia Enterprise Zone Program 2018}\label{virginia-enterprise-zone-program-2018}}

\hypertarget{introduction-1}{%
\section{INTRODUCTION}\label{introduction-1}}

This section on the Virginia Enterprise Zone Program is included because of its relevance to local taxation. Along with state grants, local enterprise zones (EZ) receive tax breaks and other incentives from local governments that must be in accordance with state and local tax law. The program is administered by the Virginia Department of Housing and Community Development (VDHCD). Each year VDHCD produces a summary report about the enterprise zone program. The current report, Virginia Enterprise Zone Program Grant Year 2018 Annual Report, has not yet been added to the web. The description that follows is based on that report.

\hypertarget{purpose-for-the-program}{%
\section{PURPOSE FOR THE PROGRAM}\label{purpose-for-the-program}}

The Virginia Enterprise Zone Program was created in 1982 to form a partnership between state and local governments to stimulate job creation, private investment, and revitalization of distressed Virginia localities. The act focused on state and local tax credits to help areas designated as enterprise zones. Cities and counties that applied for, and were granted the designation, were able to receive tax credits for businesses situated in the zones. Currently, there are 46 designated enterprise zones in Virginia.

In 2005 the General Assembly passed the Enterprise Zone Grant Act (§ 59.1-538), modifying the program to transition from tax credits to grants. A zone will receive an initial ten-year designation period, with two five-year renewals possible (§ 59.1-542.E). In addition, the number of zones will be reduced to 30 as many of the older zones expire.

The program is meant to target areas which have the greatest need and in which the greatest impact will be made. Consequently, the ranking of applications requires that 50 percent of an application's suitability rest on a given measure of local economic distress. The application ranks the locality over the most recent three-year period for its average unemployment rate, its average median adjusted gross income on all returns, and the average percentage of public school students receiving free or reduced-price lunches.

Only cities and counties can apply for the zone designation (§ 59.1-542). Towns are considered part of the county acreage. Cities and counties can jointly apply for designation, provided that the proposed zone meets program standards. A locality can choose to put a zone where it best fits local economic development needs. There may be three zones per locality and each zone may be composed of three non-contiguous areas.

\hypertarget{program-grants}{%
\section{PROGRAM GRANTS}\label{program-grants}}

There are two grants associated with the program: job creation grants and real property investment grants. Job creation grants are supposed to encourage the creation of higher quality jobs (§ 59.1-547). If a business within the zone meets a certain job creation threshold, provides health benefits and pays at least 175 percent of the federal minimum wage for the positions under consideration, it can receive a grant of up to \$500 per year for each position. A business that meets all the above conditions and pays at least 200 percent of the federal minimum wage can receive up to \$800 per year for each position.

Real property investment grants are meant to encourage creation or renovation of facilities within the enterprise zone (§ 59.1-548). The grants may be applied to commercial, industrial or mixed-use buildings, paying up to 20 percent of the cost of qualifying real property. For property investments of less than \$5 million, grants of up to \$100,000 per building or facility are available for qualifying real property. For property investments of \$5 million or more, grants may reach \$200,000 for qualifying property. Qualifying real property generally includes costs associated with the physical preparation and physical items such as excavation, grading, paving, driveways, roads, sidewalks, demolition, painting, sheetrock, carpentry and more. Costs that do not qualify include those for furnishings, appraisal, legal services, closing services, insurance and more.

\hypertarget{local-incentives}{%
\section{LOCAL INCENTIVES}\label{local-incentives}}

In addition to the state grants are the incentives provided by localities to businesses within enterprise zones. A locality may offer any incentive as long as it is permissible under federal and state law and as long as it is applied uniformly within the zone (§ 59.1-543). Incentives may include reduced property taxes, both real and personal, within the zone, partial exemptions for rehabilitated real estate within the zone, reduced permit and user fees, and more.

The current edition of Tax Rates does not carry a table listing the local incentives in enterprise zones for 2018 because the information is provided in the appendix of VDHCD's annual report. The following text table lists the years in which the current zones are scheduled to expire.

\begin{table}

\caption{\label{tab:unnamed-chunk-2}Year Enterprise Zones (EZ) Are Scheduled to Expire}
\centering
\begin{tabular}[t]{r|r}
\hline
Year & Number\\
\hline
0 & 0\\
\hline
\end{tabular}
\end{table}

Source: Virginia Department of Housing and Community Development, Grant Year 2018 Annual Report: Virginia Enterprise Zone Program. Provided by the DHCD to the author.

*The information for this section came from the Virginia Department of Housing and Community Development. See \url{http://www.dhcd.virginia.gov/index.php/business-va-assistance/startingexpanding-a-business/virginia-enterprise-zone-vez-business.html}

\hypertarget{fiscal-content-information-on-local-web-sites}{%
\chapter{Fiscal Content Information on Local Web Sites}\label{fiscal-content-information-on-local-web-sites}}

Because the web is such an inexpensive way to provide fiscal information, it has moved from being a backup source to a primary source. For that reason, we include a section in the survey asking localities to provide information on what budget, financial and tax information they carry on the web.

The first question was about the existence of a locality web site. If the answer was affirmative, then we were interested in knowing if the locality carried information about its budget, tax rates, capital programs, utilities, land book, geographic information system (GIS) mapping, and audit (technically called the Comprehensive Annual Financial Report or CAFR). There were eight questions about these topics.

\textbf{\emph{Table 26.1}} lists the answers from the respondents. The text table summarizes the fiscal content information for those localities that answered affirmatively the question of whether there was a web site.

All cities and counties have web sites. Of the towns, 108 that answered the survey had a site. Many more localities maintain a web site now than in 2003, the first year we asked for information about web sites. In that year only 18 cities, 26 counties and 19 towns reported they had a web site.

Currently, 32 cities and 79 counties, about three-fourths of each, show web information on their proposed budget. Forty-nine towns reported having the proposed budget on their sites. Higher numbers of cities, counties, and towns reported showing adopted budgets on the web, with 37 cities, 89 counties, and 80 towns reporting listing them.

Large majorities of cities (36), counties (91), and towns (94) with web sites showed tax rates. Utility rate schedules were shown by 34 cities, 49 counties and 87 towns. Not all jurisdictions maintain their own systems, a fact that should be considered in evaluating web sites. Capital improvement programs are shown by 35 cities, 54 counties, and 28 towns. In many cases capital programs may be reported as part of the adopted budget instead of as a separate category.

\begin{Shaded}
\begin{Highlighting}[]
\FunctionTok{library}\NormalTok{(tidyverse)}
\end{Highlighting}
\end{Shaded}

\begin{verbatim}
## -- Attaching packages --------------------------------------- tidyverse 1.3.0 --
\end{verbatim}

\begin{verbatim}
## v ggplot2 3.3.3     v purrr   0.3.4
## v tibble  3.1.1     v dplyr   1.0.5
## v tidyr   1.1.3     v stringr 1.4.0
## v readr   1.4.0     v forcats 0.5.0
\end{verbatim}

\begin{verbatim}
## -- Conflicts ------------------------------------------ tidyverse_conflicts() --
## x dplyr::filter() masks stats::filter()
## x dplyr::lag()    masks stats::lag()
\end{verbatim}

\begin{Shaded}
\begin{Highlighting}[]
\FunctionTok{library}\NormalTok{(knitr)}
\end{Highlighting}
\end{Shaded}

\begin{table}

\caption{\label{tab:unnamed-chunk-2}Resources Available on Locality Websites, 2019}
\centering
\begin{tabular}[t]{r|r|r|r|r}
\hline
Item & Cities & Counties & Towns & Total\\
\hline
0 & 0 & 0 & 0 & 0\\
\hline
\end{tabular}
\end{table}

By law, all localities must provide public access to the land book --- the local listing of individual land parcels by owner and the assessed value of the land and improvements. Such access is greatly enhanced when it can be provided on the web. A majority of cities (34) and counties (84) now provide convenient web access to this important information. Most with web access also provide corollary geographic information system (GIS) mapping. Relatively few towns reported web inclusion of the land book or GIS mapping, a reflection of the fact that towns generally rely on their host counties for real property assessments.

Almost three-fourths of the cities and over half the counties with web sites reported showing their latest comprehensive annual fi nancial report (CAFR). Thirty-two cities, 61 counties, and 44 towns reported doing so.

\begin{table}

\caption{\label{tab:unnamed-chunk-3}Table 26.1 Fiscal Content Information on Local Websites, 2019}
\centering
\begin{tabular}[t]{r|r|r|r|r|r|r|r|r|r}
\hline
Locality & Have A Website? & Proposed Budget & Adopted Budget & Tax Rates & Capital Improvement Programs & Utility Charges & Landbook Information & GIS Mapping & Audit (CAFR)\\
\hline
0 & 0 & 0 & 0 & 0 & 0 & 0 & 0 & 0 & 0\\
\hline
\end{tabular}
\end{table}

\hypertarget{references}{%
\chapter*{References}\label{references}}
\addcontentsline{toc}{chapter}{References}

This online edition of the \emph{Virginia Local Tax Rates} survey report was prepared with the \textbf{bookdown} package \citep{R-bookdown}, which is built on top of R Markdown and \textbf{knitr} \citep{xie2015} using the R programming language \citep{R-base}.

\hypertarget{refs}{}
\begin{CSLReferences}{0}{0}
\end{CSLReferences}

\hypertarget{appendix-appendix}{%
\appendix}


\hypertarget{tax-rates-questionnaire}{%
\chapter{2019 Tax Rates Questionnaire}\label{tax-rates-questionnaire}}

\begin{Shaded}
\begin{Highlighting}[]
\CommentTok{\# questionnaire}
\end{Highlighting}
\end{Shaded}

\hypertarget{list-of-respondents-and-non-respondents}{%
\chapter{List of Respondents and Non-Respondents}\label{list-of-respondents-and-non-respondents}}

\begin{Shaded}
\begin{Highlighting}[]
\CommentTok{\# List of Respondents and Non{-}Respondents to 2019 Tax Rates Questionnaire$\^{}a$}
\end{Highlighting}
\end{Shaded}

\hypertarget{secn:Appendix-C}{%
\chapter{Percentage Share of Total Local Taxes from Specific Sources, FY 2018}\label{secn:Appendix-C}}

\begin{Shaded}
\begin{Highlighting}[]
\CommentTok{\# Percentage Share of Total Local Taxes from Specific Sources, FY 2018}
\end{Highlighting}
\end{Shaded}

\hypertarget{population-estimates-for-virginias-cities-counties-and-towns-july-1-2018}{%
\chapter{Population Estimates for Virginia's Cities, Counties, and Towns, July 1, 2018}\label{population-estimates-for-virginias-cities-counties-and-towns-july-1-2018}}

\begin{Shaded}
\begin{Highlighting}[]
\CommentTok{\# Population Estimates for Virginia\textquotesingle{}s Cities, Counties, and Towns, July 1, 2018}
\end{Highlighting}
\end{Shaded}


  \bibliography{book.bib,packages.bib}

\end{document}
